\documentclass[12pt]{article}
\usepackage[pass]{geometry}
\usepackage{amssymb}
\usepackage{amsmath}
\usepackage{amsthm}
\usepackage{mathrsfs}
\usepackage{mathtools}
\usepackage{tikz-cd}

\newtheorem{thm}{Theorem}[section]
\newtheorem{lem}[thm]{Lemma}
\newtheorem{cor}[thm]{Corollary}
\newtheorem{defn}[thm]{Definition}

\newcommand\defeq{\mathrel{\overset{\makebox[0pt]{\mbox{\normalfont\tiny\sffamily def}}}{=}}}
\newcommand\note{$\longrightarrow$\ \textbf}

\title{A Partial Characterization of $\square_\kappa$ for Plus-One Premice}
\author{Andreas Stewart Voellmer}
\date{}

\begin{document}
\pagenumbering{gobble}
\maketitle


\begin{center}
A dissertation submitted in partial satisfaction of the\\

\smallskip

requirements for the degree of\\

\smallskip

Doctor of Philosophy\\

\smallskip

in\\

\smallskip

Logic and Methodology of Science\\

\smallskip

in the \\

\smallskip

Graduate Division\\

\smallskip

of the\\

\smallskip

University of California, Berkeley\\

\bigskip

\bigskip

\bigskip

\bigskip

Committee in Charge:\\

\bigskip

Professor John R. Steel\\

Professor Leo Harrington\\

Professor Maciej Zworski\\

\bigskip

\bigskip

\bigskip

\bigskip

\bigskip

Spring 2017





\end{center}

\newpage\null\newpage

\pagenumbering{arabic}

\begin{center}

\textbf{Abstract}\\

\bigskip

A Partial Characterization of $\square_\kappa$ for Plus-One Premice\\

\smallskip

by\\

\smallskip

Andreas Stewart Voellmer\\

\smallskip

Doctor of Philosophy in Logic and Methodology of Science\\

\smallskip

University of California, Berkeley\\

\smallskip

Professor John R. Steel, Chair\\



\end{center}

\bigskip

We develop and refine the theory of plus-one premice, first introduced by Neeman and Steel in \cite{PIPM} and \cite{FSPIPM}.  This culminates in a Condensation Lemma for iterable plus-one premice.  We then apply Condensation to the construction of $\square_\kappa$ sequences in these premice; this is similar to Schimmerling and Zeman's $\square_\kappa$ construction in \cite{zeman square proof}, but the presence of long extenders complicates both the techniques and the results.  Our main result is that for plus-one premice with finitely many long generators, $\square_{\kappa , 2}$ holds exactly when $\kappa$ is neither subcompact nor the successor of a $1$-subcompact cardinal.

\newpage
\pagenumbering{roman}
\tableofcontents

\newpage

\section{Introduction}
\pagenumbering{arabic}

A long-standing goal of inner model theory has been the construction of extender models $L[E]$ which include long extenders on their sequences.  There are a number of technical challenges that arise when one moves from short extenders to long extenders; many basic facts, such as the Comparison Lemma for premice with short extenders, seem to fail in the more general long extender context.  Neeman and Steel discovered a partial solution to these problems with their theory of plus-one premice, first presented in \cite{PIPM} and \cite{FSPIPM}.  In the present document we develop the fine structure theory of plus-one premice in detail.  Section 2 includes the basic definitions and lemmas about preservation of premousehood under various embeddings.  Sections 3, 4, and 5 develop the theory of iteration trees and comparison needed for the Condensation Lemma, which constitutes the entirety of Section 6.  The remaining sections constitute a partial characterization of the combinatorial principle $\square_\kappa$ in plus-one premice (see Section \ref{prelim section} for the definition of $\square_\kappa$).\\

The importance of $\square_\kappa$ in fine structural inner models lies partly in its applications to determining the consistency strength of various set-theoretic hypotheses; in many cases a principle, such as a forcing axiom, is known to imply failures of $\square_\kappa$.  Understanding the extent of $\square_\kappa$ in inner models can then help us gauge the ``distance" between those inner models and the outer universe in which the principle holds.  Additionally, characterization of $\square_\kappa$ in inner models is often viewed as a ``test question" for how well set theorists understand these models.\\

Schimmerling and Zeman proved in \cite{zeman square proof} that for extender models constructed with short extenders, $\square_\kappa$ holds at exactly those $\kappa$ which are not subcompact (see Section \ref{prelim section} for the definition of subcompactness).  Our main result is that for iterable plus-one premice with finitely many long generators, $\square_{\kappa, 2}$ holds at exactly those $\kappa$ which are neither subcompact nor the successor of a $1$-subcompact cardinal.\\

The construction we give is in the general context of plus-one premice with arbitrarily many long generators, but at certain points we require the assumption that the largest generator of one of our long extenders is a successor generator; new techniques are needed for limit generators.  It is our hope that the present proof can serve as the ``successor case" in a fully general $\square_\kappa$ construction for plus-one premice, in addition to being a complete proof in the finite-generator context, where the ``successor case" is the only case.\\

It also follows from our techniques that $\square_\kappa$ holds for all $\kappa < \mu$ in a plus-one premouse $M$, where $\mu$ is the least measurable cardinal of $M$; this is because there are no pluripotent levels below $\mu$, so no protomice arise in the construction (see Section \ref{prelim section}), and one can essentially repeat the well-known proof of $\square_\kappa$ for all $\kappa$ in $L$, due to Jensen.\\

\section{Preliminaries}

In this section we will present the basic finestructural notions used throughout the proof.  We then define \textit{potential premice} and describe the (easy) conditions under which potential premousehood is preserved by embeddings.  Finally we will define \textit{premice} simpliciter, and describe the (rather complex) conditions under which premousehood is preserved.\\

\subsection{Fine Structure}

We begin by establishing some notation.  $\mathcal{P}(X)$ is the powerset of $X$, and $|X|$ is the cardinality of $X$.  $\textbf{ON}$ is the class of ordinal numbers, and $o(M) = \textbf{ON} \cap M$ for any set $M$.  We assume familiarity with the fine structure theory for short extender premice, as presented in, e.g., \cite{zeman book} and \cite{ZS finestructure}.\\

 We define extenders as in \cite{ZS finestructure}.\\
 
\begin{defn} \label{extender}

For an acceptable $J$-structure $M$, $E = \langle E_a \mid a \in [ \nu ]^{< \omega} \rangle$ is a $( \kappa , \nu )$-\textit{extender over $M$ with critical points $\langle \mu_a \mid a \in [ \nu ]^{< \omega } \rangle$} if the following conditions hold:\\
 
 (1) (Ultrafilter property) For each $a \in [ \nu ]^{< \omega }$ we have that $E_a$ is an ultrafilter on the set $\mathcal{P}( [ \mu_a ]^{|a|} ) \cap M$ which is $\kappa$-complete w.r.t. sequences in $M$; moreover, $\mu_a$ is the least $\mu$ such that $[ \mu ]^{|a|} \in E_a$.\\
 
 (2) (Coherency) For $a, b \in [ \nu ]^{< \omega}$ with $a \subseteq b$ and for $X \in \mathcal{P} ( [ \mu_a ]^{|a|}) \cap M$ we have that $X \in E_a \Leftrightarrow X^{ab} \in E_b$. ($X^{ab}$ is defined below.)\\
 
 (3) (Uniformity) $\mu_{\{ \kappa \} } = \kappa$.\\
 
 (4) (Normality) Let $a \in [\nu]^{< \omega}$ and $f: [\mu_a ]^{|a|} \longrightarrow \mu_a$ with $f \in M$.  If
 
 \[
 \{ u \in [\mu_a ]^{|a|} \mid f(u) < \text{max}(u) \} \in E_a
 \]
 
 then there is some $\beta < \text{max}(a)$ such that
 
 \[
 \{ u \in [\mu_a ]^{|a \cup \{ \beta \} | } \mid f^{a , a \cup \{ \beta \} } (u) = u_\beta^{a \cup \{ \beta \} } \} \in E_{a \cup \{ \beta \} } \ .
 \]
 
 \end{defn}
 
 \bigskip
 
 We say that the \textit{space} of $E$ is sup$( \{ \mu_a  \mid a \in [ \nu ]^{< \omega } \}$.  $E$ is called \textit{short} if $space(E) = \kappa$; otherwise $E$ is \textit{long}.  The \textit{domain} of $E$, or $dom(E)$, is $\mathcal{P}( space(E))^M$.  We will also have to consider extenders which are \textit{not} total over their models, i.e., not all subsets of $space(E)$ are measured by the ultrafilters $E_a$.  In this case $dom(E)$ is just the collection of subsets which are measured by the $E_a$'s.  (Note that if $\mu_a = \mu_b$, then $E_a$ and $E_b$ measure exactly the same subsets of it.)\\
 
 Let $b = \{ \beta_1 < \ ... \ < \beta_n \}$, and let $a = \{ \beta_{j_1} < \ ... \ < \beta_{j_m} \} \subseteq b$.  If $u = \{ \xi_1 < \ ... \ < \xi_n \}$ then we write $u_a^b$ for $\{ \xi_{j_1} < \ ... \ < \xi_{j_m} \}$; we also write $u_{\beta_i}^b$ for $\xi_i$.  If $X \in \mathcal{P}( [ \mu_a ]^{|a|} )$, then we write $X^{ab}$ for $\{ u \in [ \mu_b ]^{|b|} \mid u_a^b \in X \} $.  Finally, if $f$ has domain $[ \mu_a ]^{|a|}$ then we write $f^{a , b}$ for that $g$ with domain $[ \mu_b ]^{|b|}$ such that $g (u) = f ( u_a^b )$.\\
 
 $i_E$ is the ultrapower embedding associated with $E$ (see, e.g., \cite{ZS finestructure}).  We write ``$x = [ a , f]_E^M$" to denote that $x$ is the object in the ultrapower with representing function $[a, f]$.  Let $\kappa_E = crit(i_E)$ and $\lambda_E = i_E(\kappa_E)$.  For $\xi < \nu$, $E \restriction \xi$ is the $( \kappa_E , \xi )$-extender which is the restriction of $E$ to ordinals $< \xi$, that is, $(E \restriction \xi )_a = E_a$ whenever $a \in [ \xi ]^{< \omega }$.\\
 
 Generally we will use $\vec{E}$ for the extender-sequence used to build our $J$-structures, and $G$ for the top extender.\\


We will follow the conventions of \cite{NITCIS} to describe the finestructure of the $J$-structures we work with.  $J_\alpha^A$ is the $J$-structure of height $\alpha$ constructed relative to the predicate $A$, as in \cite{ZS finestructure}.  We define the first projectum $\varrho_1(M)$ and standard parameter $p_1(M)$ of an acceptable $J$-structure $M$ as usual (see \cite{ZS finestructure}).  If $M$ is $n$-sound, we can form the $n$-th reduct, and then define $\varrho_{n+1}(M)$ and $p_{n+1}(M)$ as the first projectum \& standard parameter of the $n$th reduct.  We will adopt a convention whereby every premouse $M$ has $k(M) \leq \omega$ associated with it; this is $M$'s \textit{degree of soundness.}  More explicitly: for a $J$-structure $M$ with degree of soundness $k(M)$, we have that $M$ is $k$-sound, $\varrho(M)$ is the $k+1$th projectum of $M$, and $p(M)$ is the $k+1$th standard parameter.  $M | \langle \mu , l \rangle $ is $M$ cut at $\mu$ (keeping the last extender predicate at $\mu$), considered as an $l$-sound premouse (i.e., $k(M | \langle \mu , l \rangle ) = l$).  $M || \langle \mu , l \rangle$ is the same, but considered as a structure with \textit{no} top extender predicate.\\

A \textit{$Q$-property} is one which is preserved by embeddings which are $\Sigma_1$-elementary and cofinal.  In other words, $Q$-properties can be expressed by a formula of the form $\forall x \in \textbf{ON} \ \exists y \in \textbf{ON} \ (y > x \  \& \ \phi (y))$, where $\phi$ is $\Sigma_1$.  Such a formula is called a \textit{$Q$-formula}.\\

A $\Sigma_1^{(n)} (M)$-formula is a $\Sigma_1$-formula over the $n$-th standard reduct of $M$.  (This is equivalent to the notion of an $r \Sigma_{n+1} (M)$-formula; see \cite{zeman book} for more details.)  Similarly, $\Sigma_0^{(n)} (M)$ and $Q^{(n)} (M)$ formulas are $\Sigma_0$- or $Q$-formulas over the $n$-th standard reduct.\\

$Hull_{n+1}^M (X)$ is the unique substructure of $M$ with universe $=$ the set of all $a$ such that $a =$ unique $b$ such that $M \models \phi [ b , s]$ for some $s \in X^{< \omega}$ and some $\Sigma_1^{(n)}$-formula $\phi$ in the language of $M$ (which will usually be the language of premice defined below, but will on occasion be the language of coherent structures-- this will always be clear from the context).\\

$\mathcal{H}_{n+1}^M (X) =$ transitive collapse of $Hull_{n+1}^M (X)$.\\

\begin{defn} \label{solidity}
Let $M$ be a $J$-structure (in the language of premice or coherent structures-- see below), and $p_{n+1} (M) = \langle \alpha_0 ... \alpha_k \rangle$; then\\

$a)$ $M$ is $n+1$-solid at $\alpha_i$ iff $Th_{n+1}^M (\alpha_i \cup \{ \alpha_0 ... \alpha_{i-1} \} ) \in M$,\\

$b)$ $M$ is $n+1$-universal iff $\mathcal{P} ( \varrho_{n+1} (M) ) \cap M \subseteq \mathcal{H}_{n+1}^M (\varrho_{n+1} (M) \cup p_{n+1} (M))$.
\end{defn}

Our official definition of solidity is the existence of a certain $\Sigma_{n+1}$ theory in the model.  This is easily seen to be equivalent to the presence of the corresponding $\Sigma_{n+1}$-hull in the model; and this latter characterization is our official definition of a solidity witness.\\

\begin{defn} \label{solidity witness}
Let $M$ be a $J$-structure (in the language of premice or coherent structures-- see below), and $p = p_{n+1} (M) = \langle \alpha_0 ... \alpha_k \rangle$; fix $\alpha_i$ and let $q = \alpha_0 ... \alpha_{i-1}$.  Then\\

$a)$ The standard solidity witness for $\alpha_i$ is $W_{\alpha_i , p}^M = \mathcal{H}_{n+1}^M ( \alpha_i \cup \{ \alpha_0 ... \alpha_{i-1} \}$; note that, for $\xi_1 ... \xi_\ell < \alpha_i$ and $\bar{q} = $ the collapse of $q$ in this hull, we have

\[
M \models \phi ( \xi_1 ... \xi_\ell , q) \ \Leftrightarrow \ W_{\alpha_i , p}^M \models \phi ( \xi_1 ... \xi_\ell , \bar{q}) \ .
\]

$b)$ A generalized solidity witness for $\alpha_i$ is a pair $\langle Q , r \rangle$, where $Q \supset \alpha_i$ is an acceptable $J$-structure, $r \in Q$ is a finite set of ordinals, and for $\xi_1 ... \xi_\ell < \alpha_i$ we have

\[
M \models \Phi_n ( i, \xi_1 ... \xi_\ell , q ) \  \rightarrow \ Q \models \Phi (i, \xi_1 ... \xi_\ell , r ) \ .
\]

\end{defn}

The following well-known lemma explains the importance of solidity witnesses:

\begin{lem}
\textbf{(from \cite{zeman book})}

Suppose $M$ is solid, $p \in M$ is a finite set of ordinals such that $p \cap \alpha = \emptyset$ and $M = h_{n+1}^M ( \alpha \cup p )$ for some $n \in \omega$.  If every $\nu \in p$ has a generalized solidity witness with respect to $M$ and $p$, that is, an element of $M$, then $p = p (M) - \alpha$.
\end{lem}

\smallskip

The following lemma shows that generalized solidity witnesses are preserved under embeddings, and their existence is equivalent to the existence of standard witnesses.\\

\begin{lem}
\textbf{(from \cite{zeman book})}

$M$ contains the standard solidity witness $W_{\alpha_i , p}^M$ for $\alpha_i$ if and only if $M$ contains some generalized solidity witness $\langle Q , r \rangle$ for $\alpha_i$.  Furthermore, the statement ``$\langle Q , r \rangle$ is a generalized solidity witness for $\alpha_i$ with respect to $M$ and $p$" is $\Pi_1^{(n)}$.
\end{lem}

Hence, the property of being a generalized solidity witness is preserved upwards under $\Sigma_1^{(n)}$-maps and downward under $\Sigma_0^{(n)}$-maps.  Thus solidity is the key to proving that the standard parameter of $M$ is preserved under ultrapower embeddings.\\






\begin{defn} \label{elementary}
Let $M$ and $N$ be $J$-structures with $n = k(M) = k(N)$, and $\pi: M \longrightarrow N$; then $\pi$ is \textit{weakly elementary} iff\\

\indent $(1)$ $\pi$ is $\Sigma_1^{(n)}$ elementary on some set $X$ cofinal in $\varrho_n (M)$,\\

\indent $(2)$ $\pi ( p_k (M)) = p_k (N)$ for all $k \leq n$,\\

\indent $(3)$ $\pi ( \varrho_k (M)) = \varrho_k (N)$ for $k < n$, and $sup ( \pi `` \varrho_n (M)) \leq \varrho_n (N) \leq \pi (\varrho_n (M))$.\\

\end{defn}


We say that $\pi$ is \textit{elementary} iff $\pi$ is weakly elementary and fully $\Sigma_1^{(n)}$ elementary.\\

If $k(M) \neq k(N)$, then we say $\pi : M \longrightarrow N$ is (weakly) elementary iff it is a (weak) near $n$-embedding, where $n = min(k(M), k(N))$.  In this sense, the natural map from the core of $M$ to $M$ is elementary.  So are the maps along branches of iteration trees, and copy maps.\\

Letting $n = k(M)$, the core of $M$ is the transitive collapse $C_{n + 1} (M)$ of $Hull_{n+1}^M (\varrho (M) \cup \{ p (M) , q \} )$, where $q$ is as above, with $k (C_{n+1} (M))$ set to $n + 1$.  We say the core exists if it behaves well in the sense that it agrees with $M$ up to their common value for $\varrho (M)^+$, and $p_{n + 1} (C_{n+1} (M))$ is solid.  For the plus-one premice we work with, the $1$-core requires a slightly different definition; see Definition \ref{alternate 1-core} below.\\

\begin{defn} \label{initial segment}
If $P$ and $Q$ are $J$-structures, then $P \unlhd Q$ iff there are $\mu$ and $l$ such that $P = Q | \langle \mu , l \rangle$.  Also, $P \lhd Q$ iff $P \unlhd Q$ and $P \neq Q$.\\
\end{defn}

Thus if $P$ and $Q$ have the same universe, but $k (P) < k (Q)$, then $P \lhd Q$.  Also, if $P$ is passive and $Q$ is active at $o(P)$, then it is not the case that $P \unlhd Q$.  So for example, if $Q$ is active, then $Q || o(Q) \ntrianglelefteq Q$, where $Q || o(Q)$ is $Q$ with its last extender predicate removed.\\





\subsection{Potential Plus-One Premice}

Our definitions of potential plus-one premouse and of plus-one premouse are very close to those in \cite{PIPM} and \cite{FSPIPM}, but with one small difference in each definition.  For potential plus-one premice we have replaced Steel's `weak ISC' with the yet weaker `short ISC'.  (For full plus-one premice the weak ISC is still demanded, just as in \cite{PIPM}.)  This is because the weak ISC may not be preserved by embeddings which are $\Sigma_0$ and cofinal, while short ISC is preserved by these embeddings; thankfully, the short ISC is all we need to carry out elementary computations with these structures.  For plus-one premice, our definition goes beyond Steel's by adding a constant symbol $\dot{\gamma}$ for the largest whole initial segment in the type long-B case.  This is needed when taking $\Sigma_1$ hulls of plus-one premice, if we wish for the resulting structure to be a plus-one premouse.\\


\begin{defn} \label{ppm}
A potential plus-one premouse (or simply `potential premouse') is a J-structure $N$ constructed from a sequence $\vec{E}$ of extenders such that if $(M, G)$ is a level of $N$, and $G \neq 0$, then either:\\

	$(1)$ $G$ is a short extender over $M$, and $(M, G)$ satisfies the Jensen conditions for short extender premice as in \cite{zeman book}, Chapter 9 (for instance, $M = \text{Ult}(M, G) | (\lambda_G ^+)^{\text{Ult}(M, G)}$),\\
	
	$(2)$ $G$ is long, and\\
	
		\indent \indent $(a)$ $M = \text{Ult}(M, G) | (\lambda_G ^+)^{\text{Ult}(M, G)}$ (\textit{coherency}),\\
		
		\indent \indent $(b)$ $G \restriction \lambda_G \in M$ (\textit{short initial segment condition}),\\
		
		\indent \indent $(c)$ $G$ has a largest long generator $\nu_G = \nu_M$,\\
		
		\indent \indent $(d)$ $\dot{G}$ is the symbol for an amenable predicate over $M$ as described below (\textit{weak amenability}):\\
		
		For each $\xi < (\kappa_G^{++})^M$, let
		
\[
G_\xi = \{ (a, X) \mid a \in [ \lambda_G \cup \{ \nu_G \} ] ^{< \omega} \wedge X \in (M | \xi ) \wedge a \in i_G^M (X) \} \ .
\]
	
	Then let $\gamma_\xi$ be the least ordinal such that $G_\xi \in (M | \gamma_\xi )$.   We define our official predicate $\dot{G}$ as

\[
\dot{G}^M \defeq \{ (\gamma , a, X) \mid \gamma < o(M) \wedge \exists \xi ( \gamma_\xi \leq \gamma \wedge (a, X) \in G_\xi \} \ .
\]

\end{defn}

	
Note that the $\gamma_\xi $'s in $c)$ above all exist, and are cofinal in $o(M) = ( \lambda_G^+ )^{Ult_0 (M, G) }$: for any $\xi < (\kappa_G^{++})^M$, $(M | \xi)$ can be put in ordertype $(\kappa_G^+)^M$ inside $M$; call this enumeration $Y = \langle Y_\alpha \mid \alpha < \kappa_G^+ \rangle$.  Now in $Ult(M, G)$, we can apply Comprehension to $i_G (Y)$ to learn which of its coordinates contain which finite sets of generators $a$.  Then we must pull back this set by the short part of $G$ in order to get our desired extender-fragment $G_\xi$.  (This is why we demanded $b)$ in the definition of potential premouse.  With short extenders this pullback is not needed, but with long extenders we often need to know how ordinals $< \kappa^+$ are stretched in order to carry out routine computations.)  A standard coherency argument then gives $G_\xi \in M$.  Also, the $G_\xi$'s of this form are cofinal in $o(M)$, since any $A \subseteq \nu + 1$ in $M$ can be computed from some $G_\xi$.  (By coherency, $A = [a, f]_E^M$ for some $[a , f]$, and $A$ is rudimentary in any $G_\xi$ such that $f \in J_\xi^{\vec{E}}$.)
	
\bigskip	
	
	Potential premousehood is \textit{almost} a $Q$-property (that is, preserved under $\Sigma_0$ cofinal embeddings); the only clause which may not be preserved is totality of the top extender.\\
	
\begin{lem} \label{ppm up-pres} 
If $M$ is a potential premouse, and $i : M \longrightarrow N$ is $\Sigma_0$-elementary and cofinal, and furthermore $N$ is passive or has total top extender $H$, then $N$ is a potential premouse.  Also, if $G = \dot{G}^M$ is long, then $i (\nu_{G}) = \nu_{H}$.

\end{lem}
	
\textbf{Proof:} The Lemma for short top extenders is proved in \cite{zeman book}, so we focus on long extenders.  It is routine to express $2(a)$ and $2(b)$ of \ref{ppm} as $Q$-properties.  For $2(c)$, recall that being a generator is $\Pi_1$, so $i (\nu_{G})$ is a generator of $H$.  Now if $\nu_G < \xi < o(M)$, there is a wellorder $W$ of $\lambda_G$ of ordertype $\xi$.  It is then represented in the ultrapower as $W = [a \cup \{\nu \} , f ]_G^M$, for $a \subseteq \lambda_G$.  Then

\[
i(W) = i_{H} (i(f))(i(a), i(\nu_G))
\]

will uniformly witness that there are no $H$-generators between $i(\nu_G)$ and $i(\xi)$.  Because $i$ is cofinal, we have that $i(\nu_G)$ is the largest generator of $H$.\\

For $2(d)$, the fact that \textit{all} $\xi < (\kappa_G^{++})^N$ are measured by fragments of $H$ is actually $\Pi_2$; we are assuming totality of $H$, though, so this is not a problem. $\blacksquare$\\
	
\textbf{Remark:} Note that $\Sigma_2$ elementarity is sufficient for preservation of potential premousehood, since the $\Pi_2$ fact that the top extender is total will be preserved.  Also, if our embedding $i$ is continuous at $\kappa_G^+$ if $G$ is short, or $\kappa_G^{++}$ if $G$ is long, totality of the top extender (and hence potential premousehood) are preserved.\\
	
	
	
\subsection{Plus-One Premice}

\begin{defn} \label{weak ISC} 
A potential plus-one premouse satisfies the \textit{weak initial segment condition} (weak ISC) if $\forall \xi < \nu_G (G \restriction \xi \in M)$.
\end{defn}

A complicating feature of plus-one premice is that we do not demand the \textit{full} initial segment condition for $G$; in particular, the longest initial segment $G \restriction \nu_G$ can be missing from the model.  (Because our premice will satisfy weak ISC, this will only occur when $\nu_G$ is a limit of generators.)  In this case, however, we demand that the missing initial segment is definable over the premouse in a canonical way; we call such premice `type $Z_1$'. \\

\begin{defn} \label{type $Z_1$}
A potential plus-one premouse $(M, G)$ is type $Z_1$ iff $G$ is long, $(M , G)$ satisfies the weak ISC, and there is a short extender $F$ indexed at $\nu_G$ satisfying:\\

	1) $\lambda_G = \lambda_F$,
	
	2) $\kappa_G < \kappa_F$,
	
	3) $(\kappa_F ^+ )^M$ is not the space of an extender on the $M$-sequence,
	
	4) for cofinally many $\gamma < (\kappa_F ^+ )^M$, $i_F (E_\gamma ^M) \subseteq G$,
	
	5) $(\nu_G ^+)^{\text{Ult}(M, F)} = (\nu_G ^+)^{\text{Ult}(M, G \restriction \nu_G)} \defeq \eta$, and $\text{Ult}(M, F) | \eta = \text{Ult}(M, G \restriction \nu_G) | \eta$.
\end{defn}
	
	
If $(M,G)$ is type $Z_1$, as witnessed by $F = E_{\nu_G} ^M$, then we can define a long extender $\bar{G}$ over $M$ as follows:\\

\[
\text{For } \gamma < (\kappa_F ^+)^M \ , \  E_\gamma ^M \subseteq \bar{G} \Longleftrightarrow  i_F ^M (E_\gamma ^M) \subseteq G \ .
\]

\begin{lem}
If $M$ is a type $Z_1$ premouse, then $G \restriction \nu_G \notin M$.
\end{lem}

\textbf{Proof:} First note that $M$ and $Ult(M , \bar{G})$ agree up to their common value of $\kappa_F^{++}$.  This is because $i_F$ sends $\kappa_F^{++}$ cofinally into $(\nu_G^+)^{Ult (M , F)} = \eta$, and condition $(5)$ of $Z_1$-ness guarantees that this is the same as $Ult (M , G \restriction \nu_G ) | \eta$.  This agreement pulls down by $i_F^{-1}$ to imply that $Ult(M , \bar{G})$ agrees with $M$ up to their common $\kappa_F^{++} = \delta$.  Now, note that $\bar{G}$ collapses $\delta$, since $\bar{G}$ only has generators below $(\kappa_F^+)^M$, and so defines a surjection from $(\kappa_F^+)^M$ onto $\delta$.  Therefore $\bar{G} \notin M$.  But if $G \restriction \nu_G$ were in $M$, we could define $\bar{G}$ within a level of $M$, exactly as we defined it above.  So $\bar{G}$ would be in $M$ as well.  This proves the Lemma. $\blacksquare$\\


So $\bar{G}$ and $G \restriction \nu_G$ are both missing from $M$, but definable over $(M, G)$.\\

\begin{defn} \label{pseudoindex}
We say that $\bar{G}$ is pseudo-indexed at $\alpha$ in $L[E]$ if there is a $Z_1$ level $(M, G)$ with stretching-extender $F$ such that $\bar{G}$ is the long extender described above (namely, for $\gamma < (\kappa_F ^+)^M$, $E_\gamma ^M \subseteq \bar{G}$ iff $i_F ^M (E_\gamma ^M) \subseteq G$) and $\alpha = (\kappa_F)^{++}$.
\end{defn}

There is a parallel between the type $Z_1$ levels of long extender constructions and type $Z$ short extenders.  Recall that a short extender $E$ over $M$ is type $Z$ iff $E$ has a largest generator $\nu$ which is a limit of generators, and $(\nu^+)^{Ult(M, E)} = ( \nu^+ )^{Ult (M , E \restriction \nu )}$.  This leads to a conflict in which both $E$ and $E \restriction \nu$ ``should" be indexed at the same spot on the sequence.  More specifically, $E \restriction \nu$ is put on our sequence, and its presence on the sequence prevents $E$ from being put on afterwards, since $E \restriction \nu$ collapses $(\nu^+)^{Ult(M, E \restriction \nu )}$ and so cannot be in $Ult(M , E)$.  The standard way to resolve this is by giving preference to $E \restriction \nu$, and adding the condition that no type $Z$ extenders occur on our sequences.  Analogously, in the type $Z_1$ case described above, the presence of $F$ on the $M$-sequence prevents $G \restriction \nu$ from being put on afterwards, because $Ult ( M, G \restriction \nu ) \models$ ``$\nu$ is a cardinal", while $F$ collapses $\nu$.  We resolve this by giving preference to $F$ and adding a condition that extenders of the form $G \restriction \nu$ do not occur on our sequence.\\

The reason this conflict between $F$ and $G \restriction \nu$ arose in the first place was that we neglected to put $\bar{G}$ on our sequence, when it perhaps ``should" have been added at $(\kappa_F^{++} )^M$.  (If $\bar{G}$ were in the model already, then $G \restriction \nu$ would be added to the model along with $F$.)  More explicitly, $Ult(M , \bar{G})$ agrees with $M$ up to their common value for $\kappa_F^{++}$, but $\bar{G}$ collapses $(\kappa_F^{++})^M$.  So that would be the natural place for $\bar{G}$ to be indexed.  (Hence the terminology that $\bar{G}$ is ``pseudo-indexed" there.)  We declined to do so, however, because it is important for long extender comparison that active levels of $L[\vec{E}]$ always project to their $\lambda$ (the image of the critical point of their top extender).  The details of this can be seen in the Comparison Lemma presented below, but in brief, the issue is that iteration trees with long extenders can have generators moved along the branches.  However, the only moving generators will be the long generators, above $\lambda$ of their models.  If we know that all our mice project to their $\lambda$, this effectively gives the ordinals below $\lambda$ (which are \textit{not} moving along iteration tree branches) a measure of control over the less-well-behaved parts of the mice.\\

Any long extender mouse $(M, G)$ where $G$ has a largest long generator $\nu$ which is below $\lambda_G^+$ will necessarily project to $\lambda_G$.  By only putting long extenders with largest generators onto the sequence, we guarantee that comparison will work as desired.  However, this comes at the price of neglecting our initial segments which have no largest generators.  The way out of this difficulty is to notice that these neglected initial segments still canonically find their way into the model of their own accord, as the $\bar{G}$ of a $Z_1$ level.\\



We must also consider a different kind of initial segment, \textit{whole initial segments}, and demand that all of these are in our model:\\

\begin{defn} \label{JISC} Let $(M, G)$ be a potential plus-one premouse, $\eta < \lambda_G$, and\\

	$H = G \restriction \eta$, if $G$ is short,\\
	
	$H = G \restriction (\eta \cup \{ \nu_G \} )$, if G is long.\\
	
	We say that $H$ is \textit{whole} iff $i_H (\kappa_H) = \eta$.  The $\eta$ of a whole initial segment will be called a \textit{cutpoint} of $G$.  We say that $(M, G)$ has the Jensen Initial Segment Condition (JISC) iff whenever $H$ is a whole initial segment of $G$, then $H \in M$.
\end{defn}


\begin{defn} \label{PFS}
Let $M$ be a potential plus-one premouse; we say that $M$ has projectum-free spaces (PFS) iff whenever $G$ is a long extender on the $M$-sequence,\\

	1) if $G$ is total on $M$ and there is $k$ such that $\varrho_k (M) \leq (\kappa_G ^+)^M$, then for the least such $ k$, we have $\varrho_k (M ) \leq \kappa_G$,\\
	
	2) if $G$ is total on $M$, and $M$ is active with short last extender $H$ such that $\kappa_H = \kappa_G$, then $\varrho_1 (M) > (\kappa_G ^+ ) ^M$.
\end{defn}

\begin{defn} \label{Dodd-solid PIPM} 
For $(M, G)$ with $G$ long, we say that $(M, G)$ is Dodd-solid iff $(G \restriction \nu_G ) \in M$.
\end{defn}


We are finally ready for the definition of premouse.\\


\begin{defn} \label{premouse}
Let $M$ be a potential plus-one premouse; then $M$ is a plus-one premouse with $k(M) = 0$ iff:\\
	
	1) every proper initial segment of $M$ is fully sound, and has projectum-free spaces,\\
	
	2) every initial segment of $M$ satisfies the Jensen ISC,\\
	
	3) $M$ has projectum-free spaces,\\
	
	4) if $(N, G)$ is an initial segment of $M$ with $G$ long, then either $(N, G)$ is Dodd-solid, or $(N, G)$ is of type $Z_1$.\\
	
	A plus-one premouse $M$ with $k(M) > 0$ must in addition be $k$-sound, and we follow the conventions described above when dealing with such structures.
\end{defn}
	
\textbf{Remark:}  Every plus-one premouse satisfies the weak ISC, because both possibilities in clause $4)$ imply weak ISC.\\




\begin{defn}
The $\Sigma_0$-code of a plus-one premouse $M$ is a J-structure $M$ as above, in the language $\mathfrak{L}_0$  consisting of predicates $\dot{E}$ for the extender sequence and $\dot{G}$ for the top extender, as well as constant symbols $\dot{\nu}$ for the largest generator $\nu_G$ in the case where $G$ is long (we have $\dot{\nu}$ refer to the empty set otherwise), and $\dot{\gamma}$ for the index of the longest whole initial segment, if there is one (again $\dot{\gamma}$ refers to the empty set otherwise).
\end{defn}

\bigskip

There is a technical problem we must now consider, that requires a different definition of the $1$-core for certain plus-one premice.  If $M = ( |M| , G)$ is a $1$-sound type $Z_1$ plus-one premouse with $F = E_{\nu_G}$ the stretching-extender, it is important for all ultrapowers of $M$ to be continuous at $\kappa_F^+ = \mu$.  (If an ultrapower map were discontinuous at $\mu$, the type $Z_1$ conditions would be violated in the resulting ultrapower, and we would no longer have the weak initial segment condition.)  In the definition of type $Z_1$ premice, Definition \ref{type $Z_1$}, we demanded that $\kappa_F^+$ is not the space of an extender on the $M$-sequence; this was intended to address this problem, by ensuring that we never take ultrapowers of $M$ that are discontinuous at $\mu$.  However, we may still be forced to consider such ultrapowers if $\mu$ has $\Sigma_1^M$-cofinality equal to $\tau < \mu$, where $\tau$ is measurable in $M$.  Then $Ult_1 (M , H)$ will be discontinuous at $\mu$ if $H$ is a measure on $\tau$.  It turns out, though, that in this case we can encode the entire $\Sigma_1$-theory of $M$ as an amenable predicate on $M | \mu$.  This is because $\mu$ is regular in $M$, but we are assuming there is a $\Sigma_1$-definable cofinal map $f: \tau \longrightarrow \mu$.  For $\xi < \tau$, let $\alpha_\xi$ be least such that the value $f(\xi) = \zeta$ is witnessed by $(M|| \alpha_\xi , G \restriction \alpha_\xi)$.  Since $f$ is $\Sigma_1$-definable but missing from $M$, it must be that $\langle \alpha_\xi \ | \ \xi < \tau \rangle$ is cofinal in $o(M)$.  So $o(M)$ has $\Sigma_1$-cofinality $\tau$, as does $\mu$.  This means we can define the $1$-core of $M$ as $(M | \mu , A)$ for an appropriate amenable predicate $A$.  Then $1$-ultrapowers of this core will lift $\mu$ pointwise to its image, and preserve premousehood.\\


\begin{defn} \label{alternate 1-core}
Let $M = ( |M| , G)$ be a $1$-sound type $Z_1$ plus-one premouse with $F = E_{\nu_G}$ and $\mu = (\kappa_F^+)^M$.  Suppose $\mu$ has $\Sigma_1^M$-cofinality $\tau < \mu$, where $\tau$ is measurable in $M$, and let $f$ be a $\Sigma_1^M$-cofinal map $f : \tau \longrightarrow \mu$.  For $\xi < \tau$, let $\alpha_\xi$ be least such that the value $f(\xi) = \zeta$ is witnessed by $(M|| \alpha_\xi , G \restriction \alpha_\xi)$.  Let $\phi_n$ be the $n$-th $\Sigma_1$-formula in the language of premice, according to some standard enumeration.  Then we define $\mathcal{C}_1 (M) = ( M | \mu , A)$, where $A$ is the amenable predicate given by

\[
( \forall \xi < \tau )(\forall \beta < \varrho_1 (M)) \ A ( f (\xi) , \beta , n) \ \text{ iff } \ (M | \alpha_\xi , G \restriction \alpha_\xi ) \models \phi_n ( \beta , p_1(N)) \ .
\]
\end{defn}

By the above discussion, $A$ is an amenable predicate which encodes the $\Sigma_1$-theory of $M$ into a structure of height $\mu$, thereby resolving our discontinuity problem.\\


\textbf{Remark:}  In the context of $L[E]$-constructions with plus-one premice, we will first need to prove solidity and universality for the ``naive $1$-core"  of $M$ given by $h_1 (\varrho_1 (M) \cup p_1 (M))$.  Once this has been done we can move to the definition of $\mathcal{C}_1 (M)$ as above.\\

Neeman and Steel developed the fine structure theory of plus-one premice in \cite{FSPIPM}, culminating in a proof that the standard parameter must be solid or ``type $Z_p$".\\

\begin{defn} \label{type Z_p}
Let $M$ be a plus-one premouse; then $M$ is of type $Z_p$ iff $p_{n+1}(M) \neq \emptyset$, and letting $\alpha$ be least in $p_{n+1} (M)$, and $r = p_{n+1} (M) - ( \alpha + 1 )$, we have\\

\indent \indent $1)$ $M$ is solid at all $\beta \in r$,\\

\indent \indent $2)$ $M$ is weakly solid at $\alpha$, and\\

\indent \indent $3)$ Letting $E = \dot{E}_\alpha^M$ and $\kappa = \kappa_E$,\\

\indent \indent \indent \indent $a)$ $E$ is short,\\

\indent \indent \indent \indent $b)$ $\varrho_{n+1} (M) = (\kappa^+)^M$, and\\

\indent \indent \indent \indent $c)$ Letting $H = \mathcal{H}_{n+1}^M ( i_E `` (( \kappa^+)^M ) \cup r )$, and $\pi: H \longrightarrow M$ be the uncollapse map,\\

\indent \indent \indent \indent \indent $i)$ $M || o(H) = H || o(H)$,\\

\indent \indent \indent \indent \indent $ii)$ $\pi \restriction ( \kappa^+)^H = i_E \restriction ( \kappa^+ )^M$, and\\

\indent \indent \indent \indent \indent $iii)$ $\pi^{-1} (r) = p_{n+1} (H)$, and $p_{n+1} (H)$ is solid.\\

In this case we call $H$ the generalized core of $M$.
\end{defn}

Generalized cores will arise in our construction when we reach a model violating projectum-free spaces.  We remove the PFS violation by taking the ultrapower by the minimal extender witnessing this violation; the result will be a $Z_p$ level.\\



\bigskip

\begin{lem} \label{solidity theorem}
\textbf{(Solidity Theorem, from \cite{FSPIPM})}
Let $M$ be a $k$-sound, $(\omega_1 + 1)$-iterable plus-one premouse satisfying long extender condensation; then\\

\indent \indent $a)$ $p_{k+1} (M)$ is either solid at all $\alpha \in p_{k+1} (M)$, or of type $Z_p$,\\

\indent \indent $b)$ $p_{k+1} (M)$ is $k+1$-universal.
\end{lem}

\bigskip


Let us elaborate briefly on the fine-structural classifications of short extender mice.  We use Jensen's fine structure as presented in \cite{zeman square proof}.  For $(M, G)$ with $G$ short, we demand that all whole initial segments are elements of the model.  We divide active short mice into three types: type $A$ mice have \textit{no cutpoints}, type $B$ mice have a \textit{largest cutpoint} $\lambda^*$, and type $C$ mice have \textit{unboundedly many cutpoints} below $\lambda_G$.  In the type $B$ case, we need a constant symbol in our language for the largest cutpoint.  In fact it can be shown in this case that the largest cutpoint is on the $M$-sequence, so we add a constant symbol $\dot{\gamma}$ to the language for premice, with the interpretation that for any type $B$ premouse, $\dot{\gamma}$ refers to the index of the largest cutpoint.  In the type $A$ or $C$ cases, we just have $\dot{\gamma}$ refer to the empty set.\\

For long extender mice, the main classification is the distinction between $Z_1$ and Dodd-solid structures.  However, we must subdivide each of these types depending on the distribution of their whole initial segments.  In particular, mice with a longest whole initial segment (and therefore a nontrivial $\dot{\gamma}$ constant) will be called `long type $B$', by analogy with the short type $B$ case.  Long types $A$ and $C$ are defined analogously: they have no whole initial segments or unbounded whole initial segments in $\lambda_G$, respectively.  In total, then, we have six types of long extender mice, corresponding to every permutation of $Z_1$ or Dodd-solid on the one hand, and long type $A$, $B$, or $C$ on the other.\\

\begin{lem} \label{projecttolambda}
If $M$ is a plus-one premouse of short type $C$ or long type $C$ (either $Z_1$ or Dodd-solid), then $\varrho_1 (M) = \lambda_M$.

\end{lem}

\textbf{Proof:}  Clearly $\varrho_1 (M) \leq \lambda_M$ (this is true for all premice).  So suppose $\varrho_1 (M) = \alpha < \lambda_M$.  Since there are unboundedly many whole initial segments below $\lambda_M$, the natural factor-maps from these whole initial segments into $M$ form a direct limit system with $M$ as direct limit, and are $\Sigma_1$-elementary (see \cite{zeman square proof} section 1.4 for details).  So we can choose a whole initial segment $M'$ such that the factor-map $\sigma: M' \longrightarrow M$ has $crit(\sigma ) > \alpha$ and $p_1 (M) \subset range(\sigma)$.  But then $Th_1^N ( \alpha \cup p_1 (M)) = Th_1^{M' } (\alpha \cup \sigma^{-1} (p(M' )) )$, and this latter set is computable from $M'$, hence is an element of $M$, contradiction. $\blacksquare$\\

We will also have to consider the \textit{language for coherent structures}, which is identical to the language for premice but without the constant symbol $\gamma$.  Sometimes we will have to consider premice as objects in this language, which will lead to different standard parameters and notions of soundness.  (The reason for this, which will emerge later in the square-construction, is that in coherent structure language we have an equivalence between the standard parameter of the structure and the Dodd parameter of the extender; in other words, the finestructure of the mouse involves no more than the generators of its extender, so taking the ultrapower by that extender ``preserves information about the mouse's finestructure".  When we introduce the $\gamma$ constant, this equivalence is disrupted: the mouse's finestructure now outstrips its extender's Dodd parameter.)\\

\begin{defn} \label{dodd parameter}
Let $M$ be an active premouse, or a $J$-structure satisfying all conditions of active premousehood except that $G_M$ is not a total extender on $M$, and such that $crit (G_M) < \varrho_1 (M)$.  The Dodd parameter $d(M)$ is the $<_{lex}$-least finite set of ordinals $d$ such that $M = h^* ( \varrho_1 (M) \cup d )$, if defined.  Here $h^*$ is the canonical $\Sigma_1$-Skolem function for $M$ computed in the language of coherent structures.
\end{defn}

See \cite{zeman square proof} for more discussion of the Dodd parameter.\\





The following lemma gives conditions under which premousehood is upward preserved by embeddings.  The way to think about its various clauses is: for short extender premice, $a)$ is the ``brute force" clause, because we have $\Sigma_2$ elementarity.  $b)$ is a slightly different ``brute force" clause also-- it requires a little additional brute force than in $a)$, but in practice we will always have that little extra.  Then $c)$, $d)$, and $e)$ are the careful detailed clauses, that require some cofinality considerations, but very little elementarity.  Finally, $f)$ is the long extender ``brute force" clause.  Now in practice, we will apply this lemma by saying: if we're taking a $0$-ultrapower, we have very little brute force, but lots of control over the cofinalities, so we use $c)$, $d)$, or $e)$.  And if we're taking a $k$-ultrapower for $k \geq 1$, we lose track of the cofinalities but we have more brute force, so we apply $a)$, $b)$, or $f)$.\\

Also note that the lemma assumes that the ultrapower is already known to satisfy projectum-free spaces. Different contexts will require rather different techniques to verify PFS, and we have found it more convenient to defer the responsibility of proving PFS to those specific contexts.\\


\begin{lem} \label{upward mouse-pres}
Suppose $M$ is a premouse with top extender $G$ and $\sigma: (M, G) \longrightarrow (N, H)$ is an embedding in the language of coherent structures (with degree of elementarity specified below).  Assume $(N, H)$ satisfies projectum-free spaces.  Suppose further that one of the following holds:\\

\indent $a)$ $M$ is of short type $A$ or $B$ and $\sigma$ is $\Sigma_2$-elementary;\\

\indent $b)$ $M$ is  of short type $C$ and $\sigma$ is $Q^{(1)}$-preserving;\\

\indent $c)$ $G$ is short, $H$ is total on $N$, and $\sigma$ is $\Sigma_0$-elementary and cofinal, and maps $\lambda_G$ cofinally into $\lambda_H$;\\

\indent $d)$ $G$ is long with $M$ Dodd-solid, $H$ is total on $N$, and $\sigma$ is $\Sigma_0$-elementary and cofinal, and maps $\lambda_G$ cofinally into $\lambda_H$;\\

\indent $e)$ $G$ is long with $M$ type $Z_1$, $H$ is total on $N$, and $\sigma$ is $\Sigma_0$-elementary and cofinal, maps $\nu_G$ cofinally into its image, and maps $\lambda_G$ cofinally into $\lambda_H$.\\

\indent $f)$ $G$ is long and $\sigma$ is $\Sigma_2$-elementary.\\

Then $N$ is a premouse of the same type as $M$.  Furthermore, if $G$ is long, then $i (\nu (G)) = \nu (H)$.
\end{lem}

\textbf{Proof:}  $a)$, $b)$, and $c)$ are proven in \cite{zeman book} (Lemma 9.1.5); there is, however, one additional clause in our definition of premouse which is not a requirement for the premice in \cite{zeman book}, namely projectum-free spaces.  We have added it as a hypothesis in this lemma.  Its verification in particular cases requires additional work.\\


For the remainder of the proof we focus only on the new cases $d)$, $e)$, and $f)$, where the top extenders are long.\\

By Lemma \ref{ppm up-pres}, $N$ is a potential premouse and $i (\nu (G)) = \nu (H)$.  There are four clauses in the definition of premouse (Definition \ref{premouse}) which we must now verify.  Clearly $1)$ is preserved by any $\Sigma_0$ cofinal embedding.  For $2)$, we need to check that all whole initial segments of $H$ are elements of $N$.  Being a cutpoint is $\Pi_1$, so if $\eta$ is a cutpoint in $M$, then $\sigma ( \eta )$ is a cutpoint in $N$.  The corresponding whole initial segments will likewise be carried upward by $\sigma$.  This, combined with the fact that $\sigma$ is cofinal in $\lambda$, is enough to preserve Jensen ISC for $N$: if $M$ had unboundedly many cutpoints below $\lambda_G$, their whole initial segments will be carried upward into an unbounded set of whole initial segments of $H$; if $M$ had a largest cutpoint, its whole initial segment will be preserved, as will the fact that nothing larger is a cutpoint; and if $M$ had no cutpoints, we again have enough elementarity (and cofinality in $\lambda_H$) to preserve this fact.\\


The verification of $3)$ is the same as in the short case.  This leaves us only with $4)$.  If $M$ is Dodd-solid, then $G \restriction \nu_G \in M$, and $\sigma ( G \restriction \nu_G) = H \restriction \nu_H $, so $N$ is Dodd-solid.  And if $M$ is type $Z_1$, then we are in case $e)$ of the current lemma, so we are assuming $\sigma$ maps $\nu_G$ cofinally into $\nu_H$.  Now since $M$ is a $Z_1$ premouse, it has a short `stretching-extender' $F$ indexed at $\nu_G$, and $F$ witnesses that $\kappa_F^+$ maps cofinally into $\nu_G$.  Since $\sigma$ is continuous at $\nu_G$, it must also then be continuous at $\kappa_F^+$.  This is enough to see that $\sigma$ preserves type $Z_1$-ness of $M$: for cofinally many $\eta < \kappa_F^+$,

\[
i_F^M (E_\eta^M ) \subseteq G
\]

and hence

\[
i_{\sigma (F)}^N ( E_{\sigma (\eta )}^N ) \subseteq H \ .
\]

This finishes the proof that $N$ is a premouse of the same type as $M$. $\blacksquare$\\


We must introduce one final property which all our premice will satisfy.\\

\begin{def} \label{LEC}
Let $M$ be a plus-one premouse; then $M$ has long extender condensation iff whenever $\alpha < o(M)$, and $\gamma$ is least such that $\dot{G}^{M | \gamma}$ is long with domain $M | \alpha$, then $\varrho_1 (M | \gamma ) \leq \alpha $ and $p_1 (M | \gamma ) \subseteq \alpha $.
\end{def}

This notion plays an important role in the Closeness proof in the next section.






\subsection{Premousehood-Preservation by Hulls}



\begin{lem} \label{downward mouse-pres}
Suppose $M$ is a premouse and $\sigma: \bar{M} \longrightarrow M$ is the uncollapse embedding arising from a $\Sigma_1^{(n)}$-hull over $M$.  Also suppose that $\bar{M}$ satisfies projectum-free spaces, and that one of the following holds:\\

\indent $a)$ $n \geq 0$ and $M$ is short type $A$ or $B$,\\

\indent $b)$ $n > 0$ and $M$ is short type $C$,\\

\indent $c)$ $n \geq 0$ and $M$ is long type $A$ or $B$ (either $Z_1$ or Dodd-solid),\\

\indent $d)$ $n > 0$ and $M$ is long type $C$ (either $Z_1$ or Dodd-solid).\\


Then $\bar{M}$ is a premouse of the same type as $M$; moreover, $\sigma ( \gamma^{\bar{M}}) = \gamma^M$.\\
\end{lem}

\textbf{Proof:}  $a)$ and $b)$ are proven in \cite{zeman book} (Lemma 9.1.7), with the only additional condition for premousehood in the present context being projectum-free spaces (Clause 3 of Definition 1.7); and we are assuming this holds for $\bar{M}$.\\

For $c)$, note that $\bar{M}$ is a potential premouse, since ppm-hood is a $\Pi_2$ fact and so is downward preserved by $\Sigma_1$ hulls.  Also note that $\bar{M}$ satisfies the Jensen ISC, because in the long type $A$ case the nonexistence of cutpoints is a $\Pi_2$ fact that is downward preserved, and in the long type $B$ case the largest cutpoint $\gamma$ is in $\sigma `` (\bar{M})$, and the nonexistence of any larger cutpoint is a $\Pi_2$ fact in the parameter $\gamma$ that is downward preserved.  Additionally, observe that $cof(\nu^{\bar{M}})$ is not the space of a total extender on the $\bar{M}$-sequence, by elementarity of $\sigma$ and the fact that this holds in $M$.  We can now apply Lemma 7 of \cite{PIPM} to conclude that $\bar{M}$ is a plus-one premouse.\\

For $d)$, the embedding is $\Sigma_2$-preserving, which is enough elementarity to guarantee that $\bar{M}$ is a premouse of the same type as $M$.  $\blacksquare$\\











\section{Closeness}






In this section we introduce iteration trees, and describe the property of \textit{closeness} of extenders in these trees.  This will enable us to prove that all models produced in iteration trees are premice.\\

\subsection{Iteration Trees \& Closeness Definition}

\begin{def} \label{iteration tree}
A $k$-maximal plus-one iteration tree $\mathscr{T}$ on $M$ is an iteration tree on a plus-one premouse $M$, with $\mathscr{T}-pred (\alpha +1) = \beta$, where $\beta$ is least such that $\kappa_{E_\alpha} < \lambda_{E_\beta} $, and $M_{\alpha + 1 } = Ult_n (M_{\alpha +1}^*, E_\alpha)$, where $M_{\alpha +1}^* = M_\beta | \tau$, for $\tau$ least such that $\lambda_{E_\beta} \leq \tau$ and either $\tau = o(M_\beta )$ or $\varrho_\omega (M_\beta | \tau ) \leq space(E_\alpha)$, and $n$ is least such that $\varrho_{n+1} (M_{\alpha +1}^* ) \leq space (E_\alpha )$, or $n = k$ if there was no dropping in model or degree in $[0, \alpha + 1]_\mathscr{T}$.
\end{def}

When discussing iteration trees on a premouse $M$, we write $\kappa_\alpha$ for $\kappa_{E_\alpha}$ and $\lambda_\alpha$ for $\lambda_{E_\alpha}$.\\


\textit{Closeness} is a key property of the extenders applied in a plus-one iteration tree:\\

\begin{def} \label{closeness}
Let $M$ be a potential plus-one premouse, and $E$ an extender with $dom(E) = (M | \alpha )$ for some cardinal $\alpha$ of $M$.  We say \textit{$E$ is close to $M$} iff:\\

	1) $E$ is short, and for all finite $a \subseteq \lambda_E$, $E_a$ is $\bf{\Sigma}_1^M$ (this is called \textit{$\bf{\Sigma}_1$-amenability}), and $E_a \cap (M | \xi ) \in M$, for all $\xi < (\kappa_E^+ )^M$ (this is called \textit{weak amenability});\\
	
	2) $E$ is long, and for $\nu = \nu_E$,
	
	\indent \indent a) for $a \in [ \lambda_G \cup \{ \nu \} ] ^{< \omega}$, $E_a$ (which may now measure subsets of $\kappa_E^+$) is $\bf{\Sigma}_1^M$ (this is called \textit{$\bf{\Sigma}_1$-amenability}), and $E_a \cap (M | \xi ) \in M$, for all $\xi < (\kappa_E^{++} )^M$ (this is called \textit{weak amenability});
	
	\indent \indent b) $(\kappa_E^+ )^M$ is the space of a total extender from the $M$-sequence (this is called \textit{space agreement}).
\end{def}
	
	

The following lemma explains the importance of the $\bf{\Sigma}_1$-amenability clause of closeness:\\

\begin{lem} If $E$ is short and $\bf{\Sigma}_1$-amenable to $M$, then all $\bf{\Sigma}_1$-definable subsets of $\mathcal{P} (\kappa_E)$ in $Ult(M, E)$ are $\bf{\Sigma}_1$ in $M$.  If $E$ is long and $\bf{\Sigma}_1$-amenable to $M$ and $E \restriction \lambda_E = F$ is a set in $Ult(M, E)$, then all $\bf{\Sigma}_1$-definable subsets of $\mathcal{P} (\kappa_E^+)$ in $Ult(M, E)$ are $\bf{\Sigma}_1$ in $M$.
\end{lem}


\textbf{Proof:} First we suppose $E$ is short.  Let $A \in \bf{\Sigma}_1^{Ult(M, E)}$ be a subset of $\mathcal{P} ( \kappa_E )$ definable from parameter $x \in Ult(M, E)$, via the formula $y \in A \Longleftrightarrow \exists z \Phi ( x, y, z )$, with $\Phi$ a $\Sigma_0$ formula.  Then $x = [a , f ]_E^M$ for some $a, f$.  Assume WLOG that $\kappa \in a$, and since $[ \{ \kappa \} , id_\kappa ] $ is a representing function for $\kappa$, by ``padding its coordinates" we get a representing function $[ a , g ]_E^M =\kappa$; more explicitly, $g: \kappa^{|a|} \longrightarrow \kappa$ is the projection onto the first coordinate (which is the position occupied by $\kappa \in a$, since it is the least element of $a$).\\

Also notice that the witness $z$  to $\Phi$ can WLOG be taken to be $Ult(M, E) | \alpha$ for some $\alpha$.  Recall that the levels of $M$ are mapped by $i_E$ cofinally into the levels of $Ult(M, E)$, so in fact we can assume our witness $z$ is $i_E (M | \alpha )$ for some $\alpha$.  $A$ is now definable over $M$ as follows:


\[
y \in A \Longleftrightarrow \exists \alpha \ ( \{ d \in \kappa^{|a|} \mid \Phi ( f(d), y \cap g (d) , M | \alpha ) \} \in E_a ) \ .
\]

(By $\bf{\Sigma}_1$-amenability, $E_{a}$ is $\bf{\Sigma}_1$-definable over $M$, so the above formula is $\bf{\Sigma}_1$.)\\

Now suppose $E$ is long.  In the short proof above, we were able to easily relate subsets of $\kappa$ in $M$ with their counterparts in $Ult(M, E)$, by stretching them \& cutting down to length $\kappa$.  In the long case, it is harder to relate a given $y \subseteq \kappa^+$ with that same $y$ in the ultrapower; it is not generally true that $y = i_E (y) \cap \kappa^+$.  However, in the context of plus-one premice, we have $F = $ (short part of $E$) as an element of the ultrapower, and we can use $F$ to relate $y \in M$ to $y \in Ult(M, E)$ as follows.

We consider $F$ as a function that maps ordinals $< \kappa_E^+$ to their images in $Ult(M,E)$.  (This is easily interdefinable with our official coding of short extenders, at least when the entire extender is an element of the model.) Since $F$ is an element in the ultrapower, it has a representing function in $M$, say $ [b , f]_E^M = F $.  Let $A \in \bf{\Sigma}_1^{Ult(M, E)}$ be a subset of $\mathcal{P} ( \kappa_E^+ )$ definable from parameter $x \in Ult(M, E)$, via the formula $y \in A \Longleftrightarrow \exists z \Phi ( x, y, z )$, with $\Phi$ a $\Sigma_0$ formula.  $x$ has some representing function $ [a, h]_E^M = x$, and WLOG we can have $b \subseteq a$.  Also WLOG we have $\kappa \in a$.  $\kappa^+$ is represented in the ultrapower as $[ \{ \kappa \} , g' ]$, where $g' : \kappa \longrightarrow \kappa$ is given by $g ( \alpha ) =$ cardinal successor of $\alpha$ in $M$.  By ``padding parameters", we get a representing function $[a, g]_E^M = \kappa^+$, where $g(d) = g' ( d_0 )$ for $d \in(\kappa^+ )^{|a|}$ (where $d_0$ is the first coordinate of $d$, corresponding to $\kappa \in a$).  Note that $f (d) $ will be a function mapping $g(d)$ into $\kappa^+$, with critical point $d_0$.  So we can build a representing function $[a , k]_E^M = y$ uniformly for $y \subseteq \kappa^+$ by setting $k (d) = f(d)^{-1} (y)$.  Then $A$ is definable over $M$ as follows:

\[
y \in A \Longleftrightarrow \exists \alpha \ ( \{ d \in (\kappa^+)^{|a|} \mid \Phi ( h (d) , k(d), M | \alpha ) \} \in E_a ) \ .
\]

(By $\bf{\Sigma}_1$-amenability, $E_{a}$ is $\bf{\Sigma}_1$-definable over $M$, so the above formula is $\bf{\Sigma}_1$.) $\blacksquare$ \\

\begin{lem} If $M$ is a premouse and $E$ is weakly amenable to $M$ (either in the short or long sense), then letting $i: M \longrightarrow Ult(M , E)$ be the ultrapower map, we have $(\mathcal{P}(\kappa_E ))^{Ult(M, E)} \subseteq M$ if $E$ is short, and $(\mathcal{P}(\kappa_E^+ ))^{Ult(M, E)} \subseteq M$ if $E$ is long.
\end{lem}

\textbf{Proof:}  This is a standard argument.  $\blacksquare$\\

We can now track where the projectum of a premouse is sent by a close ultrapower.  This will be important for checking that the ultrapower is still a premouse; specifically, it is needed to verify projectum-free spaces.\\



\begin{lem} If $M$ is an $n$-sound premouse and $E$ is close to $M$, and furthermore $\varrho_{n + 1}^M \leq \kappa_E$, then $ \varrho_{n + 1}^M = \varrho_{n+1}^{Ult_n(M, E)}$.
\end{lem}

\textbf{Proof:}  Let $i: M \longrightarrow Ult_n (M , E)$ be the ultrapower map.  Then the canonical missing subset of $\varrho_{n+1}^M$ is still definable over $Ult_n(M, E)$ by elementarity of $i$.  And it is still missing, by the above lemma on weak amenability.  Thus $\varrho_{n+1}^{Ult(M, E)} \leq \varrho_{n+1}^M$.  For the other inequality, note that $\bf{\Sigma}_1$-amenability tells us that any definable subset of an ordinal $< \varrho_{n+1}^M$ in $Ult_n(M, E)$ would also be definable over $M$, contradiction. $\blacksquare$\\


Next, note that if $M$ is a premouse and $E$ is long and close to $M$, then $\varrho_{n+1}^M \neq ( \kappa_E^+ )^M$ by Clause $3)$ in the definition of premouse (projectum-free spaces) together with clause $2b)$ in the definition of closeness (space agreement).\\









\begin{lem}  Suppose $M$ is a premouse with top extender $G$, $E$ is an extender over $M$ which is close to $M$, and $(N, H) = Ult_n ((M, G), E)$ for largest possible $n$.  Suppose further that $\varrho_{n+1}^M \leq \kappa_E$.  Let $i : (M, G) \longrightarrow (N, H)$ be the ultrapower embedding.  Then\\

\indent $a)$ $(N, H)$ is a premouse of the same type as $(M, G)$,\\

\indent $b)$ if $G$ is long, then $i (\nu (G)) = \nu (H)$.
\end{lem}



\textbf{Proof:} Potential-premouse-hood is preserved upward by $\Sigma_0$ cofinal embeddings if the target structure has total top extender (Claim 1.2), which is indeed the case here:\\

\textbf{Claim 1:} $H$ is total on $N$.\\

\textbf{Proof:}  The claim will follow if we can show that $i$ is continuous at $height(dom(G))$. If $E$ is short, then $i$ is only discontinuous at points of $M$-cofinality $\kappa_E$, a limit cardinal in $M$, but $height(dom(G))$ is always a successor cardinal in $M$.  So suppose $E$ is long; by hypothesis there is a long $E'$ on the $M$-sequence with the same space (clause $b)$ of closeness).  If $G$ is long, then again there is no problem, since $height(dom(G))$ is a double successor and $i$ is necessarily continuous there (the long extenders we work with can only be discontinuous at limit cardinals and single successors in $M$).  So the only problematic case is when $G$ is short and $\kappa_G = \kappa_{E'}$.  But then by clause $2)$ of projectum-free spaces (Definition 1.5), we must have $\varrho_1^M > (\kappa_G^+)^M$.  This means that the degree $n$ of our ultrapower is $ \geq 1$.  So $i$ is at least $\Sigma_2$ elementary; and this is enough to guarantee totality of $H$ on $N$.  $\blacksquare$\\

We also have that $i ( \varrho_{n+1}^M ) = \varrho_{n+1}^N$, by a previous Lemma.  And since $i$ is an $n$-embedding, we have that\\

\indent $i ( \varrho_k (M)) = \varrho_k (N)$ for $k < n$, and\\

\indent $sup ( i `` \varrho_n (M)) = \varrho_n (N) \leq i (\varrho_n (M))$.\\

The former fact, together with elementarity of $i$, implies that $\varrho_k (N)$ is not the space of a long extender on the $N$-sequence for $k < n$.  To see this for $k = n$, note that if $i$ is continuous at $\varrho_n(M)$ then the same considerations as for $k < n$ apply.  But if $i$ is discontinuous there, then $\varrho_n(M)$ must be $\Sigma_n$-singular in $M$, which implies that it must be a limit cardinal in $M$.  But then $sup ( i `` \varrho_n (M)) = \varrho_n (N)$ is a limit cardinal in $N$, so it cannot be the space of a long extender on the $N$-sequence.\\


Now we can apply Lemma 1.10.  If $G$ is short, then the hypotheses of $c)$ of Lemma 1.10 are satisfied, since we certainly have that $i$ is continuous at $\lambda_G$.  (We have $dom(E)$ is a cardinal of $M$, so it must be $< height(M)$.)  If $G$ is long and Dodd-solid, the hypotheses of $d)$ are satisfied.  And if $G$ is long and $Z_1$, then $i$ is continuous at $\nu_G$ by the space agreement clause of closeness.  Hence the hypotheses of $e)$ are satisfied.\\

In all cases we get that $(N, H)$ is a premouse of the same type as $(M, G)$.  $\blacksquare$\\






\subsection{Closeness Proof}








\begin{lem}
All extenders used in a $k$-maximal plus-one iteration tree are close to the models they're applied to.
 \end{lem}

\textbf{Proof:}  We go by induction on the length of iteration trees.  Let $\mathscr{T}$ on $M$ be a counterexample of minimal length.  Let $M_i$ be the final model of $\mathscr{T}$, so that $F = E_i$ chosen from the $M_i$-sequence is not close to $M_{i+1}^* = M_j | \beta$ for some $\beta$.  Verifying the weak amenability clause (in the short or long case) is easy: the relevant fragment of $F$ is in $M_i$, and it must be constructed below $\lambda_j$, since this is a limit cardinal of $M_i$ above $\kappa_i$.  Thus it is also in $M_j$, since the models agree up to $\lambda_j$.\

Now for the $\bf{\Sigma}_1$-amenability clause: Suppose that some fragment $F_a$ fails to be $\bf{\Sigma}_1$ over $M_j | \beta$.\\

{\bfseries Claim:} $j < i$. (Because extenders are trivially close to the models they came from.)\\

{\bfseries Claim:} $F$ is the top extender of $M_i$.\\

\indent \indent \textbf{Proof:} If $F$ were an element of $M_i$, indexed at $\alpha$, then $F_a$ would be a $\bf{\Sigma}_1$-definable subset of $M_i | \alpha$, hence an element of $M_i$.  But $F_a$ is a subset of $\kappa_F^+$ in the short case or $\kappa_F^{++}$ in the long case; either way it must be constructed below $\lambda_j$, since this is a limit cardinal of $M_i$ above $\kappa_i$.  Thus it is also in $M_j$, since the models agree up to $\lambda_j$. $\blacksquare$\\

Now we consider a slightly different induction hypothesis.  Let $h$ be least such that $M_h$ is active and, setting $H = \dot{G}^{M_h}$, there is a $\bf{\Sigma}_1^{M_h}$ subset of $\mathcal{P} ( space ( H ) )$ which is not $\bf{\Sigma}_1$ over $M_g$, where $g$ is least such that $\lambda_g > dom ( H )$.  Call this set $X$, and the parameter in its definition $b$.  In other words: $M_h$ is the least model such that IF we had chosen its top extender as $E_h$, there would have been a $\bf{\Sigma}_1$subset of the powerset of its space which fails to be $\bf{\Sigma}_1$ over the model we apply $E_h$ to.  Clearly $h \leq i$, since our failure of closeness exemplifies this situation.  We show that for any $h$ satisfying this condition, there must be a smaller ordinal which also satisfies the condition.  Hence our hypothesized failure of closeness at $i$ is impossible. \\


{\bfseries Claim:} $h$ is a successor ordinal, say $h = e+1$.\\

\indent \indent \textbf{Proof:} If $h$ were limit, we'd have $M_h$ as the direct limit of a sequence of lower models in the tree, cofinal in $h$.  We must have a tail-end of this direct limit sequence with $\kappa_H$ below the crit of the branch-tail (and no drops along the branch-tail), since otherwise $\kappa_H$ would be lifted above all $\lambda$'s of extenders in the tree below it, and could not be applied to a lower model $M_g$.  Choose $M_k$ sufficiently high in the direct limit chain so that the parameter $b$ is in the range of the $M_k \longrightarrow M_i$ embedding, and the critical point of this embedding is above $space (H)$.  Then the $\bf{\Sigma}_1$ definition of $X$ over $M_h$ pulls back to a $\bf{\Sigma}_1$ definition of $X$ over $M_k$.  But then $k < h$ satisfies the above condition, contradicting minimality of $h$.  $\blacksquare$\\

{\bfseries Claim:} $\kappa_H  < \kappa_e$.\\

\indent \indent \textbf{Proof:} Say $m = \mathscr{T} \text{-pred} (h)$, so that $M_h = Ult_n(M_m | \gamma , E_e )$ for largest possible $\gamma$.  Now if $\kappa_H \geq \kappa_e$, it would also have to be $\geq \lambda_e$, since it is a definable point in $M_h$ and hence in the range of the $E_e$-embedding.  But then $\kappa_F < \lambda_g$ would be impossible, since the $\lambda$'s in an iteration tree are increasing, so $\lambda_g \leq \lambda_e$. $\blacksquare$\\

{\bfseries Claim:} $X$ is $\bf{\Sigma}_1^{(M_m | \gamma )}$.\\

\indent \indent \textbf{Proof:} By induction, we may assume $E_e$ is close to $M_m | \gamma$.  $X$ is a $\bf{\Sigma}_1$-definable subset of $\mathcal{P} (\kappa_F)$ or $\mathcal{P} (\kappa_F^+)$ in $M_h$, a fortiori a subset of $\mathcal{P} (\kappa_e )$.  By our preceding Lemma, then, $X$ is $\bf{\Sigma}_1$ over $M_m | \gamma$. $\blacksquare$\\

This finishes the $\bf{\Sigma}_1$-amenability clause of the closeness proof, in the short and long cases.  Finally we address the space agreement clause in the long case.  Assume towards contradiction that $F$ is applied to $M_j | \beta$, but there is no extender on the $M_j | \beta$ sequence with the same space as $F$.\\

{\bfseries Claim:} $F$ is the top extender of $M_i$.\\

\indent \indent \textbf{Proof:} If $F$ were an element of $M_i$ indexed at some $\alpha$, then we have that $M_i | \alpha$ is a premouse satisfying long extender condensation.  For the first extender $G$ on the $M_i | \alpha$ sequence with domain equal to $dom(F)$, we then have that $G$ is constructed before $(\kappa_F^{+++})^{M_i | \alpha}$, hence before $\lambda_{M_j}$.  But $M_i$ and $M_j$ agree in this region, so $G$ is on the $M_j | \beta$ sequence, contradiction. $\blacksquare$\\

{\bfseries Claim:} $i$ is a successor ordinal, say $i = e + 1$.\\

\indent \indent \textbf{Proof:} Just like before, if $i$ were limit then the direct limit sequence approaching it must have $\kappa_i$ below the branch-tail crit at some point, or else $\kappa_i$ would be lifted above all the lower $\lambda$'s in the tree and $F$ could only be applied to $M_i$.  But then any model sufficiently high in the direct limit sequence would also cause a failure of space agreement if we chose its top extender to apply.  By induction, this is a contradiction. $\blacksquare$\\

{\bfseries Claim:} $\kappa_G  < \kappa_e$.  (The proof is exactly the same as before.) \\

Let $m = \mathscr{T} \text{-pred} (i)$.  We now have that $M_i^*$ is an initial segment of $M_m$ whose top extender has the same domain as $F$.  But then we could build a shorter tree with a failure of space agreement, by choosing that extender as $E_m$.  Contradiction.\\

This finishes the Closeness Proof. $\blacksquare$\\

















\section{Comparison}

\subsection{The Comparison Lemma}

A centrally important fact about plus-one premice is\\

\begin{thm}
Let $M$ and $N$ be iterable plus-one premice; then there are iterates $P$ of $M$ and $Q$ of $N$ such that $P \unlhd Q$ or $Q \unlhd P$.
\end{thm}

\indent \indent \textbf{Proof:} As a representative special case, let $M$ and $N$ be countable, and let $\Sigma$ and $\Gamma$ be $\omega_1 + 1$ iteration strategies for them.  Let $\mathscr{T}$ on $M$ and $\mathscr{U}$ on $N$ be plays of $\Sigma$ and $\Gamma$, where at successor steps player I has chosen least disagreements $E_\alpha^{\mathscr{T}}$ in $M_\alpha^\mathscr{T}$ and $E_\alpha^{\mathscr{U}}$ in $M_\alpha^{\mathscr{U}}$, and applied $E_\alpha^{\mathscr{T}}$ to $M_\xi^{\mathscr{T}}$, where $\xi$ is least such that $crit (E_\alpha^{\mathscr{T}}) < \lambda_{E_\xi^{\mathscr{T}}}$, and similarly for $E_\alpha^{\mathscr{U}}$.  Write $M_\alpha = M_\alpha^{\mathscr{T}}$, and $N_\alpha = M_\alpha^{\mathscr{U}}$.  Suppose toward contradiction that the process does not terminate, so that $M_{\omega_1}$ and $N_{\omega_1}$ exist.  Let

\[
\pi: X \longrightarrow V_\theta
\]

with $X$ countable transitive, $\theta$ large, and everything relevant in $ran(\pi)$.  Let $\pi(\bar{\mathscr{T}}) = \mathscr{T}$, etc.  Let $\alpha = \omega_1^P = crit(\pi)$, with $\pi(\alpha) = \omega_1$.\\

We have $M_\alpha^{\bar{\mathscr{T}}} = M_\alpha^{\mathscr{T}}$ and $N_\alpha^{\bar{\mathscr{T}}} = N_\alpha^{\mathscr{T}}$.  Thus $M_\alpha$, $N_\alpha \in X$.  Moreover

\[
\pi \restriction M_\alpha = i_{\alpha , \omega_1}^{\mathscr{T}}\ \ , \ \ \pi \restriction N_\alpha = i_{\alpha , \omega_1}^{\mathscr{U}} \ .
\]

Let $\gamma + 1$ be least in $(\alpha, \omega_1)_{\mathscr{T}}$ and $\eta + 1$ least in $(\alpha, \omega_1)_{\mathscr{U}}$, and $G = E_\gamma^{\mathscr{T}}$, $H = E_\eta^{\mathscr{U}}$ the first extenders applied after $\alpha$ on their respective branches.\\

Let

\[
P = (M_\gamma | lh(G), G) \ \ , \ \ Q = (N_\eta | lh(H), H) \ .
\]

We can stretch out $P$ and $Q$ by the short parts of their respective branch-tail extenders:

\[
P^* = Ult_0 (P, E_{i^{\mathscr{T}}_{\gamma + 1, \omega_1}} \restriction \omega_1)\ \ , \ \ Q^* = Ult_0 (Q, E_{i^{\mathscr{U}}_{\eta + 1, \omega_1}} \restriction \omega_1) \ .
\]

Let $i_0 : P \longrightarrow P^*$ and $j_0 : Q \longrightarrow Q^*$ be the canonical embeddings.  Thus $i_0 = i^{\mathscr{T}}_{\gamma + 1 , \omega_1} \restriction (M_\gamma | lh(G)$ (note $M_\gamma | lh (G) \unlhd M_{\gamma + 1}$ ), and $j_0 = i^{\mathscr{U}}_{\eta + 1 , \omega_1} \restriction (N_\eta | lh(H)$.  Let $G^*$ and $H^*$ be the last extender predicates of $P^*$ and $Q^*$, i.e.,

\[
G^* = i_0 (G)\ \ , \ \ H^* = j_0 (H)
\]

where we are really applying $i_0$ and $j_0$ to fragments of $G$ and $H$.\\

\textbf{Claim:} $G^*$ and $H^*$ are initial segments of the extender $E_\pi$ from $\pi$.\\

\indent \indent \textbf{Proof:} Clearly, $G$ and $G^*$ measure the same sets, i.e., $dom (G) = dom(G^*)$.  For $x \in dom(G)$,

\[
\pi(x) = i^{\mathscr{T}}_{\alpha , \omega_1} (x) = i^{\mathscr{T}}_{\gamma + 1, \omega_1} (i_G (x)) \ .
\]

We would like to write

\[
i_{\gamma +1 , \omega_1} (i_G (x)) = i_{\gamma +1 , \omega_1} (i_G) (i_{\gamma +1 , \omega_1} (x)) = i_{\gamma +1 , \omega_1} (i_G) (x) = i_{G^*} (x)
\]

but it is only $i_0$ that can move $i_G$ fragment-wise as an amenable predicate of $(M_\gamma | lh(G), G)$, not the full $i_{\gamma +1 , \omega_1}$.  (If $i_{\gamma +1 , \omega_1}$ is a short extender, this is not a problem.)  So let

\[
\sigma: Ult_0 ( M_{\gamma + 1} , E_{i^{\mathscr{T}}_{\gamma +1 , \omega_1}} \restriction \omega_1 ) \longrightarrow M_{\omega_1} = Ult ( M_{\gamma + 1}, E_{i^{\mathscr{T}}_{\gamma +1 , \omega_1}} )
\]

be the canonical embedding.  $\sigma \restriction \omega_1 + 1 = \text{identity}$, and so $crit(\sigma )$ is at least $( \omega_1^V)^+$ of the smaller ultrapower.  But $P = ( M_\gamma | lh (G) , G) = (M_{\gamma + 1} | lh (G) , G )$, and $lh (G) = \lambda_G^+$ in $M_{\gamma + 1}$, so

\[
P^* = ( M_{\omega_1} \restriction lh (G^*), G^* ) \ ,
\]

where $lh(G^*) \leq crit( \sigma)$.  Taking $x \in dom (G) = dom(G^*)$, we may assume $x \subseteq (\kappa_G^+)^{M_\alpha}$, and we then get

\[
\begin{split}
 \pi(x) & = i^{\mathscr{T}}_{\gamma + 1 , \omega_1} (i_G (x))\\
 & = \sigma ( i_0 (i_G (x)))\\
 & = \sigma ( i_0 (i_G) ( i_0 (x)))\\
 & = \sigma ( i_{G^*} (x)) \ .
 \end{split}
 \]
   
   Since $crit (\sigma) \geq lh (G^*)$, this gives $G^* = E_\pi \cap ( M_\alpha^{\mathscr{T}} \times [ lh (G^*)]^{< \omega} )$. $\blacksquare$\\
  
  
\textbf{Claim:} At least one of $G$ and $H$ is long.\\

\indent \indent \textbf{Proof:} Assume not.  Then $G^* = H^* = E_\pi \cap (M_{\omega_1} \times [\omega_1 ]^{< \omega} )$.  But $G = G^* \restriction \lambda_G$, and $G \notin M_{\omega_1}$, so we see that the Jensen ISC fails for $G^*$.  It held for $(M_\gamma | lh(G), G)$, by our ``short extender rules" for iteration.  It follows that

\[
\begin{split}
\lambda_G & = \text{ least $\eta$ such that }G^* \restriction \eta \text{ is whole and } G^* \restriction \eta \notin M_{\omega_1}\\
& = \text{ least $\eta$ such that }H^* \restriction \eta \text{ is whole and } H^* \restriction \eta \notin N_{\omega_1}\\
& = \lambda_H \ .
\end{split}
\]

So $H = H^* \restriction \lambda_H = G^* \restriction \lambda_G = G$, contrary to $G$ being part of a disagreement. $\blacksquare$\\

\textbf{Claim:} Both $G$ and $H$ are long.\\

\indent \indent \textbf{Proof:} Suppose $G$ is short and $H$ is long.  Then by the weak ISC, $H \restriction \lambda_H$ is on the sequence of $N_\eta | lh (H)$, and hence $j_0 (H \restriction \lambda_H) = H^* \restriction \omega_1$ is on the sequence of $N_{\omega_1} | lh(H^*)$.  But $H^* \restriction \omega_1 = G^*$, and so by $N_{\omega_1} | lh(H^*) \models \text{Jensen ISC}$, $G^* \restriction \lambda_G = G \in N_{\omega_1}$.  Since $lh(G)$ is a cardinal of $N_{\omega_1}$, this is a contradiction. $\blacksquare$\\

\textbf{Claim:} It is not the case that both $G$ and $H$ are long.\\


\indent \indent \textbf{Proof:} Otherwise, let $\nu + 1 = \nu (H)$, and $\xi + 1 = \nu (G)$.  We have

\[
i_0 : (M_\gamma | lh(G), G ) \longrightarrow (M_{\omega_1} | lh(G^* ), G^* ) \ ,
\]

\[
j_0 : (N_\eta | lh(H), H ) \longrightarrow (N_{\omega_1} | lh(H^* ), H^* ) \ ,
\]

moreover $G^*$ and $H^*$ are compatible.\\

\textbf{Subclaim:} $i_0 (\xi)$ is the largest generator of $G^*$, and $\forall \mu < i_0 (\xi)$, $G^* \restriction \mu \in M_{\omega_1} | lh(G^*)$.\\

\indent \indent \textbf{Proof:} If $G \restriction \xi \in M_\gamma | lh(G)$, then $i_0 (G \restriction \xi) = G^* \restriction i_0 (\xi) \in M_{\omega_1} | lh(G^*)$, so of course $\forall \mu < i_0 (\xi)$, $G^* \restriction \mu \in M_{\omega_1} | lh (G^*)$.  But if $G \restriction \xi \notin M_\gamma | lh(G)$, then since $( M_\gamma | lh(G), G)$ is a premouse, $M_\gamma | lh(G) \models cof (\xi) = \tau^+$, for some $\tau < \lambda_G$.  But $i_0$ comes from a short extender with critical point $\lambda_G$, so we get $i_0 (\xi) = sup (i_0 `` \xi)$.  This again shows $\forall \mu < \xi \ ( G^* \restriction \mu \in M_{\omega_1} | lh(G^*))$.\\

We need to see that $i_0 (\xi)$ is the largest generator of $G^*$.  Being a generator is $\Pi_1$, so by elementarity $i_0 (\xi)$ is still a generator.  Now for any $\eta > \xi$ in $( M_\gamma | lh(G), G)$, there is a wellorder $W \in M_\gamma | lh(G)$ of $\lambda_G$ of ordertype $\eta$, and $W = i_G (f) (a, \xi )$ for $a \subseteq \lambda_G$ and $f$ some representing function.  By cutting this wellorder down to an initial segment, we can obtain representing functions for all ordinals $\leq \eta$, so none of them can be generators.  This same reasoning applied to $i_0 (W)$ shows that no ordinals $\leq i_0 (\eta)$ can be generators of $G^*$, since the representation of $W$ in $i_G$ is carried upwards to a representation of $i_0 (W)$ in $i_{G^*}$.  Since $i_0$ is cofinal in $M_{\omega_1} | lh(G^*)$, we see that nothing above $i_0 (\xi)$ can be a generator. $\blacksquare$\\


\textbf{Subclaim:} $j_0 (\nu)$ is the largest generator of $H^*$, and $\forall \mu < j_0 (\nu)$, $H^* \restriction \mu \in N_{\omega_1} | lh(H^* )$.\\

\indent \indent \textbf{Proof:} The same. $\blacksquare$\\


\textbf{Subclaim:} $i_0 (\xi) = j_0 (\nu)$.\\

\indent \indent \textbf{Proof:} Suppose e.g. that $i_0 (\xi) < j_0 (\nu)$.  By the last subclaim, $H^* \restriction (\omega_1 \cup \{ i_0 (\xi) \} ) \in N_{\omega_1} | lh(H^* )$.  That is, $G^* \in N_{\omega_1}$.  But then $G = G \restriction ( \lambda_G \cup \{ \xi \} ) = G^* \restriction ( \lambda_G \cup \{ i_0 (\xi) \} ) \in N_{\omega_1}$, contrary to $G$ collapsing $\lambda_G^+$ of $N_{\omega_1}$. $\blacksquare$\\

\textbf{Subclaim:} $G = H$.\\

\indent \indent \textbf{Proof:} We have that

\[
\begin{split}
\lambda_G & = \text{ least $\eta$ such that } G^* \restriction (\eta \cup \{ i_0 (\xi) \} ) \text{ is a whole extender that is not in } M_{\omega_1}\\
& = \text{ least $\eta$ such that } H^* \restriction (\eta \cup \{ j_0 (\nu) \} ) \text{ is a whole extender that is not in } N_{\omega_1}\\
& = \lambda_H \ .
\end{split}
\]

But then $G = G^* \restriction ( \lambda_G \cup \{ i_0 (\xi) \} ) = H^* \restriction (\lambda_H \cup \{ j_0 (\nu) \} ) = H$, as desired. $\blacksquare$\\

Clearly $G \neq H$, because they were used in disagreements.  So we have proved our last claim. $\blacksquare$\\

These three claims add up to a contradiction, thereby proving the Comparison Lemma. $\blacksquare$\\



\section{Phalanxes and Copy Maps}


\subsection{Phalanxes}

In addition to iteration trees on premice, we must consider \textit{phalanx iterations}.\\

\begin{def} \label{phalanx}
A \textit{phalanx} $\langle M, H, \alpha \rangle$ \textit{with embedding $\sigma$} consists of a triple where $H$ and $M$ are plus-one premice and $\alpha$ is an ordinal, with $\sigma : H \longrightarrow M$ an embedding with critical point $\geq \alpha$ which is $\Sigma_0$-elementary and cardinal-preserving.
\end{def}

\begin{def} \label{iteration tree on phalanx}
An \textit{iteration tree $\mathscr{T}$ on a phalanx} $\langle M, H, \alpha \rangle$ is an iteration tree constructed according to the usual rules, except that we set $M_0 = M$, $M_1 = H$, $\lambda_0 = \alpha$, and $E_0$ is undefined.  That is, for each $\beta < length( \mathscr{T} )$, $E_\beta$ is an extender on the $M_\beta$-sequence, $\kappa_\beta$ is the critical point of $E_\beta$, and $\lambda_\beta$ is the image of $\kappa_\beta$ by $i_{E_\beta}$.  We set $M_{\beta + 1}^* = (M_{\gamma} | \delta )$ for $\gamma$ least such that $dom(E_\beta ) \subseteq (M_\gamma | \lambda_\gamma)$ and $\delta$ largest such that $E_\beta$ is a total extender on $(M_{\gamma} | \delta )$.
\end{def}


\begin{lem}
Let $\mathscr{T}$ be an iteration tree on a phalanx $\langle M, H, \alpha \rangle$ with models $M_\eta$.  Recall $M_0 = M$ and $M_1 = H$.  Then for all $\eta < length ( \mathscr{T} )$,\\

\indent $a)$ If $\mathscr{T}-pred ( \eta + 1 ) \neq 0$, then $E_\eta$ is close to $M_{\eta + 1}^*$;\\

\indent $b)$ If $M_\eta$ is above $H$ in $\mathscr{T}$, i.e., $1 <_\mathscr{T} \eta$, then $M_\eta$ is a plus-one premouse;\\

\indent $c)$ If $0 <_\mathscr{T} \gamma <_\mathscr{T} \eta$ and $M_\gamma$ is a plus-one premouse, then so is $M_\eta$.
\end{lem}

The proof of $a)$ is essentially the same as the closeness proof for premouse iterations given above.   However, that proof will break down when we are considering applications of $E_\eta$ to $M_0 = M$ for which $dom( E_\eta ) = (M | \alpha )$.  This is the reason for the qualifications in this lemma, which we will now explain in more detail.\\

There are several points in the closeness proof where we need to know that when $E_\eta$ is applied to $M_j$, then $\lambda_j$ is a limit cardinal of $M_\eta$ which is above $dom ( E_\eta )$.  In phalanx iterations, this fact remains true \textit{except} in the case just described-- for in that case we have declared $\lambda_j = \lambda_0 = \alpha$, and this is the exact height of $dom (E_\eta )$.  So, for instance, the fragments of $E_\eta$ which are relevant for $\bf{\Sigma}_1$-amenability can be coded as subsets of $dom ( M_\eta )$.  In the original closeness proof, we could argue that if such a fragment is an element of $M_\eta$, it is also an element of $M_j$.  But this can break down in a phalanx iteration when $j = 0$.  Similarly, the space agreement clause of closeness relies for its proof on an application of long extender condensation that may fail here. \\

$b)$ and $c)$ follow easily from Lemma 4 of \cite{PIPM}, which shows that ultrapowers of plus-one premice by close extenders are plus-one premice (in fact, only the space agreement clause of closeness is needed).  The proof given there focuses on premice with active long extenders, but the standard proof of premousehood-preservation for short extenders goes through verbatim; see, e.g., Lemma 1.5 of \cite{zeman square proof}.  \\

The foregoing discussion also explains the following fact (\cite{FSPIPM}, Theorem 10, Claim 3):\\

\begin{lem}
For a phalanx iteration as above, let $\mathscr{T}-pred (\eta + 1 ) = 0$, and suppose $dom (E_\eta ) = (M | \beta )$ where $\beta < \alpha$; then $M_{\eta + 1}^* = M$, and $E_\eta$ is close to $M$.  Thus $M_{\eta + 1}$ is a plus-one premouse.
\end{lem}

We also have (\cite{FSPIPM}, Theorem 10, Claim 4):\\

\begin{lem}
For a phalanx iteration as above, let $\mathscr{T}-pred (\eta + 1 ) = 0$, and suppose $dom (E_\eta ) = (M | \alpha )$, and that $\alpha$ is a cardinal of $M$; then $M_{\eta + 1}^* = M$, and $E_\eta$ is close to $M$.  Thus $M_{\eta + 1}$ is a plus-one premouse.\end{lem}

\indent \indent \textbf{Proof:} Since $\alpha$ is a cardinal of $M$, we immediately get that $M_{\eta + 1}^* = M$.  Now note that the standard closeness argument can easily establish that $E_\eta$ is close to $H$.  But we have an embedding $\sigma : H \longrightarrow M$ as part of our phalanx, which can be used to transfer the requisite closeness properties to $M$.  (Unlike in \cite{FSPIPM}, we are only assuming $\sigma$ is $\Sigma_0$-elementary and cardinal-preserving.  However, this is still enough to transfer closeness.) $\blacksquare$\\

\subsection{Lifting Trees}

We will prove iterability of the phalanxes that arise in our construction by lifting the double-rooted iteration tree $\mathscr{T}$ on the phalanx $\langle M , H , \alpha \rangle$ to a single-rooted iteration tree $\mathscr{T}^*$ on $M$.  Then we can use an iteration strategy on the lifted tree $\mathscr{T}^*$ to choose branches through the double-rooted $\mathscr{T}$.  This will work because, given an embedding $\pi : M \longrightarrow N$, with $\mathscr{T}$ an iteration tree on $M$, we can copy it to a tree $\pi \mathscr{T}$ on $N$.  Given an iteration strategy $\Sigma$ for $N$, we can then define the ``pullback strategy" $\Sigma^{\pi} ( \mathscr{T}) = \Sigma ( \pi \mathscr{T})$, which will give us choices for wellfounded branches through $\mathscr{T}$.  The situation with phananx iterations is essentially the same, but with a little extra complication due to the two roots.\\

The following lemma is used to extend $\mathscr{T}^*$ at successor steps.\\

\begin{lem} \textbf{(Shift Lemma)}
Let $\bar{M}$ and $\bar{N}$ be premice, let $\bar{\kappa} = crit (\dot{F}^{\bar{N}})$, and let

\[
\psi: \mathcal{C}_0 (\bar{N}) \longrightarrow \mathcal{C}_0 (N)
\]

be a weak $0$-embedding, and

\[
\pi : \mathcal{C}_0 (\bar{M}) \longrightarrow \mathcal{C}_0 (M)
\]

be a weak $n$-embedding.  Suppose that $\bar{M}$ and $\bar{N}$ agree below $( \bar{\kappa}^+ )^{\bar{M}}$ and $( \bar{\kappa}^+ )^{\bar{M}} \leq ( \bar{\kappa}^+ )^{\bar{N}}$, while $M$ and $N$ agree below $( \kappa^+)^{M}$ and $(\kappa^+)^M \leq ( \kappa^+ )^N$, where $\kappa = \psi (\bar{\kappa} )$.  Suppose also

\[
\pi \restriction (\bar{\kappa}^+ )^{\bar{M}} = \psi \restriction (\bar{\kappa}^+ )^{\bar{N}} \ .
\]

Let $\bar{\kappa} < \varrho_n (\bar{M})$, so that $Ult_n (\mathcal{C}_0 (\bar{M}) , \dot{F}^{\bar{N}} )$ and $Ult_n (\mathcal{C}_0 (M) , \dot{F}^{N} )$ make sense, and suppose the latter ultrapower is wellfounded.  Then the former ultrapower is wellfounded; moreover, there is a unique embedding 

\[
\sigma: Ult_n (\mathcal{C}_0 (\bar{M}) , \dot{F}^{\bar{N}}) \longrightarrow Ult_n (\mathcal{C}_0 (M) , \dot{F}^{N} )
\]

satisfying the conditions:\\

\indent $1.$ \ $\sigma$ is a weak $n$-embedding,\\

\indent $2.$ \ $Ult_n (\mathcal{C}_0 (\bar{M}) , \dot{F}^{\bar{N}})$ agrees with $\bar{N}$ below $\varrho_0 (\bar{N})$, and $Ult_n ( \mathcal{C}_0 (M) , \dot{F}^{N} )$ agrees with $N$ below $\varrho_0 (N)$,\\

\indent $3.$ \ $\sigma \restriction ( \varrho_0 (\bar{N})) = \psi \restriction ( \varrho_0 (\bar{N}))$,\\

\indent $4.$ \ the diagram

\[
\begin{tikzcd}
Ult_n (\mathcal{C}_0 (\bar{M}), \dot{F}^{\bar{N}} )
\arrow[r, "\sigma"]
& Ult_n (\mathcal{C}_0 (M), \dot{F}^N )\\
\mathcal{C}_0 (\bar{M})
\arrow[u, "i"]
\arrow[r, "\pi"]
& \mathcal{C}_0 (M) \arrow[u, "j"]
\end{tikzcd}
\]


commutes, where $i$ and $j$ are the canonical ultrapower embeddings.
\end{lem}

The standard arguments (see \cite{steel outline}) allow us to establish a weak Dodd-Jensen lemma:\\

\begin{def} \label{near embedding}
A \textit{near $k$-embedding} between premice is a weak $k$-embedding which is $r \Sigma_{k + 1}$-elementary.
\end{def}

\begin{def} \label{large}
Let $M$ and $P$ be premice; then we say that $P$ is $(M, k)$\textit{-large} iff there is a near $k$-embedding from $M$ to an initial segment of $P$.
\end{def}

\begin{def} \label{minimal}
Let $\vec{e} = \langle e_i \ | \  i < \omega \rangle$ enumerate the universe of a countable premouse $M$, and $\pi: M \longrightarrow P$ be a near $k$-embedding; then we say $\pi$ is $(k , e)$\textit{-minimal} iff whenever $\sigma$ is a near $k$-embedding from $M$ to an initial segment $N$ of $P$, then $N = P$ and either $\sigma = \pi$, or $\sigma ( e_i) >_L \pi (e_i)$, where $i$ is least such that $\sigma (e_i) \neq \pi (e_i)$.
\end{def}

Notice that if $P$ is $( M , k )$-large but no proper initial segment of $P$ is $(M, k)$-large, then there is a $(k , \vec{e})$-minimal embedding from $M$ to $P$; this embedding will be the leftmost branch through a certain tree.\\

\begin{def} \label{weak DJ property}
Let $\Sigma$ be an iteration strategy for a countable premouse $M$, and let $\vec{e} = \langle e_i \ | \ i < \omega \rangle$ enumerate the universe of $M$ in order type $\omega$; then we say $\Sigma$ has the \textit{weak Dodd-Jensen property} (relative to $\vec{e}$) iff whenever $\mathscr{T}$ is an iteration tree on $M$ played according to $\Sigma$, and $\beta < lh(\mathscr{T})$ is such that $M_\beta^{\mathscr{T}}$ is $(M, k)$-large, then $i_{0, \beta}^{\mathscr{T}}$ exists and is $(k, \vec{e})$-minimal.
\end{def}

\begin{thm} \textbf{(The Weak Dodd-Jensen Lemma)}\\
\textbf{(from \cite{weak DJ})}
Suppose $M$ is iterable, and that $\vec{e}$ enumerates the universe of $M$ in order type $\omega$; then there is a unique iteration strategy for $M$ which has the weak Dodd-Jensen property relative to $\vec{e}$.
\end{thm}











\section{Condensation}










With these preliminaries, we are ready to prove one of our main results.  The idea behind the Condensation Lemma is to use a phalanx to compare $H$ and $M$, while ensuring that the critical point of the branch-embeddings is large enough to get our desired conclusions.  In the course of this proof we must consider four anomalous cases, which correspond to different ways that the iteration tree on our phalanx could break down.  Unfortunately, we have been unable to resolve these difficulties in Anomalous Case 4, so the Condensation Lemma is stated with the hypothesis that Anomalous Case 4 does not apply.  Thankfully, this form of the Lemma suffices for all the applications which follow.  The case which we are avoiding is\\

\textbf{Anomalous Case 4:} $\alpha$ is not a cardinal of $M$, and letting $\langle \eta, k \rangle$ be lexicographically least such that $\varrho_{k+1} (M | \eta ) < \alpha$, we have that $F = \dot{F}^{M | \eta }$ is short, $\alpha = ( \kappa_F^{++} )^{M | \eta }$, $k = 0$, and there are total long extenders on the $M$-sequence with critical point $\kappa_F$.\\



\begin{thm} \textbf{(Condensation Lemma)}
Let $\langle M, H, \alpha \rangle$ be a phalanx where $M$ is an $(n , \omega_1 , \omega_1 +1 )$-iterable premouse, and suppose the associated map  $\sigma : H \longrightarrow M$ is an embedding with critical point $= \alpha$ and which is $\Sigma_0^{(n)}$-elementary and cardinal-preserving.  Suppose $H$ and $M$ have the same type ($A$, $B$, $C$, $Z_1$ or Dodd-solid).  Suppose further that $M$ and $H$ are both $(n+1)$-sound, and that $\alpha \geq \varrho_{n+1}^H$.  Finally, assume that Anomalous Case 4 does not hold.  Then one of the following conclusions is true:\\

\indent $a)$ $H = M$ and $\sigma = id$.\\

\indent $b)$ $H$ is a proper initial segment of $M$.\\

\indent $c)$ $H$ is an initial segment of $Ult_n (M, E_\alpha^M )$.\\

\indent $d)$ $H$ is an initial segment of $Ult_n (M, F)$, where $F$ is pseudo-indexed at $\alpha$ on the $M$-sequence.\\

\end{thm}

\indent \indent \textbf{Proof:} The proof is similar to the Solidity proof of \cite{FSPIPM}.  We first isolate several anomalous cases.\\

\textbf{Anomalous Case 1:} There is an extender indexed at $\alpha$ on the $M$-sequence.\\

\textbf{Anomalous Case 2:} There is an extender pseudo-indexed at $\alpha$ on the $M$-sequence.\\

\textbf{Anomalous Case 3:} $\alpha$ is not a cardinal of $M$, and letting $\langle \eta, k \rangle$ be lexicographically least such that $\varrho_{k+1} (M | \eta ) < \alpha$, we have that $p_{k + 1} (M | \eta)$ is type $Z_p$, and for $\gamma$ least in $p_{k+1} (M | \eta)$ and $F = ( \dot{E}_\gamma )^M$, $\alpha = (\kappa_F^{++} )^{M | \eta }$.\\

We have already mentioned Anomalous Case 4, which we are assuming does not apply.\\


\subsection{Non-Anomalous Case}

We begin by proving the theorem under the assumption that none of the anomalous cases apply.\\

We coiterate the phalanx $\langle M, H, \alpha \rangle$ against $M$, iterating away least disagreements.  Let $\mathscr{T}$ with models $P_\beta$ be the tree produced on the phalanx, and $\mathscr{U}$ with models $Q_\beta$ be the tree produced on $M$.  We have already seen that all models of $\mathscr{U}$ are plus-one premice.  For $\mathscr{T}$ the situation is more complicated.  By our Lemma above, all models in $\mathscr{T}$ are plus-one premice (and all extenders are close to their models) with the possible exception of extenders $E_\eta$ with domain $= M | \alpha$, which may not be close to $P_{\eta + 1}^*$ and may produce models $P_{\eta + 1}$ which fail to be plus-one premice.  By our other Lemma above, these problems can only occur when $\alpha$ is not a cardinal of $M$.  For such an extender $E_\eta$, let $\xi$ be the collapsing-level for $\alpha$ in $M$, so that $P_{\eta + 1}^* = ( M | \xi )$.  Note that if $E_\eta$ is short, we have $\alpha = ( \kappa_{E_\eta}^+ )^{P_\eta}$, and if $E_\eta$ is long, then $\alpha = ( \kappa_{E_\eta}^{++} )^{P_\eta}$. \\


\textbf{Claim:} Let $\mathscr{T}-pred (\eta + 1 )= 0$, $E = E_\eta$, and $P_{\eta + 1}^* = ( M | \xi )$, where $\xi < height(M)$.  Let $k = deg^{\mathscr{T}} (\eta + 1)$ be least such that $\varrho_{k + 1} (M | \xi ) < \alpha$.  Suppose $E$ is short; then\\

\indent $a)$ $Ult_k (M | \xi , E )$ is a plus-one premouse,\\

\indent $b)$ the ultrapower map $i: (M | \xi ) \longrightarrow Ult_k (M | \xi , E )$ is a $k$-embedding, with $\varrho_{k + 1}^{M | \xi} = \varrho_{k + 1}^{Ult ( M | \xi , E )} = \kappa_E$ and $i(p_{k + 1}^{M | \xi} ) = p_{k + 1}^{Ult_k (M | \xi , E )}$.\\



\indent \indent \textbf{Proof:}  Note that $\xi > \alpha$ because we are not in Anomalous Case 1.  Moreover, $\xi < \pi (\alpha)$.\\

\textbf{Subclaim:} For all finite $a \subseteq \lambda_E$, $E_a \in M$.\\

\indent \indent \textbf{Proof:} If $E$ is not the last extender of $P_\eta$, or the branch ending at $P_\eta$ drops somewhere, then $E_a \in H$, because $E_a$ is coded by a subset of $\alpha$.  But then $E_a = \pi (E_a) \cap (M | \alpha) \in M$.\\

If $E = \dot{F}^{P_\eta}$ and the branch to $\eta$ did not drop, then $1T \eta$.  (Since if $0T \eta$, then $crit( i^{\mathscr{T}}_{0, \eta}) \leq \kappa_E$ because $\alpha = (\kappa_E^{+})^{M | \alpha}$, so $\kappa_E$ cannot be in the range of $i^{\mathscr{T}}_{0, \eta}$, contradicting that $P_\eta$ has top extender with $crit = \kappa_E$.)  In this case, we consider $\mathscr{T}^*$.  We have $E = \dot{F}^{P_\eta}$, and $\pi_\eta (E) = \dot{F}^{P_\eta^*}$.  $E_a \subseteq H | \alpha$, and $\pi_\eta \restriction \alpha = \pi_0 \restriction \alpha$, so\\

\indent \indent $E_a = ( \dot{F}^{P_\eta^*})_{\pi_\eta (a)} \cap P_\eta^* | \alpha$.\\

But $\kappa = \pi_\eta (\kappa) < \alpha < \pi_\eta (\alpha) = (\kappa^+)^{P_\eta^*}$, so $E_a \in P_\eta^*$ by weak amenability.  Thus $E_a \in P_0^* = M$. $\blacksquare$\\

We also have that each $E_a$ is weakly amenable to $M | \xi$.  For if $\beta < \alpha$, then $E_a \cap M| \beta = E_a \cap H | \beta \in H$, since $E$ is close to $H$.  So $E_a \cap M | \beta \in M | \xi$.\\

\textbf{Subclaim:} $Ult_k (M | \xi , E )$ is a plus-one premouse.\\

\indent \indent \textbf{Proof:} If $k > 1$, the ultrapower is sufficiently elementary.  If $k = 1$ then by the discussion preceding Definition \ref{alternate 1-core}, we have defined the $1$-core in such a way that $1$-ultrapowers of this core will be continuous at $\mu$, and thereby preserve premousehood, while still preserving all $\Sigma_1$-information from $M| \xi$.  We omit further detail.  So assume $k = 0$.  If $M | \xi$ is not type $Z_1$, there is no problem, so assume that it is.  This is where we use the assumption that we are not in Anomalous Case 1; in AC1 we would get a violation of Jensen ISC when we take this ultrapower.  Let $i: M | \xi \longrightarrow Ult_0 (M | \xi, E)$ be the canonical embedding.  We have that $M | \xi \models (cof(\dot{\nu}) \text{is a successor cardinal})$, and thus $i$ is continuous at $\dot{\nu}^{M | \xi}$, since we are ultrapowering by a short extender, which is only discontinuous at points of cofinality $\kappa_E$.  It follows easily that $Ult_0 (M | \xi , E )$ is a type $Z_1$ premouse, with $\dot{\nu}^{Ult (M | \xi , E )} = i (\dot{\nu}^{M | \xi})$. $\blacksquare$\\

\textbf{Subclaim:} $\varrho_{k+1}^{Ult (M | \xi , E)} = \varrho_{k+1}^{M | \xi} = \kappa_E$.\\

\indent \indent \textbf{Proof:} If $A$ is a bounded subset of $\kappa_E$, and $A$ is $\Sigma_{k+1}^{Ult_k (M | \xi , E )}$ in the parameter $[ a, f ]_E^{M | \xi}$, then since $E_a \in M$, $A \in M$.  Thus $A \in M | \xi$.  So $\varrho_{k+1}^{Ult_k (M | \xi , E)} \geq \kappa$.  That $\varrho_{k+1}^{Ult_k (M | \xi , E)} \leq \kappa$ follows from $E_a$ being weakly amenable to $M | \xi$. $\blacksquare$\\

\textbf{Subclaim:} Let $i: M | \xi \longrightarrow Ult_k (M | \xi , E )$ be the canonical embedding; then $i$ is a $(k+1)$-embedding.\\

\indent \indent \textbf{Proof:} We consider just whether $i ( p_{k+1} (M | \xi )) = p_{k+1} (Ult_k ( M | \xi , E ))$.  We have\\

\indent \indent $p_{k+1} (M | \xi ) = \langle r_0 , u \rangle$,\\

where $u$ collects the solidity witnesses for $p_{k-1} (M | \xi )$.  $i$ is sufficiently elementary that if $b$ is a solidity witness for $\gamma$ at level $k - 1$, i.e., $b = Th_{k - 1}^{M | \xi} (\gamma \cup s )$, then $i(b)$ is a solidity witness for $i (\gamma)$ in $Ult_k (M | \xi , E)$.  So if $p_{k-1} (M | \xi )$ is not of type $Z_p$, then $i (p_{k-1} (M | \xi)) = p_{k-1} (Ult_k (M | \xi , E ))$, with solidity witnesses in $i(u)$.  Suppose now $p_{k-1} (M | \xi)$ is type $Z_p$, with least element $\gamma$.  Then let $F = (\dot{E}_\gamma)^{M| \xi}$ be the stretching extender, so that $(\kappa_F^+)^{M | \xi} = \varrho_k^{M | \xi}$.\\

We have $\kappa_E < \varrho_k^{M | \xi}$.  Moreover, there can be no $\bf{\Sigma}_k^{M | \xi}$ function from $\kappa_F$ cofinally into $(\kappa_F^+)^{M | \xi}$, because otherwise $\varrho_k^{M | \xi} \leq \kappa_F$.  Thus $i$ is continuous at $(\kappa_F^+)^{M | \xi}$.  It follows that $i(F)$ still witnesses the type $Z_p$ property for $i(\gamma)$.  So again, $p_{k-1} (Ult (M | \xi , E)) = i (p_{k-1} (M | \xi))$.\\

The argument that $i(r)$ is the $(k+1)$st standard parameter over $i(u)$ for $Ult_k (M | \xi , E)$ is similar.  If $b$ is a solidity witness for $\gamma \in r_0$, then some initial segment of $i(b)$, in its natural prewellorder, is a solidity witness for $i(\gamma)$ over $Ult_k (M | \xi , E)$.  If $r_0$ is type $Z_p$, with least element $\gamma$ and stretching extender $F = \dot{E}_\gamma^{M | \xi}$, then we have $\varrho_{k+1} (M | \xi) = \kappa_F^+$.  But we are in the case that $\varrho_{k+1} (M | \xi ) = \kappa_E$, which is a limit cardinal of $M$, so this is impossible. $\blacksquare$\\

Next we consider the case where $E$ is long.\\

\textbf{Claim:} Let $T-pred (\eta + 1 ) = 0$, $E = E_\eta$, and $(M_{\eta + 1}^*)^{\mathscr{T}} = M | \xi$, where $\xi < o(M)$.  Suppose $k = deg^{\mathscr{T}} (\eta + 1)$ is least such that $\varrho_{k+1} (M | \xi ) < \alpha$.  Suppose $E$ is long; then\\

\indent $a)$ $E$ is the first long extender on the $P_\eta$-sequence having domain $P_\eta | \alpha = M | \alpha$;\\

\indent $b)$ $E$ has exactly one long generator, call it $\gamma$;\\

\indent $c)$ $Ult_k (M | \xi , E)$ is a $k+1$-sound plus-one premouse, with\\

\indent \indent $\varrho_{k+1}^{Ult_k (M | \xi , E)} = \varrho_{k+1}^{M | \xi} = (\kappa^+ )^{M | \xi}$ \ \ and \ \ $p_{k+1} (Ult_k (M | \xi , E )) = i_E (p_{k+1} (M | \xi)) \cup \{ \gamma \}$,\\

\indent $d)$ $p_{k+1} (Ult_k (M | \xi , E))$ is of type $Z_p$.\\

\indent \indent \textbf{Proof:} Let $\kappa = \kappa_E$.  We have $\alpha = (\kappa^{++})^{M | \xi}$.  We write ``$\kappa^+$" for $(\kappa^+ )^{M | \xi} = (\kappa^+)^H = (\kappa^+)^{P_\eta}$.  We have that $\kappa^+ = \varrho_{k+1} (M | \xi)$, because $\kappa^+$ is a cardinal of $M$ and so no level $M | \xi$ can project below it for $\xi > \kappa^+$.  Since $\varrho_{k+1} (M | \xi ) < \alpha$, the only candidate for this projectum is $\kappa^+$.\\

Let $\nu = \nu(E) - 1 = \dot{\nu}^{P_\eta}$.\\

\textbf{Subclaim:} For any finite $a \subseteq \lambda_E \cup \{ \nu \}$, $E \restriction ( \kappa^+ \cup a ) \in M$.\\

\indent \indent \textbf{Proof:}  The same as in the case that $E$ is short.  If $E$ is not the top extender of $P_\eta$ or $P_\eta$ comes after a drop, then $E \restriction ( \kappa^+ \cup a )$ can be coded as a subset of $\alpha$ that must be in $H$, therefore in $M$.  If there are no drops on the branch to $P_\eta$ and $E$ is its top extender, then this branch must occur above $H$ in the phalanx.  Lifting $P_\eta$ by $\pi_\eta$, we have the desired fragment in $P^*_\eta$ by weak amenability, since $\alpha < (\kappa_E^{++})^{P^*_\eta}$.  But any subset of $\alpha$ in $P^*_\eta$ is in $P^*_0 = M$. $\blacksquare$\\

\textbf{Subclaim:} $E$ is the first long extender on the $P_\eta$-sequence with domain $H | \alpha$.\\

\indent \indent \textbf{Proof:} Let $G = \dot{F}^{P_\eta | \gamma}$ be the first such extender.  By Long Extender Condensation, $\varrho_1 (P_\eta | \gamma ) = \alpha$.  Assume that $G \neq E$; then $\gamma < o (P_\eta)$.  Thus $\gamma < (\alpha^+ )^{P_\eta}$.\\

Suppose first $\eta \geq 2$.  Then $lh(E_1)$ is a cardinal in $P_\eta$, and $lh(E1) > \alpha$, and $H || lh(E_1) = P_\eta || lh(E_1)$.  Thus $G$ is on the sequence of $H$, with $lh (G) < lh(E_1)$.  It follows that $G$ is on the sequence of $M$, and of $M | \xi$.  But $\varrho_{k+1} (M | \xi) = \kappa^+ = space (G)$, so this violates the projectum-free spaces property of $M | \xi$.\\

If $\eta = 1$, then since $G \neq E_1$, we again have that $G$ is on the sequence of $M | \xi$, contradiction. $\blacksquare$\\

\textbf{Subclaim:} $Ult_k ( M | \xi , E)$ is a plus-one premouse.\\

\indent \indent \textbf{Proof:} If $k > 1$, there is enough elementarity.  In the case $k=1$ we must consider additional details related to Definition \ref{alternate 1-core}; briefly, the idea is that we have defined the $1$-core in such a way that $1$-ultrapowers of this core will be continuous at $\mu$, and thereby preserve premousehood, while still preserving all $\Sigma_1$-information from $M| \xi$.  We omit further detail.  Suppose then that $k = 0$.\\

The proof that works when $E$ is close to $M | \xi$ will work here.  First, notice that we are not in Anomalous Case 2.  Thus if $M | \xi$ is type $Z_1$, then $\dot{\nu}^{M | \xi}$ has cofinality in $M | \xi$ different from $\kappa^+$, and thus $i_E$ is continuous at $\dot{\nu}^{M | \xi}$, and thus $i_E (\dot{\nu}^{M | \xi}) = \dot{\nu}^{Ult_0 (M | \xi , E )}$, and $(\dot{E}_{\dot{\nu}})^{Ult_0 ( M | \xi , E)}$ witnesses the type $Z_1$-ness of $Ult_0 (M | \xi , E)$.\\

Second, notice that we are not in the situation that would lead to $Ult_0 (M | \xi, E)$ being merely a protomouse, by the fact that we are not in Anomalous Case 4.\\

We omit the remaining calculations. $\blacksquare$\\

\textbf{Subclaim:} Let $i: M | \xi \longrightarrow Ult_k (M | \xi , E )$ be the canonical embedding.  Let $N = Ult_k (M | \xi , E )$; then $p_{k+1} (N)$ is of type $Z_p$, with $p_{k+1} (N) = i (p_{k+1} (M | \xi )) \cup \{ \nu \}$.\\

\indent \indent \textbf{Proof:}  Because we are not in Anomalous Case 3, $p_{k+1} (M | \xi)$ is solid.  Let $t = p_{k+1} (M | \xi)$, and $A = Th_{k+1}^{M | \xi} (\kappa^+ \cup t)$, recalling here that $\kappa^+ = \varrho_{k+1} (M | \xi)$.\\

We have that $\kappa^+ \leq \varrho_{k+1} (N)$, and that $A \notin N$, as in the proof of the above Subclaim that $E$ is first long extender with its domain.  But $A$ is $\Sigma_{k+1}^N$ in the parameters $i(t)$ and $i \restriction \kappa^+$, and $i \restriction \kappa^+$ is essentially $\dot{E}_\nu^N$.  This gives $\varrho_{k+1} (N) = \kappa^+$.\\

By solidity of $t$, and the fact that for all $\beta < \kappa^+$, $Th_{k+1}^N (i (\beta) \cup i (t)) \in N$, we get that $p_{k+1} (N) = i (t) \cup \{ \nu \}$, as desired.  The stretching extender witnessing the type $Z_p$ property is of course $(\dot{E}_\nu )^N$.  The generalized core of $N$ is $M | \xi$, so its parameter is solid. $\blacksquare$\\

This completes the proof of our claim. $\blacksquare$\\



Now that we know our phalanx iteration produces mice at all stages, we can argue that the coiteration terminates, with the standard argument establishing that one branch does not drop.  Actually this latter part requires some extra care, because we have seen that in the case of long extenders applied to $M$ with domain $\alpha$, the standard parameter of $M | \xi$ can undergo some change.  However, we can still establish the following, as in \cite{FSPIPM}:\\

\textbf{Claim:}  The comparison of $\langle M, H, \alpha \rangle$ vs. $M$ terminates.  Moreover, letting $P$ be the final model of $\mathscr{T}$ and $Q$ be the final model of $\mathscr{U}$,\\

\indent $a)$ If $P \lhd Q$, then the branch to $P$ does not drop,\\

\indent $b)$ If $Q \lhd P$, then the branch to $Q$ does not drop,\\

\indent $c)$ It is not the case that both the branch to $P$ and the branch to $Q$ drop.\\

Let $P = P_\gamma$ and $Q = Q_\delta$.  Applying weak Dodd-Jensen, we also get the following two claims as in \cite{FSPIPM}:\\

\textbf{Claim:} $1 <_\mathscr{T} \gamma$; that is, $P$ is above $H$ in $\mathscr{T}$.\\

\textbf{Claim:} $P \unlhd Q$, and $[1, \gamma ]_\mathscr{T}$ does not drop.\\


At this point, our Condensation proof diverges from the Solidity proof of \cite{FSPIPM}.\\



\textbf{Claim:}\\

\indent \indent $a)$ No extenders are applied on the $H \longrightarrow P$ branch of $\mathscr{T}$; that is, $H = P$.\\

\indent \indent $b)$ No extenders are applied on the $M \longrightarrow Q$ branch of $\mathscr{U}$; that is, $M = Q$.\\

\indent \indent \textbf{Proof:} Suppose $a)$ fails, and let $E$ be the first extender applied to $H$ on the branch to $P$.  Notice that $dom (E) = (H | \beta )$ for $\beta > \alpha$, since otherwise $E$ would be applied to $M$ in the phalanx.  If $E$ is short, this implies $\kappa_E \geq \alpha$.  But $\varrho_1^H \leq \alpha$, so this is a drop in degree on $H \longrightarrow P$, contradiction.  Now suppose $E$ is long.  If $\kappa_E \geq \alpha$, or more generally if $\varrho_1^H \leq \kappa_E$, we have a contradiction exactly as in the short case.  So we are left with the possibility that $\kappa_E$ is the cardinal predecessor of $\alpha$ and $H$ projects exactly to $\alpha$, i.e., $\alpha = ( \kappa_E^+ )^H = \varrho_1^H$.  But then since $E$ is close to $H$, there is an extender $F$ on the $H$-sequence with the same space as $E$; and this is a violation of projectum-free spaces for $H$, contradicting the fact that $H$ is a plus-one premouse.  This proves $a)$.\\

Now suppose $b)$ fails, and let $E$ be the first extender applied to $M | \xi$ for some $\xi$ on the branch to $Q$.  Then $lh(E)$ is a cardinal of $Q$.  Recall that $lh(E) < \alpha$ is impossible, since we are iterating away disagreements and $H$, $M$ agree below $\alpha$.  Also $lh(E) = \alpha$ is impossible because we are not in Anomalous Case 1.  So $lh(E) > \alpha$.  We must also have $lh(E) < height(H)$, since otherwise it would not be a disagreement in need of removal.  Now $P \lhd Q$ is impossible, since $P = H$ projects to $\alpha$, and would therefore collapse $lh(E)$ in $Q$.  So we have $P = Q$.  But then $\varrho_1^Q = \varrho_1^P \leq \alpha$, so $\varrho_1^Q \leq \kappa_E$, since otherwise it would have been lifted above $lh(E)$ by $i_E$.  (The case where $E$ is long and $\varrho_1^{M | \xi} = \kappa_E^+$ is ruled out by projectum-free spaces and closeness of $E$.)  And now we are iterating above the projectum on the branch to $Q$, so $Q$ is not sound.  This contradicts soundness of $H$, since $H = P = Q$. $\blacksquare$\\

Thus, in the non-anomalous case we have $H \unlhd M$; that is, $a)$ or $b)$ of Condensation.\\


\subsection{Anomalous Case 1}

We have that $E_\alpha^M \neq \emptyset$.  Let $G = E_\alpha^M$.\\

Again, we compare $ \langle M, H, \alpha \rangle$ with $M$, with $\mathscr{T}$ and $\mathscr{U}$ being the two trees, and $\mathscr{T}^*$ the lift of $\mathscr{T}$ under $(id, \pi ) = (\pi_0 , \pi_1 )$.  The rules for $\mathscr{T}$ and $\mathscr{U}$ are as before, and we adopt our previous notation.  That is, $P_\xi = M_\xi^{\mathscr{T} }$, $Q_\xi = M_\xi^{\mathscr{U}}$, $P_\xi^* = M_\xi^{\mathscr{T}^*}$, $\pi_\xi : P_\xi \longrightarrow P_\xi^*$, etc.\\

We have $\alpha = (\lambda_G^+ )^H$, so if $E_\xi$ has critical point $\lambda_G$, then if $E_\xi$ is short, $T-pred (\xi + 1 ) = 0$, while if $E_\xi$ is long, then $T-pred (\xi + 1) = 1$.  (Note $\lambda_{E_1} > \lambda_G$, since $lh(E_1) > \alpha$.)\\

One new problem here is that the models of $\mathscr{T}$ may fail to satisfy the Jensen ISC.  Our computations involving closeness of models in a phalanx still go through as before, so the only possible failure of Jensen ISC occurs when $E_\xi$ is short and $crit(E_\xi) = \lambda_G$.  Our rules for $\mathscr{T}$ then require that\\

\indent \indent $P_{\xi + 1} = Ult_0 (M | \alpha , E_\xi )$.\\

Here $G = \dot{F}^{M | \alpha}$, and $i_{E_\xi} (G) = \dot{F}^{P_\xi + 1}$.  $G$ is a missing whole initial segment of $i_{E_\xi} (G)$, and so $P_{\xi + 1}$ does not satisfy the Jensen ISC.\\

We deal with this problem as in \cite{FSPIPM}.  Notice first that $o(P_{\xi + 1}) = lh(E_\xi) = lh ( i_{E_\xi} (G))$, so $i_{E_\xi} (G)$ is going to be part of the least disagreement between $P_{\xi+1}$ and the current model of $\mathscr{U}$.  (All models of $\mathscr{U}$ are plus-one premice.)  So $\mathscr{T}-pred (\xi + 2) = 0$, and\\

\indent \indent $P_{\xi + 2} = Ult_0 (M , i_{E_\xi} (G) )$.\\

(Since $\lambda_G$ is a cardinal of $H$, it is a cardinal of $M$, so $i_{E_\xi} (G)$ is an extender over $M$.)  Moreover, $\lambda_\xi^{\mathscr{T}} = \lambda_{\xi + 1}^{\mathscr{T}}$, so there are no models above $P_{\xi + 1}$ in $\mathscr{T}$.  It is a dead node.  We then have\\

\textbf{Claim:} All models of $\mathscr{T}$ except those of the form $P_{\xi + 1} = Ult_0 ( M | \alpha , E_\xi )$ with $crit( E_\xi ) = \lambda_G$, $E_\xi$ short, are plus-one premice.  Moreover, except for the $i_{0, \xi + 1}^{\mathscr{T}}$ of this form, the maps of $\mathscr{T}$ preserve parameters and cores.\\

We get that comparison terminates as before.  Let $P = P_\gamma^{\mathscr{T}}$ and $Q = Q_\delta^{\mathscr{U}}$ be the last models.  We need more argument now to show\\

\textbf{Claim:} $1 T \gamma$; that is, $P$ is above $H$ in $\mathscr{T}$.\\

\indent \indent \textbf{Proof:} Suppose $P$ is above $M$.\\

\textbf{Subclaim:}\\

\indent $a)$ $P = Q$;\\

\indent $b)$ neither $[ 0 , \gamma ]_{\mathscr{T}}$ nor $[ 0 , \delta ]_{\mathscr{U}}$ drops;\\

\indent $c)$ $i_{0 , \gamma }^{\mathscr{T}} = i_{0 , \delta }^{\mathscr{U}}$.\\

\indent \indent \textbf{Proof:} The same weak Dodd-Jensen argument. $\blacksquare$\\

Now let $K = E_\eta^{\mathscr{T}}$ and $L = E_\rho^{\mathscr{U}}$ be the first extenders used in $[ 0 , \gamma ]_{\mathscr{T}}$ and $[ 0 , \delta ]_{\mathscr{U}}$ respectively.  Let\\

\indent $ (P^* , K^* ) = Ult_0 (P_\eta | lh (K), F_0 )$, \ \ and \ \ $ (Q^* , L^* ) = Ult_0 (Q_\rho | lh (L), F_1 )$,\\

where $F_0$ is the short part of the branch tail extender of $i_{\eta + 1 , \gamma}^{\mathscr{T}}$ and $F_1$ the short part of the branch tail extender of $i_{\rho + 1 , \delta}^{\mathscr{U}}$.  As before, $K^* = L^*$, and both are long.  Since $L$ had the Jensen ISC, letting $\nu^*$ be the largest long generator of $L^*$,\\

\indent $\lambda_L =$ least $\theta$ such that $L^* \restriction ( \theta \cup \{ \nu^* \}  )$ is a missing whole initial segment,\\

and \ \ $L = L^* \restriction ( \lambda_L \cup \{ \nu^* \} )$.\\

If $K$ had the Jensen ISC, this would give $K = L$, contradiction.  So assume $K$ does not have the Jensen ISC.  This implies that $K = i_{E_\xi}^{\mathscr{T}} (G)$, where $crit (E_\xi ) = \lambda_G$, and $L$ is the least whole initial segment of $K$, that is, $L = G$.  (Note $G = E_0^{\mathscr{U}}$.)  So $\rho = 0$, and $1 \in [ 0 , \delta ]_{\mathscr{U}}$.  We have\\

\indent $i_{0 , \gamma }^{\mathscr{T}} (\kappa_K ) = i_{\eta + 1 , \gamma }^{\mathscr{T}} (\lambda_K )$\\

\indent \indent $= i_{0 , \delta}^{\mathscr{U}} (\kappa_G )$\\

\indent \indent $= i_{1 , \delta}^{\mathscr{U}} (\lambda_G )$.\\

But $\lambda_G = \kappa_{E_\xi} < \lambda_K$, so $crit (i_{1, \delta}^{\mathscr{U}} ) = \lambda_G $.  Let $F$ be the first extender used in $( 1, \delta ]_{\mathscr{U}}$, say $F = E_\beta^{\mathscr{U}}$, and let $F^*$ be the stretch of $F$ by the short part of the branch tail extender of $i_{\beta + 1 , \delta }^{\mathscr{U}}$.\\

\textbf{Subclaim:} $E_\xi \restriction \lambda_{E_\xi} = F^* \restriction \lambda_{E_\xi}$.\\

\indent \indent \textbf{Proof:} $\lambda_{E_\xi} = \lambda_K$.  Both extenders have critical point $\lambda_G$, and measure subsets of $\lambda_G$ in $M | lh(G)$.  Let $A \subseteq \lambda_G$ and $A \in M | lh(G)$.  We can write\\

\indent $A = [a , f]_G = i_G (f) (a)$,\\

where $f: [\kappa_G ]^{|a|} \longrightarrow \mathcal{P} (\kappa_G )$, $f \in M | (\kappa_G^+ )^M$, and $a \subseteq \lambda_G$.  But then\\

\indent $i_{E_\xi} (A) = [ a, f ]_K = i_K (f) (a)$.\\

But also\\

\indent $i_{F^*} (A) = i_{1, \delta}^{\mathscr{U}} ( i_G (f)(a))$\\

\indent \indent \indent $= i_{1, \delta}^{\mathscr{U}} ( i_G (f)) (i_{1, \delta}^{\mathscr{U}} ( a))$\\

\indent \indent \indent $= i_{0, \delta}^{\mathscr{U}} ( f)(a)$\\

\indent \indent \indent $= i_{0, \gamma}^{\mathscr{T}} ( f)(a)$\\

\indent \indent \indent $= i_{\eta + 1, \gamma}^{\mathscr{T}} ( i_K(f))(a)$\\

\indent \indent \indent $= i_{\eta + 1, \gamma}^{\mathscr{T}} ( i_K(f)(a))$.\\

So $ i_{\eta + 1, \gamma}^{\mathscr{T}} (i_{E_\xi} (A)) = i_{F^*} (A)$.  Since $crit (i_{\eta + 1 , \gamma }^{\mathscr{T}} ) = \lambda_K$, we have proved the subclaim. $\blacksquare$\\

But then if $F$ is short, we have that $E_\xi$ is the least missing whole initial segment of $F^*$, and so is $F$, so that $E_\xi = F$, a contradiction.  If $F$ is long, then $E_\xi$ is the least missing whole initial segment of $F^* \restriction \lambda_{F^*}$, so $E_\xi \in Q_\delta$, again a contradiction.  This proves the claim. $\blacksquare$\\

Here again we diverge from the solidity proof of \cite{FSPIPM}.\\

\textbf{Claim:}\\

\indent \indent $a)$ No extenders are applied on the $H \longrightarrow P$ branch of $\mathscr{T}$; that is, $H = P$.\\

\indent \indent $b)$ Exactly one extender, namely $G$, is applied on the $M \longrightarrow Q$ branch of $\mathscr{U}$; that is, $Q = Ult(M , G)$.\\

\indent \indent \textbf{Proof:} The proof of $a)$ is exactly like in the non-anomalous case.  For $b)$, we know that $E_0^{\mathscr{U}} = G$, because $\alpha$ is necessarily the least disagreement with the current model of $\mathscr{T}$ (namely, $H$).  Now suppose toward contradiction that $E_1^{\mathscr{U}} = E$ is defined.  (Recall that, in our notation, $E$ will be applied to $Q_2^*$, the longest possible initial segment of $Q_1$.)\\

 Then $lh(E)$ is a cardinal of $Q$, and $lh(E) > \alpha$.  We must also have $lh(E) < height(H)$, since otherwise it would not be a disagreement in need of removal.  Now $P \lhd Q$ is impossible, since $P = H$ projects to $\alpha$, and would therefore collapse $lh(E)$ in $Q$.  So we have $P = Q$.  But then $\varrho_1^Q = \varrho_1^P \leq \alpha$, so $\varrho_1^Q \leq \kappa_E$, since otherwise it would have been lifted above $lh(E)$ by $i_E$.  (The case where $E$ is long and $\varrho_1^{Q_2^*} = \kappa_E^+$ is ruled out by projectum-free spaces and closeness of $E$.)  And now we are iterating above the projectum on the branch to $Q$, so $Q$ is not sound.  This contradicts soundness of $H$, since $H = P = Q$. $\blacksquare$\\

Thus, in Anomalous Case 1 we have $H \unlhd Ult_n(M, E_\alpha)$ for largest possible $n$; that is, $c)$ of Condensation.\\








\subsection{Anomalous Case 2} \label{AC2 subsection}

In this case we have that $\alpha$ is a pseudo-index of some $\bar{G}$; that is, there is a least $\xi > \alpha$ such that $\varrho_1 (M | \xi ) < \alpha$, and this $M | \xi$ is type $Z_1$, with last extender $G$ and stretching extender $F = \dot{E}_{\dot{\nu}}^{M | \xi }$; moreover,\\

\indent $M | \xi \models \alpha = \kappa_F^{++}$.\\

Let
\[
\bar{G} = \bigcup_{\eta < \dot{\nu}^{M | \xi}} i_F^{-1} `` G \restriction \eta
\]

be the extender pseudo-indexed at $\alpha$.  So $\nu (\bar{G}) = (\kappa_F^+)^M = (\kappa_F^+)^H$, and $(\nu (\bar{G})^+)^{Ult(M, G)} = \alpha$.\\

The proof is similar to the non-anomalous case, but instead of comparing $\langle M , H, \alpha \rangle$ with $M$, we compare $\langle M , H, \alpha \rangle$ with

\[
\Phi = \langle M , Ult (M, \bar{G}) , \kappa_F^+ \rangle .
\]

We let $\mathscr{T}$ be the tree on

\[
\Psi = \langle M , H, \alpha \rangle
\]

that is produced, and $\mathscr{U}$ the tree on $\Phi$.  The rules for forming $\mathscr{T}$ are similar to those in Anomalous Case 1.  Again, let $P_\eta = M_\eta^{\mathscr{T}}$, with $P_0 = M$ and $P_1 = H$.  Let $E_\eta = E_\eta^{\mathscr{T}}$.  Let $\lambda_\eta^{\mathscr{T}} = \lambda_{E_\eta^{\mathscr{T}}}$ for $\eta \geq 1$, and $\lambda_0^{\mathscr{T}} = \alpha$.  We have

\[
T-pred (\eta + 1) = \text{ least } \beta \text{ such that } dom( E_\eta) \subseteq P_\beta | \lambda_\beta .
\]

$E_\eta$ then gets applied to the longest initial segment of $P_\beta$ possible, but with the following exception.  Suppose that $crit(E_\eta ) = \kappa_F$, and $E_\eta$ is long.  As in the proof of the non-anomalous case, we can show that $E_\eta$ has exactly one long generator.  We shall set

\[
P_{\eta + 1} = (Ult_0 (M | \alpha , E_\eta ), K)
\]

where $K$ is the extender of length $lh(E_\eta)$ determined by $i_{E_\eta} \circ i_{\bar{G}}$, in this case.\\

\textbf{Claim:}  Let $crit(E_\eta) = \kappa_F$, and suppose that $E_\eta$ is long; then\\

1) $E_\eta$ has exactly one long generator,\\

2) if $K$ is the extender of length $lh(E_\eta)$ over $M$ generated by $i_{E_\eta} \circ i_{\bar{G}}$, then\\

\indent \indent a) $(Ult_0 (M| \alpha , E_\eta), K)$ is a type $Z_1$ premouse, with stretching extender $E_\eta \restriction \dot{\nu}^{(P_\eta | lh (E_\eta)}$, and\\

\indent \indent b) $(Ult_0 (M | \alpha , E_\eta), K)$ adds $\bar{G}$.\\

The proof of this claim is similar to the proof of the corresponding claim in the non-anomalous case, so we omit it.\\

We shall show that if $P_{\eta + 1} = (Ult_0 (M | \alpha , E_\eta), K)$ as above, then\\

\[
E_{\eta + 1} = K,
\]

and then of course\\

\[
P_{\eta + 2} = Ult_0 (M , K).
\]

Moreover, since $\lambda_K = \lambda_{E_\eta}$, there are no models in $\mathscr{T}$ above $P_{\eta + 1}$ -- it is a dead node.  To show this, we have to look at $\mathscr{U}$.\\

We set $Q_\eta = M_{\eta}^{\mathscr{U}}$, with $Q_0 = M$ and $Q_1 = Ult(M, \bar{G})$.  We let $\lambda_\eta^{\mathscr{U}} = \lambda_{E_\eta}$, where $F_\eta = E_\eta^{\mathscr{U}}$, for $\eta \geq 1$.  We set $\lambda_0^{\mathscr{U}} = \kappa_F^+$ by fiat.  The rules for $\mathscr{U}$ are

\[
\mathscr{U}-pred (\eta + 1) = \text{ least $\beta$ such that } dom(F_\eta) \subseteq Q_\beta | \lambda_\beta^{\mathscr{U}}.
\]

There are no exceptions, and $F_\eta$ always gets applied to the longest possible initial segment of $Q_\beta$, for $\beta = \mathscr{U}-pred (\eta + 1)$.  Note $\lambda_1^{\mathscr{U}} > \alpha$.  So if $crit(F_\eta) = \kappa_F$, and $F_\eta$ is short, then $\mathscr{U}-pred (\eta + 1) = 0$, and

\[
Q_{\eta + 1} = Ult_0 (M, F_\eta),
\]

while if $crit(F_\eta) = \kappa_F$ and $F_\eta$ is long, then $\mathscr{U}-pred (\eta + 1) = 1$, and

\[
Q_{\eta + 1} = Ult_0 (Q_1, F_\eta ).
\]

\textbf{Claim:} Let $\eta \geq 1$; then no $Q_\eta | \gamma$ adds $\bar{G}$.\\

\textbf{Proof:} $\bar{G} \notin Q_\eta$ for $\eta \geq 1$, because $Q_\eta \models `` \alpha \text{ is a cardinal"}$, when $\eta \geq 1$.  It follows that if $Q_\eta | \gamma$ adds $\bar{G}$, and $\eta \geq 1$, then $Q_\eta | \gamma = Q_\eta$, and the branch of $\mathscr{U}$ ending at $Q_\eta$ does not drop.  So assume this.  Let

\[
F^* = ( \dot{E}_{\dot{\nu}})^{Q_\eta}
\]

be the stretching extender involved in adding $\bar{G}$.  Thus $\kappa_{F^*} = \kappa_F$.  Now let

\[
j =
\begin{cases}
i_{0, \eta}^{\mathscr{U}} & \text{if } 0 \mathscr{U} \eta\\

i_{1, \eta}^{\mathscr{U}} \circ i_{\bar{G}} & \text{if } 1 \mathscr{U} \eta \text{ or } 1 = \eta .
\end{cases}
\]

Thus $j: M \longrightarrow Q_\eta$.  Since $\kappa_{F^*} \in ran(j)$, and $j (crit(j)) > \alpha$, we get $\kappa_F < crit(j)$.  But then $crit(j) > \alpha$, while the rules of $\mathscr{U}$ guarantee $crit(j) \leq \kappa_F$, contradiction. $\blacksquare$\\

This last claim implies that if $P_{\eta + 1} = (Ult_0 (M | \alpha , E_\eta) , K)$ as in the previous claim, then $E_{\eta + 1} = K$, because $P_{\eta + 1}$ does add $\bar{G}$, so it is not lined up with the current model of $\mathscr{U}$.\\

We choose branches for $\mathscr{T}$ and $\mathscr{U}$ at limit stages by lifting them to trees on $M$, and using $\Sigma$.  For $\mathscr{T}$, let $\mathscr{T}^*$ be the lift of $\mathscr{T}$ under $(\pi_0 , \pi_1 )$, where $\pi_0 = id$ and $\pi_1 : H \longrightarrow M$ is the uncollapse.  $\pi_0 \restriction \lambda_0^{\mathscr{T}} = \pi_1 \restriction \lambda_0^{\mathscr{T}}$, which is what we need.  Let $P_\eta^* = M_\eta^{\mathscr{T}^*}$, and

\[
\pi_\eta: P_\eta \longrightarrow P_\eta^*
\]

be the natural map, but with the following exception related to our special case in the definition of $\mathscr{T}$.

Namely, suppose $crit(E_\eta) = \kappa_F$, and $E_\eta$ is long.  By a previous claim, $E_\eta$ has a unique long generator $\nu$.  We have set $P_{\eta + 1} = (Ult_0 (M | \alpha , E_\eta) , K)$, where $K$ is the extender of $i_{E_\eta} \circ i_{\bar{G}}$ over $P_{\eta + 1}$.  By that claim, $P_{\eta + 1}$ is type $Z_1$, and $\nu = \dot{\nu}^{P_{\eta+1}}$.  Let $E = E_\eta$, and $E^* = \pi_\eta (E)$.  Note that $dom (E^*) = \pi_\eta (H | \alpha) = \pi_1 (H | \alpha ) = M | \pi_1 (\alpha)$, so $E^*$ is an extender over all of $M$.  We set

\[
P_{\eta + 1}^* = Ult_0 (M, E^*)
\]

and more importantly

\[
G^* = i_{E^*} (G).
\]

Since $\kappa_{G}^{++} < \kappa_F$, $\kappa_{G^*} = \kappa_G$, and $G^*$ is total on $M$.  We shall leave $\pi_{\eta + 1}$ undefined.  All we need is\\

\textbf{Subclaim:} There is a $\gamma$ such that $K = G^* \restriction ( \pi_\eta `` \lambda_E \cup \{ \gamma \} )$.\\

\textbf{Remark:} Note $\lambda_E = \lambda_K$.  A better way of saying it might be: $K \restriction ( \lambda_K \cup \{ \nu \} ) = G^* \restriction ( \pi_\eta `` \lambda_E \cup \{ \gamma \} )$.\\

The subclaim is enough to go on and get $\pi_{\eta + 2}: P_{\eta + 2} \longrightarrow P_{\eta + 2}^*$.  We don't need $\pi_{\eta + 1}$ as a map on all of $P_{\eta + 1}$, because we are never going to take an ultrapower of $P_{\eta + 1}$ in forming $\mathscr{T}$.\\

\textbf{Proof of Subclaim:} Let

\[
i_E^{M | \xi} : M | \xi \longrightarrow Ult_0 (M | \xi , E ) = R
\]

be the canonical embedding. $i_E^{M | \xi}$ is discontinuous at $\kappa_F^+$, so $R$ is not a plus-one premouse.  ($E$ is not close to $M | \xi$.) Let $F_1 = i_E^{M | \xi} (F)$, and $\nu^* = i_{F_1} (\nu )$.\\

We claim that

\[
K \restriction (\lambda_K \cup \{ \nu \} ) = i_E^{M | \xi} (G) \restriction (\lambda_K \cup \{ \nu^* \} ).
\]

Both extenders have space $(\kappa_G^+)^M$.  Let $b \in [ \lambda_K ]^{< \omega}$ and $A \in M | (\kappa_G^{++})^M$; we need to see that $( b , \nu ) \in i_K (A)$ iff $(b, \nu^* ) \in i_{G_1} (A)$, where $G_1 = i_E^{M| \xi} (G)$.  But note that $i_F (\bar{G}) \subseteq G$, so

\[
(M | \xi , G ) \models \forall u \in [ \kappa_E ]^{< \omega } \ \forall \xi < \kappa_E^+ ((u, \xi) \in i_{\bar{G}}(A) \Longleftrightarrow (u, i_F (\xi)) \in i_G (A)).
\]

The formula on the right is of the form $\psi ( \kappa_E , i_{\bar{G}} (A) , F, A)$.  That is, those are the parameters.  Applying $i_E^{M | \xi}$, we get

\[
Ult_0 ((M | \xi , G ), E) \models \forall u \in [ \lambda_E ]^{< \omega} \ \forall \xi < \lambda_E^+ ((u, \xi) \in i_E (i_{\bar{G}} (A) \Longleftrightarrow (u, i_{F_1}(\xi)) \in i_E (i_G (A))).
\]

But $i_E (i_{\bar{G}} (A)) = i_K (A)$, $i_E (i_G (A)) = i_{G_1} (i_E (A)) = i_{G_1}(A)$, and $b \in [ \lambda_E ]^{< \omega }$ and $\nu < \lambda_E^+$.  So

\[
(b, \nu ) \in i_K (A) \Longleftrightarrow (b, \nu^*) \in i_{G_1} (A),
\]

as desired.  Thus $K \restriction (\lambda_K \cup \{ \nu \} ) = i_E (G) \restriction (\lambda_K \cup \{ \nu^* \} )$.\\

But now let $\sigma : Ult_0 ((M | \xi , G ), E) \longrightarrow i_{E^*} (( M | \xi , G))$ be the natural map, given by the shift lemma:

\[
\sigma ( [a \cup \{ \nu \} , f ]_E^{M | \xi} ) = [ \pi_\eta (a) \cup \{ \gamma \} , f ]_{E^*}^M
\]

for $a \subseteq \lambda_E$.  Then $\sigma \restriction \lambda_E = \pi_\eta \restriction \lambda_E$, and we define $\sigma (\nu^* ) = \gamma$.  Then $i_E (G)$ is a subextender of $i_{E^*} (G)$ under $\sigma$, finishing the proof of the subclaim. $\blacksquare$\\

We can now define

\[
P_{\eta + 2}^* = Ult_0 (M, G^* ).
\]

Recalling that $P_{\eta + 2} = Ult_0 (M, K)$, we set

\[
\pi_{\eta + 2} ( [ a \cup \{ \nu \} , f ]_K^M ) = [ a \cup \{ \gamma \} , f ]_{G^*}^M
\]

where $\gamma$ is as in the subclaim.  We then have $\pi_{\eta + 2} \restriction \lambda_K = \pi_\eta \restriction \lambda_K$, as desired.\\

We have shown\\

\textbf{Claim:} $\langle M, H, \alpha \rangle$ is iterable by the rules described; moreover $\mathscr{T}$ is lifted to a tree $\mathscr{T}^*$ according to $\Sigma$.\\

We turn now to the iterability of $\langle M, Ult_0 (M, \bar{G}), \kappa_F^+ \rangle$.\\

We lift $\mathscr{U}$ to a tree $\mathscr{U}^*$ on $M$ as follows.  The first two models of $\mathscr{U}^*$ are

\[
Q_0^* = M \ \ \ \text{and} \ \ \ Q_1^* = Ult_0 (M, G).
\]

(Note $\kappa_G^{++} < \kappa_F$ is a cardinal of $M$.)  We define maps

\[
\sigma_\eta : Q_\eta \longrightarrow Q_\eta^*
\]

with

\[
\sigma_0 = id
\]

and

\[
\sigma_1 ( [ a, f ]_{\bar{G}}^M ) = [ i_F (a) , f ]_G^M
\]

for $a \subseteq \nu ( \bar{G}) = \kappa_F^+$ finite.  Since $\bar{G}$ is a subextender of $G$ under $i_F$, this makes sense.  Suppose now we are given $\sigma_\eta : Q_\eta \longrightarrow Q_\eta^*$, and suppose by induction that we have 

\[
( \dagger ) \ \ \text{for } 0 < \gamma \leq \eta , \ Q_\gamma^* \text{ agrees with } Q_\eta^* \text{ below } \lambda_\gamma^{\mathscr{U}^*} , \text{ and }\ 
\sigma_\gamma \restriction \lambda_\gamma^{\mathscr{U}} = \sigma_\eta \restriction \lambda_\gamma^{\mathscr{U}}.
\]

Note that we don't have this for $\gamma = 0$, because $\sigma_0$ and $\sigma_1$ only agree up to $\kappa_F$, while $\lambda_0^{\mathscr{U}} = \kappa_F^+$.  If $\mathscr{U}-pred (\eta + 1) \neq 0$, or $crit (F_\eta) \neq \kappa_F$, then we get $Q_{\eta + 1}^*$ and $F_\eta^*$ and $\sigma_{\eta + 1}$ by the Shift Lemma, in the usual way:

\[
F_\eta^* = \sigma_\eta ( F_\eta ),
\]
\[
Q_{\eta + 1}^* = Ult_k (P, F_\eta^* ),
\]

where $\gamma = \mathscr{U}-pred (\eta + 1 )$, $P = \sigma_\gamma ( (M_{\eta +1}^*)^{\mathscr{U}} )$, and $k = deg^{\mathscr{U}} (\eta + 1)$, and

\[
\sigma_{\eta + 1} ( [a, f]_{F_\eta}^{(M_{\eta + 1}^* )^{\mathscr{U}}} ) = [ \sigma_\eta (a) , \sigma_\gamma (f) ]_{F_\eta^*}^P \ \ .
\]

One can easily check that $(\dagger)$ remains true.\\

\textbf{Remark:}  In fact, for $0 < \gamma < \eta$, $Q_\eta^*$ agrees with $Q_\gamma^*$ below $lh (F_\gamma^* )$, and $\sigma_\gamma \restriction lh (F_\gamma ) = \sigma_\eta \restriction lh (F_\eta )$.  But we only use agreement up to $\lambda_{F_\gamma }$, because our trees are (generally) formed with short extender rules, as they must be because our background extenders in a plus-one construction are short.

We tend to record the agreement between models and lifting maps in an iteration using the $\lambda_{F_\gamma }$ rather than the $lh ( F_\gamma )$.  This reminds us that trying to use the additional agreement we might have to apply an extender $E$ with $crit (E) = \lambda_{F_\gamma }$ to $Q_\gamma$ (say) could lead to problems with iterability.\\

Now suppose $U-pred (\eta + 1 ) = 0$, $crit ( F_\eta) = \kappa_F $, and $F_\eta$ is short.  So

\[
Q_{\eta + 1} = Ult_0 (M , F_\eta ) \ .
\]

We cannot set $Q_{\eta + 1 }^* = Ult_0 ( Q_0^* , \sigma_\eta (F_\eta ))$ now, because although the latter ultrapower makes sense (by the agreement of $Q_0^*$ with $Q_\eta^*$ up to $lh (G)$), $\sigma_\eta$ and $\sigma_0$ do not agree far enough that we could define $\sigma_{\eta + 1}$ properly.  Instead, let

\[
j : Q_\eta^* \longrightarrow Ult_0 ( Q_\eta^* , \sigma_\eta (F_\eta ))
\]

be the canonical embedding.  Note that $F$ is on the $Q_1^*$-sequence (though not on the $Q_1$-sequence), and hence $F$ is on the $Q_\eta^*$-sequence.  We have

\[
crit(j) = \sigma_\eta (\kappa_F ) = \sigma_1 (\kappa_F) = \lambda_F \ .
\]

Set

\[
Q_{\eta + 1}^* = Ult_0 (Q_0^* , j (F)) \ ,
\]

and

\[
\sigma_{\eta + 1} ( [ a, f ]_{F_\eta }^{Q_0} ) = [ \sigma_\eta (a) , f ]_{j (F)}^{Q_0^*} \ .
\]

(Recall that $Q_0^* = Q_0 = M$.)  Since $\kappa_F = \kappa_{j(F)}$, $Q_{\eta + 1}^*$ makes sense.  Moreover, $lh(F) < (\lambda_F^+ )^{Q_\eta^*}$, so $lh (j (F)) < j ( (\lambda_F^+)^{Q_\eta^+})$, so $j(F)$ is on the $Q_\eta^*$-sequence by coherence of $\sigma_\eta (F_\eta )$ with that sequence.  Also, $\kappa_{j (F)} = \kappa_F < \lambda_G$.  Thus $Q_{\eta + 1}^*$ is a legitimate next model for $\mathscr{U}^*$.\\

We must see that $\sigma_{\eta + 1}$ is well-defined and elementary.  Let $E = F_\eta$.  $E$ is short, and $\kappa_E = \kappa_F = \kappa_{j(F)}$.  Let $a \subseteq \lambda_E$ be finite and $X \subseteq [ \kappa_E ]^{|a|}$ with $X \in M$.  It is enough to show that

\[
(a, X) \in E \ \ \text{iff} \ \ (\sigma_\eta (a), X ) \in j(F) \ .
\]

But we have

\[ 
\begin{split} 
(a, X) \in E & \text{  iff  }  a \in i_E(X)\\ 
& \text{  iff  } \sigma_\eta (a) \in \sigma_\eta (i_E (X)) \\
& \text{  iff  } \sigma_\eta (a) \in \sigma_\eta (i_E ) (\sigma_\eta (X))\\
& \text{  iff  } \sigma_\eta (a) \in j (i_F (X))\\
& \text{  iff  } \sigma_\eta (a) \in j(i_F)(X)\\
& \text{  iff  } \sigma_\eta (a) \in i_{j(F)}(X)\\
& \text{  iff  } (\sigma_\eta (a) , X) \in j(F).
\end{split}
\]

For the fourth line, notice that $\sigma_\eta (X) = \sigma_1 (X) = i_F (X)$, because $X \subseteq [\kappa_F ]^{|a|}$.  For the fifth line, note $j(X) = X$.

Clearly $\sigma_{\eta + 1} \restriction \lambda_{F_\eta} = \sigma_\eta \restriction \lambda_{F_\eta}$, since $\sigma_{\eta + 1} ( [ \{ \tau \} , id ]_{F_\eta}^M ) = [ \{ \sigma_\eta (\tau ) \} , id ]_{j (F)}^M $.\\


This completes the successor step in the formation of $\mathscr{U}^*$.  At limit stages, we use $\Sigma$ to choose a branch of $\mathscr{U}^*$, and then choose the same branch for $\mathscr{U}$.  We have shown\\

\textbf{Claim:}  $\langle M, Ult (M, \bar{G}) , \kappa_F^+ \rangle$ is iterable via the strategy described above.\\

\textbf{Claim:} The comparison of $\langle M, H, \alpha \rangle$ with $\langle M, Ult (M, \bar{G}) , \kappa_F^+ \rangle$ terminates.\\

\textbf{Proof:} As before. $\blacksquare$\\

Now let $P = P_\gamma$ and $Q = Q_\delta$ be the last models on the two sides.\\

\textbf{Claim:} It is not the case that both $P$ and $Q$ are above $M$ in their respective trees.\\

\textbf{Proof:}  Suppose they were, i.e., suppose $0 \mathscr{T} \gamma$ and $0 \mathscr{U} \delta$.\\

\textbf{Subclaim:} $P = Q$, neither $[0, \gamma]_{\mathscr{T}}$ nor $[ 0 , \delta ]_{\mathscr{U}}$ drops, and $i_{0, \gamma}^{\mathscr{T}} = i_{0, \delta}^{\mathscr{U}}$.\\

\textbf{Proof:} We use $\mathscr{T}^*$, $\mathscr{U}^*$, and the weak Dodd-Jensen property of $\Sigma$. $\blacksquare$\\

Now let $K$ and $L$ be the first extenders used in $[0 , \gamma ]_{\mathscr{T}}$ and $[0, \delta ]_{\mathscr{U}}$, and let $K^*$ and $L^*$ be their stretches by the short parts of their respective branch tails.  We may assume both $K^*$ and $L^*$ are long.  Both $K$ and $L$ had the Jensen ISC, so this gives $K = L$.  That is a contradiction.\\

\textbf{Remark:} There is the case that $K$ is the extender of $i_{E_\eta} \circ i_{\bar{G}}$.  But then $K$ adds $\bar{G}$, and we showed that no $F_{\tau}$ adds $\bar{G}$, so $K \neq L$. $\blacksquare$\\


\textbf{Claim:} It is not the case that $P$ is above $M$ and $Q$ is above $Ult (M, \bar{G} )$.\\

\textbf{Proof:} Suppose they were, i.e., $0 \mathscr{T} \gamma$ and $1 \mathscr{U} \delta$.\\

\textbf{Subclaim:} $P = Q$, neither $[0, \gamma ]_{\mathscr{T}}$ nor $[1, \delta ]_{\mathscr{U}}$ drops, and $i_{0, \gamma}^{\mathscr{T}} = i_{1, \delta}^{\mathscr{U}} \circ i_{\bar{G}}$.\\

\textbf{Proof:} Again, we use $\mathscr{T}^*$, $\mathscr{U}^*$, and the weak Dodd-Jensen property of $\Sigma$. $\blacksquare$\\

\textbf{Subclaim:} $crit (i_{1, \delta}^{\mathscr{U}}) = \lambda_{\bar{G}}$.\\

\textbf{Proof:} If not, then

\[
i_{0, \gamma}^{\mathscr{T}} (\kappa_G ) = i_{0, \delta}^{\mathscr{U}} (\kappa_G ) = i_{1, \delta }^{\mathscr{U}} (i_{\bar{G}} (\kappa_G)) = i_{\bar{G}} (\kappa_G) = \lambda_{\bar{G}} \ .
\]

But it is easy to see that $\lambda_\eta^{\mathscr{T}} > \alpha$ for all $\eta \geq 1$.  So $i_{0, \gamma }^{\mathscr{T}} (\kappa_G ) > \alpha > \lambda_{\bar{G}}$. $\blacksquare$\\

Now let $K = E_\eta$ be the first extender used in $[0, \gamma]_{\mathscr{T}}$, and let

\[
(P^* , K^* ) = Ult_0 ( (P_\eta | lh(K) , K), W_0) \ \ \text{and} \ \ (Q^* , G^* ) = Ult_0 ( (M | \kappa_F^+ , \bar{G} ) , W_1 ) \ ,
\]

where $W_0$ is the short part of the extender of $i_{\eta + 1 , \gamma}^{\mathscr{T}}$, and $W_1$ is the short part of the extender of $i_{1, \delta}^{\mathscr{U}}$.  As in comparison, we have that $P^*$ and $Q^*$ are initial segments of $P_\gamma = Q_\delta$, below $i_{0, \gamma}^{\mathscr{T}} (\kappa_G^+)^{P_\gamma}$.  Moreover, $K^*$ and $G^*$ are initial segments of the extender of $i_{0, \gamma}^{\mathscr{T}} = i_{1, \delta}^{\mathscr{U}} \circ i_{\bar{G}}$.\\

Note $K^* \notin P_\gamma$, because $K \notin P_\gamma$.  Every proper initial segment of $G^*$ is in $Q^*$, so $K^*$ is not an initial segment of $G^*$.  $K^* \neq G^*$, because $(P^* , K^* )$ is a plus-one premouse, and $G^*$ has no largest generator.  It follows that $G^* = K^* \restriction \nu^*$, where $\nu^*$ is the largest generator of $K^*$.  Thus $(P^* , K^*)$ is type $Z_1$, with stretching extender

\[
F = \dot{E}_{\nu^*}^{P^*} = \dot{E}_{\nu^*}^{P_\gamma} = \dot{E}_{\nu^*}^{Q_\delta} \ ,
\]

and satisfying

\[
K^* \restriction \nu^* = G^* \ .
\]

Now let $\tau + 1$ be least in $( 1, \delta ]_{\mathscr{U}}$, so that $crit (F_\tau) = \lambda_{\bar{G}}$ by previous subclaim, and $F_\tau$ is long by our rules for $\mathscr{U}$.  Let

\[
H = \text{ least long initial segment of $F_\tau$ on the $Q_\tau$-sequence.}
\]

(We shall show shortly that $H = F_\tau$.)  Let

\[
(R^* , H^* ) = Ult_0 (( Q_\tau | lh (H) , H) , W_2) \ ,
\]

where $W_2$ is the short part of the branch-tail extender of $i_{\tau+ 1 , \delta}^{\mathscr{U}}$.  Let also

\[
L^* = \text{ extender of } i_{H^*} \circ i_{\bar{G}} \restriction (M | \alpha ) \ .
\]

\textbf{Subclaim:}\\

\indent \indent $1)$ $K^* = L^*$,\\

\indent \indent $2)$ $E_{\nu^*}^{P_\gamma} = H^* \restriction \nu^* = W_1$, \ \ and\\

\indent \indent $3)$ $(P^* , K^*)$ adds $\bar{G}$.\\

\textbf{Proof:} The extender of $i_{H^*} \circ i_{\bar{G}}$ is just the extender of $i_{1, \delta}^{\mathscr{U}} \circ i_{\bar{G}}$, restricted to the first of its generators that is $\geq \text{sup } i_{1, \delta} ``( \nu (\bar{G}))$, plus one. But

\[
\text{sup } i_{1, \delta} `` (\nu (\bar{G}) = i_{W_1} (\nu (\bar{G})) = \nu^* \ .
\]

So

\[
\begin{split}
L^* & = \text{ trivial completion of } E_{i_{1, \delta}^{\mathscr{U}} \circ i_{\bar{G}}} \restriction (\nu^* + 1 )\\
& = \text{ trivial completion of } K^* \restriction (\nu^* + 1 )\\
& = K^*.
\end{split}
\]

The branch tail $W_2$ stretches $H \restriction \lambda_H$ into $W_1$, by calculations we have done before.  Moreover, $\text{sup } i_{1, \delta } `` (\nu(\bar{G}))$ is where the superstrong part of $H^*$ is indexed in $Ult_0 ( Ult_0 (M, \bar{G}), H^*)$.  This gives us $2)$.

Finally, $H^* \restriction \nu^*$ stretches $\bar{G}$ into $G^*$, by the way we have defined $G^*$.  This gives $3)$. $\blacksquare$\\

Recall that $K = E_\eta^{\mathscr{T}}$, where $\mathscr{T}$-pred $(\eta + 1) = 0$.  We also have $\lambda_K > \alpha$.  Since $(P^* , K^* )$ adds $\bar{G}$, we must then have our special case

\[
K = \text{ extender of } i_{E_\sigma} \circ i_{\bar{G}} \restriction (M | \alpha ) \ ,
\]

where $crit(E_\sigma ) = \lambda_{\bar{G}}$ and $E_\sigma$ has exactly one long generator.\\

\textbf{Subclaim:} $E_\sigma = H$.\\

\textbf{Proof:} Let

\[
\begin{split}
E_\sigma^* & = \text{ last extender of } Ult_0 (P_\eta , W_0 )\\
& = \text{ extender with exactly one long generator determined by } i_{\eta + 1 , \gamma}^{\mathscr{T}} \circ i_{E_\sigma} \restriction (M | \alpha ) \ .
\end{split}
\]

It is not hard to show that $E_\sigma^* = H^*$.  For both have critical point $\lambda_{\bar{G}}$, and measure subsets of $\nu(\bar{G}) = (\lambda_{\bar{G}}^+)^M$ that belong to $M | \alpha$, or equivalently, belong to $Ult (M , \bar{G})$.  Let

\[
X = i_{\bar{G}}(f)(a) \in M| \alpha \ .
\]

We then have that

\[
E_\sigma^* \restriction \nu^* = E_{\nu^*}^{P_\gamma} = H^* \restriction \nu^* \ .
\]

The second equality we have already shown, and the first comes from $E_\sigma \restriction (\text{sup } i_{E_\sigma} `` \nu(\bar{G}))$ being the stretching extender in the type $Z_1$ structure $P_\eta$, and the fact that $i_{W_0}$ preserves this.  So

\[
i_{E_\sigma^*} \restriction \nu (\bar{G}) = i_{H^*} \restriction \nu (\bar{G}) \ .
\]

But then

\[
\begin{split}
i_{E_\sigma^*} (X) \cap (\nu^* + 1 ) & = i_{E_\sigma^*} (i_{\bar{G}} (f)(a)) \cap (\nu^* + 1)\\
& = i_{E_\sigma^*} (i_{\bar{G}}(f)) (i_{E_\sigma^*}(a)) \cap (\nu^* + 1)\\
& = i_{\sigma + 1 , \gamma}^{\mathscr{T}} (i_{E_\sigma} ( i_{\bar{G}} (f)))(i_{E_\sigma^*}(a)) \cap (\nu^* + 1)\\
& = i_{0, \gamma}^{\mathscr{T}}(f)(i_{E_\sigma^*}(a)) \cap (\nu^* + 1)\\
& = i_{1, \delta}^{\mathscr{U}} \circ i_{\bar{G}}(f)(i_{H^*}(a)) \cap (\nu^* + 1)\\
& = i_{H^*}(i_{\bar{G}}(f))(i_{H^*}(a)) \cap (\nu^* + 1)\\
& = i_{H^*} (i_{\bar{G}}(f)(a)) \cap (\nu^* + 1)\\
& = i_{H^*} (X) \cap (\nu^* + 1) \ .
\end{split}
\]

The third equality from the bottom comes from $i_{\bar{G}} (f)$ being essentially a subset of $\nu (\bar{G})$, and for $z \subseteq \nu(\bar{G})$ in $M | \alpha$,

\[
i_{1, \delta}^{\mathscr{U}} (z) = i_{\tau + 1 , \delta} (i_{1, \tau + 1 }(z))
\]

so that

\[
\begin{split}
i_{1, \delta}^{\mathscr{U}} (z) \cap (\nu^* + 1) & = i_{\tau + 1 , \delta } \circ i_H (z) \cap (\nu^* + 1)\\
& = i_{H^*} (z) \cap (\nu^* + 1 ) \ .
\end{split}
\]

Thus $E_\sigma^* = H^*$.  The Jensen ISC then leads to $E_\sigma = H$.  This proves the subclaim. $\blacksquare$\\

This subclaim contradicts the fact that we were hitting disagreements.  This proves the claim. $\blacksquare$\\

By the previous two claims, $P = P_\gamma$ is above $H$ in $\mathscr{T}$.\\

Now we diverge from solidity proof in \cite{FSPIPM}.\\


\textbf{Claim:}  $P \unlhd Q$.\\

\textbf{Proof:}  If $Q \lhd P$ then the branch to $Q$ does not drop.  If $Q$ is above $M$ in $\mathscr{U}$ then Dodd-Jensen gives us a contradiction.  If $Q$ is above $Ult(M, \bar{G})$ then we get the same contradiction by first embedding $M$ into $Ult(M, \bar{G})$. $\blacksquare$\\



\textbf{Claim:}\\

\indent \indent $a)$ No extenders are applied on the $H \longrightarrow P$ branch of $\mathscr{T}$; that is, $H = P$.\\

\indent \indent $b)$ No extenders are applied in $\mathscr{U}$, so it has final model $Q = Ult(M , \bar{G})$.\\

\indent \indent \textbf{Proof:} $a)$ is proved exactly as in the non-anomalous case.  Now suppose $b)$ fails, and let $E_1^{\mathscr{U}} = E$ be the extender chosen from $Ult( M, \bar{G})$ as the least disagreement with $H$.  Then $lh(E)$ is a cardinal of $Q$, and of course $lh(E) > \alpha$.  We must also have $lh(E) < height(H)$, since otherwise it would not be a disagreement in need of removal.  Now $P \lhd Q$ is impossible, since $P = H$ projects to $\alpha$, and would therefore collapse $lh(E)$ in $Q$.  So we have $P = Q$.  But then $\varrho_1^Q = \varrho_1^P \leq \alpha$, so $\varrho_1^Q \leq \kappa_E$, since otherwise it would have been lifted above $lh(E)$ by $i_E$.  (The case where $E$ is long and $\varrho_1^{Ult(M, \bar{G})} = \kappa_E^+$ is ruled out by projectum-free spaces and closeness of $E$.)  And now we are iterating above the projectum on the branch to $Q$, so $Q$ is not sound.  This contradicts soundness of $H$, since $H = P = Q$. $\blacksquare$\\

Thus, in the non-anomalous case we have $H \unlhd Ult_n(M, \bar{G})$ for largest possible $n$; that is, $d)$ of Condensation.\\





\subsection{Anomalous Case 3}








In this case we have $\langle \xi , k \rangle$ least such that $\alpha \leq \xi$ and $\varrho_k (M | \xi ) < \alpha$, with $\alpha < \xi$.  Moreover, $\varrho_k (M | \xi)$ is of type $Z_p$.\\


Let

\[
\begin{split}
\gamma_0 & = \text{ least element of } p_{n+1} (M | \xi ) \ , \\
t_0 & = p_{n+1} (M | \xi ) - (\gamma_0 + 1) \ , \\
F & = \dot{E}_{\gamma_0}^{M} \ , \\
N & = H_{n+1}^{M | \xi } (i_F `` (\kappa_F^+ \cup t_0) \ .
\end{split}
\]

So $F$ is the stretching extender, and $N$ is the generalized core of $M | \xi$.  We have that $p_{n+1} (N) = \sigma^{-1} (t_0)$, where $\sigma: N \longrightarrow (M | \xi)$ is the uncollapse, and that $\alpha = (\kappa_F^{++})^{M | \xi} = (\kappa_F^{++})^H = (\kappa_F^{++})^N$, moreover $M | \alpha = N | \alpha = H | \alpha$.  Our goal is to reach a contradiction.\\

In this case we compare $\langle M , H, \alpha \rangle $ with $\langle M , N, \kappa_F^+ \rangle$.  Let $\mathscr{T}$ be the tree on $\langle M , H, \alpha \rangle $ and $\mathscr{U}$ the tree on $\langle M , N, \kappa_F^+ \rangle $ produced.  The rules for forming $\mathscr{T}$ are similar to those in the other anomalous cases.  Let $P_\eta = M_\eta^{\mathscr{T}}$, with $P_0 = M$ and $P_1 = H$.  Let $E_\eta = E_\eta^{\mathscr{T}}$, and $\lambda_0^{\mathscr{T}} = \alpha$.  Again

\[
\mathscr{T}\text{-pred} (\eta + 1) = \text{ least } \beta \text{ s.t. } dom (E_\eta) \subseteq P_\beta | \lambda_\beta^{\mathscr{T}} \ .
\]

Again, $E_\eta$ then gets applied to the longest initial segment of $P_\beta$ possible, except when $crit(E_\eta) = \kappa_F$, and $E_\eta$ is long.  Here $\beta = 0$, and $P_\beta = M$.  Again, we can show $E_\eta$ has exactly one long generator in this case.  We then set

\[
P_{\eta + 1} = Ult_n (N, E_\eta) \ .
\]

\textbf{Claim:} Let $crit(E_\eta) = \kappa_F$, and suppose $E_\eta$ is long; then\\

\indent \indent 1) $E_\eta$ has exactly one long generator,\\

\indent \indent 2) $Ult_n (N, E_\eta )$ is a plus-one premouse of type $Z_p$, with first standard parameter $i_{E_\eta} (p_1 (N)) \cup \{ \nu \}$, where $\nu = \dot{\nu}^{P_\eta | lh(E_\eta)}$, and stretching extender $E_\eta \restriction \nu$,\\

\indent \indent 3) $N$ is the generalized core of $Ult_n (N, E_\eta)$.\\

\textbf{Remark:} The solidity of $p_{n+1}(N)$ is used in proving 2).\\

We omit the proof of the claim.\\

In contrast to our first two anomalous cases, we cannot predict what $P_{\eta + 2}$ will be.  Moreover, $P_{\eta + 1}$ is not dead, in that $\mathscr{T}$ may have models above $P_{\eta + 1}$.

We choose branches for $\mathscr{T}$ by lifting it to $\mathscr{T}^*$ on $M$, with models $P_\beta^* = M_\beta^{\mathscr{T}^*}$.  We have copy maps

\[
\pi_\beta: P_\beta \longrightarrow P_\beta^*
\]

with $\pi_0 = id$, and $\pi_1 : H \longrightarrow M$ the uncollapse.  We have $\pi_\beta \restriction \lambda_\beta^{\mathscr{T}} = \pi_\gamma \restriction \lambda_\beta^{\mathscr{T}}$ for all $\gamma \geq \beta$.  Again, we must take some care in our special case.

So suppose $crit(E_\eta) = \kappa_F$, and $E_\eta$ has $\nu = \dot{\nu}^{P_\eta | lh(E)}$ as its unique long generator.  Let $E = E_\eta$, and $E^* = \pi_\eta (E)$.  We have that $\lambda_1^{\mathscr{T}} > \alpha$, and $\pi_\eta \restriction \lambda_1^{\mathscr{T}} = \pi_1 \restriction \lambda_1^{\mathscr{T}}$, so $dom(E^*) = M | \pi_1 (\alpha)$, so $E^*$ is an extender over all of $M$.  Let

\[
i_{E^*}^M : M \longrightarrow Ult_n (M, E^*)
\]

be the canonical embedding, and set

\[
P_{\eta + 1}^* = i_{E^*}^M (M | \xi) \ .
\]

We have the diagram\\

\[
\begin{tikzcd}
i_{E^*}^M (N) \arrow[to=2-2, "i_{E^*}^M (\sigma)"] & & \\
Ult_n (N, E) \arrow[u] \arrow[r, "\pi_{\eta + 1}"] & i_{E^*}^M (M | \xi ) \arrow[r] & Ult_n (M, E^*)\\
N \arrow[u] \arrow[r, "\sigma"] & M | \xi \arrow[u] \arrow{r} & M \arrow[u]
\end{tikzcd}
\]

where

\[
\pi_{\eta + 1} ( [a, f]_E^N) = i_{E^*}^M (\sigma) ( [ \pi_\eta (a) , f ]_{E^*}^M ) \ ,
\]

for $a \in [ \lambda_E \cup \{ \nu \} ]^{< \omega}$ and $f: [ \kappa_E^+ ]^{|a|} \longrightarrow N$ with $f \in N$.  Since $crit( \sigma) = \kappa_E = \kappa_{E^*}$, $crit(i_{E^*}^M (\sigma)) = \lambda_{E^*}$, and thus $\pi_{\eta + 1} \restriction \lambda_E = \pi_\eta \restriction \lambda_E$.  So we have the agreement of copy maps required to continue.

This completes our proof sketch for\\

\textbf{Claim:} $\langle M , H, \alpha \rangle$ is iterable by the rules described; moreover, $\mathscr{T}$ is lifted to a tree $\mathscr{T}^*$ according to $\Sigma$.\\

We now turn to $\mathscr{U}$.  Set

\[
Q_\eta = M_\eta^{\mathscr{U}} \ ,
\]

with $Q_0 = M$ and $Q_1 = N$.  Let

\[
\lambda_0^{\mathscr{U}} = \kappa_F^+  \ \ \text{and} \ \ \lambda_\eta^{\mathscr{U}} = \lambda_{E_\eta^{\mathscr{U}}}
\]

for $\eta \geq 1$.  The rules for $\mathscr{U}$ are

\[
\mathscr{U} \text{-pred} (\eta + 1) = \text{ least } \beta \text{ s.t. } dom(E_\eta^{\mathscr{U}}) \subseteq Q_\beta \restriction \lambda_\beta^{\mathscr{U}} \ .
\]

So short extenders with critical point $\kappa_F$ get applied to $M$, while long extenders with critical point $\kappa_F$ get applied to $N$.  (The latter because $\lambda_1^{\mathscr{U}} > \alpha = ( \kappa_F^{++})^{Q_\eta}$, for all $\eta \geq 1$.)\\

We lift $\mathscr{U}$ to a tree $\mathscr{U}^*$ on $M$ as follows.  The first two models of $\mathscr{U}^*$ are

\[
Q_0^* = M \ \ \text{and} \ \ Q_1^* = M | \xi \ .
\]

We define maps $\sigma_\eta : Q_\eta \longrightarrow Q_\eta^*$ with

\[
\sigma_0 = id \ \ \text{and} \ \ \sigma_1 = \sigma \ ,
\]

where $\sigma: N \longrightarrow M | \xi$ is the uncoring map.  Suppose now by induction that we have $\sigma_\eta : Q_\eta \longrightarrow Q_\eta^*$, and

\[
( \dagger ) \text{  for } 0 < \gamma \leq \eta , \ Q_\gamma^* \text{ agrees with } Q_\eta^* \text{ below } \lambda_\gamma^{\mathscr{U}^*} \ , \text{ and } \sigma_\gamma \restriction \lambda_\gamma^{\mathscr{U}} = \sigma_\eta \restriction \lambda_\gamma^{\mathscr{U}} \ .
\]

Again, we do not have this for $\gamma = 0$, because $\sigma_0$ and $\sigma_1$ agree only to $\kappa_F$, and $\lambda_0^{\mathscr{U}} = \kappa_F^+$.  We then define $Q_{\eta + 1}^*$ and $\sigma_{\eta + 1}$ so that $( \dagger )$ remains true.  This is done using the shift lemma, except when, for $F_\eta = E_\eta^{\mathscr{U}}$, $crit(F_\eta) = \kappa_F$ and $F_\eta$ is short.

So assume that.  We have $\mathscr{U} \text{-pred} (\eta + 1) = 0$, and $Q_{\eta + 1} = Ult_n (M, F_\eta)$.  Let

\[
j: Q_\eta^* \longrightarrow Ult_n (Q_\eta^* , \sigma_\eta (F_\eta ))
\]

be the canonical embedding.  Note that the stretching extender $F$ for $p_{n+1} (M | \xi)$ is on the sequence of $Q_1^* = M | \xi$, so that $F$ is on the $Q_\eta^*$-sequence (because $\alpha < \lambda_1^{\mathscr{U}}$, so that $\sigma_1 (\alpha) = (\lambda_F^{++})^{M | \xi} < \lambda_1^{\mathscr{U}^*}$).  We have

\[
crit(j) = \sigma_\eta (\kappa_F) = \sigma_1 (\kappa_F) = \lambda_F \ .
\]

Set

\[
Q_{\eta + 1}^* = Ult_0 (M, j(F)) \ \ \text{and} \ \ \sigma_{\eta + 1} ( [ a, f]_{F_\eta}^M ) = [ \sigma_\eta (a), f ]_{j(F)}^M \ .
\]

Since $\kappa_F = \kappa_{j(F)}$, $Q_{\eta + 1}^*$ makes sense.  We omit further detail.\\

This leads to\\

\textbf{Claim:} $\langle M, N , \kappa_F^+ \rangle$ is iterable via the strategy just described.\\

\textbf{Claim:}  The comparison of $\langle M, H , \alpha \rangle$ with $\langle M, N , \kappa_F^+ \rangle$ terminates.\\

Now let $P = P_\gamma$ and $Q = Q_\delta$ be the last models on the two sides.\\

\textbf{Claim:}  It is not the case that both $P$ and $Q$ are above $M$ in their respective trees.\\

\textbf{Proof:}  Suppose they were.\\

\textbf{Subclaim:}  It cannot be that both $[ 0 , \gamma]_{\mathscr{T}}$ and $[ 0 , \delta ]_{\mathscr{U}}$ drop.\\

This takes a little more argument than usual, because the branch to $P$ might have dropped to $N$.  But note\\

\textbf{Subclaim:} There is no $\eta$ such that $0 \mathscr{U} \eta$ and $N$ is a generalized core of some $Q_\eta | \gamma$.\\

\textbf{Proof:} $N$ collapses $\alpha$ definably to $\kappa_F^+$, so we would have $[ 0 , \eta ]$ does not drop, and $Q_\eta \restriction \gamma = Q_\eta$.  Let $F^*$ be the stretching extender of $Q_\eta$, so that $\kappa_F = \kappa_F^*$ and $F^* \in ran (i_{0, \eta}^{\mathscr{U}})$.  Then $crit(i_{0, \eta}^{\mathscr{U}}) \leq \kappa_F$, and $i_{0, \eta}^{\mathscr{U}} (crit (i_{0, \eta}^{\mathscr{U}})) > \alpha$, so $\kappa_{F^*} \notin ran (i_{0, \delta}^{\mathscr{U}})$, contradiction. $\blacksquare$\\

\textbf{Proof of first subclaim:} Suppose not.  Let $R$ be the last model to which we drop in model or degree along $[ 0 , \gamma ]_{\mathscr{T}}$, so that we have

\[
R = (M_{\tau + 1}^*)^{\mathscr{T}}
\]

with

\[
j = i_{\tau + 1 , \gamma }^{\mathscr{T}} \circ (i_{\tau + 1}^*)^{\mathscr{T}} : R \longrightarrow P
\]

being a $k$-embedding with $crit(j) \geq \varrho_{k+1}^R$, for some $k$ such that $R$ is $(k+1)$-sound.  Similarly, let $S$ be the last model to which we drop in model or degree along $[0 , \delta ]_{\mathscr{U}}$, so that

\[
S = (M_{\theta + 1}^*)^{\mathscr{U}}
\]

with

\[
\ell = i_{\theta + 1 , \delta}^{\mathscr{U}} \circ ( i_{\theta + 1}^*)^{\mathscr{U}} : S \longrightarrow Q
\]

being an $m$-embedding, with $crit ( \ell) \geq \varrho_{m+1}^S$, and $S$ being $(m + 1)$-sound.\\

We have that $Q$ is not $(m+1)$-sound, so $P \unlhd Q$.  By the subclaim, $N$ is not a generalized core of $Q$, so $N \neq R$, so $P$ is not $(k+1)$-sound.  Thus $P=Q$, and $m=k$.  Moreover,

\[
R = S = \mathcal{C}_{k+1} (P) = \mathcal{C}_{k+1} (Q) \ ,
\]

and

\[
j = \ell = \text{ uncoring embedding} \ .
\]

We want to show that $E_\tau^{\mathscr{T}} = E_\theta^{\mathscr{U}}$.  Let $E = E_\tau^{\mathscr{T}}$ and $G = E_\theta^{\mathscr{U}}$.  We assume that both $E$ and $G$ are long, and leave the other cases to the reader.  Let

\[
\nu = \nu (E) - 1 \ \ \ \text{and} \ \ \ \mu = \nu (G) - 1
\]

be the largest generators.  Let

\[
\kappa = crit (j) = \kappa_E = \kappa_G \ \ \ \text{and} \ \ \ \lambda = j(\kappa) \ .
\]

So

\[
i_{\tau + 1, \gamma}^{\mathscr{T}} (\lambda_E) = i_{\theta + 1 , \delta }^{\mathscr{U}} (\lambda_G ) = \lambda \ .
\]

Let $W_0 = E_{i_{\tau + 1 , \gamma}^{\mathscr{T}}} \restriction \lambda$ and $W_1 = E_{i_{\theta + 1 , \delta }^{\mathscr{U}}} \restriction \lambda$ be the two short parts of the branch tails.  Let

\[
( P || \eta_0 , E^* ) = Ult_0 (( P_\tau || lh(E) , E ) , W_0 )
\]

and

\[
( P || \eta_1 , G^* ) = Ult_0 (( Q_\theta || lh(G), G), W_1 ) \ .
\]

As in the comparison arguments, $E^*$ and $G^*$ are initial segments of the extender of $j$, so by our initial segment conditions, $\eta_0 = \eta_1$ and $E^* = G^*$.  But then $E = G$, because these are the first whole initial segments of $E^* = G^*$ that do not belong to $P = Q$. $\blacksquare$\\

\textbf{Subclaim:} Neither $[0, \gamma]_{\mathscr{T}}$ nor $[0, \delta]_{\mathscr{U}}$ drops; moreover, $P = Q$ and $i_{0, \gamma}^{\mathscr{T}} = i_{0, \delta}^{\mathscr{U}}$.\\

\textbf{Proof:} If $[ 0 , \delta ]_{\mathscr{U}}$ drops, then $Q$ is unsound, so $P \unlhd Q$ and $[ 0 , \gamma ]_{\mathscr{T}}$ does not drop.  But then $\sigma_\delta \circ i_{0 , \gamma}^{\mathscr{T}}$ maps $M$ to $Q_\delta^*$, a dropping iterate of $M$ via $\Sigma$, contrary to weak Dodd-Jensen.\\

If $[ 0 , \delta ]_{\mathscr{T}}$ drops, then $Q \unlhd P$.  This is clear if $P$ is unsound.  The alternative is that we have $\mathscr{T} \text{-pred} (\eta + 1) = 0$ with $\eta + 1 \in [ 0 , \gamma ]_{\mathscr{T}}$ and $E_\eta$ long and $crit(E_\eta) = \kappa_F$, so that $P_{\eta + 1} = Ult_0 (N, E_\eta)$ is sound, and there is no further dropping on $[ 0 , \gamma]_{\mathscr{T}}$.  But then $N$ is a generalized core of $P$, so $P \ntrianglelefteq Q$ by previous subclaim.\\

So if $[ 0 , \gamma ]_{\mathscr{T}}$ drops, then $[ 0 , \delta ]_{\mathscr{U}}$ does not, and $Q \unlhd P$.  But then $\pi_\gamma \circ i_{0, \delta}^{\mathscr{U}} : M \longrightarrow P_\gamma^*$ contradicts weak Dodd-Jensen.\\

The proof that $P = Q$ and $i_{0, \gamma}^{\mathscr{T}} = i_{0 , \delta}^{\mathscr{U}}$ is similar. $\blacksquare$\\

We can now prove the claim in just the same way that we finished the analogous proof in AC2. $\blacksquare$\\

\textbf{Claim:}  It is not the case that $P$ is above $M$ and $Q$ is above $N$.\\

\textbf{Proof:} Suppose otherwise, i.e., $0 \mathscr{T} \gamma$ and $1 \mathscr{U} \delta$.\\

\textbf{Subclaim:} $[1, \delta]_{\mathscr{U}}$ does not drop, in model or degree.\\

\textbf{Proof:} Otherwise $P \unlhd Q$ and $[ 0 , \gamma ]_{\mathscr{T}}$ does not drop.  (Note here that in the situation where $\eta + 1 \in [ 0 , \gamma ]_{\mathscr{T}}$, $crit (E_\eta) = \kappa_F$, and thus $P_{\eta + 1} = Ult_0 (N, E_\eta)$, and this is the only drop on $[0, \gamma]_{\mathscr{T}}$, then $P \ntrianglelefteq Q$.  For then $N \in Q_\beta$, for some $\beta \geq 1$, because $N$-to-$Q$ dropped.)  But then $\pi_\delta \circ i_{0 , \gamma }^{\mathscr{T}} : M \longrightarrow Q_\delta^*$ contradicts the weak Dodd-Jensen property of $\Sigma$. $\blacksquare$\\

\textbf{Subclaim:} $P = Q$.\\

\textbf{Proof:} If $P \lhd Q$, then $[ 0 , \gamma ]_{\mathscr{T}}$ does not drop.  This is because $P$ would be unsound, unless there were exactly one ``special drop" to $N$ on $[ 0 , \gamma ]_{\mathscr{T}}$, and $P = Ult_0 (N, E_{\gamma - 1} )$.  But then $N \in Q$, whereas $N$ definably collapses $\alpha$.  Thus $Q \unlhd P$, by weak Dodd-Jensen.\\

If $Q \lhd P$, then $Th_1^N (\kappa_F^+ \cup p_1 (N))$ belongs to $P$.  This is clear if $crit (i_{1, \delta}^{\mathscr{U}} ) > \alpha$, and it holds if $crit (i_{1, \delta}^{\mathscr{U}}) \leq \alpha$, for then the first extender used in $[1, \delta]_{\mathscr{U}}$ is long, with critical point $\kappa_F$, and we can use its superstrong part as usual to define $Th_1^N (\kappa_F^+ \cup p_1 (N))$ over the ultrapower.  But $Th_1^N (\kappa_F^+ \cup p_1 (N)) \notin P$, because it collapses $\alpha$. $\blacksquare$\\

Now let $\eta + 1 \in [ 0 , \gamma ]_{\mathscr{T}}$ be such that $\mathscr{T} \text{-pred} (\eta + 1) = 0$.\\

\textbf{Subclaim:}\\

\indent \indent $(a)$ $\varrho_1 (Q) = \kappa_F^+$ \ ,\\

\indent \indent $(b)$ $crit (E_\eta) = \kappa_F$ \ .\\

\textbf{Proof:} We already argued $(a)$ in the proof of the previous subclaim.  For $(b)$, we otherwise have $crit (E_\eta) < \kappa_F$.  But then $\varrho_1 (P) = \varrho_1 (Q) = \alpha$ is in the interval $( crit (E_\eta ) , i_{E_\eta}^{\mathscr{T}} (crit (E_\eta)))$, which is impossible.\\

\textbf{Subclaim:} $[ \eta + 1 , \gamma ]_{\mathscr{T}}$ does not drop.\\

\textbf{Proof:}  Otherwise $N \in P_\beta$ for some $\beta \geq 1$. $\blacksquare$\\

Since $\varrho_1 (P) = \kappa_F^+$ and $crit (E_\eta ) \leq \kappa_F$, it must be that $crit (E_\eta) = \kappa_F$ and $E_\eta$ is long; moreover, $E_\eta$ has exactly one long generator because $P$ has no long extenders with space $\kappa_F^+$ on its sequence.  Thus $P$ is of type $Z_p$, and the first core of $P$ is

\[
\mathcal{C}_1 (P) = P_{\eta + 1}
\]

and $P_{\eta + 1}$ is of type $Z_p$.  The generalized core of $P$ is $N$, because the branch $N$-to-$Q$ did not drop, so that $N$ is the generalized core of $Q$.  So

\[
P_{\eta + 1} = Ult_0 (N , E_\eta ) \ ,
\]

and for $\tau + 1$ least in $( 1 , \delta ]_{\mathscr{U}}$,

\[
Q_{\tau + 1} = Ult_0 (N , F_\tau ) \ .
\]

Again, $F_\tau$ has exactly one long generator.  We have

\[
i_{\eta + 1 , \gamma }^{\mathscr{T}} \circ i_{E_\eta} = i_{\tau + 1 , \delta}^{\mathscr{U}} \circ i_{F_\tau }
\]

because both are the uncoring map from $N$ into $P = Q$.  We then proceed as in the argument that comparison terminates to show that $E_\eta = F_\tau$.  This is a contradiction. $\blacksquare$\\

By the previous two claims, $P$  is above $H$ in $\mathscr{T}$.  If $Q$ is above $M$ in $\mathscr{U}$, then repeating the argument given at the end of the non-anomalous case yields that $P = H$ and $Q = M$, since there are no extenders indexed at $\alpha$ in our present case; so we have $H \unlhd M$, that is, $a)$ or $b)$ of Condensation.\\

If $Q$ is above $N$ in $\mathscr{U}$, we can similarly argue that no extenders were applied on the branch $N$-to-$Q$, so $Q = N$.  This means $H \unlhd N$.  But $H$ is a plus-one premouse, and $N$ violates projectum-free spaces, so $H \neq N$.  This means $H \lhd N$.  But $N$ agrees with $M | \xi$ on all proper initial segments, so we have that $b)$ of Condensation holds.\\

This proves the Condensation Lemma in all cases.  $\blacksquare$













\section{Preliminaries for the $\square_\Lambda$ Construction} \label{prelim section}







For this and all future sections, we will work inside an iterable premouse $W$ such that $W \models ZFC$.\\



\subsection{Subcompactness}

\begin{def} \label{square}
Given a cardinal $\kappa$, we say that $\langle C_\alpha ; \kappa < \alpha < \kappa^+ \wedge lim( \alpha ) \rangle$ is a $\square_\kappa$-sequence if and only if\\

\indent $a)$ Each $C_\alpha$ is a closed unbounded subset of $\alpha$;\\

\indent $b)$ $C_\beta = C_\alpha \cap \beta$ whenever $\beta \in lim (C_\alpha )$;\\

\indent $c)$ $otp (C_\alpha ) \leq \kappa$.\\

$\square_\kappa$ is the statement ``There is a $\square_\kappa$-sequence."
\end{def}\\


\begin{def} \label{subcompact}
A cardinal $\kappa$ is \textit{subcompact} if and only if given any $A \subseteq \kappa^+$, there are $\mu < \kappa$, $\bar{A} \subseteq \mu^+$ and an elementary embedding\\

\indent \indent $\sigma : \langle H_{\mu^+} , \bar{A} \rangle \longrightarrow \langle H_{\kappa^+ }, A \rangle$\\

with critical point equal to $\mu$.
\end{def}


\begin{lem} \textbf{(Jensen, refining Solovay)}
If $\kappa$ is subcompact, then $\square_\kappa$ fails.
\end{lem}

\indent \indent \textbf{Proof:} Suppose $\mathscr{C} = \langle C_\alpha ; \kappa < \alpha < \kappa^+ \wedge lim( \alpha ) \rangle$ is a $\square_\kappa$-sequence.  We can think of $\mathscr{C}$ as a subset of $\kappa^+$.  Now by subcompactness, we have an embedding\\

\indent \indent $\sigma : \langle H_{\mu^+} , \bar{\mathscr{C}} \rangle \longrightarrow \langle H_{\kappa^+ }, \mathscr{C} \rangle$\\

witnessing subcompactness of $\kappa$ relative to $\mathscr{C}$.  Note $\bar{\mathscr{C}}$ is a $\square_\mu$-sequence.  Now let $\nu = sup (\sigma `` (\mu^+))$.  $C_\nu$ is a closed unbounded subset of $\nu$, which has cofinality $\mu^+$.  However, we will show that all initial segments of $C_\nu$ must have ordertype $< \mu$, which is a contradiction.\\

Notice that there are cofinally many $\bar{\gamma} < \mu^+$ such that  $cof (\bar{\gamma}) = \omega$ and $\sigma$ is continuous at $\bar{\gamma}$ and $\sigma ( \bar{\gamma}) \in C_\nu$.  This follows from the standard closure argument and the fact that $\sigma$ is continuous at limits of cofinality $< \mu$.  Now for any such $\bar{\gamma}$, we have that $\gamma = \sigma (\bar{\gamma})$ has a club $C_\gamma = C_\nu \cap \gamma$.  And $cof( \gamma) = cof (\bar{\gamma}) = \omega$, so $otp (\bar{C}_{\bar{\gamma}}) < \mu$, and by elementarity of $\sigma$ (which has critical point $\mu$) we also have $otp (C_\gamma) < \mu$.  $\blacksquare$\\  


\begin{def} \label{1-subcompact}
A cardinal $\kappa$ is $1$-subcompact if and only if given any $A \subseteq \kappa^{++}$, there are $\mu < \kappa$, $\bar{A} \subseteq \mu^{++}$ and an elementary embedding\\

\indent \indent $\sigma : \langle H_{\mu^{++}} , \bar{A} \rangle \longrightarrow \langle H_{\kappa^{++} }, A \rangle$\\

with critical point equal to $\mu$.
\end{def}


\begin{lem}
If $\kappa$ is 1-subcompact, then $\square_{\kappa^+}$ fails.
\end{lem}

\indent \indent \textbf{Proof:} We give a slightly different proof than above.  (The current proof style can be adapted to the above subcompact case as well.)\\

Suppose $\mathscr{C} = \langle C_\alpha ; \kappa^+ < \alpha < \kappa^{++} \wedge  lim( \alpha ) \rangle$ is a $\square_{\kappa^+}$-sequence.  We can think of $\mathscr{C}$ as a subset of $\kappa^{++}$.  Now by 1-subcompactness, we have an embedding\\

\indent \indent $\sigma : \langle H_{\mu^{++}} , \bar{\mathscr{C}} \rangle \longrightarrow \langle H_{\kappa^{++} }, \mathscr{C} \rangle$\\

witnessing 1-subcompactness for $\mathscr{C}$.  Note $\bar{\mathscr{C}}$ is a $\square_{\mu^+}$-sequence.  Now let $\nu = sup (\sigma `` (\mu^{++}))$.  $C_\nu$ is a closed unbounded subset of $\nu$.  Notice that there are cofinally many $\bar{\gamma} < \mu^{++}$ such that  $cof (\bar{\gamma}) = \omega$ and $\sigma$ is continuous at $\bar{\gamma}$ and $\sigma ( \bar{\gamma}) \in C_\nu$.  This follows from the standard closure argument and the fact that $\sigma$ is continuous at limits of cofinality $< \mu$.  Now for any such $\bar{\gamma}$, we have that $\gamma = \sigma (\bar{\gamma})$ has a club $C_\gamma = C_\nu \cap \gamma$.  Thus, for $\bar{\gamma}' < \bar{\gamma}$ two such ordinals, we have $C_{\gamma '} = C_{\gamma} \cap \gamma '$.  By elementarity this agreement will transfer down to $\bar{\mathscr{C}}$, so we have an unbounded set of limit ordinals in $\mu^{++}$ whose clubs in $\bar{\mathscr{C}}$ are strict extensions of each other.  But then they must eventually have $otp > \mu^+$, so $\bar{\mathscr{C}}$ is not a $\square_{\mu^+}$-sequence, contradiction.  $\blacksquare$\\

\textbf{Remark:} The above argument in fact shows that $\Pi_1^2 \ 1$-subcompactness implies failure of $\square ( \kappa^+ )$. (Recall that $\square (\kappa^+ )$ is the statement ``every coherent sequence of clubs of length $= \kappa^{++}$ can be threaded".)  Similarly, if $\kappa$ is $\Pi_1^2$-subcompact, then $\square (\kappa)$ fails.  (See \cite{equiconsistencies} for more details on $\square (\kappa ) $ and $\Pi_1^2$-subcompactness.)\\

The above arguments can be trivially modified to prove that $\square_{\kappa , 2}$ or $\square_{\kappa^+ , 2}$ also fail when $\kappa$ is subcompact or $1$-subcompact, respectively.  ($\square_{\kappa , 2}$ is a well-known weakening of $\square_\kappa$; an official definition is given at the beginning of Section \ref{main construction section}.)\\

\textbf{Remark:} If $\kappa$ is $1$-subcompact then $\kappa$ is $\Pi_1^2$-subcompact.\\


We also have a characterization of subcompactness in terms of extenders on the sequence of $L[E]$:\\


\begin{lem} \textbf{(from \cite{zeman square proof})} In $L[E]$, suppose $\kappa$ is a cardinal such that 

\[
\{ \alpha \mid \kappa < \alpha < \kappa^+ \wedge ( L[E] | \alpha) \text{ is active} \}
\]

is stationary in $\kappa^+$.  (In other words, there are stationarily many extenders-- short or long-- indexed below $\kappa^+$.)  Then $\kappa$ is subcompact.
\end{lem}



\textbf{Proof:}  Assume $\kappa$ is not subcompact; let $A$ be the $<_{L[E]}$-least subset of $\tau = \kappa^+$ witnessing this.  Notice that $A \in J^E_{\tau^+}$ and that the set $| J^E_\tau | $ (with no top predicate) is exactly $H_\tau$.  Also notice that any elementary embedding $\sigma : ( H_{\mu^+} , \bar{A} ) \longrightarrow (H_\tau , A )$ with $\mu < \kappa$ can be coded as a bounded subset of $H_\tau$, and hence can be considered an element of $H_\tau$.  Thus, the fact that $A$ witnesses the failure of subcompactness for $\kappa$ can be expressed as $M \models \phi (A , \tau )$, where $M = L[E] \ || \ \tau^+ $ and $\phi (A , \tau )$ is the $\Sigma_1$-statement 

\[
\begin{split}
& ( \exists H) (H = | J^E_\tau | ) \text{ and } \neg ( \exists \bar{H} , \bar{A} , \sigma \in H ) \\ 
( \sigma : ( \bar{H} , \bar{A} ) \longrightarrow & ( H , A ) \text{ witnesses subcompactness for } \kappa \text{ with } A ) \ .
\end{split}
\]

\bigskip

A standard closure argument shows that the set of all ordinals $\nu < \tau$ satisfying $\tau \cap h_M (\nu \cup \{ \tau \} ) = \nu$ is closed and unbounded in $\tau$.  Since we assume that $E_\nu \neq \emptyset$ for stationarily many $\nu$, we can pick a $\nu$ satisfying both

\[
\tau \cap h_M ( \nu \cup \{ \tau \} ) = \nu \ \text{ and } \ E_\nu \neq \emptyset \ .
\]

Let $M'$ be the transitive collapse of $h_M (\nu \cup \{ \tau \} )$ and let $\sigma'$ be the associated uncollapsing map.  Then $M'$ is a passive premouse, $\varrho_1 (M' ) \leq \nu = crit (\sigma' )$ and $\sigma' (\nu ) = \tau$.  A further closure argument allows us to assume $\varrho_1 (M' ) = \nu$; this is because the set of $\nu < \tau$ satisfying

\[
( \forall \alpha < \nu ) Th_1^{M} (\alpha \cup \{ \tau \} ) \in (M | \nu )
\]

is a club in $\tau$.  If we require $\nu$ to be in this club, clearly $\varrho_1 (M') \geq \nu$.  Thus we have $\nu = ( \kappa^+ )^{M' } = \varrho_1 (M')$.\\

By the Solidity lemma of \cite{FSPIPM}, $M'$ is solid and its standard parameter is universal, so $\tilde{M} = core_\nu (M' ) =$ transitive collapse of $h_M ( \nu \cup p(M') )$ exists.  Let $\tilde{\sigma} : \tilde{M} \longrightarrow M$ be the associated core map.  Notice that $\nu = (\kappa^+)^{\tilde{M}}$, as any surjection $g: \alpha \longrightarrow \nu$ such that $g \in \tilde{M}$ and $\alpha < \nu$ would give rise to the surjection $\tilde{\sigma} (g) \restriction \alpha : \alpha \longrightarrow \nu$, which would be an element of $M'$. Consequently, the critical point of $\tilde{\sigma }$, being at least $\nu$, must be $> \nu$, as every ordinal between $\kappa$ and $\tilde{\sigma} (\nu)$ has $M'$-cardinality equal to $\kappa$.  Letting $\sigma = \sigma' \circ \tilde{\sigma}: \tilde{M} \longrightarrow M$, the above discussion can be summarized as follows:

\begin{itemize}
\item{$\varrho_1 (\tilde{M} ) = \nu$ and $\tilde{M}$ is sound;}
\item{$crit (\sigma )= \nu$ and $\sigma (\nu ) = \tau$;}
\item{$\sigma$ is $\Sigma_1$-preserving.}
\end{itemize}



These conclusions enable us to apply the Condensation lemma to $M$, $\tilde{M}$, and $\sigma$, so one of the conclusions $a) - d)$ of Condensation holds.  (Notice that Anomalous Case 4 does not apply, since the critical point $\nu$ is a local successor of a limit cardinal, and AC4 requires it to be a local double successor.)  $a)$ and $b)$ are impossible, since $M$ has an extender indexed at $\nu$ but $\tilde{M}$ does not.  $d)$ is impossible because $\nu$ is a local successor of a limit cardinal, and pseudoindices must be local double successors.  This leaves only $c)$.  Set $N = Ult (M || \nu , E_\nu )$.  Then $\tilde{M} \lhd N$.  Set $\mu = crit( E_\nu )$ and $\theta = ( \mu^+ )^M$.  Since $\tilde{M}$ projects to $\nu$, we in fact have $\tilde{M} \lhd Ult ( M || ( \theta^+ )^M , E_\nu ) = P$.\\

Observe that since $\sigma$ is $\Sigma_1$-preserving and $\sigma (\nu) = \tau$, there is a set $B \in \tilde{M}$ such that $\tilde{M} \models \phi (B , \nu )$.  Let $\tilde{A}$ be the $<_{\tilde{M}}$-least such $B$.  Then $\tilde{A}$ must be $\sigma^{-1} (A)$, since otherwise $A <_M \sigma (\tilde{A} )$ and the preservation properties of $\sigma$ would force the existence of some $\tilde{A}^*  <_{\tilde{M}} \tilde{A}$ for which $\phi ( \tilde{A}^* , \nu )$ holds in $\tilde{M}$, a contradiction.  Also notice that since $M \lhd P$, $\tilde{A}$ is the $<_P$-least set such that $P \models \phi(\tilde{A} , \nu )$.  This means that $\tilde{A}$ is $\Sigma_1$-definable over $P$ from the single parameter $\nu$; it follows that $\tilde{A}$ is in the range of the ultrapower map $i_{E_\nu } : M || (\theta^+ )^M \longrightarrow P$.  Set $\bar{A} = i_{E_\nu }^{-1} (\tilde{A} )$ and $\pi = \sigma \circ i_{E_\nu}$.  Then $\pi (( H_\theta , \bar{A} )) = (H_\tau , A )$, which witnesses subcompactness of $\kappa$ for $A$; contradiction. $\blacksquare$\\



\begin{lem} In $L[E]$, suppose $\kappa$ is a cardinal such that 

\[
\{ \alpha \mid \kappa^+ < \alpha < \kappa^{++} \wedge ( \exists \beta > \alpha \ ( L[E] | \beta \text{ is type } Z_1 \wedge \alpha = (\kappa_F^{++})^{L[E] | \beta}))
\]

is stationary in $\kappa^{++}$.  (In other words, there are stationarily many extenders \textit{pseudo-indexed} below $\kappa^{++}$.)  Then $\kappa$ is $1$-subcompact.
\end{lem}

\textbf{Proof:}  Assume $\kappa$ is not $1$-subcompact; let $A$ be the $<_{L[E]}$-least subset of $\tau = \kappa^{++}$ witnessing this.  Notice that $A \in J^E_{\tau^+}$ and that the set $| J^E_\tau | $ (with no top predicate) is exactly $H_\tau$.  Also notice that any elementary embedding $\sigma : ( H_{\mu^{++}} , \bar{A} ) \longrightarrow (H_\tau , A )$ with $\mu < \kappa$ can be coded as a bounded subset of $H_\tau$, and hence can be considered an element of $H_\tau$.  Thus, the fact that $A$ witnesses the failure of $1$-subcompactness for $\kappa$ can be expressed as $M \models \phi (A , \tau )$, where $M = L[E] \ || \ \tau^+ $ and $\phi (A , \tau )$ is the $\Sigma_1$-statement 

\[
\begin{split}
& ( \exists H) (H = | J^E_\tau | ) \text{ and } \neg ( \exists \bar{H} , \bar{A} , \sigma \in H ) \\ 
( \sigma : ( \bar{H} , \bar{A} ) \longrightarrow & ( H , A ) \text{ witnesses $1$-subcompactness for } \kappa \text{ with } A ) \ .
\end{split}
\]

\bigskip

A standard closure argument shows that the set of all ordinals $\nu < \tau$ satisfying $\tau \cap h_M (\nu \cup \{ \tau \} ) = \nu$ is closed and unbounded in $\tau$.  Since we assume that stationarily many $\nu$ are pseudoindices, we can pick a $\nu$ satisfying both

\[
\tau \cap h_M ( \nu \cup \{ \tau \} ) = \nu \ \text{ and } \ \nu \text{ is a pseudoindex.}
\]

Let $M'$ be the transitive collapse of $h_M (\nu \cup \{ \tau \} )$ and let $\sigma'$ be the associated uncollapsing map.  Then $M'$ is a passive premouse, $\varrho_1 (M' ) \leq \nu = crit (\sigma' )$ and $\sigma' (\nu ) = \tau$.  Thus $\nu = ( \kappa^{++} )^{M' }$.  As in the above Lemma, a further closure condition on $\nu$ allows us to assume without loss of generality that in fact $\varrho_1 (M' ) = \nu$.  By the Solidity lemma of \cite{FSPIPM}, $M'$ is solid and its standard parameter is universal, so $\tilde{M} = core_\nu (M' ) =$ transitive collapse of $h_M ( \nu \cup p(M') )$ exists.  Let $\tilde{\sigma} : \tilde{M} \longrightarrow M$ be the associated core map.  Notice that $\nu = (\kappa^{++})^{\tilde{M}}$, as any surjection $g: \alpha \longrightarrow \nu$ such that $g \in \tilde{M}$ and $\alpha < \nu$ would give rise to the surjection $\tilde{\sigma} (g) \restriction \alpha : \alpha \longrightarrow \nu$, which would be an element of $M'$. Consequently, the critical point of $\tilde{\sigma }$, being at least $\nu$, must be $> \nu$, as every ordinal between $\kappa^+$ and $\tilde{\sigma} (\nu)$ has $M'$-cardinality equal to $\kappa^+$.  Letting $\sigma = \sigma' \circ \tilde{\sigma}: \tilde{M} \longrightarrow M$, the above discussion can be summarized as follows:

\begin{itemize}
\item{$\varrho_1 (\tilde{M} ) = \nu$ and $\tilde{M}$ is sound;}
\item{$crit (\sigma )= \nu$ and $\sigma (\nu ) = \tau$;}
\item{$\sigma$ is $\Sigma_1$-preserving.}
\end{itemize}

These conclusions enable us to apply the Condensation lemma to $M$, $\tilde{M}$, and $\sigma$, so one of the conclusions $a) - d)$ of Condensation holds.  (Notice that Anomalous Case 4 does not apply, since cofinally many levels of $M$ project to $\kappa^+$; so if there were a total long extender $G = E_\gamma^M$ with critical point $\kappa$, as required by AC4, then any level of $M$ above $\gamma$ which projected to $\kappa^+$ would cause a violation of projectum-free spaces, so $M$ would not be a premouse.)\\

$a)$ of Condensation is impossible, because $\nu$ is a cardinal in $\tilde{M}$ but is collapsed in $M$.  $c)$ is impossible because $\nu$ is a local double successor of a limit cardinal, and extenders can only be indexed at local single-successors.  If $b)$ holds, then $\tilde{M} \lhd Q$, where $Q$ is the $Z_1$ level at which $\nu$ is witnessed to be a pseudoindex (that is, $Q$ is the collapsing-level for $\nu$).\\

So we have that either $d)$ of Condensation holds, or else $b)$ holds and $\tilde{M} \lhd Q$.  Set $E =$ the extender pseudoindexed at $\nu$, and $N = Ult (M || \nu , E )$.  Then in either of the above cases, $\tilde{M} \lhd N$.  Set $\mu = crit( E )$ and $\theta = ( \mu^{++} )^M$.  Since $\tilde{M}$ projects to $\nu$, we in fact have $\tilde{M} \lhd Ult ( M || ( \theta^+ )^M , E) = P$.\\

Observe that since $\sigma$ is $\Sigma_1$-preserving and $\sigma (\nu) = \tau$, there is a set $B \in \tilde{M}$ such that $\tilde{M} \models \phi (B , \nu )$.  Let $\tilde{A}$ be the $<_{\tilde{M}}$-least such $B$.  Then $\tilde{A}$ must be $\sigma^{-1} (A)$, since otherwise $A <_M \sigma (\tilde{A} )$ and the preservation properties of $\sigma$ would force the existence of some $\tilde{A}^*  <_{\tilde{M}} \tilde{A}$ for which $\phi ( \tilde{A}^* , \nu )$ holds in $\tilde{M}$, a contradiction.  Also notice that since $M \lhd P$, $\tilde{A}$ is the $<_P$-least set such that $P \models \phi(\tilde{A} , \nu )$.  This means that $\tilde{A}$ is $\Sigma_1$-definable over $P$ from the single parameter $\nu$; it follows that $\tilde{A}$ is in the range of the ultrapower map $i_{E } : M || (\theta^+ )^M \longrightarrow P$.  Set $\bar{A} = i_{E}^{-1} (\tilde{A} )$ and $\pi = \sigma \circ i_{E}$.  Then $\pi (( H_\theta , \bar{A} )) = (H_\tau , A )$, which witnesses $1$-subcompactness of $\kappa$ for $A$; contradiction. $\blacksquare$\\



One of the goals of our main construction below is to show that the converses of the above two lemmas hold as well: if $\kappa$ is subcompact in $L[E]$ then there must be stationarily many extenders indexed below $\kappa^+$, and if $\kappa$ is $1$-subcompact in $L[E]$ then there must be stationarily many extenders pseudo-indexed below $\kappa^{++}$.  We will discuss these implications more later.\\









\subsection{Interpolation} \label{interpolation subsection}







Now we begin our actual construction of $\square$-sequences.  Recall that we are working in a fixed iterable premouse $W$ such that $W \models ZFC$, and all statements should be understood as internal to $W$.  Let $E = \vec{E}^W$, the extender sequence of $W$, so that $J_\alpha^E = W | \alpha$.\\


The goal which we are working towards is the construction of a $\square_\Lambda$ sequence in $W$ for any $\Lambda$ which is neither subcompact, nor the successor of a 1-subcompact cardinal. We have shown above that $\square_\Lambda$ fails for such $\Lambda$, so if this goal were attained it would be a complete characterization of $\square$ in $W$.  However, there seem to be substantial technical obstacles to achieving this goal.  One of these obstacles has the effect that our sequences will in fact be $\square_{(\Lambda , 2)}$-sequences instead of the optimal $\square_{(\Lambda , 1)}$-sequences; this is the problem of ``unstable" $W$-levels described below in \ref{canonical divisor subsection}.  We are hopeful that this problem can be resolved without too much difficulty; in particular, Steel has sketched a possible solution involving the concept of ``V-divisors".  At present, however, the construction described below results in $\square_{(\Lambda , 2)}$-sequences.\\




The most difficult technical obstacle we have encountered, though, arises from levels $N$ of $W$ which are active with top extender $G$, where $G$ has a largest generator $\nu_G$ which is a limit of generators.  In the course of our construction, we will need to identify the short part $G \restriction \lambda_G$ of this extender just from the ordinal $\nu$.  (The recovery of $G \restriction \lambda_G$ from $\nu$ needs to be possible even over some different level of $W$, which may be passive, or have a top extender which is unrelated to $G$.)  It is unclear how to make this identification when $\nu_G$ is a limit of generators, because in this case there is not enough information indexed at $\nu_G$ itself to uniquely identify $G$.  However, if $\nu_G$ is a successor generator of $G$ (here we consider the first long generator to be a successor generator), then there is an initial segment of $G$ indexed at $\nu_G$, namely, $G \restriction (\nu_G - 1) = E_{\nu_G}$.  From here it is easy to identify $G \restriction \lambda_G$ as the short part of $E_{\nu_G}$.\\

Because of this, some key parts of our construction require the assumption that the long active levels $N$ which we encounter have a largest generator $\nu$ which is a successor generator.  On the one hand, this assumption will always be met if all long extenders in $W$ have finitely many long generators.  Thus one of our main results is a $\square_{\Lambda, 2}$ construction for $\Lambda$ as above, under the ``smallness assumption" that the extenders of $W$ have $< \omega$ long generators.  On the other hand, our construction can be considered as dealing with the successor-generator cases of a complete $\square_\Lambda$ construction for plus-one premice.  Because of this, we do \textit{not} assume at the outset that all long extenders of $W$ have $< \omega$ long generators.  Instead, we will go as far as possible without any smallness assumption on $W$, and then explicitly highlight the points at which we assume $\nu_G$ is a successor generator.\\




Now we embark on the actual construction. Fix a cardinal $\Lambda$ which is not subcompact and which is not the successor of a 1-subcompact cardinal.  So there are nonstationarily many extenders indexed between $\Lambda$ and $\Lambda^+$, and nonstationarily many extenders pseudo-indexed between $\Lambda$ and $\Lambda^+$.  We attempt to construct a $\square_\Lambda$-sequence.\\


We start by forming a closed unbounded set $\mathscr{S}$ in $\Lambda^+$ of levels which are `local successors' of $\Lambda$, i.e., fully elementary substructures of $J_{\Lambda^+}^E$.  Closing upwards by an appropriate Skolem function, we can easily obtain a club of such substructures of $J_{\Lambda^+}^E$; however, in general the structures obtained in this way will condense to levels of $W$ only if that level of $W$ does not have a top extender.  Because of our assumption that there are nonstationarily many extenders indexed below $\Lambda^+$, however, we can work inside a club where \textit{no} levels have top extenders.  Likewise, our assumption that there are nonstationarily many pseudoindices lets us work inside a club where no levels are pseudoindices.  Therefore we have $\mathcal{S}$ a club in $\Lambda^+$ such that all $\tau \in \mathcal{S}$ satisfy:\\

\indent $a)$ $\Lambda$ is the largest cardinal in $J_\tau^E$;\\

\indent $b)$ $J_\tau^E$ is a fully elementary substructure of $J_{\Lambda^+}^E$;\\

\indent $c)$ $E_\tau = \emptyset$;\\

\indent $d)$ $\tau$ is not a pseudoindex.\\

\textbf{Remark:}  Our ``smallness assumption" that the long extenders of $W$ have $< \omega$ long generators will have the consequence that there are no type $Z_1$ levels in $W$, hence no pseudoindices, and so condition $d)$ above is superfluous in that context.  As promised, though, we go as far as possible in the full generality of arbitrary plus-one premice.\\


For $\tau \in \mathcal{S}$, set $N_\tau =$ the collapsing-level for $\tau$ in $W$, and $k(\tau) =$ the least $k$ such that $\varrho_{k+1}( N_\tau )  = \Lambda$, so that $N_\tau = W  |  \langle \beta , k(\tau) \rangle$ for $\beta = o (N_\tau )$.  So $\tau$ is a cardinal in $N_\tau$ but is collapsed by a $\bf{\Sigma}_{1}^{(k)}$-definable function over $N_\tau$.  Note that no levels above $\Lambda$ can project below $\Lambda$ because it is a $W$-cardinal, and the first level collapsing $\tau$ must project to $\Lambda$ because there are no cardinals between $\Lambda$ and $\tau$ in $J_\tau^E$ or any higher level.\\

\begin{lem}
For $N_\tau$ as above, $cof (o(N_\tau^{(k)})) = cof (\tau)$  (recall that $N_\tau^{(k)}$ is the $k$-th standard reduct of $N_\tau$).
\end{lem}

\indent \indent \textbf{Proof:} Since $N_\tau^{(k)}$ collapses $\tau$ to $\Lambda$, it has a $\bf{\Sigma}_1$ bijection $f: \Lambda \longrightarrow \tau$.  Each value $f(\alpha) = \beta$ of this bijection is witnessed by some point $x_\beta$ in $N_\tau^{(k)}$, for $\beta < \tau$.  Now observe that if $cof ( N_\tau^{(k)}) > cof(\tau)$, we could take the supremum of the levels at which the $x_\beta$'s appear, and this level would be an element of $N_\tau^{(k)}$.  But then the entire bijection $f$ would be an element of $N_\tau^{(k)}$, contradicting that $\tau$ is a cardinal of $N_\tau^{(k)}$.\\

If $cof(\tau) > cof(N_\tau^{(k)})$, there would be a particular level of $N_\tau^{(k)}$ which contained $x_\beta$'s for cofinally many $\beta$'s in $\tau$.  But this would be enough to get a bijection from $\Lambda$ to $\tau$, because each ordinal $\beta < \tau$ has cardinality $\Lambda$ (as witnessed within $J_\tau^E$).  So again $\tau$ cannot be a cardinal in $N_\tau^{(k)}$, contradiction. $\blacksquare$\\


We would like to define canonical club subsets of each $\tau$ by taking certain hulls of $N_\tau$ and applying Condensation to show that these hulls are levels of $W$ below $\tau$.  We will define an \textit{interpolant} of $N_\tau$ to be a structure with a certain weak embedding into $N_\tau$, and ideally we can show that there is a club in $\tau$ of levels which are interpolants of $N_\tau$.  This is essentially how Jensen first proved $\square_\Lambda$ for all $\Lambda$ in $L$.  However, the presence of short and long extenders on our sequence derails this proof at certain points.  One major problem is that if $N_\tau$ is \textit{pluripotent} (see below), the interpolants of $N_\tau$ will not be premice at all, so they certainly do not condense to lower levels of $W$.\\


\begin{defn} \label{short pluripotent}
For $N_\tau$ a level of $W$ as above, we say $N_\tau$ is \textit{short pluripotent} if and only if $N_\tau$ is active with short top extender $G$, $\kappa_G < \Lambda$, and $k(\tau) = 0$.
\end{defn}

\begin{defn} \label{long pluripotent}
For $N_\tau$ a level of $W$ as above, we say $N_\tau$ is \textit{long pluripotent} if and only if $N_\tau$ is active with long top extender $G$, $\kappa_G^+ < \Lambda$ (note that $\kappa_G^+ \neq \Lambda$ by projectum-free spaces), and $k(\tau) = 0$.
\end{defn}

Pluripotent levels of both varieties will lead to \textit{protomice} when we try to shrink them down via interpolations.  We now give the description of this interpolation process:\\



\begin{defn} \label{interpolant}

Let $M = (|M| , G)$ be a sound coherent structure; then a $J$-structure $\tilde{M} = (| \tilde{M} | , \tilde{G})$ is an interpolant of $M$ if there is a map $\sigma : \tilde{M} \longrightarrow M$ that is $\Sigma_0^{k(N)}$-preserving with respect to the language of coherent structures and such that $\sigma (p (\tilde{M}) = p (M)$, and for every $\alpha \in p(M)$, there is a generalized solidity witness $W_\alpha^{M , p(M)}$ in the range of $\sigma$, and $\varrho (\tilde{M}) = \varrho (M)$.  In this case we call $\sigma$ the interpolation embedding.
\end{defn}


We will use the following Lemma for interpolations of pluripotent levels and protomice, in addition to non-pluripotent levels of $W$.  In the former cases, we will always have $k = 0$.\\

\begin{lem} \label{cofinal interpolants}
Let $M = (|M| , G)$ be a coherent structure with $k = k(M)$ and $\varrho ( M) = \varrho_{k+1} (M) = \Lambda$, where $\Lambda$ is a cardinal of $W$.  Assume $M$ is $(k+1)$-sound and $(k+1)$-solid and that $\varrho_k (M) > \Lambda$.  Further, assume $(\Lambda^+ )^M = \tau$ is an ordinal in $M$ and that $\tau$ has uncountable cofinality (in $W$).  Then for any $\beta < \tau$, there is an interpolant $\bar{M}$ of $M$ with interpolation embedding $\sigma$ such that $crit(\sigma) = (\Lambda^+)^{\bar{M}}= \bar{\tau} > \beta$, and $\sigma ( \bar{\tau}) = \tau$.
\end{lem}

\textbf{Proof:}  We take a fully elementary countable hull $X' \prec_{\Sigma_\omega} M$ such that $\{ \beta \} \in X'$.  Let $X$ be the transitive collapse of $X'$ with $\pi: X \longrightarrow M$ the uncollapse map, and $F$ the extender of length $\Lambda$ derived from $\pi$.  Now take $\bar{M} = Ult_k (X , F )$.  We claim $\bar{M}$ is an interpolant of $M$ with the desired properties, and the interpolation embedding of $\bar{M}$ into $M$ is the natural factor map $\sigma: \bar{M} \longrightarrow M$ given by $ \sigma (i_F^X ( f) ( a )) = \pi (f)  (a)$ for $f \in X^{(k)}$ and $a \in \Lambda^{< \omega}$.\\

$\Sigma_0^{(k)}$-elementarity of $\sigma$ follows easily from the basic properties of ultrapowers (see \cite{ZS finestructure}). To see that the critical point is $\bar{\tau}$, note that the embedding is $id$ on $\Lambda \cup \{ \Lambda \}$, so the critical point is $\geq \bar{\tau}$; and also,  $cof(\bar{\tau}) =  \omega$, since $\pi^{-1} ( \tau)$ has countable cofinality and is regular in $X^{(k)}$, so $Ult_k (X , F )$ is continuous there.  But $\tau$ itself has uncountable cofinality, so the critical point is $\leq \bar{\tau}$.  Now we need to see that $\sigma (p (\bar{M}) = p (M)$.  But $p(X) = \pi^{-1}`` (p(M))$ by $\Sigma_\omega$ elementarity of this hull, and $i_F (p(X)) = p(\bar{M})$ by a standard argument, which we give in the paragraph below.  It then follows that $\sigma (p (\bar{M}) = p (M)$, because $\sigma \circ i_F = \pi$.  Finally note that there are solidity witnesses for every $\alpha \in p(M)$ in the range of $\pi$, and hence in the range of $\sigma$, by elementarity of $\pi$.\\ 

 The proof will be complete once we show that $i_F (p(X)) = p(\bar{M})$.  First observe that $\Lambda \leq \varrho_{k+1} (\bar{M})$, because $\Lambda$ is a cardinal of $W$.  And because $\bar{M}$ is generated by $i_F`` (X) \cup \Lambda = i_F`` (\varrho_{k+1} (X) \cup p(X)) \cup \Lambda = i_F''(p(X)) \cup \Lambda$, we can see that $i_F (p(X))$ is a good parameter for $p(\bar{M})$.  Finally, notice that $X$ is solid (by elementarity, since $M$ was solid), and therefore $i_F ( \{ p(X) \})$ is a top segment of $p (\bar{M})$.  But it is a good parameter, so it is the standard parameter. $\blacksquare$\\


For the remainder of this subsection we shall use the notation of the above definition, and in addition we let $i = i_F : X \longrightarrow H$ be the ultrapower map.\\


In the proof above, we form interpolants of $M$ by first taking a countable hull $X$, then ``fattening it up" by taking an ultrapower of length $\Lambda$.  Note that if $M$ is a premouse, there is no difference between interpolants formed using the language of premice and those formed using language of coherent structures, because the constant symbols which are present in the premouse language but not CS-language all refer to objects which are automatically in the hulls we take.  In particular, $\dot{\gamma}$ and $\dot{\nu}$, if they are defined for $M$, will be put into $X$ because it is fully elementary (in either language).\\

Note that $o(H^{(k)})$ will have countable cofinality, since it is an ultrapower of the countable set $X$.\\

Now we consider interpolants of our collapsing-levels $N_\tau$.\\


\begin{lem} \label{non-pluripotent interpolants condense to W-levels}
Suppose $H$ is an interpolant of $N_\tau$ formed as in the proof of \ref{cofinal interpolants} with interpolation embedding $\sigma$, and letting $crit(\sigma) = \bar{\tau}$, suppose $\bar{\tau}$ is not an index or pseudoindex in $W$.  Suppose further that $N_\tau$ is not pluripotent (short or long).  Then $H$ is a level of $W$.
\end{lem}

\textbf{Proof:} We will show that $H$ is a plus-one premouse of the same type as $N_\tau$.  Then we can apply Condensation to $H$.  Notice that Anomalous Case 4 does not apply, since if there were a total long extender $G = E_\gamma^{N_\tau}$ on the $N_\tau$-sequence such that $\kappa_G^{++} = \bar{\tau}$, we would have $\Lambda = (\kappa_G^+)^{N_\tau}$.  But then $N_\tau$ violates projectum-free spaces, contradiction.\\


Clauses $c)$ and $d)$ of Condensation are impossible, since they require $crit(\sigma)$ to be an index or pseudoindex, respectively; so we have $a)$ or $b)$, so the conclusion holds.\\

So the proof reduces to the problem of showing that $H$ is a premouse of the same type as $N_\tau$.  Obviously the fully elementary hull $X$ is a premouse of the same type as $N_\tau$, because all conditions of premousehood are expressible by some first-order formulas.  So we want to show that the ultrapower embedding $i_E: X \longrightarrow H$ preserves premousehood.  For this we will apply Lemma 1.10, so we must verify that its hypotheses hold.  We will first check the various assumptions of clauses $a)$-$f)$, and save the verification that $H$ has projectum-free spaces for last.  Recall that $i$ is an ultrapower map by an extender of length $\Lambda$.\\

First note that $i$ is $\Sigma_0^{(k)}$-elementary and cofinal.  Now if $N_\tau$ is passive then so is $X$, and $H$ is trivially a premouse; so assume $N_\tau$ is active with top extender $G$.  Also if $k > 0$ then $i$ is $\Sigma_2$-preserving-- in fact it is $Q^{(1)}$-preserving.  This is enough to verify $a)$, $b)$, or $f)$ of Lemma 1.10, except for the projectum-free spaces clause for $H$, which we defer for the moment.\\


Now suppose $k = 0$.  Because $N_\tau$ is not pluripotent, we have that $\kappa_G \geq \Lambda$ if $G$ is short, and $( \kappa_G^+ )^{N_\tau} \geq \Lambda$ if $G$ is long (this can be concisely written as ``$dom(G) > \Lambda$").  These facts transfer down to $X$: letting $\bar{\Lambda}$ be the preimage of $\Lambda$ in $X$, and $\bar{G}$ the top extender of $X$, $dom (\bar{G} ) > \bar{\Lambda}$, and $dom (\bar{G})$ is a successor cardinal of $X$.  Now since $k(N_\tau ) = 0$, $i$ is a $0$-ultrapower map, and its only discontinuities are at points whose $X$-cofinality is $< \bar{\Lambda}$.  In this case $i$ is continuous at $dom (\bar{G})$, so $H$ has a total top extender.  Also $\lambda_{\bar{G}} > \bar{\Lambda}$, so $i$ will map $\lambda_{\bar{G}}$ continuously into its image.  Finally, if $X$ is type $Z_1$, $\nu_{\bar{G}}$ has $X$-cofinality equal to $(\kappa_F^+ )^X > \kappa_{\bar{G}}^+ \geq \bar{\Lambda}$ (because of non-pluripotence), so again $i$ is continuous at $\nu_{\bar{G}}$. This verifies all the hypotheses for $c)$, $d)$, or $e)$ of Lemma 1.10, except for the projectum-free spaces assumption.\\

The proof will be complete if we can show that $H$ has projectum-free spaces.  First we consider $1)$ of \ref{PFS}.  Note that $\varrho_{k+1}(H) = \varrho_\omega (H) = \Lambda$, because there is a definable $\Sigma_{k+1}$ surjection from $\Lambda \cup i (p(X))$ onto $H$ (so $\varrho_{k+1}(H) \leq \Lambda$), and $\Lambda$ is an $L[E]$-cardinal, where our construction takes place (so $\varrho_\omega (H) \geq \Lambda$).  If $\Lambda$ were the space of a long extender on the $H$-sequence, then letting $\sigma : H \longrightarrow N_\tau$ be the interpolation embedding, note that $crit (\sigma) > \Lambda$, so $\sigma$ would send this long extender to an extender on the $N_\tau$ sequence with space $= \Lambda$.  But $\Lambda$ was $\varrho_{k+1}(N_\tau)$, so this would violate projectum-free spaces for $N_\tau$.  All that remains is to show that the $\varrho_m (H)$ for $m \leq k$ do not violate projectum-free spaces.  But note that $i : X \longrightarrow H$ is a $k$-embedding, so we can follow the proof of projectum-free spaces preservation that we used in Lemma 2.7.  In particular, for $m < k$, $i (\varrho_m(X)) = \varrho_m(H)$, so the absence of PFS violations at those projecta in $H$ follows by $\Sigma_1$-elementarity of $i_E$; and for $m = k$, if $i$ is continuous at $\varrho_k(X)$ the same argument applies, whereas if $i$ is discontinuous at $\varrho_k(X)$ then $\varrho_k (X)$ must be $\Sigma_k$-singular in $X$ and therefore a limit cardinal of $X$.  Then $\varrho_k (H) = sup ( i `` \varrho_k (X))$ is a limit cardinal of $H$ and therefore cannot cause a violation of $1)$ of PFS.\\

It is easy to see that $H$ satisfies $2)$ of PFS, because $X$ satisfies it: if $k > 0$ this follows from elementarity, and if $k = 0$ then $X$ does not have a short top extender $\bar{G}$ such that there is a long $F$ on the $X$-sequence with $\kappa_{\bar{G}} = \kappa_F$ and $\varrho_1 (X) \leq \kappa_{\bar{G}}^+$.  This is a $\Pi_1$-statement about $X$, so it holds in $H$ as well.  $\blacksquare$\\











\section{Short Protomice}

We now describe the (much more complicated) situation that arises when $N_\tau$ is short or long pluripotent.  The interpolants of $N_\tau$ will then  fail to be premice; instead they will be \textit{protomice}, and we must invoke a procedure to turn them into levels of $W$ before we can use them in the definition of $C_\tau$.  We will refer to the interpolated protomouse as $M$, and its associated premouse as $N$. \\

\subsection{Fine Structure for Short Protomice}

\begin{defn} \label{short protomouse}
A short protomouse $M = (| M |, \tilde{G})$ is a $J$-structure, considered in the language of coherent structures, such that\\

$a)$ $| M |$ is a passive premouse with $k( |M|) = 0$,\\

 $b)$ $\tilde{G}$ is a short extender over $|M|$ that it is not total on $|M|$; more precisely, there is an ordinal $\theta < \kappa_{\tilde{G}}^+$ such that $\tilde{G}$ measures exactly the subsets of $\kappa_{\tilde{G}}$ in $M | \theta$, and $\theta = (\kappa_{\tilde{G}}^+)^{M| (\theta)}$;\\
 
 $c)$ $M$ satisfies the coherency condition $Ult_n ( M || \theta , \tilde{G} ) = |M|$,\\
 
 
  $d)$ $\varrho_1 (M)$ is not the space of a long extender on the sequence of $Ult_n(N^* , \tilde{G})$, where $\langle N^* , n \rangle$ is the collapsing-level for $\theta$ in $M$.\\
\end{defn}

\bigskip

\textbf{Remark:} $b)$ and $c)$ imply that $i_{\tilde{G}}``(\theta)$ will be cofinal in $o(M)$.\\

In all our dealings with protomice, we will use the language of coherent structures.  This means we will be working with their Dodd parameters $d(M)$, and $\varrho_1(M)$ in the above definition is the Dodd projectum. Also, we only need to consider $\Sigma_1$-definability for protomice (in the language of coherent structures), and never need to talk about $\Sigma_1^{(n)}$-definability over protomice for $n > 0$.\\



\begin{defn} \label{theta of short protomouse}
Given a short protomouse $M$ with top extender $\tilde{G}$, let $\theta^M = dom(\tilde{G})$.  (Note $\theta^M < (  \kappa_{\tilde{G}}^+ )^M$.)  Also, let $(N^* )^M$ be the collapsing-level for $\theta$ in $M$; so $(N^*)^M = \langle N^* , n \rangle$ where $n$ is such that $\kappa_{\tilde{G}} = \varrho_{n+1} ( N^* ) < \varrho_n (N^* )$.
\end{defn}


We will see later that pluripotent levels of $W$ interpolate to yield protomice; this is a major problem for our $\square$ construction.  Schimmerling and Zeman solved this problem in \cite{zeman square proof} by showing that these protomice can nevertheless be canonically associated with levels of $W$, and the associations are sufficiently uniform that we can translate the finestructural properties of protomice back and forth with the levels of $W$.\\

Schimmerling and Zeman's basic idea is that given a protomouse $(M, \tilde{G})$ which is an interpolant of a level of $W$, we should apply $\tilde{G}$ to the longest initial segment of $M$ which it measures.  In our notation, the structure produced will be $N = Ult((N^*)^M , \tilde{G})$, and we will call it the ``associated ppm" of $M$.  $N$ can then be shown to be a premouse, and in fact with a bit of diagram-chasing we can show that if $(M, \tilde{G})$ was derived from a level of $W$, then $N$ is a level of $W$.  This will, hopefully, be the level of $W$ canonically associated with our protomouse $M$.  It is of critical importance, though, that the process of transforming $(M , \tilde{G})$ into $N$ is \textit{canonically reversible}: that is, the level $N$, viewed on its own, must ``know that it is a transformed protomouse", and must provide us with a recipe for reversing the transformation and reconstructing that protomouse.  Fortunately, the process \textit{is} reversible-- there is a simple procedure by which $N$ can reconstruct the protomouse $(M , \tilde{G})$.  Unfortunately, there may be \textit{many different} protomice which $N$ can `reconstruct'.  The different recipes which a level of $W$ can use to reconstruct protomice are called \textit{divisors}.  Our task, then, is to find \textit{canonical divisors} for levels of $W$, and at the same time to refine our interpolation process in such a way that the protomice it produces are exactly those which correspond to the canonical divisors.  This is all successfully accomplished in Schimmerling and Zeman's paper \cite{zeman square proof}.\\

In the context of plus-one premice, matters are complicated by the existence of both short protomice and long protomice that can arise in interpolations.  We begin by reproducing the treatment of short protomice from \cite{zeman square proof}, but in the richer context of plus-one premice.\\


\begin{defn} \label{short associated ppm}
Given a short protomouse $M$ with top extender $\tilde{G}$ and $(N^*)^M = \langle N^* , n \rangle$, we define the associated ppm of $M$ to be $Ult_n (N^* , \tilde{G} )$, and we call $i_{\tilde{G}}$ the associated ppm embedding.
\end{defn}

\textbf{Remark:} It is easy to see, using \ref{ppm up-pres}, that the associated ppm of a short protomouse $M$ is in fact a potential premouse.  This is because the ultrapower map $i_{\tilde{G}}$ is only discontinuous at $\kappa_{\tilde{G}}$, which is a limit cardinal of $N^*$; so by the remark following \ref{ppm up-pres}, if the associated ppm has a top extender then it must be total.\\






\textbf{Remark:} There is, unfortunately, a very technical case-splitting which we must consider.  If $N^* = (N^*)^M$ is passive or has a top extender $H$ such that $\lambda_H > \kappa_{\tilde{G}}$, then \ref{upward mouse-pres} can be used to show that $N = Ult_n (N^* , \tilde{G} )$ is a premouse of the same type as $N^*$.  However, if $N^*$ has a top extender $H$ such that $\lambda_H = \kappa_{\tilde{G}}$, then $i_{\tilde{G}}$ will be discontinuous at $\lambda_H = \lambda (N^*)$.  This disrupts the application of \ref{upward mouse-pres}, which essentially relies on the embedding being continuous at $\lambda (N^*)$.  We will see in \ref{short protomouse condensation} that it is still possible in this case to show that $N$ is a premouse; however, it may not be of the same type as $N^*$.  Because of this, $i_{\tilde{G}}$ will not be an $n$-embedding in the language of premice when $\lambda_H = \kappa_{\tilde{G}}$, but merely an $n$-embedding in the language of coherent structures.  Many of the lemmas that follow have two parts, corresponding to this case-splitting.\\




Now we describe the finestructural relations between a protomouse $M$ and its associated ppm $N$.\\


\begin{lem} \label{translation1}
\textbf{(Parameter-Less Simulation of Definable Singletons of $N$ from within Protomouse)}\\
\textbf{(from \cite{zeman square proof})}\\
Let $M = (|M| , \tilde{G})$ be a short protomouse.  Let $\kappa = \kappa_{\tilde{G}}$, $ \lambda = \lambda_{\tilde{G}}$, and $N^* = (N^*)^M$.  Let $i: N^* \longrightarrow  N$ be the associated ppm embedding.  Let $\zeta < \lambda$, $b \subset \lambda$ be finite, and $r =  i (p (N^* ))$, so $min(r) \geq \lambda$ if $r$ is nonempty.  Then $\{ \zeta \}$ is $(\Sigma_1^{(n)})^N$-definable from $r$, $b$, and an ordinal $\xi < \kappa$ if and only if there is an $f: \kappa \longrightarrow \kappa$ in $N^*$ such that $\zeta = i (f)(b)$.  Likewise, $y \subset \lambda$ is $(\Sigma_1^{(n)})^N$-definable from $r$, $b$, and an ordinal $\xi < \kappa$ if and only if there is an $f: \kappa \longrightarrow \mathcal{P} ( \kappa )$ in $N^*$ such that $y = i (f)(b)$.\\


Here $(\Sigma_1^{(n)})^N$-definability is in the language of premice if $i$ is an $n$-embedding in the language of premice, and in the language of coherent structures if $i$ is an $n$-embedding in that language.
\end{lem}

The proof is the same whether $i$ is in the language of premice or coherent structures; the difference arises only in which version of {\L}o{\'s}'s Theorem applies to $i$.  In what follows we prove both versions at once.  We focus on the case with a single ordinal $\zeta$; the proof for $y \subset \lambda$ is similar.\\

We begin with the forward implication.  Suppose $\zeta$ is the unique object such that $N \models (\exists z) \psi ( z , \zeta , r , b )$ where $\psi$ is a $\Sigma_0^{(n)}$-formula.  Fix a $\delta^* < \varrho_n^{N^*}$ large enough such that, setting $\delta = i(\delta^* )$, there is a $z \in N^{(n)} | \delta$ witnessing this existential statement; such a $\delta^*$ exists since $i$ is cofinal in $N^{(n)}$.  Define a partial map $f: [ \kappa ]^{|b|} \longrightarrow \kappa$ as follows:

\[
f(x) = \text{ the unique } \xi < \kappa \text{ such that } N^* \models (\exists z \in (N^*)^{(n)} | \delta^* ) \psi (z , \xi , p (N^* ) , x ) \ .
\]

Then $f$, being a $(\bf{\Sigma}_0^{(n)})^N$ subset of $\kappa < \varrho_n (N^* )$, is an element of $N^*$.  Applying $i$, it follows that $i (f)(x)$, if defined, is the unique ordinal $\xi < \lambda$ such that we have $(\exists z \in N^{(n)} | \delta ) \psi (z, \xi , r , x )$ in $N$.  But for $x = b$, we know that $i (f) (x)$ is defined, so $i (f)(b) = \zeta$.  Obviously, $f$ can be turned into a total function on $[ \kappa ]^{|b|}$ by setting $f(x) = 0$ whenever $f(x)$ is undefined by the above definition.\\

To see the converse implication, suppose $\zeta = i (f)(b)$ for $f$, $b$ as above; since $N^*$ is sound, there is a $\xi < \kappa$ such that $f = h_{n+1}^{N^*} ( \xi , p (N^*))$.  The preservation properties of $i$ then give $i(f) = h_{n+1}^N ( \xi , r)$, so $\zeta$ can be defined in a $\Sigma_1^{(n)}$-fashion over $N$ as follows:

\[
(\exists g ) ( g = h_{n+1}^N (\xi , r ) \ \& \ \zeta = g (b)) \ .
\]

$\blacksquare$\\

The following lemma shows that definable subsets of $\lambda$ in $N$ are also definable in $M$; thus it is a stronger version of the above lemma, since it applies to arbitrary definable subsets instead of just singletons.  However, it requires $\theta$ as a parameter in the definition over $M$.  It will turn out that the embeddings we consider between different protomice do \textit{not} preserve $\theta$, so we will have to make do with \ref{translation1} in those contexts.  We will primarily need \ref{translation2} in showing that the projectum of a protomouse is the same as that of its associated premouse.\\


\begin{lem} \label{translation2}
\textbf{(Simulation of Definable Classes of $N$ from within Protomouse, in Parameter $\theta$)}\\
\textbf{(from \cite{zeman square proof})}\\
Let $(M , \tilde{G})$ be a short protomouse.  Let $N^*$ be the longest initial segment of $M$ on which $\tilde{G}$ is total.  Let $\kappa = \kappa_{\tilde{G}}$, $ \lambda = \lambda_{\tilde{G}}$, and $i: N^* \longrightarrow Ult_n (N^* , \tilde{G} ) = N$ be the largest possible fine ultrapower map.  Let $r =  i (p (N^* ))$, so $min(r) \geq \lambda$ if $r$ is nonempty.  Then we have:\\

\indent \indent $a)$ If $i$ is an $n$-embedding in the language of premice, then for any $\Sigma_1^{(n)}$-formula $\phi ( v_0 ... v_\ell )$ in the language of premice, there is a $\Sigma_1$-formula $\phi^* ( v_0 ... v_\ell )$ in the language of coherent structures such that for every tuple $x_1 ... x_\ell \in J_\lambda^E$, we have

\[
N \models \phi ( r , x_1 ... x_\ell ) \ \text{ if and only if } \ M \models \phi^* ( \theta , x_1 ... x_\ell ) \ .
\]

\indent \indent $b)$ If $i$ is an $n$-embedding in the language of coherent structures, then for any $\Sigma_1^{(n)}$-formula $\phi ( v_0 ... v_\ell )$ in the language of coherent structures, there is a $\Sigma_1$-formula $\phi^* ( v_0 ... v_\ell )$ in the language of coherent structures such that for every tuple $x_1 ... x_\ell \in J_\lambda^E$, we have

\[
N \models \phi ( r , x_1 ... x_\ell ) \ \text{ if and only if } \ M \models \phi^* ( \theta , x_1 ... x_\ell ) \ .
\]


\end{lem}


\indent \indent \textbf{Proof:}  First we consider $a)$.  We may assume $x_1 ... x_\ell$ are all ordinals $< \lambda$, since each $x \in J_\lambda^E$ is $\Sigma_1$-definable from an ordinal $< \lambda$.  

Suppose $\phi$ is of the form $(\exists z) \psi (z , vo ... v_\ell )$ where $\psi$ is $\Sigma_0^{(n)}$.  Then $N \models \phi (r, x_1 ... x_\ell )$ if and only if $(\exists u \in (N^*)^{(n)} ) [ N \models ( \exists z \in \pi (u) ) \psi ( z , r , x_1 ... x_\ell ) ]$.  Using {\L}o{\'s}'s Theorem, this can be expressed in a $\Sigma_1$-fashion over $M$ as

\[
( \exists Q , p^* , \kappa , a , y , u , m) \phi_0^* (Q , p^* , \kappa , a , y , u , m , \theta , x_1 ... x_\ell )
\]

where $\phi_0^* (Q , p^* , \kappa , a , y , u , m , \theta , x_1 ... x_\ell )$ is the conjunction of the following statements:\\

\begin{itemize}
\item{ $Q$ is an initial segment of $M$ and $m \in \omega$,}
\item{ $\theta = (\kappa^+)^Q$, $\varrho_{m+1} (Q) = \kappa < \varrho_m (Q)$, $p^* = p (Q)$ and $u \in Q^{(n)}$,}
\item{ $a = \{ \langle \eta_1 ... \eta_\ell \rangle \in \kappa \ | \ Q \models (\exists z \in u ) \psi (z , p^* , \eta_1 ... \eta_\ell ) \}$,}
\item{ $y = F(a)$ and $\langle x_1 ... x_\ell \rangle \in y$.}
\end{itemize}


 
 \bigskip

The proof for $b)$ is the same, but only allows translations in the language of coherent structures because this is the version of {\L}o{\'s}'s Theorem that applies in that case. $\blacksquare$\\

The next lemma shows how the associated ppm of $M$ can still ``see" $M$ to some extent.  This allows us to translate back and forth between protomice and their associated ppm, which is necessary for the finestructural computations which follow.\\







\begin{lem} \label{translation3}
\textbf{(Simulation of Short Protomouse From its Associated Premouse)}
\textbf{(from \cite{zeman square proof})}
Let $M$ be a short protomouse with associated ppm $N$, and $i = i_{\tilde{G}}: N^* \longrightarrow N$ the ultrapower map.  Let $\phi ( v_1 ... v_\ell )$ be a $\Sigma_1$-formula in the language of coherent structures.  Then we have\\

\indent \indent $a)$ If $i$ is an $n$-embedding in the language of premice, then there is a $\Sigma_1^{(n)}$-formula $\psi ( v, v' , v_0 ... v_\ell )$ in the language of premice and an ordinal $\xi_0 < \kappa$ such that for every $x_1 ... x_\ell \in M$, 

\[
M \models \phi ( x_1 ... x_\ell ) \ \textit{ if and only if } \ N \models \psi ( r , \xi_0 , \kappa , x_1 ... x_\ell ) \ .
\]

\indent \indent $b)$ If $i$ is an $n$-embedding in the language of coherent structures, then there is a $\Sigma_1^{(n)}$-formula $\psi ( v_0 ... v_\ell )$ in the language of coherent structures such that for every $x_1 ... x_\ell \in M$,

\[
M \models \phi ( x_1 ... x_\ell ) \ \textit{ if and only if } \ N \models \psi ( \kappa , x_1 ... x_\ell ) \ .
\]

\end{lem}

\indent \indent \textbf{Proof:}  First we focus on $a)$.  Since $\phi$ is a $\Sigma_1$-formula, $M \models \phi (x_1 ... x_\ell)$ if and only if there is an ordinal $\zeta < o(M)$ such that $ \langle J^E_\zeta , G \cap J^E_\zeta \rangle \models \phi (x_1 ... x_\ell )$.  Fixing an ordinal $\xi_0 < \kappa$ such that $h_{n+1}^N ( \xi_0 , r ) = \lambda (\kappa , q)$, this can be expressed over $N$ in the parameters $r , \kappa , \xi_0$ as

\[
( \exists \zeta , \lambda , \beta_1 , \beta_2 , f , G , Q ) \ \psi ( \zeta , \lambda , \beta_1 , \beta_2 , f , G , Q , r , \xi_0 , \kappa , x_1 ... x_\ell )
\]

where $\psi$ is the conjunction of the following statements:\\

\begin{itemize}

\item{$\beta_1 , \beta_2 < \kappa$ and $\zeta < \lambda^+$ ;}
\item{$\zeta = h_{n+1}^N ( \beta_1 , r )$ , $f = h_{n+1}^N ( \beta_2 , r )$ , and $\lambda = h_{n+1}^N ( \xi_0 , r )$ ;}
\item{$f : \lambda \xrightarrow{onto} \mathcal{P} (\lambda ) \cap J^E_\zeta$ ; }
\item{$G = \{ \langle f ( \alpha ) \cap \kappa , f (\alpha ) \rangle | \alpha < \kappa \}$ and $Q = \langle J^E_\zeta , G \rangle$ ;}
\item{$Q \models \phi (x_1 ... x_\ell )$ . }
 

\end{itemize}

In other words, $\psi$ asserts that there is a level $J^E_\zeta$ and a surjection $f : \lambda (\kappa , q ) \xrightarrow{onto} \mathcal{P} (\lambda ) \cap J^E_\zeta$ which are both in the range of $i$ (so that their collapses in $N^*$ will be, respectively, a level of $N^*$ and a surjection from $\kappa$ onto that level's $\mathcal{P} (\kappa )$).  From here we can describe a fragment of the protomouse-extender, called $G$, as the set of ordered pairs of subsets of $\kappa$ in this level of $N^*$ and the subsets of $\lambda (\kappa , q)$ which they stretch to by $i$.  (This is trivially equivalent to our official description of extender-predicates.)  The resulting structure $Q$ is a level of $M$.\\

The proof of $b)$ is almost identical, but $\psi$ now requires fewer parameters; recall that in this case, we must have $\lambda ( \kappa , q) = \lambda_N$, so $r$ is empty, and also we can talk about $\lambda (\kappa , q)$ in the formula $\psi$ without requiring $\xi_0$ as a parameter.  $\blacksquare$\\


We can now relate the finestructural properties of $M$ and $N$.  Recall that for a protomouse $M$, $W_{\alpha , s}^M = \mathcal{H}_{1}^M (\alpha \cup \{ s \} )$; and for a premouse $\langle N , n \rangle $, $W_{\alpha , s}^N = \mathcal{H}_{n+1}^N ( \alpha \cup \{ s \} ) $.\\


\begin{lem} \label{short finestructure computation}
\textbf{(from \cite{zeman square proof})}
Let $(|M| , \tilde{G})$ be a short protomouse and $N$ the associated premouse, with ultrapower embedding $i$, and let $n$ be the degree of soundness of $N^* = (N^*)^M$ (so $i$ is an $n$-ultrapower map).  Set $o(M) = \eta$, $\kappa = \kappa_{\tilde{G}}$, $\lambda = \lambda_{\tilde{G}}$, and $r = i(p(N^*))$ (so $min(r) \geq \lambda$ if $r$ is nonempty).  Then\\

$a)$ $\varrho_1 (M) = \varrho_{n+1} (N)$.\\

Denote this common value by $\varrho$.  Granting that $\kappa < \varrho$, the following holds:\\

$b)$ $p_{n+1} (N) \cap \lambda = d(M)$.\\

$c)$ $M$ is $1$-sound if and only if $N$ is $(n+1)$-sound.\\

$d)$ Let $s$ be a finite subset of $\lambda$ and $\theta \leq \alpha < \lambda$.  Then $W_{\alpha , s \cup r}^N = Ult_n (N^* , G )$, where $G$ is the top extender of $W_{\alpha , s}^M$ (so $G$ has the same ultrapower as $\tilde{G} \restriction ( \text{coordinates in } \alpha \cup s)$).  Moreover, the associated ultrapower embedding is precisely the uncollapsing map associated with the $\Sigma_1^{(n)} ( W_\alpha^N )$-hull of $\kappa \cup  \bar{r}$, where $\bar{r}$ is the preimage of $r$ under the canonical witness map.\\

$e)$ $M$ is $1$-solid if and only if $N$ is $(n+1 )$-solid.
\end{lem}

\textbf{Proof:}  Note first that neither $\varrho_1 (M)$ nor $\varrho_{n+1} (N)$ is larger than $\lambda$.  First we show $a)$.  If $A$ is a $\Sigma_1 (M)$-relation in $p_1 (M)$ then by \ref{translation3} there is a $\Sigma_1^{(n)} (N)$-relation $A^*$ in $d (M)$, $r$, and $\theta$ such that $A^*$ agrees with $A$ up to $\varrho_1 (M)$.  Choose $A$ such that $A \cap \varrho_1 (M) \notin M$.  Then $A^* \cap \varrho_1 (M)$ is not a member of $J_{o(M)}^E$ and therefore of $N$; this follows from the fact that $o(M)$ is a cardinal in $N$.  Thus, $\varrho_{n+1}(N) \leq \varrho_1 (M)$.  The dual argument using \ref{translation2} yields the converse, which proves $a)$.\\

From now on suppose that $\kappa < \varrho$.  We now prove $b)$.  The ordinal $\varrho$, being smaller than $\lambda$, is a cardinal in both $M$ and $N$.  It follows that $\theta < \varrho$.  Given $A$ as above, by \ref{translation3} there is a $\Sigma_1^{(n)}$-relation $A^*$ such that $A(\xi) \leftrightarrow A^* ( d (M) , r , \theta , \xi )$ whenever $\xi < \lambda$.  From $A^*$ we obtain a new subset of $\varrho$ which is $\Sigma_1^{(n)}(N)$ in $d (M) \cup r$, so $r \cup (p_{n+1} (N) \cap \lambda) = p_{n+1} (N) \leq_{lex} r \cup d (M)$ and, consequently, $p_{n+1} (N) \cap \lambda \leq_{lex} d (M)$.  As before, the dual argument yields the converse, which proves $b)$.\\

If $M$ is $1$-sound, then every $\xi < \lambda$ is $\Sigma_1 (M)$-definable from $d(M) = p_{n+1} (N) \cap \lambda$ and a parameter less than $\varrho$.  Thus, every $\xi < \lambda$ is $\Sigma_1^{(n)}(N)$-definable from $r \cup (p_{n+1} (N) \cap \lambda) = p_{n+1}(N)$ and parameters less than $\varrho$.  In other words, $N = Hull_{n+1}^N ( \varrho \cup \{ p_{n+1} (N) \} )$.  Thus, $p_{n+1} (N) \in R_{n+1}(N)$, so $N$ is $(n + 1)$-sound.  The converse follows again by the dual argument, which proves $c)$.\\

Next we prove $e)$ from $d)$.  To see that the $1$-solidity of $M$ implies the $(n+1)$-solidity of $N$, notice that $W_{\alpha}^M = \mathcal{H}_1^M (\alpha \cup \{ d(M) \} )$ can be encoded into a $\bf{\Sigma_1} (M)$ subset $A$ of $\alpha$.  Such an $A$ is in $J_{\eta}^E$ by acceptability, and $W_{\alpha}^M$ can be reconstructed from $A$ inside $J_{\eta}^E$.  But then also $W_{\alpha}^N$ is in $J_{\eta}^E$ by $d)$.  For the converse use again the dual argument.\\

Finally we show $d)$.  Let $\bar{\sigma}: W_{\alpha , s}^M \longrightarrow M$ be the canonical witness map, $\bar{\lambda} = \lambda_G$ and $\bar{\eta} = o ( W_{\alpha , s}^M)$.  Since $\bar{\sigma}$ is $\Sigma_1$-preserving, $dom( G) = \mathcal{P} ( \kappa ) \cap J_{\theta}^E$, so $G$ can be applied to $N^*$ (notice that $N^*$ is an initial segment of $W_{\alpha , s}^M$).  Let $W = Ult_n (N^* , G )$ and $\bar{\pi} : N^* \longrightarrow W$ the associated ultrapower map.  Clearly $\bar{\pi}$ is $\Sigma_0^{(n)}$-preserving and cofinal, and $W = Hull_{n+1}^W ( \bar{\lambda} \cup \{ \bar{r} \} )$, where $\bar{r} = \bar{\sigma} ( p (N^*))$.  We can also define a natural factor-map $\sigma : W \longrightarrow N$ by $\sigma ( \bar{\pi} (f)(a)) = i (f) ( \bar{\sigma} (a))$ for $a \in [ \bar{\lambda} ]^{< \omega}$.  It is easy to see that $\sigma$ is $\Sigma_0^{(n)}$-preserving and cofinal, by {\L}o{\'s}'s Theorem for $\Sigma_0^{(n)}$-formulas together with the cofinality of $\bar{\sigma}$ and $i$.  Also, $i = \sigma \circ \bar{\sigma}$ and $crit (\sigma ) \geq \bar{\eta}$.  It follows that $\sigma ( \bar{r}) = r$, $\sigma \restriction \alpha = id$, and, letting $\bar{s}$ be the $\bar{\sigma}$-preimage of $s - ( \alpha + 1)$, also $\sigma ( \bar{s}) = \bar{\sigma} ( \bar{s}) = s - (\alpha + 1 )$.  As each $\zeta < \bar{\lambda}$ is $\Sigma_1 (W_{\alpha , s }^M )$-definable from $\bar{s}$ and parameters below $\alpha$, \ref{translation3} allows us to conclude that $W = Hull_{n+1}^W ( \alpha \cup \{ \bar{r} \cup \bar{s} \} )$, so $Hull_{n+1}^N ( \alpha \cup \{ r \cup s - ( \alpha + 1 ) \} ) = range( \sigma)$.  But this means that $W = W_{\alpha , r \cup s }^N$ and $\sigma$ is the associated witness map. $\blacksquare$\\ 







\subsection{Short Divisors}

We now look at how the ppm $N$ associated to $M = (|M| , \tilde{G})$ can recover $M$.  We want to see that $N^*$ can be identified as a $\Sigma_1^{(n)}$-hull in $N$, using the standard parameter of $N$ to help identify it.  If we can do this, the protomouse-extender $\tilde{G}$ can be recovered as the extender of the corresponding uncollapse map.\\



\begin{defn} \label{short divisor}
Let $N = N_{\bar{\tau}}$ be the collapsing-level for $\bar{\tau}$ in $W$, with $\varrho_{n+1}(N) = \Lambda < \varrho_n (N)$ and $\bar{\tau} = (\Lambda^+ )^N$.  Then a pair $( \kappa , q )$ is a \textit{short divisor} of $N$ if $\kappa$ is a cardinal $< \Lambda$, and there is an ordinal $\lambda (\kappa , q) $ with $\Lambda < \lambda (\kappa , q) < \varrho_n(N)$ such that, letting $r = p (N) - q$, the following conditions hold:\\


\indent \indent $a)$ $q = p(N) \cap \lambda (\kappa , q)$,\\

\indent \indent $b)$ $\mathcal{H}_{n+1}^N ( \kappa \cup r ) \cap \varrho_n (N)$ is cofinal in $\varrho_n^N$,\\

\indent \indent $c)$ $\lambda ( \kappa , q) $ is the least ordinal in $\mathcal{H}_{n+1}^N ( \kappa \cup r ) - \kappa$.\\

In addition, we say $(\kappa , q)$ is a strong short divisor if it satisfies:\\

\indent \indent $d)$ $\mathcal{P} (\kappa ) \cap \mathcal{H}_{n+1}^N ( \kappa \cup r ) = \mathcal{P} (\kappa ) \cap \mathcal{H}_{n+1}^N ( \kappa \cup p(N) )$.
\end{defn}


We refer to $\mathcal{H}_{n+1}^N ( \kappa \cup r )$ as the \textit{divisor-hull} associated with a short divisor $(\kappa, q)$.  Conditions $a)$, $b)$, and $c)$ above constitute the definition of divisor in \cite{zeman square proof}.  However, the authors of \cite{zeman square proof} exclusively consider ``strong divisors" in their $\square_\Lambda$ construction, which have the additional clause $d)$.  Thus our ``strong short divisors" are the ``strong divisors" of \cite{zeman square proof}.  We only need to consider the notion of non-strong short divisors in order to develop the theory of strong short divisors in this subsection; after this subsection we will only be using strong short divisors.\\

Note that the uncollapse map $\pi$ from a divisor-hull $\mathcal{H}_{n+1}^N ( \kappa \cup r )$ into $N$ is $\Sigma_0^{(n)}$-elementary and cofinal.\\

\begin{lem}
If $N$ is a level of $W$ and $(\kappa , q)$ is a short divisor of $N$ with divisor-hull $N^*$, then $\kappa$ is a limit cardinal of $W$, and is inaccessible in $N^*$.
\end{lem}

\textbf{Proof:}  First we show that $\kappa$ is a limit cardinal in $N^*$.  Clearly it is a cardinal.  If $\kappa = ( \eta^+ )^{N^*}$, then $N^* \models$ ``$\eta$ is the largest cardinal in $J_\kappa^E \ $".  Letting $\pi: N^* \longrightarrow N$ be the uncollapse map, $N \models$ ``$\eta$ is the largest cardinal in $J_{\pi (\kappa )}^E$".  But $\pi (\kappa ) = \lambda (\kappa , q )$, and there are certainly cardinals greater than $\eta$ below it, e.g., $\Lambda$; contradiction.\\

It is now immediate that $\kappa$ is a limit cardinal in $W$, because $\pi$ is $\Sigma_1$-elementary with critical point $\kappa$, and so the cofinal sequence of $N^*$-cardinals below $\kappa$ are all preserved as cardinals $< \Lambda$ in $N$, and $N$ agrees with $W$ below $\Lambda$.  Finally we note that $\kappa$ must be regular in $N^*$, hence inaccessible, because it is the critical point of $\pi$.  $\blacksquare$\\


\begin{lem}
If $N$ is a level of $W$ and $(\kappa , q)$ is a short divisor of $N$ with divisor-hull $N^*$, then $N^*$ is a proper initial segment of $N$.
\end{lem}

\textbf{Proof:} Let $\pi: N^* \longrightarrow N$ be the uncollapse map associated with the divisor-hull.  Thus $crit(\pi) = \kappa$ and $\pi ( \kappa ) = \lambda ( \kappa , q) = \lambda$.  Set $s = \pi^{-1} (r)$ (recall $r$ is a top segment of $p (N)$, namely, the part above $\lambda$).  Then $N^* = Hull_{n+1}^{N^*} (\kappa \cup s)$; combining this with the previous lemma, we conclude that

\[
\varrho_\omega (N^*) = \varrho_{n + 1} (N^*) = \kappa < \varrho_n (N^*) \ ,
\]

and also that $s$ is a very good parameter for $N^*$.\\

 We would like to see that $N^*$ is a premouse of the same type as $N$ by applying Lemma 1.12.  First we show that $N^*$ satisfies projectum-free spaces.  Clearly for $m > n$, $\varrho_m (N^*)$ is not a space, since we have just seen that it is a limit cardinal, namely $\kappa$.  For $m \leq n$, we know that $\pi$ is an $n$-embedding, so we can follow the argument given in Lemma 2.7 to show that $\varrho_m$ does not cause a PFS violation.\\

Now if $N$ is type $A$ or $B$ (short or long varieties), then clauses $a)$ or $c)$ of Lemma 1.12 can be applied immediately.  If $N$ is type $C$ (short or long), we need to verify that $n > 0$.  But if $n = 0$, then $\Lambda = \varrho_1 (N)$, and $\varrho_1$ of any type $C$ premouse is always the largest cardinal of that premouse (by \ref{projecttolambda}).  This contradicts the fact that $\Lambda < \lambda$ in the definition of a divisor, since $\lambda$ is a cardinal of $N$.  Thus $N^*$ is a premouse of the same type as $N$.\\

\textbf{Claim:} $N^*$ is sound.\\

\textbf{Proof:} If $(\kappa^+)^{N^*}$ does not exist, then $N^*$ is easily seen to be sound: if $s$ is empty then $N^*$ is trivially sound, because in this case $N^* = Hull_{n+1}^{N^*} (\kappa ) $ and we know that $\kappa = \varrho_{n+1} (N^* )$.  If $s$ is nonempty and $t < s$ is the standard parameter for $N^*$ then by universality $t$ generates $s$, since they are both below $\kappa^+$.  But then $t$ is a very good parameter, so $N^*$ is sound.\\

Now assume $(\kappa^+ )^{N^*}$ exists.  Recall that $\pi$ sends its critical point $\kappa$ to $\lambda$, and that $N = Hull_{n+1}^N (\lambda \cup r) = Hull_{n+1}^N (\lambda \cup \pi `` (N^*))$.  Letting $G$ be the extender of length $\lambda$ derived from $\pi$, it follows that $\pi = i_G$.  (That is, the short extender derived from $\pi$ is sufficient to capture the full embedding.)  This implies that $\pi$ is only discontinuous at points of cofinality $\kappa$ in $(N^*)^{(n)}$.  In particular, it is continuous at $\theta = (\kappa^+)^{N^*}$.\\
 
 We want to see that $s$ is the standard parameter of $N^*$.  Suppose $t <_{lex} s$ were the standard parameter; by the Solidity theorem of \cite{FSPIPM}, we have that $t$ is universal.  That is, $N^* | \theta \ \subseteq \ Hull_{n+1}^{N^*} (\kappa \cup t )$.  But then $Hull_{n+1}^N (\kappa \cup \pi (t))$ is cofinal in $\lambda^+$ (using the fact that $\pi$ is continuous at $\theta$).  However, $\pi (t) <_{lex} r$, so $Th_{n+1}^N (\pi (t) \cup \lambda ) \in N$, by solidity of $r$.  This theory codes a collapse of $(\lambda^+)^N$, which is a contradiction. $\blacksquare$\\

We have shown that $\pi : N^* \longrightarrow N$ is a $\Sigma_0^{(n)}$-elementary and cofinal embedding, with critical point $\kappa = \varrho_{n+1} (N^* )$, and that $N^*$ is a sound premouse of the same type as $N$.  This is all we need in order to apply Condensation.  (Notice that Anomalous Case 4 does not apply, since the critical point $\kappa$ is a limit cardinal, and AC4 requires it to be a local double successor.)\\

Clause $a)$ of Condensation is obviously impossible, since $N^*$ projects to $\kappa$ while $N$ does not. Clauses $c)$ and $d)$ require $\kappa$ to be a successor cardinal in $N$, which it is not.  Thus clause $b)$ must hold.  This proves the Lemma. $\blacksquare$\\


We are finally ready to describe the protomouse associated with this divisor $(\kappa , q )$.  Because $N^*$ projects to $\kappa$, it must be the collapsing-level in $L[E]$ for $\theta = (\kappa^+ )^{N^*}$.  (It is possible that $\kappa$ is the largest cardinal in $N^*$, in which case $\theta = height(N^* )$.)\\


\begin{defn} \label{protomouse associated with divisor}
For $N$ a level of $W$ and $(\kappa , q)$ a short divisor of $N$ with $\pi : N^* \longrightarrow N$ the uncollapse map associated with the divisor-hull, let $\theta = (\kappa^+ )^{N^*}$ and $\eta = (( \lambda (\kappa , q ) )^+)^{N} = \pi (\theta )$.  (If $\theta = height (N^* )$ then $\eta = height (N)$.)  We define $N (\kappa , q) = (J_\eta^E , G )$, where $G$ is the extender of length $\lambda (\kappa , q )$ derived from $\pi$, and call $N (\kappa , q)$ the protomouse associated with the divisor $(\kappa , q)$.
\end{defn}

Note that $crit(G) = \kappa$ and $dom (G) = \theta$.  Thus $N (\kappa , q)$ is a short protomouse, since $\theta < ( \kappa^+ )^{L[E]}$.\\



\begin{lem} \label{short associated associated is original}
If $N$ is a level of $W$ and $N (\kappa , q)$ is the protomouse associated with a divisor of $N$, then the associated ppm of $N (\kappa , q)$ is $N$.
\end{lem}

\textbf{Proof:}  The associated ppm of $N(\kappa , q )$ is given by $Hull_{n+1}^N ( \kappa \cup \{ r \} \cup \lambda (\kappa, q))$, because the associated ppm map is the ultrapower map by a short extender of length $\lambda$.  But this hull is the full premouse $N$, since $N$ is sound and $\varrho_{n+1} (N) = \Lambda < \lambda (\kappa , q)$ and $p _{n+1}(N) = r \cup q$ with $q \subset \lambda(\kappa , q)$. $\blacksquare$\\

\textbf{Remark:}  Lemma \ref{short associated associated is original} implies that in the situation described, $d(N(\kappa , q)) = q$.  This is because Lemma \ref{short finestructure computation} clause $b)$ tells us that $p(N) \cap \lambda = d(N(\kappa , q))$.\\

\textbf{Remark:}  It follows from \ref{short associated associated is original} and \ref{short finestructure computation} that any protomouse $N(\kappa , q)$ associated with a divisor of a level of $W$ must be a sound short protomouse.\\


There is another important fact about short divisors which we will need in our $\square_\Lambda$ construction.  We say that two short divisors of a given $W$-level $N$ \textit{overlap} if neither one's divisor-hull is contained in the other's divisor-hull.  The point of our definition of \textit{strong} short divisors is to guarantee that there can never be two overlapping divisors of our $W$-levels $N$; thus, we can order the strong short divisors of $N$ by the inclusion ordering on their divisor-hulls.  We turn now towards proving this fact.\\


\begin{lem} \label{characterization of strongness}
\textbf{(From \cite{zeman square proof})}
Let $N$ be a level of $W$ and $(\kappa , q)$ a short divisor of $N$.  Let $r = p(N) - q$.  Let $N^* = \mathcal{H}_{n+1}^N ( \kappa \cup r )$ and $P =  \mathcal{H}_{n+1}^N ( \kappa \cup p(N) )$.  Then the following are equivalent:\\

$a)$ $(\kappa , q)$ is a strong short divisor; that is, $\mathcal{P} (\kappa ) \cap N^* = \mathcal{P} (\kappa ) \cap P$.\\

$b)$ $N^* = \mathcal{H}_{n+1}^{P} (\varrho (P) \cup p(P) )$.\\

$c)$ $| p(P)| = | p (N^*) | = |r|$.
\end{lem}

\textbf{Proof:}  Recall that $\kappa < \Lambda$ by Definition \ref{short divisor}.  Now let $\pi : N^* \longrightarrow N$ and $\pi' : P \longrightarrow N$ be the associated uncollapsing maps, and $\pi^* = \pi'^{-1} \circ \pi : N^* \longrightarrow P$.  Both $\pi'$ and $\pi^*$ are $\Sigma_0^{(n)}$-preserving and cofinal, and $N^*$ and $P$ have the same $(n+1)$-st projectum $\kappa$.  The map $\pi'$ is sufficiently preserving to guarantee that $P$ is a premouse.  By the Solidity Theorem of \cite{FSPIPM}, $P$ is solid and its standard parameter is universal. (Since $\kappa$ is a limit cardinal, $P$ cannot be type $Z_p$, because $Z_p$ levels always project to a successor cardinal.)\\

Setting $p' = \pi^* (p_{N^*})$, solidity implies that the parameter $p'$ is a top segment of $p_{P}$.  To see that $a)$ implies $b)$, notice first that $p'$ is a good parameter for $P$.  This follows by the standard argument: pick a set $A$ that is $\Sigma_1^{(n)}$ over $N^*$ in parameter $p_{N^*}$, such that $A \cap \kappa \notin N^*$.  Letting $A'$ be $\Sigma_1^{(n)}$ over $P$ in parameter $p'$ by the same definition, we have $A' \cap \kappa = A \cap \kappa$, so $A' \cap \kappa \notin P$ by $a)$.  Hence $p' \leq p (P )$, and $p' \geq p (P)$ follows from Solidity.\\

Clause $c)$ is an immediate consequence of $b)$, so it remains only to derive $a)$ from $c)$.  We have seen that $p (P)$ is a lengthening of $p'$; clause $c)$ then implies that they are equal.  So $N^*$ is $\mathcal{H}_{n+1}^{P} (\kappa \cup p(P)$, and $a)$ follows from the universality of $p(P)$. $\blacksquare$\\


The authors of \cite{zeman square proof} showed that strongness of a short divisor can be characterized over the associated protomouse $N(\kappa , q)$ in a simple way, which we now describe.\\

\begin{defn} \label{theta is closed in M}
Let $M = ( |M| , G )$ be a short protomouse or short pluripotent level of $W$; we say that an ordinal $\theta < \theta_M$ is closed in $M$ relative to $d(M)$ if and only if for every $f: \kappa \longrightarrow \mathcal{P}(\kappa)$ from $M || \theta$ and every $\xi < \kappa$, we have $i_G (f) (d(M) , \xi ) \cap \kappa \in (M || \theta )$.
\end{defn}

Notice that if $\theta < \theta (M)$ is a limit of ordinals that are closed in $M$, then $\theta$ itself is closed in $M$.  Thus, the set of all $\theta < \theta (M)$ which are closed in $M$ is a closed set.  Notice also that if $M$ is a pluripotent level of $W$ then $\theta_M = (\kappa^+)^M$ is trivially closed in $M$.\\

\begin{lem} \label{strongness is the same as theta closed}
\textbf{(From \cite{zeman square proof})}
Let $N$ be a level of $W$ and $(\kappa , q)$ a short divisor of $N$.  Then $(\kappa , q)$ is strong (that is, it satisfies $d)$ of Definition \ref{short divisor}) if and only if $\theta (\kappa , q)$ is closed in $N (\kappa , q)$.
\end{lem}

\textbf{Proof:} Lemma \ref{translation1} implies that for each $x \subset \lambda$ we have $x \in Hull_{n+1}^N (\kappa \cup p(N))$ just in case $x$ is of the form $i_G (f)(q , \xi )$ for some $\xi < \kappa$ and $f: \kappa \longrightarrow \mathcal{P}(\kappa )$ from $dom (G)$, where $G$ is the top extender of $N (\kappa , q)$.  So $(\kappa , q)$ satisfies $d)$ if and only if $x \cap \kappa \in N^* ( \kappa , q)$ for each such $x$ which, in turn, is equivalent to the requirement that $i_G (f)(q , \xi) \cap \mu \in (M || \theta)$. $\blacksquare$\\

\begin{defn} \label{set of strong divisors}
Let $N = N_\tau$ be the collapsing-level for $\tau$ in $W$, as described at the beginning of \ref{interpolation subsection}, and let $q$ be a bottom segment of $p(N)$.  Then $\mathcal{D}^*_q (N) = \{ \kappa < \Lambda | (\kappa , q) \text{ is a strong short divisor of } N \}$.
\end{defn}



\begin{lem} \label{strong short divisors closed and bounded}
\textbf{(From \cite{zeman square proof})}
$\mathcal{D}^*_q (N)$ is closed and bounded in $\Lambda$.
\end{lem}

\textbf{Proof:} First we show that it is closed.  Given a limit point $\kappa$ of $\mathcal{D}^*_q (N)$, the pair $( \kappa , q)$ is clearly a short divisor of $N$.  Suppose for contradiction that $(\kappa , q)$ is not strong.  Let $\pi : N^* \longrightarrow N$ be the uncollapse map associated with the divisor-hull of $(\kappa, q)$, $\pi' : N' \longrightarrow N$ the uncollapse associated with $N' =  \mathcal{H}_{n+1}^N ( \kappa \cup p(N) )$, and $\pi^* = ( \pi')^{-1} \circ \pi$.  Let further $r'$ and $p'$ be the $\pi'$-preimage of $r$ and $p(N)$, respectively.  Then both $N^*$ and $N'$ are premice, as in \ref{characterization of strongness}.  Also as before, we observe that $p(N')$ is a lengthening of $r'$.\\

Granting that $(\kappa , q)$ is not strong, $p (N')$ is a \textit{proper} lengthening of $r'$, by \ref{characterization of strongness}.  Let $\alpha = max ( p(N') - r' )$.  As $N'$ is solid or type $Z_p$ (by the Solidity Theorem of \cite{FSPIPM}), $W_\alpha \in N'$, where $W_\alpha$ is the standard solidity witness for $\alpha$ with respect to $N'$ and $p(N')$.  Let $r'_\alpha$ be the preimage of $r'$ under the canonical witness map $\sigma'_\alpha : W_\alpha \longrightarrow N'$.  Pick $\xi < \kappa$ such that $\langle \alpha , W_\alpha , r'_\alpha \rangle = h_{n+1}^{N'} ( \xi , p' )$; this is possible since $p'$ is a very good parameter for $N'$.  Pick further a strong divisor $( \bar{\kappa} , q )$ for $N$ such that $\xi < \bar{\kappa } < \kappa$.  Let $N'(\bar{\kappa}) = \mathcal{H}_{n+1}^N ( \bar{\kappa} \cup p(N) )$.  Let $\bar{\pi}' : N'(\bar{\kappa}) \longrightarrow N$ be the uncollapsing map, and $\bar{p}' = (\bar{\pi}')^{-1} (p(N))$.  Notice that $\sigma = (\pi')^{-1} \circ \bar{\pi}'$ is a $\Sigma_1^{(n)}$-preserving embedding of $N' (\bar{\kappa })$ into $N' (\kappa )$ which sends $\bar{p}'$ to $p'$.  By our choice of $\bar{\kappa}$, the range of $\sigma$ contains $\alpha$, $W$, and $r'_\alpha$.  Let $\phi (x_1 ... x_\ell )$ be a $\Sigma_1^{(n)}$-formula.  Since $W$ is the standard solidity witness for $\alpha$ with respect to $N'$ and $p(N')$, the premouse $N'$ satisfies the $\Pi_2^{(n)}$-statement

\[
( \forall \xi_1 ... \xi_\ell < \alpha ) (\phi ( \xi_1 ... \xi_\ell , r') \ \leftrightarrow \ W_\alpha \models \phi ( \xi_1 ... \xi_\ell , r'_\alpha )) \ .
\]

Since $\Pi_2^{(n)}$-statements are downward preserved under $\sigma$, the same is true of $\bar{\alpha}$, $\bar{r}'$, $\bar{Q}$ and $\bar{r}'_\alpha$ in $N'(\bar{\kappa})$, where $\bar{\alpha}$, $\bar{Q}$ and $\bar{r}'_\alpha$ are the $\sigma$-preimage of $\alpha$, $W$, and $p'_\alpha$, respectively.  Thus, if $A$ is any set of ordinals which is $\Sigma_1^{(n)}$-definable over $N'(\bar{\kappa})$ in parameter $\bar{r}'$ then $A \cap \bar{\kappa}$, being $\Sigma_1^{(n)}$-definable over $\bar{Q}$ in parameter $\bar{r}'_\alpha$, is an element of $N'(\bar{\kappa})$.  It follows that $\bar{r}' \neq p(N'(\bar{\kappa}))$.  By \ref{characterization of strongness}, $(\bar{\kappa} , q)$ is not strong after all, contradiction.\\


Now we show that $\mathcal{D}^*_q (N)$ is bounded in $\Lambda$.  Suppose it was unbounded; then by the above argument, $( \Lambda , q)$ would satisfy all the conditions of a strong short divisor of $N$, except for the clause in Definition \ref{short divisor} that $\kappa < \Lambda$.  In particular, letting $N^* = \mathcal{H}_{n+1}^N ( \Lambda \cup r)$ and $N' = \mathcal{H}_{n+1}^N ( \Lambda \cup p(N))$, we would have $\mathcal{P} (\Lambda ) \cap N^* = \mathcal{P} (\kappa ) \cap N'$.  But $N' = N$ by soundness of $N$, so $\mathcal{P} (\Lambda) \cap N = \mathcal{P} ( \Lambda) \cap N^*$.  In the remainder of this proof, we will reach a contradiction by showing that $N^*$ must be missing a subset of $\Lambda$ which is in $N$.\\

First we show that $q$ is nonempty.  Suppose $q$ were empty; then $Hull_{n+1}^N (\Lambda \cup r) = N$ by soundness, but this contradicts $c)$ in Definition \ref{short divisor}.  So let $x = max (q)$.  Then the $\Sigma_{n+1}^N$-theory of $N^* = \mathcal{H}_{n+1}^N (\Lambda \cup r)$ can be computed in $N$ from the solidity witness for $x$, since $N$ is solid or $Z_p$; and this theory will therefore be an element of $N$.  (If $N$ is $Z_p$ and $x = min (p (N))$, recall that $N$ still satisfies weak solidity at $x$.  Note that $x > \Lambda$, and taking any $\alpha$ with $\Lambda < \alpha < x$, we can still compute the $\Sigma_{n+1}^N$-theory of $N^* = \mathcal{H}_{n+1}^N (\Lambda \cup r)$ from the weak solidity witness for $\alpha$.)  But $N^*$ projects to $\Lambda$, so there is $A \subset \Lambda$ which is missing from $N^*$ but definable over $N^*$, and therefore an element of $N$.  This is the desired contradiction. $\blacksquare$\\


The reason Schimmerling and Zeman introduced strong short divisors in \cite{zeman square proof} is the following fact:\\

\begin{lem} \label{strong short divisor has no short overlaps}
\textbf{(From \cite{zeman square proof})}
Let $(\kappa , q)$ be a strong short divisor of $N$ and $\bar{q}$ be a proper bottom segment of $q$.  Then there is no $\bar{\kappa} < \kappa$ such that $(\bar{\kappa} , \bar{q})$ is a short divisor of $N$.
\end{lem}

\textbf{Proof:}  Suppose otherwise; let this be witnessed by $(\bar{\kappa} , \bar{q})$.  Let $\bar{r} = p (N) - \bar{q}$; clearly $\bar{r}$ is a proper lengthening of $r$ and $\alpha = max (q)$ is an element of $\bar{r} - \bar{q}$.  Now letting $\pi (\bar{\kappa} , \bar{q})$ be the uncollapse associated to the divisor-hull of $(\bar{\kappa} , \bar{q})$, observe that $h_{n + 1}^N (\bar{\kappa} \cup \bar{r}) = range (\pi (\bar{\kappa} , \bar{q}))$ contains a generalized witness $\langle Q , t \rangle$ for $\alpha$ with respect to $N$ and $p(N)$, so $\langle Q , t \rangle$ must be in the range of the larger embedding $\pi' : N' \longrightarrow N$, where $N' = \mathcal{H}_{n+1}^N ( \kappa \cup p(N))$.  Notice that $\langle Q , t \rangle$ is a generalized solidity witness for $\alpha$ with respect to $N$ and $r = p(N) - q$.  Then $\langle Q' , t' \rangle = ( \pi')^{-1} ( \langle Q , t \rangle )$ is a generalized witness for $( \pi')^{-1} (\alpha ) \geq \kappa$ with respect to $N'$ and $r' = ( \pi')^{-1} (r)$.  It follows that $p(N')$ must be a proper lengthening of $r'$, which contradicts strongness by Lemma \ref{characterization of strongness}. $\blacksquare$\\

We can now show that there is a particular strong short divisor of $N$ which has a strictly larger divisor-hull than any other strong short divisor of $N$.\\

\begin{defn} \label{canonical short divisor}
Let $N = N_\tau$ be the collapsing-level for $\tau$ in $W$, and suppose there is at least one strong short divisor of $N$.  Let $q$ be the smallest bottom segment of $p(N)$ such that $\mathcal{D}^*_q (N)$ is nonempty.  Let $\kappa$ be the maximal element of $\mathcal{D}^*_q (N)$ (there is a maximal element, by Lemma \ref{strong short divisors closed and bounded}).  Then we say $(\kappa, q)$ is the canonical strong short divisor of $N$.
\end{defn}

\textbf{Remark:}  We will discuss the significance of canonical strong short divisors for our $\square_\Lambda$ construction in \ref{canonical divisor subsection}.  For now we record an important feature of these divisors in the following lemma.\\

\begin{lem} \label{canonical short biggest hull}
If $(\kappa, q)$ is the canonical strong short divisor of $N$, then any other strong short divisor $(\kappa' , q')$ of $N$ must have $q \subseteq q'$ and $\kappa' \leq \kappa$; in other words, the divisor-hull corresponding to $(\kappa' , q')$ is a subset of the divisor-hull corresponding to $(\kappa , q)$.
\end{lem}

\textbf{Proof:} Clearly $q \subseteq q'$, because Definition \ref{canonical short divisor} took the smallest $q$ for which there are any strong short divisors of $N$.  Now if $q = q'$ then $\kappa' < \kappa$, again because Definition \ref{canonical short divisor} took the largest $\kappa$ for which $(\kappa , q)$ is a strong short divisor.  On the other hand, if $q \subset q'$ then we must have $\kappa' \leq \kappa$, because otherwise Lemma \ref{strong short divisor has no short overlaps} gives a contradiction. $\blacksquare$\\





\subsection{Short Protomouse Condensation}


The following lemma shows how short protomice arise from interpolations of short pluripotent levels.\\



\begin{lem} \label{short pluripotent interpolates to short protomouse}
If $N_\tau$ is short pluripotent and $M$ is an interpolant of $N_\tau$, then $M$ is a sound short protomouse.
\end{lem}

\textbf{Proof:} Let $X$ be the fully elementary countable hull from the interpolation procedure.  The ultrapower embedding $i_E: X \longrightarrow M$ is $\Sigma_0$-elementary and cofinal, which is enough to verify $a)$ and $c)$ of \ref{short protomouse}.  To see that the top extender $\tilde{G}$ of $M$ is not total, note that because $(N_\tau , G)$ was short pluripotent, $\kappa_G < \Lambda$.  So $(\kappa_G^+ )^{N_\tau} = (\kappa_G^+)^M$, because these models agree up to $\Lambda$ (indeed slightly beyond it).  However, in $N_\tau$ the amenable predicate coding $G$ is split into fragments which are cofinal in the height of $N_\tau$; specifically, letting $G_\xi$ be the fragment measuring $\mathcal{P} (\kappa_G) \cap (N_\tau | \xi)$ for $\xi < \kappa_G^+$, we have that $\{ G_\xi \ | \ \xi < \kappa_G^+ \}$ is cofinal in the height of $N_\tau$.  But $\tilde{G} = \bigcup_{\eta < \kappa_G^+} i_E ( \pi^{-1} (G_\eta ))$, and since $\pi$ is not cofinal in $o(N_\tau )$, letting $\alpha = sup(\pi `` (X))$ and $\theta = sup ( \{ \xi \ | \ G_\xi < \alpha \} )$, we have $\theta < \kappa_G^+$ and $\tilde{G} = \bigcup_{\eta < \theta} i_E ( \pi^{-1} (G_\eta ))$.  This shows that $\tilde{G}$ is not total on $M$, and verifies all conditions of short protomousehood except for $d)$ of \ref{short protomouse}; and this last clause follows from the fact that $\varrho_{1} (M) = \Lambda$, since $\Lambda$ is not the space of an extender in $N_\tau$ by PFS, and this fact is preserved by $\sigma$.  The fact that $M$ is sound also follows easily from the preservation properties of interpolation embeddings. $\blacksquare$\\

\begin{lem} \label{associated short protomouse interpolates to short protomouse}
If $N_\tau$ is the collapsing-level for $\tau$ in $W$ (following the conventions set at the beginning of Section \ref{interpolation subsection}), and $(\kappa , q)$ is a short divisor of $N_\tau$ with associated protomouse $M_\tau = N_\tau (\kappa , q )$, then any interpolant $M$ of $M_\tau$ is a short protomouse.
\end{lem}

\textbf{Proof:} The proof is similar to the one we just gave for \ref{short pluripotent interpolates to short protomouse}, but now the target structure is a short protomouse $M_\tau$ instead of a short pluripotent level of $W$, which in fact makes things easier.  As before, let $X$ be the fully elementary countable hull from the interpolation procedure.  The ultrapower embedding $i_E: X \longrightarrow M$ is $\Sigma_0$-elementary and cofinal, which is enough to verify $a)$ and $c)$ of \ref{short protomouse}.  $b)$ follows easily from the fact that $(\kappa_G^+)^M = (\kappa_G^+)^{M_\tau}$ and the fact that $b)$ holds in $M_\tau$.  The rest of the lemma follows exactly as in \ref{short pluripotent interpolates to short protomouse}.  (Note that $M_\tau$ is sound and solid, by \ref{short finestructure computation} and the fact that $M_\tau$ is the protomouse associated with a sound and solid level of $W$.) $\blacksquare$\\



\begin{lem} \label{short protomouse condensation}
\textbf{(Short Protomouse Condensation)}
Let $M_\tau = (| M_\tau | , G)$ be either a short pluripotent level of $W$, or the protomouse associated with a divisor $(\kappa , q)$ of $N_\tau$, where $N_\tau$ is the collapsing-level for $\tau$ in $W$.  Let $M = (|M | , \tilde{G})$ be an interpolant of $M_\tau $ such that the interpolation embedding has critical point $\bar{\tau} \in \mathcal{S}$ (see Section \ref{interpolation subsection}), and let $N$ be the associated ppm of $M$.  Then $N$ is a level of $W$.
\end{lem}

\textbf{Remark:} The cases listed above are \textit{not} exclusive-- that is, pluripotent levels can have divisors.  In our construction we will take interpolants of protomice associated with divisors (short or long) whenever there are such divisors; pluripotent levels can be thought of as a ``limiting case"  of divisors when there are no divisors proper, since the interpolants of a pluripotent level are similar to the interpolants of protomice.\\


\indent \indent \textbf{Proof of \ref{short protomouse condensation}:}  By either \ref{short pluripotent interpolates to short protomouse} or \ref{associated short protomouse interpolates to short protomouse}, we have that $M$ is a short protomouse with $\varrho_{1}(M) = \Lambda$.  Let $\sigma: M \longrightarrow M_\tau$ be the interpolation embedding.  Now we describe how $N$ can be embedded into an $L[E]$-level.  If $M_\tau$ is a protomouse then let $i_G : (N^*)^{M_\tau} \longrightarrow N_\tau$ be the associated ppm embedding.  Similarly, if $M_\tau$ is a short pluripotent level then $Ult_m (M_\tau | \kappa_G^+ , G) = ( M_\tau || o(M_\tau) )$, where $m$ is largest possible so that this ultrapower is defined; this is a basic property of coherent structures.  In this case we set $(N^*)^{M_\tau} = M_\tau | \kappa_G^+$, and again we have $i_G : (N^*)^{M_\tau} \longrightarrow N_\tau$.\\

There is a natural embedding $\pi: i_{\tilde{G}}(N^* ) \longrightarrow i_G ( N^* )$, given by 

\[
\pi ( i_{\tilde{G}} (f) (\alpha)) = i_G (f)(\sigma (\alpha )) \ .
\]

\smallskip

Thus we have the diagram


\[
\begin{tikzcd}
( M_\tau , G ) & \\
\bigtriangledown & \\
(N^*)^{M_\tau} \arrow[r, "i_G"] & Ult ((N^*)^{M_\tau} , G ) = N_\tau \\
& \bigtriangledown \\
( M , \tilde{G} ) \arrow[to=1-1, bend left=50, "\sigma"] & i_{G}(N^* ) \\
\bigtriangledown & \\
N^* \arrow[r, "i_{\tilde{G}}"] & N \arrow[to=5-2 , "\pi"]
\end{tikzcd}
\]
 
 \smallskip
 
Note that $i_{G}(N^* )$ is a level of $W$, and $\pi$ is $\Sigma_{n+1}$-elementary.  Also, $crit (\pi ) = crit (\sigma ) = \tilde{\tau }$. \\

We would like to see that $N$ is a plus-one premouse, by applying Lemma 1.10.\\

\textbf{Claim:} $N$ satisfies projectum-free spaces.\\


\textbf{Proof:}  For $k < n$, $i_{\tilde{G}} (\varrho_k (N^*)) = \varrho_k (N)$, and by elementarity we can conclude that none of these projecta cause a failure of PFS.  For $k = n$, if $i_{\tilde{G}}$ is continuous at $\varrho_n(N^*)$ the same argument applies, whereas if $i_{\tilde{G}}$ is discontinuous at $\varrho_n(N^*)$ then $\varrho_n (N^*)$ must be $\Sigma_n$-singular in $N^*$ and therefore a limit cardinal of $N^*$.  Then $\varrho_n (N) = sup ( i_{\tilde{G}} `` \varrho_n (N^*))$ is a limit cardinal of $N$ and therefore cannot cause a violation of PFS.  Finally, we have by the previous lemma that $\varrho_{n+1}(N) = \varrho_\omega (N) = \Lambda$.  But $\Lambda$ cannot be the space of an extender on the $N$-sequence, since $\pi ( \Lambda ) = \Lambda$ would then be the space of an extender on the $i_{\bar{G}} ( N^* )$-sequence, that is, on the $L[E]$-sequence; and $i_{\bar{G}} ( N^* ) \lhd N_\tau$, which projects to $\Lambda$, so we would then have a PFS violation in $N_\tau$.  Clause $b)$ of projectum-free spaces follows easily by elementarity of $i_{\tilde{G}}$: if $N$ has a short top extender $F$ such that $\kappa_F = \kappa_H$ for a total long extender $H$ on the $N$-sequence, and $\varrho_1 (N) \leq \kappa_F$, then $i_{\tilde{G}}^{-1} (\kappa_F )$ has these same properties in $N^*$, contradicting that $N^*$ is a premouse.  $\blacksquare$\\



 If $n \geq 1$ then $i_{\tilde{G}}$ is $Q^{(1)}$-preserving, a fortiori $\Sigma_2$-preserving, so we can apply clauses $a)$, $b)$, or $f)$ of Lemma 1.10, depending on the mouse-type of $N^*$, to conclude that $N$ is a plus-one premouse of the same type as $N^*$.\\

If $n = 0$ we would like to apply $c)$, $d)$, or $e)$ of Lemma 1.10.  If $N^*$ is passive this is trivial, so assume it is active with top extender $H^*$.  Also, if $N^*$ is type $Z_1$ with stretching-extender $F$, then $\kappa_{\tilde{G}} \neq (\kappa_F^+ )^{N^* }$, because $\kappa_{\tilde{G}}$ is a limit cardinal in $N^*$.  For the remainder of the proof we must consider two cases.\\

\textbf{Case 1:} $\kappa_{\tilde{G}} \neq \lambda_{H^*}$.  In this case $i_{\tilde{G}}$ is continuous at $\lambda_{H^*}$, since we are taking a coarse ultrapower and $\lambda_{H^*}$ is a regular cardinal in $N^*$, while $i_{\tilde{G}}$ is only discontinuous at points of $N^*$-cofinality $\kappa_{\tilde{G}}$.  Similarly, $i_{\tilde{G}}$ is continuous at $\nu_{N^*}$ in the $Z_1$ case, because $\nu_{N^*}$ has cofinality $(\kappa_F^+ )^{N^* }$, and $i_{\tilde{G}}$ is continuous there.  This verifies all the hypotheses of $c)$, $d)$, or $e)$ of Lemma 1.10.\\

\textbf{Case 2:} $\kappa_{\tilde{G}} = \lambda_{H^*}$.  In this case $i_{\tilde{G}}$ is discontinuous at $\lambda_{H^*}$ and we cannot apply Lemma 1.10.  It is still clear that $N$ is a potential premouse, by Claim 1.2.  Let $i_{\tilde{G}} (H^* ) = H$ be the top extender of $N$.  If $N^*$ is long and Dodd-solid then so is $N$, by an easy elementarity argument.  Similarly, if $N^*$ is long and type $Z_1$ then so is $N$, by elementarity combined with the fact that $i_{\tilde{G}}$ is continuous at $\nu_{N^* }$.  The only remaining clause we need to verify for $N$ to be a premouse is the Jensen initial segment condition (JISC).  The fact that $i_{\tilde{G}}$ is discontinuous at $\lambda_{H^*}$ disrupts our attempt to push the Jensen initial segments of $N^*$ upwards to $N$; in fact, $N$ might have a different $ABC$-mousetype than $N^*$.  We must instead look at an upward embedding $\pi': N \longrightarrow N'$, where $N'$ is an $L[E]$-level or Jensen initial segment thereof, and draw the JISC downward from there.  For this we will need to know that $\pi'$ is continuous at $\lambda_H$.\\

If $\pi : N \longrightarrow i_{G} (N^* )$ sends $\lambda_H$ cofinally to $\lambda ( i_{G} (N^* ) )$, set $N' = i_{G} (N^* )$, $H' =$ top extender of $i_{G} (N^* )$, and $\pi' = \pi$.  Otherwise, set $\tilde{H} = $ top extender of $i_{G} (N^* )$, and $\lambda' = sup (\pi`` (\lambda_H )) < \lambda_{\tilde{H}}$. (In this case $N'$, $H'$, and $\pi'$ are defined after the following Claim.)\\

\textbf{Claim:} $\lambda'$ is a cutpoint of $\tilde{H}$.\\

\textbf{Proof:}  For any $f: \kappa_{\tilde{H}}^{n} \longrightarrow \kappa_{\tilde{H}}$ with $f \in i_{G} (N^* )$, and any $a \in (\lambda' )^{< \omega }$ if $H$ is short, or $a \in (\lambda' \cup \nu_H )^{< \omega }$ if $H$ is long, we need to see that $i_{\tilde{H}} (f)(a)  < \lambda' $.  WLOG assume $f$ is monotonically increasing (on all $n$ of its coordinates).  Note that $f \in N$ as well, since $crit (\pi ) > \Lambda \geq dom (H)$.  And because $\lambda' = sup (\pi`` (\lambda_H ))$, there is $b \in range (\pi )$ such that $b(i) > a(i)$ for $i < n$.  Thus $i_{\tilde{H}} (f)(a) \leq i_{\tilde{H}} (f)(b) = \beta \in range (\pi)$.  And $\beta < \lambda'$ because it is in $range (\pi)$.  This proves that $\lambda'$ is a cutpoint. $\blacksquare$\\

Then set $H' = \tilde{H} \restriction \lambda'$ and $N' =$ the coherent structure associated with $H'$ (specifically, $N' = ( i_{G} (N^* ) || (\lambda')^+ , H' )$).  Letting $j : N' \longrightarrow i_{G} (N^* )$ be the natural factor-map corresponding to the whole initial segment, note that $range(\pi ) \subseteq range (j)$ (because the projectum and standard parameter of $N$ are $\leq \lambda_H$), so we can define $\pi' : N \longrightarrow N'$ by setting $\pi' = j^{-1} \circ \pi$.  In either case, we now have $\pi' : N \longrightarrow N'$ sending $\lambda_H$ cofinally into its image.\\

\textbf{Claim:} $N$ is not type $C$ (short or long).\\

\textbf{Proof:}  We know that $N = h_1^N (\Lambda \cup p (N))$, and that $\lambda_H > \Lambda$.  If any $\alpha > sup (\Lambda \cup p (N))$ was a cutpoint of $H$, we would have $h_1^N (\Lambda \cup p (N)) \subseteq \alpha$, contradiction. $\blacksquare$\\

Being a cutpoint is a $\Pi_1$-fact about an ordinal $\alpha$.  Furthermore, for any cutpoint $\alpha$ of $N'$, $(\pi' )^{-1} `` (\alpha )$ is a cutpoint of $N$; this is because for any $f: \kappa_H^{n} \longrightarrow \kappa_H$ with $f \in N$, and any $a \in ((\pi' )^{-1} `` (\alpha ))^{< \omega }$, we have $i_{H'} (f)(\pi(a)) = \beta < \alpha$ because $\alpha$ is a cutpoint.  By elementarity, $\beta \in range(\pi )$ and $i_H (f)(a) = \pi^{-1} (\beta ) < (\pi' )^{-1} `` (\alpha )$.\\

It follows from the above discussion, along with the $\lambda$-cofinality of $\pi'$, that $N$ has no cutpoints if and only if $N'$ has no cutpoints, and that $N$ has a largest cutpoint $\alpha$ if and only if $N'$ has a largest cutpoint.  In the latter case we need to see that $(H \restriction \alpha ) \in N$.  Note that $\pi' (\alpha ) = \alpha'$ is a cutpoint of $H'$, by elementarity.\\

\textbf{Claim:} $\pi' ((\alpha^+)^{Ult ( N , H \restriction \alpha )}) = (\alpha'^+ )^{Ult (N' , H' \restriction \alpha' )}$.\\

\textbf{Proof:} $\pi'$ is continuous at the point in question, because it has cofinality $= (\kappa_H^+ )^N = ( \kappa_{H'}^+ )^{N'} < crit (\pi' )$. $\blacksquare$\\

By the JISC for $N'$, we know that $H' \restriction \alpha'$ is indexed on the $N'$-sequence at $(\alpha'^+ )^{Ult (N' , H' \restriction \alpha' )}$.  But we have just seen that this is in the range of $\pi'$, so $H' \restriction \alpha'$ is in the range as well.  But then $\pi^{-1} ( H' \restriction \alpha' ) = H \restriction \alpha$ by elementarity (being an initial segment of the model's top extender is a $\Pi_1$-property).  This finishes the proof that $N$ satisfies the JISC, and therefore is a premouse.\\  

We are finally ready to apply Condensation to the $\pi' : N \longrightarrow N'$ embedding.  We know that $N'$ is an iterable premouse of the same type as $N$ ($N'$ is iterable because it is either an $L[E]$-level or embeddable into one), that $N$ is sound, and that $\varrho (N) < crit(\pi )$.  This allows us to apply the Condensation Lemma to $\pi : N \longrightarrow i_G (N^*)$.  Notice that Anomalous Case 4 does not apply, since if there were a total long extender $R$ on the $N_\tau$-sequence such that $\Lambda = (\kappa_R^+)^{N_\tau}$, then $N_\tau$ violates projectum-free spaces, contradiction.  (And clearly if there are no such extenders on the $N_\tau$ sequence, there are none on the $i_G (N^*)$-sequence.)\\

Also recall that $crit (\pi) = \bar{\tau} \in \mathcal{S}$, so $\bar{\tau}$ is not an index or pseudoindex.  It follows that $c)$ and $d)$ of the Condensation Lemma are impossible.  Because $\pi$ has a critical point $\bar{\tau}$, it is not the identity, so $a)$ of Condensation is impossible.  This leaves $b)$, and thereby proves the lemma.  $\blacksquare$\\








\section{Long Protomice}














We have a similar situation with long pluripotent levels $N_\tau$, in which their interpolants turn out to be \textit{long protomice}.\\

\subsection{Fine Structure for Long Protomice}


\begin{defn} \label{long protomouse}
A long protomouse $M = (| M |, \tilde{G})$ is a $J$-structure, considered in the language of coherent structures, such that\\

$a)$ $| M |$ is a passive premouse with $k( |M|) = 0$,\\

 $b)$ $\tilde{G}$ is a long extender over $|M|$ that is not total on $|M|$; more precisely, there is an ordinal $\theta$ with $\kappa_{\tilde{G}}^+ < \theta < \kappa_{\tilde{G}}^{++}$ such that $\tilde{G}$ measures exactly the subsets of $\kappa_{\tilde{G}}^+$ in $M | \theta$, and $\theta = (\kappa_{\tilde{G}}^{++})^{M| \theta}$;\\
 
 $c)$ $M$ satisfies the coherency condition $(Ult_n ( M || \theta , \tilde{G} ) ) | o(M) = |M|$,\\
 
 $d)$ $\kappa_{\tilde{G}}^+ < \varrho_1 (M)$,\\
 
 $e)$ $\varrho_1 (M)$ is not the space of an extender on the sequence of $Ult_n(N^* , \tilde{G})$, where $\langle N^* , n \rangle$ is the collapsing-level for $\theta$ in $M$;\\
 
 $f)$ $\tilde{G}$ has a largest generator $\nu = \nu^M$,\\
 
 $g)$ $\tilde{G} \restriction \lambda_{\tilde{G}}$ is on the $M$-sequence (short initial segment condition).

\end{defn}

\bigskip


\textbf{Remark:} Note that the short initial segment condition $g)$ demands that our long protomice still have the \textit{total} short part of $\tilde{G}$ on their sequence.  Also, it is a basic fact about plus-one premice that this short extender must be indexed at the first long generator of $\tilde{G}$.  Also note that $b)$ and $c)$ imply that $i_{\tilde{G}}``(\theta)$ will be cofinal in $o(M)$.\\

Again, we will use the language of coherent structures to deal with long protomice.  This means we will be working with their Dodd parameters $d(M)$, and $\varrho_1(M)$ in the above definition is the Dodd projectum. Also, as in the short case, we only need to consider $\Sigma_1$-definability for long protomice (in the language of coherent structures), and never need to talk about $\Sigma_1^{(n)}$-definability over long protomice for $n > 0$.\\

\textbf{Remark:} The Dodd parameter of a long protomouse always has largest element $\nu^M$, since this is the largest generator of $\tilde{G}$.  Since $\nu^M$ has cardinality $\lambda_{\tilde{G}}$, all other elements of the Dodd parameter will be $< \lambda_{\tilde{G}}$.\\



\begin{defn} \label{theta of long protomouse}
Given a long protomouse $M = ( |M| , \tilde{G} )$, let $\theta^M = dom(\tilde{G})$.  (Note $\kappa_{\tilde{G}}^+ < \theta^M < (  \kappa_{\tilde{G}}^{++} )^M$.)  Also, let $(N^* )^M$ be the collapsing-level for $\theta$ in $M$; so $(N^*)^M = \langle N^* , n \rangle$ where $n$ is such that $\kappa_{\tilde{G}^+} = \varrho_{n+1} ( N^* ) < \varrho_n (N^* )$.
\end{defn}

\begin{defn} \label{mu of long protomouse}
Given a long protomouse $M = ( |M| , \tilde{G} )$, let $\mu^M$ be the least long generator of $\tilde{G}$.
\end{defn}

\textbf{Remark:} Condition $g)$ of \ref{long protomouse} implies that the short extender $\tilde{G} \restriction \lambda_{\tilde{G}}$ is indexed at $\mu^M$ on the $M$-sequence.

\bigskip

One new problem which arises for long protomice is that $Ult_n (N^* , \tilde{G} )$ will not always be a potential premouse-- it may be a short protomouse.  This will happen exactly when $M$ is type $2$, as defined below:\\

\begin{defn} \label{long protomouse type}
We say a long protomouse $M = (|M| , \tilde{G})$ is type $2$ if $(N^* )^M = \langle  N^* , n \rangle$ is active with short top extender $D$, $n = 0$, and $\kappa_D = \kappa_{\tilde{G}}$.  Otherwise we say $M$ is type $1$.
\end{defn}

We will deal with long protomice of type $2$ in the next subsection.  For now we prove some general finestructural lemmas, which will be applicable in both the type $1$ and type $2$ cases.  In the type $1$ case the associated embedding will be elementary in the language of premice, while in the type $2$ case we will need to consider $N^*$ in the language of coherent structures, and will consider its Dodd parameter instead.\\

\begin{defn} \label{long associated ppm}
Given a long protomouse $M$ of type $1$ with top extender $\tilde{G}$ and $(N^*)^M = \langle N^* , n \rangle$, we define the associated ppm of $M$ to be $Ult_n (N^* , \tilde{G} )$, and we call $i_{\tilde{G}}$ the associated embedding.
\end{defn}

\begin{lem}
The associated ppm $N$ of a type $1$ long protomouse $M$ is a potential premouse.
\end{lem}

\textbf{Proof:}  By the remark following \ref{ppm up-pres}, it suffices to show that if $N$ has a top extender, it is total on $N$.  This is immediate if $n > 0$ or $N^*$ is passive; and if $n = 0$ and $N^*$ has top extender $E$ then we need to know that the associated ppm embedding $i_{\tilde{G}}$ is continuous at $dom(E)$.  But $i_{\tilde{G}}$ is only discontinuous at $\kappa_{\tilde{G}}$, which is a limit cardinal of $N^*$ and hence not the domain of $E$, and $\kappa_{\tilde{G}}^+$, which can only be the domain of $E$ if $E$ is short and $\kappa_E = \kappa_{\tilde{G}}$.  But this is exactly the situation where $M$ is of type $2$. $\blacksquare$\\




\textbf{Remark:} Fortunately, the technical case-splitting which we had to consider in the Remarks following \ref{short associated ppm} in the short protomouse case does not arise with long protomice.  Recall that this problem arose because the associated ppm embedding $i: N^* \longrightarrow N$ was discontinuous at $\lambda (N^*)$.  In the long protomouse case, however, if $N^*$ is active then $\lambda (N^*)$ is a limit cardinal in $N^*$ and also the largest cardinal in $N^*$, hence it must be $> \kappa_{\tilde{G}}^+$, and $i$ is continuous there.\\


We now prove analogues of the finestructure translation lemmas \ref{translation1}, \ref{translation2}, and \ref{translation3} from the short case.



\begin{lem} \label{long translation1}
\textbf{(Parameter-Less Simulation of Definable Singletons of $N$ from within Long Protomouse)}\\

Let $M = (|M| , \tilde{G})$ be a long protomouse.  Let $\kappa = \kappa_{\tilde{G}}$, $ \nu = \nu_{\tilde{G}}$, and $N^* = (N^*)^M$.  Let $i: N^* \longrightarrow  N$ be the associated embedding.  Let $\zeta < \nu + 1$, $b \subset (\nu + 1 )$ be finite, and $r =  i (p (N^* ))$, so $min(r) \geq \lambda^+ > \nu$ if $r$ is nonempty.  Then $\{ \zeta \}$ is $(\Sigma_1^{(n)})^N$-definable from $r$, $b$, and some finite $c \subset i`` (\kappa^+)$ if and only if there is an $f: \kappa^+ \longrightarrow \kappa^+$ in $N^*$ such that $\zeta = i (f)(b)$.\\


Here we consider all parameters and definability relations to be in the language of premice if $M$ is type $1$, and in the language of coherent structures if $M$ is type $2$.
\end{lem}

\textbf{Proof:} We begin with the forward implication.  Suppose $\zeta$ is the unique object such that $N \models (\exists z) \psi ( z , \zeta , r , c, b )$ where $\psi$ is a $\Sigma_0^{(n)}$-formula.  Fix a $\delta^* < \varrho_n^{N^*}$ large enough such that, setting $\delta = i(\delta^* )$, there is a $z \in J_\delta^E$ witnessing this existential statement; such a $\delta^*$ exists since $i$ is cofinal in $N^{(n)}$.  Let $c^* = i^{-1}(c)$.  Define a partial map $f: \kappa^+ \longrightarrow \kappa^+$ as follows:

\[
f(x) = \text{ the unique } \xi < \kappa^+ \text{ such that } N \models (\exists z \in J_{\delta^*}^E ) \psi (z , \xi , p (N^* ) , c^* ,  x ) \ .
\]

Then $f$, being a $(\bf{\Sigma}_0^{(n)})^N$ subset of $\kappa^+ < \varrho_n (N^* )$, is an element of $N^*$.  Applying $i$, it follows that $i (f)(x)$, if defined, is the unique ordinal $\xi < \lambda$ such that we have $(\exists z \in J_\delta^E) \psi (z, \xi , r , c, x  )$ in $N$.  But for $x = b$, we know that $i (f) (x)$ is defined, so $i (f)(b) = \zeta$.  Obviously, $f$ can be turned into a total function on $\kappa^+$ by setting $f(x) = 0$ whenever $f(x)$ is undefined by the above definition.\\

To see the converse implication, suppose $\zeta = i (f)(b)$ for $f$, $b$ as above; since $N^*$ is sound, there is a finite $c^* \subset \kappa^+$ such that $f = h_{n+1}^{N^*} ( c^* , p (N^*))$.  The preservation properties of $i$ then give $i(f) = h_{n+1}^N ( i( c^* ) , r)$, so $\zeta$ can be defined in a $\Sigma_1^{(n)}$-fashion over $N$ as follows:

\[
(\exists g ) ( g = h_{n+1}^N (i( c^* ) , r ) \ \& \ \zeta = g (b)) \ .
\]

$\blacksquare$\\



\begin{lem} \label{long translation2}
\textbf{(Simulation of Definable Classes of $N$ from within Long Protomouse, in Parameter $\theta$)}\\

Let $M = (|M| , \tilde{G})$ be a long protomouse.  Let $ \nu = \nu_{\tilde{G}}$ and $i: N^* \longrightarrow  N$ be the associated ppm embedding.  Let $r =  i (p (N^* ))$, so $min(r) \geq \lambda^+$ if $r$ is nonempty.  Then  for any $\Sigma_1^{(n)}$-formula $\phi ( v_0 ... v_\ell )$ in the language of premice, there is a $\Sigma_1$-formula $\phi^* ( v_0 ... v_\ell )$ in the language of coherent structures such that for every tuple $x_1 ... x_\ell \in M | (\nu + 1 )$, we have

\[
N \models \phi ( r , x_1 ... x_\ell ) \ \text{ if and only if } \ M \models \phi^* ( \theta , x_1 ... x_\ell ) \ .
\]

Here we consider all parameters and definability relations to be in the language of premice if $M$ is type $1$, and in the language of coherent structures if $M$ is type $2$.

\end{lem}


\indent \indent \textbf{Proof:}  We may assume $x_1 ... x_\ell$ are all ordinals $< (\nu + 1)$.  Suppose $\phi$ is of the form $(\exists z) \psi (z , vo ... v_\ell )$ where $\psi$ is $\Sigma_0^{(n)}$.  Then $N \models \phi (r, x_1 ... x_\ell )$ if and only if $(\exists u \in (N^*)^{(n)} ) [ N \models ( \exists z \in \pi (u) ) \psi ( z , r , x_1 ... x_\ell ) ]$.  Using {\L}o{\'s}'s Theorem, this can be expressed in a $\Sigma_1$-fashion over $M$ as

\[
( \exists Q , p^* , \kappa , a , y , u , m) \phi_0^* (Q , p^* , \kappa , a , y , u , m , \theta , x_1 ... x_\ell )
\]

where $\phi_0^* (Q , p^* , \kappa , a , y , u , m , \theta , x_1 ... x_\ell )$ is the conjunction of the following statements:\\

\begin{itemize}
\item{ $Q$ is an initial segment of $M$ and $m \in \omega$,}
\item{ $\theta = (\kappa^{++})^Q$, $\varrho_{m+1} (Q) = \kappa < \varrho_m (Q)$, $p^* = p (Q)$ and $u \in Q^{(n)}$,}
\item{ $a = \{ \langle \eta_1 ... \eta_\ell \rangle \in \kappa^+ \ | \ Q \models (\exists z \in u ) \psi (z , p^* , \eta_1 ... \eta_\ell ) \}$,}
\item{ $y = i_{\tilde{G}}(a) \restriction (\nu +1)$ and $\langle x_1 ... x_\ell \rangle \in y$.}
\end{itemize}


 
 This proves the lemma. $\blacksquare$\\







\begin{lem} \label{long translation3}
\textbf{(Simulation of Long Protomouse From its Associated Premouse)}

Let $M = (|M| , \tilde{G})$ be a long protomouse.  Let $\kappa = \kappa_{\tilde{G}}$, $\lambda = \lambda_{\tilde{G}}$, $ \nu = \nu_{\tilde{G}}$, $\mu = \mu_{\tilde{G}}$, and $i: N^* \longrightarrow  N$ be the associated ppm embedding.  Further, let $Z =$ the image of $\kappa^+$ under $i_{E_\mu}$; that is, since $\mu$ indexes the short part of $\tilde{G}$, $Z = i``(\kappa^+ )$.  Finally let $r =  i (p (N^* ))$, so $min(r) \geq \lambda^+$ if $r$ is nonempty.  Then if $\phi ( v_1 ... v_\ell )$ is a $\Sigma_1$-formula in the language of coherent structures, there is a $\Sigma_1^{(n)}$-formula $\psi ( v, v' , v^*, v_0 ... v_\ell )$ in the language of premice and an ordinal $\xi_0 \in Z$ such that for every $x_1 ... x_\ell \in M$, 

\[
M \models \phi ( x_1 ... x_\ell ) \ \textit{ if and only if } \ N \models \psi ( r , \xi_0 , \mu , \nu , x_1 ... x_\ell ) \ .
\]


Here we consider all parameters and definability relations to be in the language of premice if $M$ is type $1$, and in the language of coherent structures if $M$ is type $2$.

\end{lem}

\indent \indent \textbf{Proof:}  Since $\phi$ is a $\Sigma_1$-formula, $M \models \phi (x_1 ... x_\ell)$ if and only if there is an ordinal $\zeta < o(M) = \lambda^+$ such that $ \langle M || \zeta , G \cap (M || \zeta) \rangle \models \phi (x_1 ... x_\ell )$.  Fixing an ordinal $\xi_0 \in Z$ such that $h_{n+1}^N ( \xi_0 , r ) = \lambda^+$, this can be expressed over $N$ in the parameters $r , \mu , \nu , \xi_0$ as

\[
( \exists Z , \zeta , \lambda^+ , \beta_1 , \beta_2 , f , G , Q ) \ \psi ( Z , \zeta , \lambda^+ , \beta_1 , \beta_2 , f , G , Q , r , \xi_0 , \mu , \nu , x_1 ... x_\ell )
\]

where $\psi$ is the conjunction of the following statements:\\

\begin{itemize}

\item{$Z = i_{E_\mu}``( crit (E_\mu)^+ )$ ;}
\item{$\beta_1 , \beta_2 \in Z$ and $\zeta < \lambda^+$ ;}
\item{$\zeta = h_{n+1}^N ( \beta_1 , r )$ , $f = h_{n+1}^N ( \beta_2 , r )$ , and $\lambda^+ = h_{n+1}^N ( \xi_0 , r )$ ;}
\item{$f : \lambda^+ \xrightarrow{onto} \mathcal{P} (\lambda^+ ) \cap (N || \zeta)$ ; }
\item{$G = \{ \langle \text{transitive collapse of }(f ( \alpha ) \cap Z ) , f (\alpha ) \restriction (\nu + 1) \rangle | \alpha < \kappa^+ \}$ and $Q = \langle N || \zeta , G \rangle$ ;}
\item{$Q \models \phi (x_1 ... x_\ell )$ . }
 

\end{itemize}

In other words, $\psi$ asserts that there is a level $N || \zeta$ and a surjection $f : \lambda^+  \xrightarrow{onto} \mathcal{P} (\lambda^+ ) \cap (N || \zeta)$ which are both in the range of $i$ (so that their collapses in $N^*$ will be, respectively, a level of $N^*$ and a surjection from $\kappa^+$ onto that level's $\mathcal{P} (\kappa^+ )$).  From here we can describe a fragment of the protomouse-extender, here called $G$, as the set of ordered pairs of subsets of $\kappa^+$ in this level of $N^*$ and the subsets of $\lambda^+$ which they stretch to by $i$, chopped down to length $\nu + 1$.  (This is trivially equivalent to our official description of long extender predicates.)  The resulting structure $Q$ is a level of $M$.  There is an additional detail here which did not occur in the short extender case; namely, to recover a subset $X$ of $\kappa^+$ from the subset $Y$ of $\lambda^+$ which it stretches to, we must take the transitive collapse of $Y \cap Z$, where $Z$ is as defined at the beginning of the proof. $\blacksquare$\\


We can now relate the finestructural properties of $M$ and $N$, analogously to Lemma \ref{short finestructure computation} from the short case.  Recall that for a long protomouse $M$, $W_{\alpha , s}^M = \mathcal{H}_{1}^M (\alpha \cup \{ s \} )$; and for a premouse $\langle N , n \rangle $, $W_{\alpha , s}^N = \mathcal{H}_{n+1}^N ( \alpha \cup \{ s \} ) $.\\

\textbf{Remark:} Recall from the discussion at the beginning of \ref{interpolation subsection} that at certain points in our construction we will require an assumption that the largest generator $\nu$ of our long plus-one premice is a successor generator, and therefore that it indexes an initial segment of the top extender.  We will explicitly mention this assumption wherever it is used.\\

\begin{lem} \label{long finestructure computation}


Let $M = (|M| , \tilde{G})$ be a long protomouse.  Let $\kappa = \kappa_{\tilde{G}}$, $\lambda = \lambda_{\tilde{G}}$, $ \nu = \nu_{\tilde{G}}$, $\mu = \mu_{\tilde{G}}$, and let the associated ppm embedding $i: N^* \longrightarrow  N$ be an $n$-embedding (that is, $k(N^*) = n$).  Further, let $Z =$ the image of $\kappa^+$ under $i_{E_\mu}$; that is, since $\mu$ indexes the short part of $\tilde{G}$, $Z = i``(\kappa^+ )$.  Finally let $r =  i (p (N^* ))$, so $min(r) \geq \lambda^+$ if $r$ is nonempty, and set $\eta = o(M) = (\lambda^+)^N$.  Then\\

$a)$ $\varrho_1 (M) = \varrho_{n+1} (N)$.\\

Denote this common value by $\varrho$.  Granting that $\kappa^+ < \varrho$ and that $\nu$ is a successor generator of $\tilde{G}$, the following holds:\\

$b)$ $p_{n+1} (N) \cap \lambda^+ = d(M)$.\\

$c)$ $M$ is $1$-sound if and only if $N$ is $(n+1)$-sound.\\

$d)$ Let $s$ be a finite subset of $\nu + 1$ and $\alpha$ be an ordinal with $\theta \leq \alpha \leq (\nu + 1)$.  Then $W_{\alpha , s \cup r}^N = Ult_n (N^* , G )$, where $G$ is the top extender of $W_{\alpha , s}^M$ (so $G$ has the same ultrapower as $\tilde{G} \restriction ( \text{coordinates in } \alpha \cup s)$).  Moreover, the associated ultrapower embedding is precisely the uncollapsing map associated with the $\Sigma_1^{(n)} ( W_\alpha^N )$-hull of $Z \cup  \bar{r}$, where $\bar{r}$ is the preimage of $r$ under the canonical witness map.\\

$e)$ $M$ is $1$-solid if and only if $N$ is $(n+1 )$-solid.\\

Here we consider all parameters and definability relations to be in the language of premice if $M$ is type $1$, and in the language of coherent structures if $M$ is type $2$.

\end{lem}

\textbf{Proof:}  Note that $N$ is generated by $(\nu + 1) \cup i`` (N^*) = (\nu + 1) \cup r \cup Z$.  Because $r$ is finite and $Z \subset \mu$, it follows that $\varrho_{n+1} (N) \leq \mu$, and in fact $\varrho_{n+1} (N) \leq \lambda$, since $\mu$ has $N$-cardinality $\lambda$.  Also note $\varrho_1 (M) \leq \lambda$, because $\tilde{G}$ gives a $\Sigma_1^M$-surjection from $\lambda$ onto $M$.\\

First we show $a)$.  If $A$ is a $\Sigma_1 (M)$-relation in $p_1 (M)$ then by \ref{translation3} there is a $\Sigma_1^{(n)} (N)$-relation $A^*$ in $p_1 (M)$, $r$, $\mu$, $\nu$, and some $\xi_0 \in Z$ such that $A^*$ agrees with $A$ up to $(\nu + 1 ) > \varrho_1 (M)$.  Choose $A$ such that $A \cap \varrho_1 (M) \notin M$.  Then $A^* \cap \varrho_1 (M)$ is not an element of $N$; this follows from the fact that $\eta = o(M)$ is a cardinal in $N$.  Thus, $\varrho_{n+1}(N) \leq \varrho_1 (M)$.  The dual argument using \ref{translation2} yields the converse, which proves $a)$.\\



From now on suppose that $\kappa^+ < \varrho$ and that $\nu$ is a successor generator.  We now prove $b)$.  The ordinal $\varrho$, being $\leq \lambda$, is a cardinal in both $M$ and $N$.  It follows that $\theta < \varrho$.  Given $A$ as above, by \ref{translation3} there is a $\Sigma_1^{(n)}$-relation $A^*$ and some $\xi_0 \in Z$ such that $A(\xi) \leftrightarrow A^* ( d (M) , r , \mu, \nu, \xi_0 , \xi )$ whenever $\xi < \lambda$.  From $A^*$ we obtain a new subset of $\varrho$ which is $\Sigma_1^{(n)}(N)$ in $d (M) \cup  r \cup \{ \mu \} \cup \{ \nu \} \cup \{ \xi_0 \}$.  But note that $\nu \in d(M)$, and that $\mu$ is $\Sigma_1^{(n)}(N)$-definable from $\nu$ because of our assumption that $\nu$ is a successor generator and therefore indexes a total fragment of $\tilde{G}$.  We also have that $\xi_0 \in Z$ is $\Sigma_1^{(n)}(N)$-definable from $\mu \cup \varrho$, because all elements of $Z$ are definable from $\mu \cup \kappa^+$ (recall that $Z = i_{E_\mu}`` (\kappa^+)$) and we are supposing that $\kappa^+ < \varrho$.  It follows that the parameters $\mu$, $\nu$, and $\xi_0$ were not really needed in the above definition, so that in fact our new subset of $\varrho$ is $\Sigma_1^{(n)}(N)$ in $d (M) \cup  r $.  We have shown that $p_{n+1} (N) \leq_{lex} r \cup d (M)$ and, consequently, $p_{n+1} (N) \cap \lambda^+ \leq_{lex} d (M)$.  As before, the dual argument yields the converse, which proves $b)$.\\


If $M$ is $1$-sound, then every $\xi < (\nu + 1)$ is $\Sigma_1 (M)$-definable from $d(M) = p_{n+1} (N) \cap \lambda^+$ and a parameter less than $\varrho$.  Thus, every $\xi < (\nu + 1)$ is $\Sigma_1^{(n)}(N)$-definable from $r \cup (p_{n+1} (N) \cap \lambda^+) = p_{n+1}(N)$ and parameters less than $\varrho$. But recall that $N$ is generated by $(\nu + 1) \cup r \cup Z = (\nu + 1) \cup r$.  In other words, $N = Hull_{n+1}^N ( \varrho \cup \{ p_{n+1} (N) \} )$.  Thus, $p_{n+1} (N) \in R_{n+1}(N)$, so $N$ is $(n + 1)$-sound.  The converse follows again by the dual argument, which proves $c)$.\\



Next we prove $e)$ from $d)$.  To see that the $1$-solidity of $M$ implies the $(n+1)$-solidity of $N$, notice that $W_{\alpha}^M = \mathcal{H}_1^M (\alpha \cup \{ d(M) \} )$ can be encoded into a $\bf{\Sigma_1} (M)$ subset $A$ of $\alpha$.  Such an $A$ is in $J_{\eta}^E$ by acceptability, and $W_{\alpha}^M$ can be reconstructed from $A$ inside $J_{\eta}^E$.  But then also $W_{\alpha}^N$ is in $J_{\eta}^E$ by $d)$.  For the converse use again the dual argument.\\



Finally we show $d)$.  Let $\bar{\sigma}: W_{\alpha , s}^M \longrightarrow M$ be the canonical witness map, $\bar{\nu} = \nu_G$ and $\bar{\eta} = o ( W_{\alpha , s}^M)$.  Since $\bar{\sigma}$ is $\Sigma_1$-preserving, $dom( G) = \mathcal{P} ( \kappa^+ ) \cap (M | \theta)$, so $G$ can be applied to $N^*$ (notice that $N^*$ is an initial segment of $W_{\alpha , s}^M$).  Let $W = Ult_n (N^* , G )$ and $\bar{\pi} : N^* \longrightarrow W$ the associated ultrapower map.  Clearly $\bar{\pi}$ is $\Sigma_0^{(n)}$-preserving and cofinal, and $W = Hull_{n+1}^W ( (\bar{\nu} + 1 ) \cup \{ \bar{r} \} )$, where $\bar{r} = \bar{\sigma} ( p (N^*))$.  We can also define a natural factor-map $\sigma : W \longrightarrow N$ by $\sigma ( \bar{\pi} (f)(a)) = i (f) ( \bar{\sigma} (a))$ for $a \in [ \bar{\nu} + 1 ]^{< \omega}$.  It is easy to see that $\sigma$ is $\Sigma_0^{(n)}$-preserving and cofinal, by {\L}o{\'s}'s Theorem for $\Sigma_0^{(n)}$-formulas together with the cofinality of $\bar{\sigma}$ and $i$.  Also, $i = \sigma \circ \bar{\pi}$ and $crit (\sigma ) \geq \bar{\eta}$.  It follows that $\sigma ( \bar{r}) = r$, $\sigma \restriction \alpha = id$, and, letting $\bar{s}$ be the $\bar{\sigma}$-preimage of $s - ( \alpha + 1)$, also $\sigma ( \bar{s}) = \bar{\sigma} ( \bar{s}) = s - (\alpha + 1 )$.  As each $\zeta < (\bar{\nu} + 1)$ is $\Sigma_1 (W_{\alpha , s }^M )$-definable from $\bar{s}$ and parameters below $\alpha$, \ref{translation3} allows us to conclude that $W = Hull_{n+1}^W ( \alpha \cup \{ \bar{r} \cup \bar{s} \} )$, so $Hull_{n+1}^N ( \alpha \cup \{ r \cup s - ( \alpha + 1 ) \} ) = range( \sigma)$.  But this means that $W = W_{\alpha , r \cup s }^N$ and $\sigma$ is the associated witness map.\\


There is a difficulty which might arise in the above argument: if $N^*$ is a type $Z_p$ premouse and $k(N^*) = 0$, then $i_{\tilde{G}}$ will be discontinuous at $\kappa^+$, and $Ult_0 (N^* , \tilde{G})$ may not be weakly solid.  We can solve this problem in the same way we dealt with Anomalous Case $2$ in the Condensation Lemma.  In particular, an argument similar to that given in Section \ref{AC2 subsection} shows that $\tilde{G}$ has only a single long generator $\nu_{\tilde{G}}$.  Now instead of applying $\tilde{G}$ to $N^*$ we will apply it to the generalized core of $N^*$.  More precisely, letting $H$ be the generalized core of $N^*$ as described in Definition \ref{solidity theorem}, we form $N = Ult_0 ( H, \tilde{G})$.  Again following the argument of Section \ref{AC2 subsection}, we conclude that $N$ is a type $Z_p$ premouse with stretching-extender $\tilde{G} \restriction \lambda_{\tilde{G}}$.  Thus, even in this case we have the requisite solidity for $N$.  We omit further detail.\\



$\blacksquare$\\ 









\subsection{Long Protomice of Type $2$}








The procedure for dealing with long protomice of type $2$ is more complex than that for type $1$.  The problem is that $P = Ult(N^* , \tilde{G})$ is not a premouse in this case, because it has non-total top extender $F$.  Thus we must take a \textit{second} ultrapower of the longest measured initial segment by this partial top extender; and this time we can show that the ultrapower is a premouse.\\

Recall that when $M$ is a 
 of type $2$, it has a long top extender $\tilde{G}$ with domain $\theta$, where $\theta$ is a local $\kappa^{++}$, and the collapsing-level for $\theta$ is $N^*$, where $N^*$ has short top extender $D$ with $\kappa_D = \kappa_{\tilde{G}} = \kappa$, and $k(N^*) = 0$.\\



\begin{lem} \label{lpt2 first ultrapower}
If $M = (|M| , \tilde{G})$ is a long protomouse of type $2$, then $P = Ult_0 (N^* , \tilde{G} )$ satisfies all the conditions for short protomousehood in Definition \ref{short protomouse}, except possibly $d)$.
\end{lem}

\textbf{Proof:}  Since $n = 0$ in this case, clearly $k(P) = 0$.  Let $i = i_{\tilde{G}}$ be the ultrapower map; then the top extender $F$ of $P$ is given by $F = i``(D)$, where $i``(D)$ is the pointwise image of the fragments of $D$.  Since $dom ( D) = \kappa^+$, we have $dom(F) = sup ( i `` ( \kappa^+)) = \mu_{\tilde{G}}$, where $\mu_{\tilde{G}}$ is the least long generator of $\tilde{G}$.  But $i$ is discontinuous at $\kappa^+ $, so $\mu_{\tilde{G}} < i(\kappa^+ ) = (\kappa_F^+)^P $.  Hence $F$ is not total on $P$.\\

The remaining conditions of short protomousehood, other than $d)$, follow easily by $\Sigma_1$-elementarity of $i$. $\blacksquare$\\

\begin{defn} \label{associated short quasi-protomouse}
For $M$ a long protomouse of type $2$, the associated short quasi-protomouse of $M$ is $Ult_0 (N^* , \tilde{G} )$.
\end{defn}

\begin{lem}
Let $M = (|M|, \tilde{G} )$ be a long protomouse of type $2$ with associated short quasi-protomouse $P = ( |P|, F )$.  Let $\kappa = \kappa_{\tilde{G}}$ and $\lambda = \lambda_{\tilde{G}}$.  Then $P | ( \lambda^+ )^P = |M|$, $dom(F) = \mu_{\tilde{G}}$, and $E_{\mu_{\tilde{G}}}^P = \tilde{G} \restriction \lambda$.  Thus the collapsing-level for $\mu$ in $P$ is $P | \mu$ itself, and $\varrho_1 (P | \mu) = \lambda$.
\end{lem}

\textbf{Proof:}  The fact that $P | ( \lambda^+ )^P = |M|$ is an immediate consequence of the coherency condition for $M$, that is, $c)$ of \ref{long protomouse}.  We saw in the proof of \ref{lpt2 first ultrapower} that $dom(F) = \mu_{\tilde{G}}$.  The fact that $E_{\mu_{\tilde{G}}}^P = \tilde{G} \restriction \lambda$, and therefore that $P | \mu$ collapses its own height in a $\Sigma_1$-way, follows from the Remark after \ref{long protomouse}. $\blacksquare$\\

We now apply $F$ to the longest initial segment of $P$ which it measures; we have just seen that this initial segment is $Q = P | \mu = (P || \mu , \tilde{G} \restriction \lambda )$, and that $k(Q) = 0$.\\

\begin{defn} \label{associated ppm type 2}

Let $M = (|M|, \tilde{G} )$ be a long protomouse of type $2$ with associated short quasi-protomouse $P = ( |P|, F )$, and let $Q = P | \mu$ be the collapsing-level in $P$ for $dom (F)$.  Then the associated ppm of $M$ is $Ult_0 ((P | \mu ) , F)$.
\end{defn}

\begin{lem} \label{double unmorph is a ppm}
Let $M = (|M|, \tilde{G} )$ be a long protomouse of type $2$ and $N$ its associated ppm.  Then $N$ is a potential premouse.
\end{lem}

\textbf{Proof:}  We apply \ref{ppm up-pres} to the embedding $i_F : Q \longrightarrow N$.  Note that $N$ has a total top extender $H$, since $Q$ has top extender $\tilde{G} \restriction \lambda$ with domain $= \kappa^+ < crit (F) = \lambda$.  So $H$ has domain $(\kappa^+)^Q = (\kappa^+ )^N$.  The rest follows immediately. $\blacksquare$\\

Note that $ i_F : Q \longrightarrow N$ is discontinuous at $\lambda = \lambda_{\tilde{G}}$, so we cannot apply Lemma $\ref{upward mouse-pres}$ to see that $N$ is a premouse.  We will have to verify the ISC for $N$, and thereby show that it is a premouse, by considering its upward embedding into a level of $W$.\\

\begin{lem} \label{type 2 simulation of F}
Let $M = (|M|, \tilde{G} )$ be a long protomouse of type $2$, $P = ( |P|, F )$ its associated short quasi-protomouse, and $N = ( |N| , H )$ its associated ppm.  Then $|P| = |N|$, and $F$ is $\Sigma_1$-definable over $N$ in the parameter $\mu$, where $\mu$ is the least long generator of $\tilde{G}$.
\end{lem}

\textbf{Proof:}  The fact that $|P| = |N|$ is an easy consequence of the coherency of $F$, which was proved in \ref{lpt2 first ultrapower}, together with the fact that $o(Q) = dom(F) = \mu$.  To see that $F$ is $\Sigma_1$-definable over $N$ from $\mu$, we observe that $H$ is the stretch of $\tilde{G} \restriction \lambda$ by $F$, and that $\tilde{G} \restriction \lambda$ is still indexed at $\mu$ on the $N$-sequence, by coherency.  Therefore $N$ can define $F$ by factoring its top extender $H$ into the part up to $\lambda$, which corresponds to $\tilde{G} \restriction \lambda$, and the part up to $\lambda_H$, which corresponds to $F$.  More formally, we define the relation $R ( X, Y )$ over $N$, in parameter $\mu$, as follows (recall that $E_\mu = \tilde{G} \restriction \lambda$):

\[
N \models R ( X , Y ) \ \text{ iff } \ ( \exists \xi < \kappa_H^+ ) \  [ ( i_{E_\mu} (N | \xi ) = X) \ \& \ i_H ((N | \xi) = Y ) ] \ .
\]

Then $N \models R (X , Y)$ if and only if $X \in i_{E_\mu}`` ( \{ (N | \xi ) \ | \ \xi < \kappa_H^+ \} )$ and $i_F (X) = Y $.  Since $i_{E_\mu}`` (\{ (N | \xi ) \ | \ \xi < \kappa_H^+ \} )$ is a cofinal subset of $dom (F)$, the relation $R$ is easily seen to be equivalent to the predicate $F$. $\blacksquare$\\



We also need to track the parameters and projecta of the models under consideration.\\

\begin{lem}
Let $M = (|M|, \tilde{G} )$ be a long protomouse of type $2$, with $\kappa = \kappa_{\tilde{G}}$ and $\lambda = \lambda_{\tilde{G}}$, and let $N = ( |N| , H )$ be its associated ppm.  Then $\varrho_1 (N) = \varrho_1 (M)$, and $p(N) =  d(M) \cup r$, where, letting $i: N^* \longrightarrow P$ be the associated embedding, $r = i (d(N^*))$.
\end{lem}

\textbf{Remark:}  In this case $H$ is a type $C$ premouse, so $d (N) = p(N)$.\\

\textbf{Proof:} Recall that $\varrho_1 (P) = \varrho_1 (M)$ and $d(P) =  d(M) \cup r$, by Lemma \ref{long finestructure computation}.  Letting  $Q = P | \mu$ be the collapsing-level in $P$ for $dom (F)$, recall that $\varrho_1 (Q) = \lambda$ (in fact $\varrho_\omega (Q) = \lambda$, because $\lambda$ is a cardinal in $P$).  It follows that $p (Q)$ is empty, because $\lambda = \lambda_{\tilde{G} \restriction \lambda}$, and $Q$ has top extender $\tilde{G} \restriction \lambda$.\\

The same argument that we used in Lemma \ref{long finestructure computation} for clause $d)$ implies that $N$ has solidity witnesses for all elements of $d(P)$, which implies $d(P) \leq_{\text{lex}} p(N)$.  Also, the canonical missing subset of $P$, $Th_1^P (\varrho_1(P) \cup d(P))$, is $\Sigma_1$-definable over $N$, because of our previous lemma that $F$ is $\Sigma_1$-definable over $N$, and $|P| = |N|$.  Finally note that $\varrho_1 (N) \geq \varrho_1(P)$, because any definition of a subset of some $\alpha < \varrho_1 (P)$ in $N$ can easily be pulled back to a definition of that same subset of $\alpha$ over $P$ (recall here that $N = Ult_0 ((P | \mu ) , F )$, and the critical point of this ultrapower map $= \lambda > \varrho_1 (P)$), and $\mathcal{P}(\alpha)^P = \mathcal{P}(\alpha)^N$.  This implies $d(P) \geq_{\text{lex}} p(N)$.\\

We have shown that $\varrho_1 (N) = \varrho_1 (M)$ and $p(N) =  d(M) \cup r$, as desired. $\blacksquare$\\









\begin{lem} \label{type 2 recovering $N^*$}
Let $M = (|M|, \tilde{G} )$ be a long protomouse of type $2$, $P = ( |P|, F )$ its associated short quasi-protomouse, and $N = ( |N| , H )$ its associated ppm.  Then $N^* = Hull_1^{(|P|, F)} (Z \cup r )$, where $Z = i_{E_\mu} `` (\kappa_H^+ )$ and $r = p(N) - (\lambda_\mu^+ )$.
\end{lem}

\textbf{Proof:} We know that $N^* = Hull_1^P (Z \cup i_{\tilde{G}} (d(N^*)))$, and that $i_{\tilde{G}} (d(N^*)) = p(N) - (\lambda_\mu^+ ) = r$.  The lemma follows immediately. $\blacksquare$\\

Essentially, what the above lemmata allow us to do is recover the original type $2$ protomouse $M$ from its associated ppm $N$, via a similar process to the type $1$ case.  Within $N$, we first define the predicate $F$, which is the top extender of $P = Ult_0 (N^* , \tilde{G})$.  Then $N^*$ can be recovered as a $\Sigma_1$-hull in this language, and $\tilde{G}$ recovered as the uncollapse map; thus we can get back to the protomouse $M = (|M|, \tilde{G} )= ( N || i_{\tilde{G}}(\kappa^+) , \tilde{G})$.  This will be the key to our description of type $2$ long divisors in a later subsection.  First, though, we turn to the more basic description of long divisors which yield type $1$ long protomice.\\









\subsection{Long Divisors}










\textbf{Remark:}  The following definition is crucial to our $\square_\Lambda$ construction, and it is only appropriate under the ``smallness assumption" described at the beginning of \ref{interpolation subsection}.  For the fully general $\square_\Lambda$ construction, some different notion of long divisor is presumably needed as well.\\

\begin{defn} \label{long divisor}
Let $N = N_{\bar{\tau}}$ be the collapsing-level for $\bar{\tau}$ in $W$, with $\varrho_{n+1}(N) = \Lambda < \varrho_n (N)$ and $\bar{\tau} = (\Lambda^+ )^N$.  Then an ordinal $\nu \in p(N)$ is a \textit{long divisor} of $N$ if the following conditions hold:\\

\indent \indent $a)$ There is an extender $E_\nu$ on the $N$-sequence such that, letting $\kappa = \kappa_{E_\nu}$ and $\lambda = \lambda_{E_\nu}$, we have $\kappa^+ < \Lambda < \lambda$, and $\lambda^+ = (\lambda^+)^N < \varrho_n (N)$,\\

Letting $r = p (N) - (\nu + 1)$, $E_\mu = E_\nu \restriction \lambda$ be the short part of $E_\nu$ (indexed at $\mu$ on the $N$-sequence), and $Z = i_{E_\nu}``(\kappa^+) =  i_{E_\mu}``(\kappa^+)$, we have:\\



\indent \indent $b)$ $Hull_{n+1}^N ( Z \cup r ) \cap \varrho_n (N)$ is cofinal in $\varrho_n^N$,\\

\indent \indent $c)$ $Hull_{n+1}^N ( Z \cup r ) \cap \lambda^+ = Z$,\\

\indent \indent $d)$ $\lambda^+$ is not the space of an extender on the $N$-sequence,\\


\end{defn}


We refer to $\mathcal{H}_{n+1}^N ( Z \cup r )$ as the \textit{divisor-hull} associated with a long divisor $\nu$.\\

Our long divisors are in fact quite similar to the short divisors considered in \cite{zeman square proof}, except that they are of course designed to correspond to long protomice.  The key difference between long protomice and short protomice is that long protomice have a collapsing-level $N^*$ for $\theta$ which projects to $\kappa_{\tilde{G}}^+$, and their $\theta$ is a local $\kappa^{++}$ instead of a local $\kappa^+$.  Thus when we form the associated ppm $N = Ult_n (N^* , \tilde{G})$ of a long protomouse, with embedding $i$, and we want to recover $N^*$ as a hull in $N$, our job is more complex than in the short protomouse case.  We need to take the $\Sigma_1^{(n)}$-hull of $i``(\kappa^+ ) \cup i(p(N^*))$, and it is quite difficult to make ``guesses" at $i``(\kappa^+ )$ from the perspective of $N$.  However, the task becomes much more manageable if we assume that $\tilde{G}$ had a largest generator $\nu_{\tilde{G}}$ which was a successor generator; for then there must be a ``total fragment" of $\tilde{G}$ indexed at $\nu$ (that is, a total long extender which agrees with $\tilde{G} \restriction \nu$ on their common domain), and from this we can easily identify $i``(\kappa^+ )$, since $\tilde{G}$ agrees with this fragment on $\kappa^+$.  In addition, $\nu_{\tilde{G}}$ is the largest element of the Dodd parameter of our long protomouse $M$, so $\nu_{\tilde{G}}$ will be an element of the standard parameter of $N$.  Thus our task reduces to the problem of guessing an element of the standard parameter to serve as our candidate for $\nu_{\tilde{G}}$; we have seen that it determines $i``(\kappa^+ )$, and it also determines $i(p(N^*))$ as the top part of $p(N)$, that is, $p(N) - (\nu+1 )$.  Our candidate long divisors, then, are the elements of the standard parameter of $N$ such that we can treat them as candidates for $\nu_{\tilde{G}}$ in the way just described.  In the above definition, our guess at $i``(\kappa^+ )$ is called $Z$, and our guess at $i(p(N^*))$ is called $r$.\\

\textbf{Remark:} Note that the uncollapse map $\pi$ from a divisor-hull $N^* = \mathcal{H}_{n+1}^N ( Z \cup r )$ into $N$ is $\Sigma_0^{(n)}$-elementary and cofinal.  Also note that $N^* = \mathcal{H}_{n+1}^{N^*} (\kappa^+ \cup \pi^{-1}(r))$, so $\varrho_{n+1}(N^*) \leq \kappa^+$.\\

\textbf{Remark:} One could define a notion of ``strong long divisors", by analogy with strong short divisors.  In the present work, however, we have not found any need for this concept, because as we will see, long divisors cannot overlap in the way that short divisors could (recall that the point of the `strongness' condition for short divisors was to rule out overlaps).\\


\begin{lem}
If $N$ is a level of $W$ and $\nu$ is a long divisor of $N$ with $\kappa = \kappa_{E_\nu}$, then $(\kappa^+)^N$ is a cardinal of $W$.
\end{lem}

\textbf{Proof:} Recall that $\Lambda$ is a cardinal of $W$, and by $a)$ of Definition \ref{long divisor}, $(\kappa^+)^N < \Lambda$.  The conclusion is immediate. $\blacksquare$\\


\begin{lem} \label{long divisor hull condensation}
If $N$ is a level of $W$ and $\nu$ is a long divisor of $N$ with $N^*$ the transitive collapse of the divisor-hull associated with $\nu$, then $N^*$ is a proper initial segment of $N$, and $p (N^*) = \pi^{-1} (r)$, where $r = p(N) - q$ and $\pi$ is the associated uncollapse map.
\end{lem}

\textbf{Proof:}  Let $\pi: N^* \longrightarrow N$ be the uncollapse map associated with the divisor-hull.  Thus $crit(\pi) = \kappa_{E_\nu} = \kappa$, $\pi ( \kappa ) =\lambda_{E_\nu} = \lambda$, $\pi``(\kappa^+) = Z$, and $\pi ( \kappa^+ ) = \lambda^+$.  
Set $s = \pi^{-1} (r)$ (recall $r$ is a top segment of $p (N)$, namely, the part above $\nu$).  Then $N^* = Hull_{n+1}^{N^*} (\kappa^+ \cup s)$; combining this with the fact that $\kappa^+$ is a $W$-cardinal, we conclude that

\[
\varrho_\omega (N^*) = \varrho_{n + 1} (N^*) = \kappa^+ < \varrho_n (N^*) \ ,
\]

and also that $s$ is a very good parameter for $N^*$.\\

 We would like to see that $N^*$ is a premouse of the same type as $N$ by applying Lemma \ref{downward mouse-pres}.  First we show that $N^*$ satisfies projectum-free spaces.\\
 

 For $m > n$, $\varrho_m (N^*) = \kappa^+$ is not a space, since clause $d)$ of Definition \ref{long divisor} guarantees that $\lambda^+$ is not a space in $N$, and this fact transfers down to $N^*$.  For $m \leq n$, we know that $\pi$ is an $n$-embedding, so we can follow the argument given in Lemma 2.7 to show that $\varrho_m$ does not cause a PFS violation.\\

Now if $N$ is type $A$ or $B$ (short or long varieties), then clauses $a)$ or $c)$ of Lemma \ref{downward mouse-pres} can be applied immediately.  If $N$ is type $C$ (short or long), we need to verify that $n > 0$.  But if $n = 0$, then $\Lambda = \varrho_1 (N)$, and $\varrho_1$ of any type $C$ premouse is always the largest cardinal of that premouse (by \ref{projecttolambda}).  This contradicts the fact that $\Lambda < \lambda$ in clause $a)$ of Definition \ref{long divisor}, since $\lambda$ is a cardinal of $N$.  Thus $N^*$ is a premouse of the same type as $N$.\\

\textbf{Claim:} $N^*$ is sound.\\


\textbf{Proof:} If $(\kappa^{++})^{N^*}$ does not exist, then $N^*$ is easily seen to be sound: if $s$ is empty then $N^*$ is trivially sound, because in this case $N^* = Hull_{n+1}^{N^*} (\kappa^+ ) $ and we know that $\kappa^+ = \varrho_{n+1} (N^* )$.  If $s$ is nonempty and $t < s$ is the standard parameter for $N^*$ then by universality $t$ generates $s$, since they are both below $\kappa^{++}$.  But then $t$ is a very good parameter, so $N^*$ is sound.\\

Now assume $(\kappa^{++} )^{N^*}$ exists.  Recall that $\pi$ sends $\kappa^+$ to $\lambda^+$, and that $N = Hull_{n+1}^N (\lambda^+ \cup r) = Hull_{n+1}^N (\lambda^+ \cup \pi `` (N^*))$.  Letting $G$ be the extender of length $\lambda^+$ derived from $\pi$, it follows that $\pi = i_G$.  (That is, the long extender of length $\lambda^+$ derived from $\pi$ is sufficient to capture the full embedding.)  This implies that $\pi$ is only discontinuous at points of cofinality $\kappa$ and $\kappa^+$ in $(N^*)^{(n)}$, since this is the space of $G$.  In particular, it follows that $\pi$ is continuous at $\theta = (\kappa^{++})^{N^*}$, since $\theta$ is $\Sigma_1^{(n)}$-regular in $N^*$.\\
 
 We want to see that $s$ is the standard parameter of $N^*$.  Suppose $t <_{lex} s$ were the standard parameter; by the Solidity theorem of \cite{FSPIPM}, we have that $t$ is universal.  That is, $N^* | \theta \ \subseteq \ Hull_{n+1}^{N^*} (\kappa^+ \cup t )$.  But then $Hull_{n+1}^N ( Z \cup \pi (t))$ is cofinal in $\lambda^{++}$ (using the fact that $\pi$ is continuous at $\theta$).  However, $\pi (t) <_{lex} r$, so $Th_{n+1}^N (\pi (t) \cup \lambda^+ ) \in N$, by solidity of $r$.  This theory codes a collapse of $(\lambda^{++})^N$, which is a contradiction. $\blacksquare$\\

 
 
We have shown that $\pi : N^* \longrightarrow N$ is a $\Sigma_0^{(n)}$-elementary and cofinal embedding, and that $N^*$ is a sound premouse of the same type as $N$.  However, we cannot immediately apply the condensation lemma, because $crit (\pi) = \kappa$, but $N^*$ projects to $\kappa^+$.  Instead, we must look at the model 

\[
Q = \mathcal{H}_{n+1}^{N} ( \mu \cup \{ r \} ) \ .
\]

\textbf{Claim:} $Q$ is a sound premouse of the same type as $N$.\\

\textbf{Proof:}  We have already seen that $N^*$ is a sound premouse of the same type as $N$.  But notice that $Q = Ult_n (N^* , F)$, where $F$ is the extender of length $\mu$ derived from $\pi$.  (This follows immediately from the definition of $Q$ as the collapse of a certain hull in $N$.)  Further, notice that $F$ has no long generators, because its length $\mu$ is equal to $sup ( \pi`` (\kappa^+))$.  In other words, $F$ is a short extender, so it is only discontinuous at points of cofinality $= \kappa$ in $N^*$.  This means we can apply Lemma \ref{upward mouse-pres} to $i_F : N^* \longrightarrow Q$ to conclude that $Q$ is a plus-one premouse of the same type as $N^*$.  Let us verify that the hypotheses of the lemma are met.  Since $\kappa$ is the only discontinuity of $i_F$, and $\kappa$ is not the largest cardinal of $N^*$ (recall $\kappa^+ \in N^*$), certainly $i_F$ is continuous at $\lambda_{N^*}$ if $N^*$ is active.  Also, if $\nu_{N^*}$ is defined then its cofinality is a successor cardinal of $N^*$; since $\kappa$ is a limit cardinal, $i_F$ is continuous here too.  Finally we need to check projectum-free spaces for $Q$, and for this we will first need to prove that

\[
\varrho_{n+1} (Q) = \mu \ .
\]

The fact that $\varrho_{n+1} (Q) \leq \mu$ follows immediately from the definition of $Q$ as a hull of $\mu \cup \{ r \}$.  To see that $\varrho_{n+1} (Q) \geq \mu$, recall that for any $\alpha < \kappa^+$, $Th^{N^*}_{n+1} (\alpha \cup r ) \in N^*$, because $\kappa^+ = \varrho_{n+1} (N^*)$.  Then $Th^{Q}_{n+1} (i_F (\alpha ) \cup i_F(r) ) \in Q$, because it can easily be recovered from $i_F ( Th^{N^*}_{n+1} (\alpha \cup r ))$.  Since $i_F$ maps $\kappa^+$ cofinally into $\mu$, this proves that no ordinal $\alpha < \mu$ can be the projectum of $Q$, as desired.\\

Projectum-free spaces for $Q$ now follows immediately from the fact that $N^*$ satisfied projectum-free spaces, and $i_F$ is continuous at $\varrho_{n+1} (N^*) = \kappa^+$.  So we have shown that $Q$ is a plus-one premouse of the same type as $N$.\\

Now we apply the Condensation Lemma to the uncollapse map $\sigma : Q \longrightarrow N$.  It is easy to see that $crit(\sigma) = \mu$, and also that $E_\mu^N \notin Q$, since this extender would collapse $\mu$.  It follows that clause $c)$ of Condensation must hold, that is, $Q$ is an initial segment of $Ult ( N , E_\mu ) = R$.  (Notice that Anomalous Case 4 does not apply, since the critical point $\mu$ is a local successor of a limit cardinal, and AC4 requires it to be a local double successor.)\\

Now recall that $\theta$ is the $\kappa^{++}$ of $N^*$, and because it is $\Sigma_{n+1}$-regular in $N^*$ we have that $i_F$ is continuous at $\theta$.  Let $\theta' = i_F (\theta )$.  Then $\theta' = ( \mu^+ )^{Q}$, but $Q$ collapses $\theta'$; therefore $Q$ is the collapsing-level in $R$ for $\theta'$.  Also notice that $F = i_{E_\mu}$, since both are short extenders with critical point $\kappa$ which map $\kappa^+$ to $Z$.  (It is true that the extender of $\pi : N^* \longrightarrow N$ could not be applied to $N$ because it is only partial on $(\kappa^{++})^N$, but $F$ is the short part of this extender, which is total on $\kappa^+$.)  By elementarity of the full ultrapower map $i_{E_\mu} = i_F : W \longrightarrow Ult ( W , E_\mu )$, we can conclude that whatever level of $W$ was the collapsing-level for $i_F^{-1} (\theta') = \theta$ is mapped via $i_F$ to $Q$.  But we know exactly what structure is mapped to $Q$ via $i_F$: it is $N^*$.  Therefore $N^*$ is the collapsing-level for $\theta$ in $W$, and hence $N^*$ is a proper initial segment of $N$. $\blacksquare$\\









We now describe the long protomouse associated with a long divisor $\nu$.  Because $N^*$ projects to $\kappa^+$, it must be the collapsing-level in $W$ for $\theta = (\kappa^{++} )^{N^*}$.  (It is possible that $\kappa^+$ is the largest cardinal in $N^*$, in which case $\theta = o(N^* )$.)\\


\begin{defn} \label{protomouse associated with long divisor}
For $N$ a level of $W$ and $\nu$ a long divisor of $N$ with $\pi : N^* \longrightarrow N$ the uncollapse map associated with the divisor-hull, let $\kappa = \kappa_{E_\nu}$ and $\lambda = \lambda_{E_\nu}$.  Let $\theta = (\kappa^{++} )^{N^*}$ and $\eta = ( \lambda^+)^{N} = \pi (\kappa^+ )$.  We define $N (\nu) = (J_\eta^E , \dot{G} )$, where $\dot{G}$ is the long extender of length $\lambda^+$ derived from $\pi$, considered as an amenable predicate as described in \ref{ppm}; that is, for each $\xi < \theta$, let
		
\[
G_\xi = \{ (a, X) \mid a \in [ \lambda \cup \{ \nu \} ] ^{< \omega} \ \& \ X \in (N^* | \xi ) \ \& \ a \in \pi (X) \} \ .
\]
	
	Then let $\gamma_\xi$ be the least ordinal such that $G_\xi \in (N | \gamma_\xi )$.   We define our official predicate $\dot{G}$ as

\[
\dot{G} \defeq \{ (\gamma , a, X) \mid \gamma < \eta \ \& \ \exists \xi \ ( \gamma_\xi \leq \gamma \ \& \ (a, X) \in G_\xi \} \ .
\]


 We call $N (\nu)$ the long protomouse associated with the divisor $\nu$.
\end{defn}

\begin{lem}
For $N$ a level of $W$ and $\nu$ a long divisor of $N$ with associated long protomouse $N (\nu) = (J_\eta^E , \tilde{G} )$, we have that $N (\nu)$ is a long protomouse of type $1$.
\end{lem}


\textbf{Proof:} Let $\pi : N^* \longrightarrow N$ be the uncollapse map associated with the divisor-hull.  We must verify the clauses of Definition \ref{long protomouse}.  $a)$ is trivial. $b)$ follows from the fact that $N^*$ is a level of $N$ projecting to $\kappa^+$.  $c)$ follows from the definition of $N (\nu)$ as a level of $N$ with top predicate $\tilde{G}$; this guarantees that the ultrapower of $N^*$ by $\tilde{G}$ coheres with $N (\nu)$, since $\tilde{G}$ is the extender derived from $\pi$, and certainly $\pi (N^*)$ coheres with $N (\nu)$.\\

$d)$ follows immediately from Definition \ref{long divisor} clause $a)$, together with the fact that $\varrho_1 (N(\nu)) = \varrho_{n+1} (N) = \Lambda$; to see this latter fact, note that $\varrho_1 (N (\nu)) \geq \Lambda$ because $\Lambda$ is a cardinal in $W$; to see that $\varrho_1 (N (\nu )) \leq \Lambda$, observe that any subset of $\Lambda$ which was definable over $N$ is still definable over $N(\nu)$, by the exact same argument used to prove Lemma \ref{long translation2}.  (We cannot literally apply Lemma \ref{long translation2} here, because we do not yet know that $N(\nu)$ is a long protomouse; however, we know enough about $N (\nu)$ to run this argument.)\\

$e)$ also follows immediately from the fact that $\varrho_1 (N(\nu)) = \varrho_{n+1} (N) = \Lambda$, and the fact that $Ult_n (N^* , \tilde{G}) = N$, which of course satisfies projectum-free spaces; to see that $Ult_n (N^* , \tilde{G}) = N$, note that $Ult_n (N^* , \tilde{G}) = \mathcal{H}_{n+1}^N ( Z \cup \{ r \} \cup \lambda^+)$.  But this hull is the full premouse $N$, since $N$ is sound and $\varrho_{n+1} (N) = \Lambda < \lambda^+$ and $p _{n+1}(N) = r \cup \{ \nu \} \cup q$ with $( \{ \nu \} \cup q ) \subset \lambda^+$.\\

$f)$ follows from the fact that $p(N) = r \cup \{ \nu \} \cup q$; this tells us that $Ult_n (N^* , \tilde{G} \restriction (\nu + 1)) = \mathcal{H}_{n+1}^N ( Z \cup \{ r \} \cup (\nu + 1 )) = N$, since this hull contains the projectum and parameter of $N$.  But also we know that $Ult_n (N^* , \tilde{G} \restriction \nu) = \mathcal{H}_{n+1}^N ( Z \cup \{ r \} \cup \nu )$ does not contain the ordinal $\nu$, because $\nu \in p(N)$.  (In fact $Hull_{n+1}^N ( Z \cup \{ r \} \cup \nu )$ cannot contain any ordinals between $\nu$ and $\lambda^+$, since then it would also contain $\nu$.)  This is exactly the definition of $\nu$ being the largest generator of $\tilde{G}$.\\

Finally, $g)$ follows immediately from the fact that $\tilde{G} \restriction \lambda_{\tilde{G}} = E_{\nu} \restriction \lambda_{E_\nu}$, and this latter extender is indexed on the $N$-sequence below $\eta$, and therefore is on the $M$-sequence.\\

To see that $N (\nu )$ is type $1$, we can simply observe that if it were type $2$ then $Ult_n ( N^* , \tilde{G} )$ would have a non-total top extender predicate; but we know that $Ult_n ( N^* , \tilde{G} ) = N$, which is a premouse.  $\blacksquare$\\




\begin{lem} \label{long associated associated is original}
If $N$ is a level of $W$ and $N (\nu)$ is the protomouse associated with a long divisor $\nu$ of $N$, then the associated ppm of $N (\nu)$ is $N$.
\end{lem}

\textbf{Proof:}  The ideas of this proof were already presented in the above Lemma, but we repeat it here for convenience.  Letting $Z = $ the image of $\kappa^+$ under the associated ppm embedding, recall that the associated ppm of $N(\nu)$ is given by $Hull_{n+1}^N ( Z \cup \{ r \} \cup \lambda^+)$.  But this hull is the full premouse $N$, since $N$ is sound and $\varrho_{n+1} (N) = \Lambda < \lambda^+$ and $p _{n+1}(N) = r \cup \{ \nu \} \cup q$ with $( \{ \nu \} \cup q ) \subset \lambda^+$. $\blacksquare$\\

\textbf{Remark:}  It follows from \ref{long associated associated is original} and \ref{long finestructure computation} that any protomouse $N(\nu)$ associated with a long divisor of a level of $W$ must be a sound long protomouse.\\











\subsection{Type $2$ Long Divisors}










We now consider how to recover long protomice of type $2$ from their associated ppms.  Recall that the associated ppm $N$ of a type $2$ long protomouse necessarily has $k = 0$, and has a short top extender $H$ with $\kappa_H < \varrho_1 ( N)$; thus, in fact, $N$ will be a short pluripotent level of $W$ (although this observation does not seem to be highly relevant).\\


In Lemma \ref{type 2 simulation of F} and \ref{type 2 recovering $N^*$} we described how the associated ppm of a long type $2$ protomouse $M$ can $\Sigma_1$-define the $\Sigma_1$-definability relation of $M$'s associated short quasi-protomouse in parameter $\mu$, and thereby recover $N^*$ and $M$.  We will define $\nu$ to be a long divisor of a $W$-level $N$ when this situation holds.


\begin{defn} \label{type 2 long divisor}


Let $N = N_{\bar{\tau}} = (|N| , H )$ be the collapsing-level for $\bar{\tau}$ in $W$, with $\varrho_{1}(N) = \Lambda$ and $\bar{\tau} = (\Lambda^+ )^N$.  Then an ordinal $\nu \in p(N)$ is a type $2$ long divisor of $N$ if the following conditions hold:\\


\indent \indent $a)$ There is an extender $E_\nu$ indexed at $\nu$ on the $N$-sequence such that, letting $\kappa = \kappa_{E_\nu}$ and $\lambda = \lambda_{E_\nu}$, we have $\kappa^+ < \Lambda < \lambda$, $\lambda^+ = (\lambda^+)^N < \varrho_n (N)$, and $\kappa = \kappa_H$,\\

Now let $r = p (N) - (\nu + 1)$, $q = p (N) \cap \nu$, and $E_\mu = E_\nu \restriction \lambda$ be the short part of $E_\nu$ (indexed at $\mu$ on the $N$-sequence), and let $Z = i_{E_\nu}``(\kappa^+) =  i_{E_\mu}``(\kappa^+)$.  Furthermore let $R$ be the $\Sigma_1^N$-relation given by

\[
N \models R ( X , Y ) \ \text{ iff } \ ( \exists \xi < \kappa^+ ) \  [ ( i_{E_\mu} (N | \xi ) = X) \ \& \ ( i_H (N | \xi) = Y ) ] \ .
\]

and let $P = (|N|, R)$.  (Since $R$ is $\Sigma_1^N$, any $\Sigma_1^P$ relation is $\Sigma_1^N$ as well.)\\


Then we have:\\



\indent \indent $b)$ $Hull_{n+1}^P ( Z \cup r ) \cap \varrho_n (N)$ is cofinal in $\varrho_n^N$,\\

\indent \indent $c)$ $Hull_{n+1}^P ( Z \cup r ) \cap \lambda^+ = Z$,\\

\indent \indent $d)$ $\lambda^+$ is not the space of an extender on the $N$-sequence.\\


\end{defn}


We refer to $Hull_{n+1}^P ( Z \cup r )$ as the \textit{divisor-hull} associated with a long divisor $\nu$.\\

\textbf{Remark:}  In the above definition we are simply carrying out the process described in Lemma \ref{type 2 recovering $N^*$}, the only difference being that now we are dealing with an arbitrary $W$-level $N$, and cannot assume that it came from a type $2$ long protomouse as we did in Lemma \ref{type 2 recovering $N^*$}.  Notice that the above definition is very close to Definition \ref{long divisor}, except that our divisor-hull is now a hull of $P$ instead of $N$.\\

\textbf{Remark:}  The above definition relies on the fact that $\Sigma_1^P$ relations are $\Sigma_1^N$ in the parameter $\mu$.  Note that the converse holds as well: $\Sigma_1^N$ relations are $\Sigma_1^P$ in the parameter $\mu$, because the top extender of $N$ can be defined over $P$ by the relation

\[
P \models H ( X , Y ) \ \text{iff } \ ( \exists \xi < \mu ) \ [ ( i_{E_\mu} ( X) = (P | \xi ) ) \ \& \ ( i_R ( P | \xi ) = Y ) ] \ .
\]

So $N$ and $P$ are interdefinable, using the parameter $\mu$ in either direction.\\


\begin{lem}
Let $\nu$ be a type $2$ long divisor of $N = (|N| , H)$ and let $\tilde{G}$ be the long extender of length $(\lambda_{E_\nu}^+)^N = \lambda^+$ derived from the uncollapse map associated with the divisor-hull, considered as an amenable predicate (see \ref{protomouse associated with long divisor}).  Then $N ( \nu ) = ((N || \lambda^+ ), \tilde{G})$ is a type $2$ long protomouse; we call it the type $2$ long protomouse associated with the divisor $\nu$.
\end{lem}

\textbf{Proof:} Let $\pi : N^* \longrightarrow P$ be the uncollapse map associated with the divisor-hull.  We must verify the clauses of Definition \ref{long protomouse}, and further verify that this long protomouse is type $2$.  $a)$ of Definition \ref{long protomouse} is trivial.  $b)$ follows from the fact that $N^*$ is a level of $N$ projecting to $\kappa^+$, as in Lemma \ref{long divisor hull condensation}.  To verify the remaining clauses we will first need to see that $Ult_n (N^* , \tilde{G}) = P$.  Note that $Ult_n (N^* , \tilde{G}) = \mathcal{H}_{n+1}^P ( Z \cup \{ r \} \cup \lambda^+)$.  But this hull is the full $P$, since $\mu < \lambda^+$ is in the hull, so by the above Remark we have $\mathcal{H}_{n+1}^P ( Z \cup \{ r \} \cup \lambda^+) = \mathcal{H}_{n+1}^N ( Z \cup \{ r \} \cup \lambda^+)$.  Since $N$ is sound and $\varrho_{n+1} (N) = \Lambda < \lambda^+$ and $p _{n+1}(N) = r \cup \{ \nu \} \cup q$ with $( \{ \nu \} \cup q ) \subset \lambda^+$, it follows that every element of $|N| = |P|$ is in this hull.\\

$c)$ follows from the definition of $N (\nu)$ as a level of $N$ with top predicate $\tilde{G}$; this guarantees that the ultrapower of $N^*$ by $\tilde{G}$ coheres with $N (\nu)$, since $\tilde{G}$ is the extender derived from $\pi$, and certainly $\pi (N^*) = P$ coheres with $N (\nu)$.  $d)$ follows immediately from Definition \ref{type 2 long divisor} clause $a)$, together with the fact that $\varrho_1 (N(\nu)) = \varrho_{n+1} (N) = \Lambda$; to see this latter fact, note that $\varrho_1 (N (\nu)) \geq \Lambda$ because $\Lambda$ is a cardinal in $W$; to see that $\varrho_1 (N (\nu )) \leq \Lambda$, observe that any subset of $\Lambda$ which was definable over $N$ is still definable over $N(\nu)$, by the exact same argument used to prove Lemma \ref{long translation2}.  (We cannot literally apply Lemma \ref{long translation2} here, because we do not yet know that $N(\nu)$ is a long protomouse; however, we know enough about $N (\nu)$ to run this argument.  Also, the definition may require $\nu$ as a parameter, since the top extender of $P$ and the top extender of $N$ are interdefinable in parameter $\nu$.)\\

$e)$ also follows immediately from the fact that $\varrho_1 (N(\nu)) = \varrho_{n+1} (N) = \Lambda$, and the fact that $Ult_n (N^* , \tilde{G}) = P$, which has the same extender-sequence as $N$, except for its top extender, and therefore satisfies projectum-free spaces.  (Since its top extender has critical point $\lambda > \Lambda$, this top extender does not violate $e)$.)\\  



$f)$ follows from the fact that $p(N) - \lambda  = d(P) - \lambda =  r \cup \{ \nu \}$; this tells us that $Ult_n (N^* , \tilde{G} \restriction (\nu + 1)) = \mathcal{H}_{n+1}^P ( Z \cup \{ r \} \cup (\nu + 1 )) = P$, since this hull contains the projectum and parameter of $P$.  But also we know that $Ult_n (N^* , \tilde{G} \restriction \nu) = \mathcal{H}_{n+1}^P ( Z \cup \{ r \} \cup \nu )$ does not contain the ordinal $\nu$, because $\nu \in d(P)$.  (In fact $Hull_{n+1}^P ( Z \cup \{ r \} \cup \nu )$ cannot contain any ordinals between $\nu$ and $\lambda^+$, since then it would also contain $\nu$.)  This is exactly the definition of $\nu$ being the largest generator of $\tilde{G}$.\\

Finally, $g)$ follows immediately from the fact that $\tilde{G} \restriction \lambda_{\tilde{G}} = E_{\nu} \restriction \lambda_{E_\nu}$, and this latter extender is indexed on the $N$-sequence below $\eta$, and therefore is on the $M$-sequence.\\

To see that $N (\nu )$ is type $2$, recall that $k(N) = 0$, so $k(P) = k(N^*) = 0$.  Since $P$ has top extender $R$ with critical point $\lambda = \lambda_{\tilde{G}}$, clearly $\pi^{-1}`` (R) = $ (top extender of $N^*$ ) will have critical point $\kappa_{\tilde{G}}$.  This verifies that $N(\nu)$ is a type $2$ long protomouse.  $\blacksquare$\\


\begin{lem}
Let $N$ have a type $2$ long divisor $\nu$, and let $M$ be the type $2$ long protomouse associated with $\nu$.  Then the associated ppm of $M$ is $N$.
\end{lem}

\textbf{Proof:}  Many of the ideas of this proof were already presented in the above Lemma, but we repeat them here for convenience.  Letting $Z = $ the image of $\kappa^+$ under the associated short quasi-protomouse embedding, recall that the associated short quasi-protomouse of $N(\nu)$ is given by $Hull_{n+1}^P ( Z \cup \{ r \} \cup \lambda^+)$.  By the Remark following \ref{type 2 long divisor} and the fact that $\nu < \lambda^+$, this hull is the full model $P$ (here we use also that $N$ is sound, $\varrho_{n+1} (N) = \Lambda < \lambda^+$, and $p _{n+1}(N) = r \cup \{ \nu \} \cup q$ with $( \{ \nu \} \cup q ) \subset \lambda^+$.)  Finally, the associated ppm of $M$ is $Ult_0 ( (P | \mu), R)$, and we would like to see that this ultrapower is equal to $N$.  It is easy to see that the ultrapower has the same ordinals and indeed the same underlying set as $P$, by coherency of $R$; all that remains is to show that its top extender $H$ is the same as $G_N$, the top extender of $N$.  Note that $H$ is given by $i_R `` ( E_\mu )$, so $i_H = i_R \circ i_{E_\mu}$.  But $R$ was defined such that $i_R \circ i_{E_\mu} = i_{G_N}$.  This shows that $Ult_0 ( (P | \mu), R) = N$ and completes the proof.  $\blacksquare$\\
 
 







\subsection{Long Protomouse Condensation}











\begin{lem} \label{long pluripotent interpolates to long protomouse}
If $N_\tau$ is long pluripotent and $M$ is an interpolant of $N_\tau$, then $M$ is a long protomouse (possibly of type $2$), with $\varrho_{n+1}(M) = \Lambda$.  Further, the interpolation embedding $\sigma : M \longrightarrow N_\tau$ sends the Dodd parameter $d(M)$ to the Dodd parameter of $N_\tau$.
\end{lem}

\indent \indent \textbf{Proof:}  Almost exactly the same as in the short pluripotent case, Lemma \ref{short pluripotent interpolates to short protomouse}.  All properties of potential premousehood are upward preserved by $\pi: X \longrightarrow M$, except for totality of the top extender $\tilde{G}$, which will fail. This is because $N_\tau$ has top extender $G$ with $dom(G) = (\kappa_G^++)^{N_\tau} \leq \Lambda$ by long pluripotence; and since $M$ agrees with $N_\tau$ up to a little past $\Lambda$, that entire domain needs to be measured by $\tilde{G}$ in order for it to be total.  But $M$ is a non-cofinal hull in $N_\tau$, so there is some bound on the subsets of $\kappa_G^+$ which are measured by $\tilde{G}$.\\

The remaining properties of long protomice follow easily by elementarity of the embedding $\sigma$. $\blacksquare$\\


We also have

\begin{lem} \label{associated long protomouse interpolates to long protomouse}
If $N_\tau$ is the collapsing-level for $\tau$ in $W$ (following the conventions set at the beginning of \ref{interpolation subsection}), and $\nu$ is a long divisor or type $2$ long divisor of $N_\tau$ with associated protomouse $M_\tau = N_\tau (\nu )$, then any interpolant $M$ of $M_\tau$ is a long protomouse (possibly of type $2$) with $\varrho_{1}(M) = \Lambda$.
\end{lem}

\textbf{Proof:}  The proof is similar to the one we just gave for Lemma \ref{long pluripotent interpolates to long protomouse}, but now the target structure is a long protomouse $M_\tau$ instead of a long pluripotent level of $W$, which in fact makes things easier; it is a straightforward verification of the clauses of Definition \ref{long protomouse}, which follow by elementarity of $\sigma$. We omit further detail.  $\blacksquare$\\

\textbf{Remark:}  Recall that whether a long protomouse $M$ is type $2$ depends on whether its collapsing-level $N^*$ for its $\theta$ has a top extender satisfying certain conditions (and also whether $k(M) = 0$).  Because $N^*$ is \textit{not} preserved under interpolation embeddings, type $2$-ness of $M$ will not be preserved either.  In other words, a type $1$ long protomouse could have type $2$ interpolants, and vice versa.\\

Finally we prove the analogue of Lemma \ref{short protomouse condensation} for long protomice.\\




\begin{lem} \label{long protomouse condensation}
\textbf{(Long Protomouse Condensation)}
Let $M_\tau = (| M_\tau | , G)$ be either a long pluripotent level of $W$, or the protomouse associated with a long divisor or type $2$ long divisor $\nu$ of $N_\tau$, where $N_\tau$ is the collapsing-level for $\tau$ in $W$.  Let $M = (|M | , \tilde{G})$ be an interpolant of $M_\tau $ such that the interpolation embedding has critical point $\bar{\tau} \in \mathcal{S}$ (see Section \ref{interpolation subsection}), and let $N$ be the associated ppm of $M$.  Then $N$ is a level of $W$.
\end{lem}

\textbf{Proof:}  By either \ref{long pluripotent interpolates to long protomouse} or \ref{associated long protomouse interpolates to long protomouse}, we have that $M$ is a long protomouse (possibly of type $2$) with $\varrho_{1}(M) = \Lambda$.  Let $\sigma: M \longrightarrow M_\tau$ be the interpolation embedding.  Now we describe how $N$ can be embedded into a level of $W$.  First we consider the case where $M_\tau$ is either long pluripotent, or the protomouse associated with a long divisor.  (Afterwards we will consider the case where $M_\tau$ is the type $2$ long protomouse associated with a type $2$ long divisor.)\\

If $M_\tau$ is a type $1$ long protomouse then let $i_G : (N^*)^{M_\tau} \longrightarrow N_\tau$ be the associated ppm embedding.  Similarly, if $M_\tau$ is a long pluripotent level then $(Ult_n (M_\tau | \kappa_G^{++} , G) | \lambda_G^+ ) = ( M_\tau || o(M_\tau) )$, where $n$ is largest possible so that this ultrapower is defined.  In this case we set $(N^*)^{M_\tau} = M_\tau | \kappa_G^{++}$, and we have $i_G : (N^*)^{M_\tau} \longrightarrow Ult( ( N^*)^{M_\tau} , G)$.  By coherency of $G$, this ultrapower agrees with our pluripotent level $M_\tau$ up through $o(M_\tau)$.\\



There is a natural embedding $\pi: i_{\tilde{G}}(N^* ) \longrightarrow i_G ( N^* )$, given by 

\[
\pi ( i_{\tilde{G}} (f) (\alpha)) = i_G (f)(\sigma (\alpha )) \ .
\]

\smallskip

Thus we have essentially the same diagram as in the short case:


\[
\begin{tikzcd}
( M_\tau , G ) & \\
\bigtriangledown & \\
(N^*)^{M_\tau} \arrow[r, "i_G"] & Ult ((N^*)^{M_\tau} , G ) \\
& \bigtriangledown \\
( M , \tilde{G} ) \arrow[to=1-1, bend left=50, "\sigma"] & i_{G}(N^* ) \\
\bigtriangledown & \\
N^* \arrow[r, "i_{\tilde{G}}"] & N \arrow[to=5-2 , "\pi"]
\end{tikzcd}
\]
 
 \smallskip
 
 
 In the case where $M_\tau$ is a long protomouse, we have by hypothesis that $Ult ((N^*)^{M_\tau} , G ) = N_\tau$ is a level of $W$.  If $M_\tau$ is a pluripotent level of $W$, then $Ult ((N^*)^{M_\tau} , G )$ is not a level of $W$, but it agrees with $W$ up through $o(M_\tau) = (\lambda_G^+)^{Ult ((N^*)^{M_\tau} , G )}$.  Note that $i_{G}(N^* )$ is a level of $Ult ((N^*)^{M_\tau} , G )$, and $\pi$ is $\Sigma_{n+1}$-elementary.  Also, $crit (\pi ) = crit (\sigma ) = \tilde{\tau }$. \\


We would like to see that $N$ is a plus-one premouse, by applying Lemma \ref{upward mouse-pres}.  If $M$ is a type $2$ long protomouse, then following the remark after Lemma \ref{double unmorph is a ppm}, we can argue that $N$ satisfies the Jensen ISC by an argument just like that given in Lemma \ref{short protomouse condensation} when $i_{\tilde{G}}$ is discontinuous at $\lambda_{N^*}$.  We omit further detail.  So assume $M$ is a type $1$ long protomouse.  The verification that $N$ satisfies projectum-free spaces goes exactly as in the Short Protomouse Condensation Lemma \ref{short protomouse condensation}.\\



 If $n \geq 1$ then $i_{\tilde{G}}$ is $Q^{(1)}$-preserving, a fortiori $\Sigma_2$-preserving, so we can apply clauses $a)$, $b)$, or $f)$ of Lemma \ref{upward mouse-pres}, depending on the mouse-type of $N^*$, to conclude that $N$ is a plus-one premouse of the same type as $N^*$.\\

If $n = 0$ we would like to apply $c)$, $d)$, or $e)$ of Lemma \ref{upward mouse-pres}.  If $N^*$ is passive this is trivial, so assume it is active with top extender $H^*$.  Notice $\lambda_{H^*} > \kappa_{\tilde{G}}^+$, because $\lambda_{H^*}$ is the largest cardinal in $N^*$ and a limit cardinal.  This means $i_{\tilde{G}}$ is continuous at $\lambda_{H^*}$, since its only discontinuities are at points of $N^*$-cofinality $\kappa_{\tilde{G}}$ and $\kappa_{\tilde{G}}^+$.  Also notice that $N$ has a total top extender, because we are assuming $M$ is a long protomouse of type $1$.\\

The final condition of Lemma \ref{upward mouse-pres} to check is that $i_{\tilde{G}}$ is continuous at $\nu_{H^*}$ in the case where $N^*$ is a type $Z_1$ premouse.  So consider the case where $\kappa_{\tilde{G}}^+ = \kappa_F^+$, where $F = E_{\nu_H*}$ is the stretching-extender for the type $Z_1$ premouse $N^*$ as described in Definition \ref{type $Z_1$}.  We will treat this case similarly to Anomalous Case $2$ in the Condensation Lemma.  In particular, a similar argument to the one given in Section \ref{AC2 subsection} shows that $\tilde{G}$ has only a single long generator $\nu_{\tilde{G}}$.  Now instead of applying $\tilde{G}$ to $N^*$ we will apply it to the ``extender being added" by the $Z_1$ level $N^*$.  More precisely, letting $\bar{H^*}$ be as in Definition \ref{type $Z_1$}, we form $N = Ult_0 ( (( N^* | \theta ) , \bar{H^*} ) , \tilde{G})$.  Again, following the argument of Section \ref{AC2 subsection}, we conclude that $N$ is a type $Z_1$ premouse with stretching-extender $\tilde{G} \restriction \lambda_{\tilde{G}}$.  Thus, even in this case we have that $N$ is a plus-one premouse with a natural factor-map $\pi : N \longrightarrow i_G (N^*)$.\\

It now follows from Lemma \ref{long finestructure computation} that $N$ is sound and that $\varrho_{n+1} (N) = \Lambda$.  This allows us to apply the Condensation Lemma to $\pi : N \longrightarrow i_G (N^*)$.  Notice that Anomalous Case 4 does not apply, since if there were a total long extender $R$ on the $N_\tau$-sequence such that $\Lambda = (\kappa_R^+)^{N_\tau}$, then $N_\tau$ violates projectum-free spaces, contradiction.  (And clearly if there are no such extenders on the $N_\tau$ sequence, there are none on the $i_G (N^*)$-sequence.)\\


Recall that $crit (\pi) = \bar{\tau} \in \mathcal{S}$, so $\bar{\tau}$ is not an index or pseudoindex.  It follows that $c)$ and $d)$ of the Condensation Lemma are impossible.  Because $\pi$ has a critical point $\bar{\tau}$, it is not the identity, so $a)$ of Condensation is impossible.  This leaves $b)$.  We have shown that $N \lhd  i_G (N^*) \unlhd Ult ((N^*)^{M_\tau} , G )$.  In the case where $M_\tau$ is a long protomouse, $Ult ((N^*)^{M_\tau} , G )$ is a level of $W$, and we have proved the lemma.  In the case where $M_\tau$ is long pluripotent, $Ult ((N^*)^{M_\tau} , G )$ agrees with $W$ up to its $\lambda_{\tilde{G}}^+$.  But notice that in this case $N$ is a proper initial segment of $Ult ((N^*)^{M_\tau} , G )$ which projects to $\Lambda$, and $\Lambda < \lambda_{\tilde{G}}^+$.  So $N$ is in the region where we have agreement with $W$, and again $N$ is a level of $W$.  We have now proved the lemma in the cases where $M_\tau$ is either long pluripotent, or the protomouse associated with a long divisor.\\


As promised, we now consider the case where $M_\tau$ is the type $2$ long protomouse associated with a type $2$ long divisor.  In this case $P = (|P| , R ) = Ult ((N^*)^{M_\tau} , G )$ is not a level of $W$, but rather the short quasi-protomouse associated with $M_\tau$, and by hypothesis we have $N_\tau = $ (the associated ppm of $M_\tau$) is a level of $W$.  Recall that in this case $|P| = |N_\tau|$, so any proper initial segment of $P$ is a level of $W$.  Now we can repeat the above proof exactly, with the only difference being that $i_{G}(N^* ) \lhd P$ in the above diagram, instead of $i_{G}(N^* ) \lhd Ult ((N^*)^{M_\tau} , G )$.  (In fact, strictly speaking it is still the case that $P = Ult ((N^*)^{M_\tau} , G )$, but we want to highlight the fact that $P$ is not a premouse in this case, and that no problem is caused by this.)  Thus $\pi$ is an embedding from $N$ into a proper initial segment of $P$, hence a level of $W$, and we can apply the Condensation Lemma exactly as before. $\blacksquare$\\








\subsection{Canonical Divisors} \label{canonical divisor subsection}












We now move towards the definition of the $C_\tau$ sequences and the proof of $\square_\Lambda$.  Recall the conventions fixed at the beginning of \ref{prelim section} and \ref{interpolation subsection}; in particular, we are working in a fixed iterable premouse $W$ such that $W \models ZFC$, and all statements should be understood as internal to $W$.  Also we have fixed a cardinal $\Lambda$ which is not subcompact and which is not the successor of a 1-subcompact cardinal.  Finally, we have $\mathcal{S}$ a club in $\Lambda^+$ such that all $\tau \in \mathcal{S}$ satisfy:\\

\indent $a)$ $\Lambda$ is the largest cardinal in $J_\tau^E$;\\

\indent $b)$ $J_\tau^E$ is a fully elementary substructure of $J_{\Lambda^+}^E$;\\

\indent $c)$ $E_\tau = \emptyset$;\\

\indent $d)$ $\tau$ is not a pseudoindex.\\


For $\tau \in \mathcal{S}$, set $N_\tau =$ the collapsing-level for $\tau$ in $W$, and $k(\tau) =$ the least $k$ such that $\varrho_{k+1}( N_\tau )  = \Lambda$, so that $N_\tau = W  |  \langle \beta , k(\tau) \rangle$ for $\beta = o (N_\tau )$.\\

We would like to associate a canonical divisor, and its associated protomouse, to each of our levels $N_\tau$ by choosing the divisor of $N_\tau$ with the largest divisor-hull out of all the candidate divisors.  Essentially the reason for this choice is that protomice corresponding to larger divisor-hulls are ``visible" from the perspective of protomice corresponding to smaller divisor-hulls; this fact will be the key to our proof that the interpolants we consider in the $\square_\Lambda$ construction are the canonical protomice associated with their local successor cardinals.  We will go into more detail shortly, but in brief, the form of the argument is as follows: We start with a canonical protomouse $M_\tau$ at some level $\tau$, and consider its interpolated protomice, which correspond to lower levels $\bar{\tau}$.  We need to see that a tail-end of these interpolants are the canonical protomice associated with their $\bar{\tau}$'s.  Suppose not; then there are cofinally many $\bar{\tau} < \tau$ such that the canonical protomouse $M_{\bar{\tau}}$ has a larger divisor-hull than the one corresponding to our interpolant.  But these larger-hull $M_{\bar{\tau}}$'s are ``visible" from within our interpolants, and we can use the embedding-chain through our interpolants to push the existence of a larger-divisor-hull protomouse up the chain and show that our original protomouse $M_\tau$ can likewise ``see" the existence of a larger divisor-hull of $N_\tau$.  But then $M_\tau$ was not the canonical divisor of $N_\tau$ after all, contradiction.\\

It is essential in the above argument that there is a \textit{largest} divisor-hull of each $N_{\bar{\tau}}$, because a protomouse can only ``see" other protomice if they correspond to strictly larger divisor-hulls.  If a level $N_{\bar{\tau}}$ has no largest divisor-hull, but merely a collection of incomparable ones, then there is no way to guarantee that our interpolated protomouse corresponding to $\bar{\tau}$ is the canonical $M_{\bar{\tau}}$.  (The supposition that it is not the canonical $M_{\bar{\tau}}$ would no longer imply that it can ``see" the true $M_{\bar{\tau}}$, so we have nothing to push up the chain to $N_\tau$.)  We will show that in most cases there is a divisor with strictly largest divisor-hull in this sense, but there will be some cases in which the largest-hull short divisor and the largest-hull long divisor have ``overlapping" divisor-hulls; such levels are called \textit{unstable}.  Thus we will have to content ourselves with not one but two canonical protomice for unstable levels $N_\tau$, and our proof will in fact yield a $\square_{\Lambda , 2}$-sequence instead of a full $\square_\Lambda$-sequence.  We will also treat levels where a type $2$ long divisor $\nu$ has $\nu = max(q)$ for the canonical short divisor $(\kappa , q)$ as unstable levels.\\

\begin{lem}
If $\nu$ is a type $2$ long divisor of $N$, then $\nu$ is not a long divisor of $N$.
\end{lem}

\textbf{Proof:} Any type $2$ long divisor $\nu$ has an extender $E_\nu$ with the same critical point $\kappa_{E_\nu}$ as the top extender of $N$.  This means that any $\Sigma_1$-hull of $N$ must include the point $\{ \kappa_{E_\nu} \}$, since it is definable over $N$.  But then $\nu$ cannot be a long divisor, since clause $c)$ of Definition \ref{long divisor} implies that $\{ \kappa_{E_\nu} \}$ is not in the corresponding divisor-hull. $\blacksquare$\\

So the long divisors of $N$ and the type $2$ long divisors of $N$ are disjoint subsets of $p(N)$.\\

\begin{defn} \label{unstable level}
Let $N$ be a level of $W$ with a canonical short divisor $(\kappa , q)$ and one or more long or type $2$ long divisors.  Let $\nu \in p(N)$ be least such that $\nu$ is a long or type $2$ long divisor.  We say $N$ is unstable if and only if $\nu = max (q)$ and, if $\nu$ is a long divisor, $\kappa_{E_\nu} < \kappa$.
\end{defn}

We are finally ready to define the canonical divisor associated with a level of $W$.\\

\begin{defn} \label{canonical divisor}
Let $N$ be a level of $W$ with at least one strong short divisor, long divisor, or type $2$ long divisor, and suppose $N$ is not unstable.  Let $(\kappa , q)$ be the canonical strong short divisor if there is one, and $\nu \in p(N)$ be the least long or type $2$ long divisor if there is one.  (At least one of these is defined.)  Then\\

$a)$ If $\nu$ is undefined or if $max(q) < \nu$, we say the short divisor $(\kappa , q)$ is the canonical divisor of $N$.\\

$b)$ If $(\kappa , q)$ is undefined or if $\nu \leq max (q)$, we say the long or type $2$ long divisor $\nu$ is the canonical divisor of $N$.
\end{defn}

Likewise we define the canonical protomouse:

\begin{defn} \label{canonical protomouse}
Let $N$ be a level of $W$ with at least one strong short divisor, long divisor, or type $2$ long divisor, and suppose $N$ is not unstable.  Then\\

$a)$ If the canonical divisor of $N$ is a strong short divisor $(\kappa , q)$, we say the short protomouse associated with $(\kappa , q)$ is the canonical protomouse associated with $N$.\\

$b)$ If the canonical divisor of $N$ is a long divisor or type $2$ long divisor $\nu$, we say the long protomouse (type $1$ or $2$) associated with $\nu$ is the canonical protomouse associated with $N$.\\
\end{defn}


For $N_\tau$ the collapsing-level for $\tau$ in $W$, we want to define $M_\tau$ to be the associated canonical protomouse.  However, we also need to treat pluripotent levels as a ``limiting case" of protomice in which the top extenders are in fact total.\\

\begin{defn} \label{associated M}
Let $N_\tau$ be the collapsing-level for $\tau$ in $W$, and suppose $N$ is not unstable.  Then\\

$a)$ If $N_\tau$ has an associated canonical protomouse, we define this protomouse to be $M_\tau$.\\

$b)$ If $N_\tau$ does not have an associated canonical protomouse but $N_\tau$ is pluripotent (short or long), we define $M_\tau = N_\tau$.\\

$c)$ If $N_\tau$ does not have an associated canonical protomouse and $N_\tau$ is not pluripotent, we say that $M_\tau$ is undefined.

\end{defn}

For unstable levels we have two associated canonical protomice:\\

\begin{defn} \label{two associated M's for unstable levels}
Let $N_\tau$ be the collapsing-level for $\tau$ in $W$, and suppose $N$ is unstable, with canonical short divisor $(\kappa , q)$ and $\nu = max (q)$ a long or type $2$ long divisor.  Then we set\\

\begin{itemize}
\item{$M_\tau^{short} = N ( \kappa , q)$; this is called the canonical short protomouse associated with $N_\tau$;}
\item{$M_\tau^{long} = N ( \nu)$; this is called the canonical long protomouse associated with $N_\tau$ if $\nu$ is a long divisor, and the canonical type $2$ long protomouse associated with $N_\tau$ if $\nu$ is a type $2$ long divisor.}
\end{itemize}
\end{defn}

Another important fact about the distribution of divisors of a premouse is that a type $2$ long divisor will \textit{never} be the canonical divisor, unless it is the \textit{only} divisor or we are in the unstable case.  In other words, type $2$ long divisors can never ``beat" other divisors; the other divisor will always be canonically chosen (or, in the unstable case, they might be ``tied" as canonical divisors).\\

\begin{lem} \label{type 2 long never beat other divisors}
Let $\nu$ be a type $2$ long divisor of $N$; then there are no long divisors or type $2$ long divisors $\nu' \in p(N) - ( \nu + 1 )$, and there are no strong short divisors $( \kappa' , q' )$ with $\nu < max( q')$.
\end{lem}

\textbf{Proof:}  Let $\kappa = \kappa_{E_\nu} = \kappa_{G_N}$, as demanded in clause $a)$ of Definition \ref{type 2 long divisor}.  Because $\kappa$ is the critical point of $G_N$, it will automatically be in all $\Sigma_1$-hulls of $N$.  Now suppose $\nu' \in p(N) - ( \nu + 1 )$ were a long divisor of $N$; then the divisor-hull associated with $\nu'$ must contain no ordinals between $\kappa_{\nu'}$ and $\Lambda$, by clause $c)$ of Definition \ref{long divisor}.  Since we have just shown $\kappa$ must be in this hull, it follows that $\kappa < \kappa_{\nu'}$.  Now recall that the divisor-hull associated with $\nu$ is $Hull^P_{n+1} (Z \cup r )$, using the notation of Definition \ref{type 2 long divisor}; and this hull does not contain any ordinals between $\kappa$ and $\Lambda$, hence does not contain $\kappa'$.  But $\nu' \in r$, and $\kappa'$ is definable from $\nu'$ as the critical point of $E_{\nu'}$; contradiction.\\

The argument for long type $2$ $\nu'$ is similar: suppose $\nu' \in p(N) - ( \nu + 1 ) = r$ were long type $2$.  Then the extender $E_{\nu'}$ has critical point $\kappa$.  Now the divisor-hull associated with $\nu$ is $Hull^P_{n+1} (Z \cup r )$, using the notation of Definition \ref{type 2 long divisor}; and this hull does not contain any ordinals between $\kappa$ and $\Lambda$.  But $E_{\nu'}$ is still indexed at $\nu' \in r$, so $\kappa$ is definable and therefore in the divisor-hull, contradiction.\\

Finally, suppose $( \kappa' , q' )$ is a strong short divisor of $N$ such that $\nu < max( q')$.  As before, it must be that $\kappa' > \kappa$, since $\kappa$ is automatically in the divisor-hull of $( \kappa' , q' )$.  Let $\alpha = max(q')$.  Let $N' = \mathcal{H}_{n+1}^N ( \kappa' \cup p(N))$, with $\sigma'$ the associated uncollapse map.  By Lemma \ref{characterization of strongness}, $p (N') = (\sigma')^{-1} (r')$; in particular, there is no solidity witness for $\alpha \in p(N)$ in $Hull_{n+1}^N ( \kappa' \cup p(N))$, for this would transfer downwards to a generalized solidity witness for $\alpha \in p(N')$.  But the divisor-hull associated with $\nu$, namely $Hull^P_{n+1} (Z \cup r )$ in the notation of Definition \ref{type 2 long divisor}, does contain a solidity witness for $\alpha \in p(P)$, since $\alpha \in r$.  This solidity witness is the $\Sigma_{n+1}^P$-theory of $\alpha \cup r'$.  Also note that $Hull^P_{n+1} (Z \cup r ) \subset Hull_{n+1}^N ( \kappa' \cup p(N))$, because the top extender of $P$ is $\Sigma_1$-definable from the top extender of $N$ together with the parameter $\nu$ (see Definition \ref{type 2 long divisor}), and $\nu \in p(N)$.  In other words, although $P$ and $N$ have different top extenders, they are interdefinable using the parameter $\nu$; so a $\Sigma_{n+1}$-hull over $P$ is contained in a $\Sigma_{n+1}$-hull of the same set over $N$, assuming $\nu$ is one of the points in the $N$-hull.  Now we would like to reach a contradiction from the fact that a solidity witness for $\alpha$ is in the smaller $Hull^P_{n+1} (Z \cup r )$ but no such witness is in the larger $Hull_{n+1}^N ( \kappa' \cup p(N))$.  Unfortunately, the meaning of ``solidity witness" is slightly different in these two contexts, because one is a theory in $P$ and the other is a theory in $N$.  This can be easily remedied by again using the fact that $P$ and $N$ have top extenders which are interdefinable using the parameter $\nu$, and $\nu < \alpha$ so it is a point in the solidity witness $Th_{n+1}^P ( \alpha \cup r')$.  In other words, $Th_{n+1}^N ( \alpha \cup r')$ can be easily computed from $Th_{n+1}^P ( \alpha \cup r' )$, so this latter solidity witness is in $N'$, contradiction. $\blacksquare$\\


Finally, we show that strong short divisors cannot ``overlap" long or type $2$ long divisors, except in the unstable case.\\

\begin{lem} \label{no long-short overlaps except unstable}
Let $(\kappa , q)$ be a strong short divisor of $N$, and let $r = p(N) - q$.  Then any long or type $2$ long divisor $\nu$ with $\nu \in r$ must have $\kappa_{E_\nu} < \kappa$.  Also, if $\nu \in q$ is a long divisor of $N$, then either $\nu = max(q)$ or else $\kappa_{E_\nu} \geq \kappa$.
\end{lem}

\textbf{Proof:}  First suppose $\nu \in r$ is a long or type $2$ long divisor.  Then the divisor-hull associated with $(\kappa , q)$ contains the ordinal $\nu$, hence the extender $E_\nu$, hence its critical point $\kappa_{E_\nu}$.  But there are no points in this divisor-hull between $\kappa$ and $\Lambda$, so we must have $\kappa_{E_\nu} < \kappa$.\\

Now suppose $\nu \in q$ is a long divisor with $\nu < max (q)$.  Suppose towards contradiction that $\kappa_{E_\nu} < \kappa$; the argument is similar to the one just given in Lemma \ref{type 2 long never beat other divisors}.  Let $\alpha = max(q')$.  Let $N' = \mathcal{H}_{n+1}^N ( \kappa \cup p(N))$, with $\sigma'$ the associated uncollapse map.  By Lemma \ref{characterization of strongness}, $p (N') = (\sigma')^{-1} (r)$; in particular, there is no solidity witness for $\alpha \in p(N)$ in $Hull_{n+1}^N ( \kappa \cup p(N))$, for this would transfer downwards to a generalized solidity witness for $\alpha \in p(N')$.  But the divisor-hull associated with $\nu$, namely $Hull^N_{n+1} (Z \cup (p(N) - (\nu + 1)))$, where $Z = i_{E_\nu}``(\kappa_{E_\nu}^+)$, does contain a solidity witness for $\alpha \in p(N)$, by Lemma \ref{long divisor hull condensation}.  We have shown that a solidity witness for $\alpha \in p(N)$ is in the smaller $Hull^N_{n+1} (Z \cup (p(N) - (\nu + 1)))$ but no such witness is in the larger $Hull_{n+1}^N ( \kappa \cup p(N))$.  Contradiction.  $\blacksquare$\\






\section{Main $\square_\Lambda$ Construction} \label{main construction section}








\subsection{Defining the Sequence}


Recall the description of $\mathcal{S}$ at the beginning of \ref{canonical divisor subsection}.  Under the ``smallness assumption" described at the beginning of \ref{interpolation subsection}, we construct a $\square_{\Lambda , 2}$-sequence from $\mathcal{S}$.\\

\begin{thm}
There is a sequence $\mathcal{C} = \langle \mathfrak{C}_\tau | \tau \in \mathcal{S} \rangle$ such that each $\mathfrak{C}_\tau$ contains either one or 
two sets $C_\tau \in \mathfrak{C}_\tau$, satisfying:\\

\indent $a)$ $C_\tau \subseteq ( \mathcal{S} \cap \tau )$ is closed;\\

\indent $b)$ $C_\tau$ is unbounded in $\tau$ whenever $\tau$ is a limit point of $S$ and $cof(\tau) > \omega$;\\

\indent $c)$ $C_\tau \cap \bar{\tau} \in \mathfrak{C}_{\bar{\tau}}$ whenever $\bar{\tau} \in C_\tau$ (this is called coherency);\\

\indent $d)$ $otp (C_\tau ) \leq \kappa$.
\end{thm}

\bigskip

Let

\[
\mathcal{S}^0 = \mathcal{S} - \mathcal{S}^1 \ ;
\]

\[
\mathcal{S}^1 = \{ \tau \in \mathcal{S} \ | \ M_\tau \text{ is defined } \} \ .
\]

A sequence of this form is technically called a $\square'_{(\Lambda , 2)}$-sequence, because $b)$ is phrased only for $\tau$ of uncountable cofinality; it is typical for square constructions in inner models to produce such $\square'$-sequences, which can easily be turned into $\square$-sequences by combinatorial manipulations.  See \cite{zeman square proof} for details.\\


Many of the definitions and explanations in what follows are taken from \cite{zeman square proof}, with some mild adaptations to the context of plus-one premice; we reiterate these parts for the sake of completeness, and because the broader plus-one premice context requires additional consideration at certain points.  The reader familiar with \cite{zeman square proof} can focus attention on the parts of the construction that involve long divisors and long protomice.\\




\begin{defn} \label{B-tau sequence on S^0}
Given $\tau \in \mathcal{S}^0$, $B_\tau$ is the set of all $\bar{\tau} \in \mathcal{S}^0 \cap \tau$ satisfying:\\

\begin{itemize}
\item{$N_{\bar{\tau}}$ is a premouse of the same type as $N_\tau$;}
\item{$n(\tau) = n(\bar{\tau})$;}
\item{There is a map $\sigma_{\bar{\tau} \tau } : N_{\bar{\tau}} \longrightarrow N_\tau$ that is $\Sigma_0^{n(\tau)}$-preserving with respect to the language of premice and such that}
\end{itemize}

\indent \indent $a)$ $\bar{\tau} = crit (\sigma_{\bar{\tau} \tau })$ and $\sigma_{\bar{\tau} \tau } (\bar{\tau}) = \tau $;\\

\indent \indent $b)$ $\sigma_{\bar{\tau} \tau } (p ( N_{\bar{\tau}}) = p ( N_\tau)$;\\

\indent \indent $c)$ for each $\alpha \in p( N_\tau )$ there is a generalized solidity witness $Q_\tau (\alpha )$ for $\alpha$ with respect to $N_\tau$ and $p(N_\tau)$ such that $Q_\tau (\alpha ) \in range ( \sigma_{\bar{\tau} \tau } )$.\\
\end{defn}

We say that $\mathcal{B} = \langle B_\tau \ | \ \tau \in \mathcal{S} \rangle$.\\

In other words, $B_\tau$ is the set of all $\bar{\tau} \in \mathcal{S}^0 \cap \tau$ such that $N_{\bar{\tau}}$ is an interpolant of $N_\tau$, and the interpolation map sends $\bar{\tau}$ to $\tau$.\\

Solidity witnesses in clause $c)$ are, of course, computed with respect to the language of premice.  The map $\sigma_{\bar{\tau} \tau }$ always has a critical point since, by the restrictions imposed on the elements of $\mathcal{S}$, each $N_\tau$ is strictly longer than $W | \tau$.  Notice that $\sigma_{\bar{\tau} \tau }$ is uniquely determined; this follows from soundness of $N_{\bar{\tau}}$.  Given any $x \in N_{\bar{\tau}}$, there is a $\xi < \Lambda$ such that $x = h_{n+1}^{N_{\bar{\tau}}} (\xi , p (N_{\bar{\tau}}))$; as $\Sigma_1^{(n)}$-statements are upward preserved under $\Sigma_0^{(n)}$-embeddings and Skolem functions have absolute definitions, $\sigma (x) = h_{n+1}^{N_{\tau}} (\xi , p (N_\tau))$ for any map $\sigma$ satisfying the above definition.  Notice also that clause $c)$ does not have any influence on the uniqueness of $\sigma_{\bar{\tau} \tau }$.  One further important fact about $\sigma_{\bar{\tau} \tau }$ is that $\sigma_{\bar{\tau} \tau }$ is \textit{not} $\Sigma_1^{(n(N_\tau)}$-preserving, hence

\[
\sigma_{\bar{\tau} \tau } \text{ is non-cofinal at the $n(N_\tau)$-th level.}
\]

Otherwise, $\sigma_{\bar{\tau} \tau }$ would be $\Sigma_1^{(n(N_\tau)}$-preserving, so $h_{n+1}^{N_{\bar{\tau}}} (\xi , p (N_{\bar{\tau}}))$ would be defined if and only if $h_{n+1}^{N_{\tau}} (\xi , p (N_\tau))$ would.  By the soundness of $N_\tau$, we would have $range ( \sigma_{\bar{\tau} \tau }) = N_\tau$, and from this the obviously false conclusion $N_{\bar{\tau}} = N_\tau$.  Finally we note that clause $c)$ in the above definition will be used only once in the entire construction, namely in the proof that $C_{\tau}$ is closed, to make sure certain direct limits are sound.  The condition in $c)$ cannot be strengthened to a requirement that \textit{standard} solidity witnesses be members of the corresponding ranges; this follows from the fact that we are forced to deal with embeddings with very weak preservation properties which typically do not map standard witnesses to standard witnesses.\\

\begin{defn} \label{short B-tau sequence on S^1}
Given $\tau \in \mathcal{S}^1$, if the canonical divisor of $N_\tau$ is a short divisor $(\kappa , q)$, then $B_\tau$ is the set of all $\bar{\tau} \in \mathcal{S}^1 \cap \tau$ satisfying:

\begin{itemize}
\item{One of the canonical divisors of $N_{\bar{\tau}}$ is a short divisor $(\bar{\kappa} , \bar{q})$ such that $\kappa = \bar{\kappa}$ and $|q| = |\bar{q}|$;}
\item{There is a map $\sigma_{\bar{\tau} \tau } : M_{\bar{\tau}} \longrightarrow M_\tau$ (or $\sigma_{\bar{\tau} \tau } : M_{\bar{\tau}}^{short} \longrightarrow M_\tau$) that is $\Sigma_0$-preserving with respect to the language of coherent structures and such that}
\end{itemize}

\indent \indent $a)$ $\bar{\tau} = crit (\sigma_{\bar{\tau} \tau })$ and $\sigma_{\bar{\tau} \tau } (\bar{\tau}) = \tau $;\\

\indent \indent $b)$ $\sigma_{\bar{\tau} \tau } (\bar{q}) = q$;\\

\indent \indent $c)$ for each $\alpha \in \bar{q}$ there is a generalized solidity witness $Q_\tau (\alpha )$ for $\alpha$ with respect to $M_\tau$ and $q$ such that $Q_\tau (\alpha ) \in range ( \sigma_{\bar{\tau} \tau } )$.\\

If $N_\tau$ is unstable, then we define $B_\tau^{short}$ as above, with $M_\tau^{short}$ in place of $M_\tau$.


\end{defn}

In other words, $B_\tau$ for $\tau$ such that $M_\tau$ is a short protomouse is the set of all $\bar{\tau} \in \mathcal{S}^1 \cap \tau$ such that (one of the two) $M_{\bar{\tau}}$ is an interpolant of $M_\tau$, and the interpolation map sends $\bar{\tau}$ to $\tau$.  If $N_\tau$ is unstable then we use this definition for \textit{one} of the canonical $B_\tau$ sequences.\\


\begin{defn} \label{long B-tau sequence on S^1}
Given $\tau \in \mathcal{S}^1$, if the canonical divisor of $N_\tau$ is a long divisor or type $2$ long divisor $\nu$, then $B_\tau$ is the set of all $\bar{\tau} \in \mathcal{S}^1 \cap \tau$ satisfying:

\begin{itemize}
\item{One of the canonical divisors of $N_{\bar{\tau}}$ is a long or type $2$ long divisor $\bar{\nu}$ such that $|p(N_\tau) \cap \nu| = |p(N_{\bar{\tau}} \cap \bar{\nu} )|$;}
\item{There is a map $\sigma_{\bar{\tau} \tau } : M_{\bar{\tau}} \longrightarrow M_\tau$ (or $\sigma_{\bar{\tau} \tau } : M_{\bar{\tau}}^{long} \longrightarrow M_\tau$) that is $\Sigma_0$-preserving with respect to the language of coherent structures and such that}
\end{itemize}

\indent \indent $a)$ $\bar{\tau} = crit (\sigma_{\bar{\tau} \tau })$ and $\sigma_{\bar{\tau} \tau } (\bar{\tau}) = \tau $;\\

\indent \indent $b)$ $\sigma_{\bar{\tau} \tau } (\bar{\nu}) = \nu$;\\

\indent \indent $c)$ for each $\alpha \in d(M_{\bar{\tau}})$ there is a generalized solidity witness $Q_\tau (\alpha )$ for $\alpha$ with respect to $M_\tau$ and $d(M_\tau)$ such that $Q_\tau (\alpha ) \in range ( \sigma_{\bar{\tau} \tau } )$.\\

If $N_\tau$ is unstable, then we define $B_\tau^{long}$ as above, with $M_\tau^{long}$ in place of $M_\tau$.



\end{defn}


In other words, $B_\tau$ for $\tau$ such that $M_\tau$ is a long protomouse is the set of all $\bar{\tau} \in \mathcal{S}^1 \cap \tau$ such that (one of the two) $M_{\bar{\tau}}$ is an interpolant of $M_\tau$, and the interpolation map sends $\bar{\tau}$ to $\tau$.  If $N_\tau$ is unstable then we use this definition for \textit{one} of the canonical $B_\tau$ sequences.\\

Notice also that in the definitions of our canonical sequences, we treat type $1$ long protomice and type $2$ long protomice together.  They will be related to levels of $W$ in slightly different ways, but we do not distinguish between them while constructing elementary chains of protomice.\\







\begin{lem} \label{range inclusion}
Let $\tau \in \mathcal{S}^i$ for $i \in \{ 0 , 1 \}$ and let $\tau^* < \bar{\tau}$ satisfy all the requirements of the definition of $B_\tau$, $B_\tau^{short}$, or $B_\tau^{long}$, except possibly clause $c)$.  Then $range (\sigma_{\tau^* \tau} ) \subset range ( \sigma_{\bar{\tau} \tau } )$.
\end{lem}

\textbf{Proof:} It follows from the elementarity of our maps that if $h_{n+1}^{N_{\tau^*}} ( \xi , p ( N_{\tau^*} ))$ is defined then also  $h_{n+1}^{N_{\bar{\tau}}} ( \xi , p ( N_{\bar{\tau}} ))$ is defined, whenever $\xi < \Lambda$.  (To be more precise, we need to see that $sup ( ( \sigma_{\tau^* \tau }) `` \varrho_n (N_{\tau^*}) ) < sup ( ( \sigma_{\bar{\tau} \tau }) `` \varrho_n (N_{\bar{\tau}}) )$; see Lemma $3.2$ of \cite{zeman square proof} for details.)  

It follows that $\sigma_{\tau^* \tau } (h_{n+1}^{N_{\tau^*}} ( \xi , p ( \tau^*))) = h_{n+1}^{N_\tau} ( \xi , p ( N_\tau )) = \sigma_{\bar{\tau} \tau } (h_{n+1}^{N_{\tau}} ( \xi , p ( \bar{\tau})))$ whenever $h_{n+1}^{N_{\tau^*}} ( \xi , p ( \tau^*))$ is defined.  Since $N_{\tau^*}$ is sound, this completes the proof. $\blacksquare$\\

\begin{lem} \label{almost coherency of B}
Let $\tau \in \mathcal{S}^i$ for $i \in \{ 0 , 1 \}$ and $\bar{\tau} \in B_\tau$, $B_\tau^{short}$, or $B_\tau^{long}$.  Then $B_\tau \cap \bar{\tau} = B_{\bar{\tau}} - min (B_\tau)$, and the corresponding statements hold for $B_\tau^{short}$ and $B_\tau^{long}$.
\end{lem}

\textbf{Proof:}  We treat the case $\bar{\tau} \in B_\tau$, since the other cases will be identical.  Suppose $\tau \in \mathcal{S}^0$ (the proof for $\tau \in \mathcal{S}^1$ is the exact same).  Pick a $\tau^* \in B_\tau \ \cap \ \bar{\tau}$.  We first show that $\tau^* \in B_{\bar{\tau}}$.  By the previous lemmata, $range ( \sigma_{\tau^* \tau} ) \subset  range ( \sigma_{\bar{\tau} \tau} )$, so we can define a map $\sigma: N_{\tau^*} \longrightarrow N_{\bar{\tau}}$ by $\sigma = ( \sigma_{\bar{\tau} \tau})^{-1} \circ \sigma_{\tau^* \tau} $.  It is routine to verify that $\sigma$ satisfies all requirements on $\sigma_{\tau^* \bar{\tau}}$ except possibly clause $c)$, which we verify now.\\

Given $\bar{\alpha} \in p_{\bar{\tau}}$, let $\alpha = \sigma_{\bar{\tau} \tau} (\bar{\alpha} )$ and $\sigma_{\tau^* \bar{\tau}} (\alpha^* ) = \bar{\alpha}$.  The definition of $B_\tau$ guarantees that we have a generalized witness $Q_{\tau} (\alpha)$ for $\alpha$ with respect to $N_\tau$ and $p_\tau$ in the range of $\sigma_{\tau^* \tau}$, and we know that the $\sigma_{\bar{\tau} \tau}$-preimage $\bar{Q} ( \bar{\alpha} )$ of $Q_\tau (\alpha)$ is in the range of $\sigma$.  But ``$Q$ is a generalized witness for $\alpha$ with respect to $N$ and $p$" is a $\Pi_1^{(n)}$-statement, so it is downward preserved under $\Sigma_0^{(n)}$-maps.  It follows that $\bar{Q} ( \bar{\alpha} )$ is a generalized witness for $\bar{\alpha}$ with respect to $N_{\bar{\tau}}$ and $p_{\bar{\tau}}$.  This proves that $\tau^* \in B_{\bar{\tau}}$ and thus the inclusion $\subset$.\\

Let $\tau' = min (B_\tau )$.  Pick a $\tau^* \in B_{\bar{\tau}} - \tau'$ that is larger than $\tau'$.  Define an embedding $\sigma : N_{\tau^*} \longrightarrow N_\tau$ by $\sigma = \sigma_{\bar{\tau} \tau} \circ \sigma_{ \tau^* \bar{\tau}}$.  Again, $\sigma$ meets all requirements in the above definition except possibly clause $c)$, which suffices to conclude that $\sigma = \sigma_{\tau^* \tau}$.  Regarding $c)$, if $Q (\alpha ) \in range ( \sigma_{\tau' \tau } )$ is a generalized witness for $\alpha \in p_\tau$ with respect to $N_\tau$ and $p_\tau$, then $Q (\alpha )$ is in the range of $\sigma_{\tau^* \tau}$ by the previous lemma, so $\tau^* \in B_\tau$.  This proves the inclusion $\supset$. $\blacksquare$\\






By the above Lemma, the sequence $\mathcal{B}$ is almost coherent; the only deficiency of $\mathcal{B}$ is that the initial segments of $B_\tau$, $B_\tau^{short}$, or $B_\tau^{long}$ might grow as $\tau$ decreases.  This can be fixed by adding all potential initial segments to each $B_\tau$.  For $\tau \in \mathcal{S}$ we set

\begin{itemize}
\item{$\tau (0) = \tau$;}
\item{$\tau ( i+1) = min ( B_{\tau (i)})$ if there is a unique $B_{\tau (i)}$, $min ( B^{short}_{\tau (i)})$ if $\tau$ had a single canonical short divisor, and $min ( B^{long}_{\tau (i)})$ if $\tau$ had a single canonical long divisor;}
\item{$\ell_\tau =$ the least $i$ such that $B_{\tau (i)} = \emptyset$.}
\end{itemize}

Likewise, if $N_\tau$ is unstable, we define $\tau (i)^{short}$ and $\tau (i)^{long}$ exactly as above, but with the short or long canonical divisor respectively in place of the single canonical divisor of $N_\tau$.\\

The number $\ell_{\tau}$ (or the pair $\ell_{\tau}^{short}$ and $\ell_{\tau}^{long}$) is defined for every $\tau \in \mathcal{S}$, otherwise we would have an infinite decreasing sequence of ordinals.  We are now ready to define a fully coherent sequence $\mathcal{B}^* = \langle B_\tau^* \ | \ \tau \in \mathcal{S} \rangle$.  Given any $\tau \in \mathcal{S}$,\\

\begin{itemize}
\item{$B_\tau^* = B_{\tau (0)} \cup \ ... \ \cup B_{\tau ( \ell_\tau - 1 )}$;}
\item{$\sigma_{\bar{\tau} \tau }^* = \sigma_{ \tau (1) \tau (0) } \circ \ ... \circ \sigma_{\tau (j) \tau ( j-1)} \circ \sigma_{\bar{\tau} \tau (j) } $ whenever $\bar{\tau} \in B_\tau^*$ and $j$ is such that $\bar{\tau} \in B_{\tau (j)}$.}
\end{itemize}

We also define $(B_\tau^{short})^*$ and $(B_\tau^{long})^*$ in the obvious way, with $\tau (i)^{short}$ or $\tau (i)^{long}$ in place of $\tau (i)$.\\

 \begin{lem}
 $\mathcal{B}^*$ is a coherent sequence.
 \end{lem}
 
 \textbf{Proof:} Pick a $\tau \in \mathcal{S}$ and a $\bar{\tau} \in B_\tau^*$.  Suppose for simplicity that neither $\tau$ nor $\bar{\tau}$ is unstable.  (If they are, the proof is exactly the same, with superscripts in the appropriate places.)  Assume without loss of generality that $\bar{\tau} > \tau ( \ell_\tau - 1 ) = min ( B_\tau^* )$.  We first observe
 
 \[
 min ( B_{\bar{\tau}}) \in B_\tau^* \ \ \text{ and } \ \ B_{\bar{\tau}} = B_\tau^* \cap [ min (B_{\bar{\tau}} ) , \bar{\tau} ) \ .
 \]
 
 This follows from Lemma \ref{almost coherency of B}.  Let $j$ be such that $\bar{\tau} \in B_{\tau (j)}$.  Then $\tau (j + 1 ) \in B_{\bar{\tau}}$, and either $\tau ( j+1) = min ( B_{\bar{\tau}})$ or else $B_{\bar{\tau}} \cap \tau ( j+1)$ is a tail-end of $B_{\tau ( j + 1 ) } \subset B_\tau^*$.  This proves that $min ( B_{\bar{\tau}}) \in B_\tau^*$.  To see that $B_{\bar{\tau}} = B_\tau^* \cap [ min (B_{\bar{\tau}} ) , \bar{\tau} )$, we observe that $B_{\bar{\tau}}$ agrees with $B_{\tau (j)}$ on $[ \tau (j+1) , \bar{\tau} )$ and $B_{\bar{\tau}}$ agrees with $B_{\tau ( j+1)}$ on $[ min ( B_{\bar{\tau}} ) , \tau ( j+1 ) )$.\\
 
 Define $\bar{\tau} (i)$ from $\bar{\tau}$ the same way $\tau (i)$ was defined from $\tau$.  Let $\bar{\ell} = \ell_{\bar{\tau}}$.  Using the facts we just proved, we inductively show that $B_{\bar{\tau} (i)}$ is a segment of $B_{\tau}^*$ for all $i < \bar{\ell}$.  It follows that $B_{\bar{\tau}}^*$ is a (not necessarily initial) segment of $B_\tau^*$.  But $B_{\bar{\tau}}^*$ must be in fact an initial segment of $B_\tau^*$; otherwise $\bar{\tau} ( \bar{\ell} - 1 ) > \tau ( \ell_\tau - 1 )$ which would mean that $B_{\bar{\tau} ( \bar{\ell} - 1 )}$ is nonempty.  Contradiction.  $\blacksquare$\\
 
 
Since each $\sigma_{\bar{\tau} \tau }^*$ is the unique $\Sigma_0^{(n)}$-preserving map from $N_{\bar{\tau}}$ to $N_{\tau}$ with critical point $\bar{\tau}$ sending $p_{\bar{\tau}}$ to $p_\tau$, Lemma \ref{range inclusion} guarantees that $range ( \sigma_{\tau^* \tau }^*) \subset range ( \sigma_{\bar{\tau} \tau }^*)$ whenever $\tau^* < \bar{\tau}$ are in $\mathcal{S}^0$.  On $\mathcal{S}^1$, the situation is analogous.  Thus we can define $\sigma_{\tau^* \bar{\tau} }^*$ by $\sigma_{\tau^* \bar{\tau} }^* = ( \sigma_{\bar{\tau} \tau }^*)^{-1} \circ \sigma_{\tau^* \tau }^*$ for any $\tau^* \leq \bar{\tau}$ from $B_\tau^* \cup \{ \tau \}$.  It follows immediately that $\sigma_{\tau^* \bar{\tau} }^* : N_{\tau^*} \longrightarrow N_{\bar{\tau}}$ is the unique map that is $\Sigma_0^{(n)}$-preserving with respect to the language for premice, has critical point $\tau^*$, and sends $\tau^*$ to $\bar{\tau}$ and $p_{\tau^*}$ to $p_{\bar{\tau}}$ whenever $\tau \in \mathcal{S}^0$ and $\tau^* \leq \bar{\tau}$ are in $B_\tau^* \cup \{ \tau \}$.  Also, $\sigma_{\tau^* \bar{\tau} }^* : M_{\tau^*} \longrightarrow M_{\bar{\tau}}$ is the unique map that is $\Sigma_0$-preserving with respect to the language for coherent structures, has critical point $\tau^*$, and sends $\tau^*$ to $\bar{\tau}$ and $q_{\tau^*}$ to $q_{\bar{\tau}}$ whenever $\tau \in \mathcal{S}^1$ and $\tau^* \leq \bar{\tau}$ are in $B_\tau^* \cup \{ \tau \}$.\\

To complete the definition of the square sequence, we will make use of the following crucial lemma, whose proof will constitute the next subsection.

\begin{lem} \label{B-tau is a club on a tail-end}
For every $\tau \in \mathcal{S}$ of uncountable cofinality, $B_\tau$, $B_\tau^{short}$, and $B_\tau^{long}$ are closed unbounded subsets of $\tau$ on a tail-end (if they are defined).  In other words, there is a $\bar{\tau} < \tau$ such that $B_\tau - \bar{\tau}$ is closed and unbounded in $\tau$; likewise for $B_\tau^{short}$ and $B_\tau^{long}$.
\end{lem}
 
 
 To each $\tau \in \mathcal{S}$, let $\beta_\tau$ be the least $\beta \in B_\tau^* \cup \{ \tau \}$ such that $B_\tau^* - \beta$ is closed in $\tau$.  Define $\beta_\tau^{short}$ and $\beta_\tau^{long}$ analogously.  The ordinal $\beta_\tau$ (or $\beta_\tau^{short}$ and $\beta_\tau^{long}$, if $\tau$ is unstable) is always defined, and if $cof( \tau) > \omega$, then $\beta_\tau < \tau$, as follows from the previous lemma and the fact that $B_\tau$ is a tail-end of $B_\tau^*$.  Set
 
\[
\begin{split}
& C_\tau^* = B_\tau^* - \beta_\tau \ ,\\
& (C_\tau^{short})^* = (B_\tau^{short})^* - \beta_\tau^{short} \ ,\\
& (C_\tau^{long})^* = (B_\tau^{long})^* - \beta_\tau^{long}\\
\end{split}
\]
 
and then define $\mathfrak{C}^*_\tau = \{ C_\tau^* \}$ if $\tau$ has a single canonical divisor, else $\mathfrak{C}^*_\tau = \{ (C_\tau^{short})^* , (C_\tau^{long})^* \}$ if $\tau$ is unstable.  Finally we set
 
 \[
  \mathcal{C}^* = \langle \mathfrak{C}^*_\tau \ | \ \tau \in \mathcal{S} \rangle \ .
 \]
 
 Given $\bar{\tau} \in C_\tau^*$, we know that $\bar{\tau} \in B_\tau^*$, $\bar{\tau} \geq \beta_\tau$ and that $B_\tau^*$, $B_{\bar{\tau}}^*$ cohere.  It follows that $\beta_{\bar{\tau}} = \beta_\tau$.  Consequently, $C_{\bar{\tau}}^* = B_{\bar{\tau}}^* - \beta_{\bar{\tau}} = B_\tau^* \cap \bar{\tau} - \beta_\tau = C_\tau^* \cap \bar{\tau}$; likewise for $(C_\tau^{short})^*$ and $(C_\tau^{long})^*$.  The sequence $\mathcal{C}^*$ thus satisfies all requirements on a $\square'_{\Lambda , 2}$-sequence with the only exception that the sets $C_\tau^*$, $(C_\tau^{short})^*$ and $(C_\tau^{long})^*$ might have large order type.\\

We now observe that for $\tau \in \mathcal{S}^1$, the order type of $C_\tau^*$, $(C_\tau^{short})^*$ and $(C_\tau^{long})^*$ are always small.  Consider $C_\tau^*$ without loss of generality.  To each $\tau^* < \bar{\tau}$ from $C_\tau^*$ we have the map $\sigma_{\tau^* \bar{\tau} }^* : M_{\tau^*} \longrightarrow M_{\bar{\tau}}$ which is the unique map with the preservation properties considered above.  It is easy to see that $dom(F_{\tau^*}) \subset dom( F_{\bar{\tau}})$ and this inclusion is strict, since $\sigma_{\tau^* \bar{\tau} }^*$ is non-cofinal.  Hence $\theta_{\tau^*} < \theta_{\bar{\tau}}$.  It follows that $\theta_{\bar{\tau}} \longrightarrow \bar{\tau}$ is a strictly monotonic enumeration of $C_\tau^*$ with domain contained in $\theta_\tau$.  But $\theta_\tau \leq \kappa^+ \leq \Lambda$ for short protomice, and $\theta_\tau \leq \kappa^{++} \leq \Lambda$ for long protomice (of type $1$ or $2$), so in all cases we have $otp(C_\tau^*) \leq otp ( \theta_\tau) \leq \Lambda$.\\

The above discussion shows that for $\tau \in \mathcal{S}^1$, our construction already yields a $\square'_{\Lambda}$-sequence.  For $\tau \in \mathcal{S}^0$ this is not obvious, and to arrange that the order types are small, we will replace the sets $C_\tau^*$ by suitably chosen subsets.\\


Let $X_\tau (\xi)$ be the $\Sigma_1^{(n)}$-hull of $\{ \xi , p_\tau \}$ in $N_\tau$.  We define sequences $\langle \tau_\iota \rangle $, $\langle \xi_\iota^\tau \rangle$ in the following way:\\

\begin{itemize}
\item{$\tau_0 = min ( C_\tau^* \cup \{ \tau \} )$;}
\item{$\xi_\iota^\tau = $ the least $\xi < \Lambda$ such that $X_\tau (\xi)$ is not contained in $range( \sigma_{\tau_\iota \tau }^* )$;}
\item{$\tau_{\iota + 1} = $ the least $\bar{\tau} \in C_\tau^*  \cup  \{ \tau \}$ such that $X_\tau ( \xi_\iota^\tau )$ is contained in $range ( \sigma_{\bar{\tau} \tau}^*)$;}
\item{$\tau_\gamma = sup \{ \tau_\iota \ | \ \iota < \gamma \}$ for limit $\gamma$;}
\item{$\iota_\tau =$ the least $\iota$ such that $\tau_\iota = \tau$.}
\end{itemize}

If $\tau_\iota < \tau$, then $\xi_\iota^\tau$ is always defined; just choose a $\xi < \Lambda$ such that $\tau_\iota = h_\tau ( \xi , p_\tau )$ and observe that $X_\tau (\xi)$ is not contained in $range(\sigma_{\tau_\iota \tau }^*)$, as $\tau_\iota = crit(\sigma_{\tau_\iota \tau }^*)$.  Set

\[
C'_\tau = \{ \tau_\iota \ | \ \iota < \iota_\tau \} \ .
\]

\begin{lem}
$\langle C'_\tau \ | \ \tau \in \mathcal{S}^0 \rangle$ is a $\square'_\Lambda$-sequence on $\mathcal{S}^0$.
\end{lem}

\textbf{Proof:}  Given $\iota < \iota_\tau$, the above definition immediately yields that $\tau_{\iota + 1} > \tau_\iota$, and it is easy to see that $C'_\tau$ is closed.  Furthermore, if $\tau$ does not have cofinality $\omega$, $C'_\tau$ is unbounded, since each $X_\tau ( \xi_\iota^\tau )$ is countable and therefore contained in $range ( \sigma_{\bar{\tau} \tau }^* )$ for sufficiently large $\bar{\tau} \in C_\tau^*$ (recall that $C_\tau^*$ is closed unbounded in $\tau$ in this case).  We next observe that if $\bar{\iota} < \iota$, then $X_\tau ( \xi_\iota^\tau )$ is not contained in $range ( \sigma^*_{\tau_{\bar{\iota}} \tau})$, as $range ( \sigma^*_{\tau_{\bar{\iota}} \tau}) \subset range ( \sigma^*_{\tau_\iota \tau})$.  So $\xi_\iota^\tau < \xi_{\bar{\iota}}^\tau$ would contradict the minimality of $\xi_{\bar{\iota}}^\tau$.  Furthermore, since $X_\tau ( \xi_{\bar{\iota}}^\tau) \subset range ( \sigma^*_{\tau_{\bar{\iota} + 1} \tau}) \subset range ( \sigma^*_{\tau_\iota \tau})$, the ordinals $\xi_\iota^\tau$ and $\xi_{\bar{\iota}}^\tau$ must be distinct.  This proves that $\langle \xi_\iota^\tau \ | \ \iota < \iota_\tau \rangle$ is a \textit{strictly} increasing sequence of ordinals smaller than $\Lambda$.  As an immediate consequence we have $\iota_\tau \leq \Lambda$ and that $\iota \longrightarrow \tau_\iota$ is a strictly monotonic enumeration of $C'_\tau$.  So $otp (C'_\tau) \leq \Lambda$ for each of the possibly two $C'_\tau$ sequences.  It only remains to prove that the sequence $\langle C'_\tau \ | \ \tau \in \mathcal{S}^0 \rangle$ is coherent.\\

Pick a $\bar{\tau} \in C'_\tau$.  Assume that $\bar{\tau} > min (C'_\tau )$; otherwise there is nothing to prove.  Since $\bar{\tau} \in C_\tau^*$, we know that $C_{\bar{\tau}}^* = C_\tau^* \cap \bar{\tau}$.  By induction on $\iota$ we show that $\bar{\tau}_\iota = \tau_\iota$ whenever $\iota < \iota_{\bar{\tau}}$.  For $\iota = 0$, this follows immediately, and the same applies to limit $\iota$.\\

It remains to prove that $\bar{\tau}_{\iota + 1} = \tau_{\iota + 1}$, granted that this equality holds with $\iota$ in place of $\iota + 1$.  Here we use the following fact.\\

\textbf{Claim:}  If $X_\tau (\xi) \subset range ( \sigma_{\bar{\tau} \tau}^* )$, then $X_\tau (\xi) = ( \sigma_{\bar{\tau} \tau}^* )`` X_{\bar{\tau}} (\xi )$.\\

\textbf{Proof:}  The inclusion $\supset$ follows immediately, as $\Sigma_1^{(n)}$-statements are upward preserved under $\sigma_{\bar{\tau} \tau}^*$.  Now suppose $y \in X_\tau (\xi)$.  This means that there is an $i \in \omega$ and $\zeta < \varrho_\tau$ such that $( \exists u \in S^E_\zeta ) H_\tau ( u , y , \langle i , \xi \rangle , p_\tau )$; take $\zeta$ to be the least such.  Then $\zeta$ is uniquely characterized by the following $\Sigma_0^{(n)}$-statement:

\[
( \exists u \in S_\zeta^E ) H_\tau ( u , y, \langle i , \xi \rangle , p_\tau ) \ \text{ and }
\]

\[
( \forall \nu \in S_\zeta^E ) (S_\zeta^E = S ( \nu )) \longrightarrow ( \forall u \in \nu ) \neg H_\tau ( u, y, \langle i , \xi \rangle , p_\tau ) \ .
\]

Since $y = h_\tau ( \langle i, \xi \rangle , p_\tau )$, substituting $h_\tau ( \langle i , \xi \rangle , p_\tau )$ for $y$ yields that $\zeta$ is $\Sigma_1^{(n)}$-definable over $N_\tau$ from $\xi$ and $p_\tau$, so $\zeta \in X_\tau (\xi )$.  By our assumption that $X_\tau (\xi) \subset range (\sigma_{\bar{\tau} \tau}^*)$, there are $\bar{y}$, $\bar{\zeta} \in N_{\bar{\tau}}$ such that $y = \sigma_{\bar{\tau} \tau}^* (\bar{y})$ and $\zeta = \sigma_{\bar{\tau} \tau}^* (\bar{\zeta})$.  Then the above formula holds in $N_{\bar{\tau}}$ with $\bar{\tau}$, $\bar{\zeta}$, $\bar{y}$ and $p_{\bar{\tau}}$ in place of $\tau$, $\zeta$, $y$, and $p_{\tau}$; this witnesses that $(\exists u ) H_{\bar{\tau}} ( u, \bar{y}, \langle i , \xi \rangle , p_{\bar{\tau}} )$, i.e., that $\bar{y} = h_{\bar{\tau}} ( \langle i , \xi \rangle , p_{\bar{\tau}} ) \in X_{\bar{\tau}} (\xi)$.  This proves the $\subset$ inclusion and therefore the Claim.  $\blacksquare$\\


Now suppose $\tau_\iota = \bar{\tau}_\iota < \bar{\tau}$.  Recall that $\sigma_{{\tau'} \tau}^* = \sigma_{\bar{\tau} \tau}^* \circ \sigma_{\tau'  \bar{\tau} }^*$ for any $\tau' < \bar{\tau}$.  By the $\supset$ direction of the above Claim, for any $\xi < \Lambda$ we have

\[
X_\tau ( \xi ) \subset range (\sigma_{{\tau_\iota} \tau}^* ) \longrightarrow X_{\bar{\tau}} (\xi ) \subset range ( \sigma_{\tau_\iota \bar{\tau}}^*) \ .
\]

It follows that $\xi_\iota^{\bar{\tau}} \geq \xi_\iota^\tau$.  Since we are assuming that $\bar{\tau} \in C'_\tau$ and $\tau_\iota < \bar{\tau}$, we have $\tau_{\iota + 1} \leq \bar{\tau}$, so $X_\tau ( \xi_\iota^\tau ) \subset range (\sigma_{\bar{\tau} \tau}^*)$.  Using the full strength of the Claim, we obtain

\[
X_\tau ( \xi_\iota^\tau ) = (\sigma_{\bar{\tau} \tau}^*)`` X_{\bar{\tau}} (\xi_\iota^\tau) \ ,
\]

so $X_{\bar{\tau}} ( \xi_\iota^\tau) \not\subset range (\sigma_{\bar{\tau}_\iota \bar{\tau} }^*)$.  It follows that $\xi_\iota^{\bar{\tau}} = \xi_\iota^\tau$.  Letting $\xi_\iota$ be this common value, the above formula guarantees that for every $\tau' \in [ \bar{\tau}_\iota , \bar{\tau} ] \cap C_\tau^*$,

\[
X_\tau ( \xi_\iota ) \subset range (\sigma_{\tau' \tau}^*) \ \leftrightarrow \ X_{\bar{\tau}} (\xi_\iota) \subset range ( \sigma_{\tau' \bar{\tau}}^*) \ ,
\]

which in turn implies that $\bar{\tau}_{\iota + 1} = \tau_{\iota + 1}$.  $\blacksquare$\\

It is now obvious that if we define $\mathcal{C} = \langle \mathfrak{C}_\tau \ | \ \tau \in \mathcal{S} \rangle$ by

\[
\mathfrak{C}_\tau =
\begin{cases}
\{ C'_\tau \} , & \text{if } \tau \in \mathcal{S}^0 \\
\mathfrak{C}_\tau^* , &  \text{if } \tau \in \mathcal{S}^1\\
\end{cases}
\]

then $\mathcal{C}$ is a $\square'_{\Lambda, 2}$-sequence on $\mathcal{S}$.  To complete the construction, we have to give the proof of Lemma \ref{B-tau is a club on a tail-end}.  The next two subsections are devoted to this task.\\








\subsection{Proof of Lemma \ref{B-tau is a club on a tail-end} for $\tau \in \mathcal{S}^0$}









Let $\tau \in \mathcal{S}^0$ be a limit point of $\mathcal{S}$ with uncountable cofinality.  We first define an approximation $D$ to $B_\tau$.  The set $D$ is the set of all $\bar{\tau} \in \mathcal{S} \ \cap \ \tau$ satisfying:\\


\begin{itemize}
\item{$N_{\bar{\tau}}$ is a premouse of the same type as $N_\tau$;}
\item{$n(\tau) = n(\bar{\tau})$;}
\item{There is a map $\sigma_{\bar{\tau} \tau } : N_{\bar{\tau}} \longrightarrow N_\tau$ that is $\Sigma_0^{n(\tau)}$-preserving with respect to the language of premice and such that}
\end{itemize}

\indent \indent $a)$ $\bar{\tau} = crit (\sigma_{\bar{\tau} \tau })$ and $\sigma_{\bar{\tau} \tau } (\bar{\tau}) = \tau $;\\

\indent \indent $b)$ $\sigma_{\bar{\tau} \tau } (p ( N_{\bar{\tau}})) = p ( N_\tau)$;\\

\indent \indent $c)$ for each $\alpha \in p( N_\tau )$ there is a generalized solidity witness $Q_\tau (\alpha )$ for $\alpha$ with respect to $N_\tau$ and $p(N_\tau)$ such that $Q_\tau (\alpha ) \in range ( \sigma_{\bar{\tau} \tau } )$.\\


The only difference between $D$ and $B_\tau$ is that we also allow ordinals from $\mathcal{S}^1$ to be elements of $D$; that is, our interpolant might not be canonical.  Later we prove that there are only boundedly many such ordinals in $\tau$.  Obviously, $B_\tau \subset D$.\\

\begin{lem}
$D$ is unbounded in $\tau$.
\end{lem}

\textbf{Proof:}  Recall that $\tau \in \mathcal{S}^0$, so $N_\tau$ is not pluripotent (short or long).  By Lemma \ref{non-pluripotent interpolants condense to W-levels}, any interpolant of $N_\tau$ such that the interpolation embedding has critical point $\bar{\tau} \in \mathcal{S}$ will be a level of $W$; this is because all points in $\mathcal{S}$ are neither indices nor pseudoindices, so the hypotheses of Lemma \ref{non-pluripotent interpolants condense to W-levels} hold.  Now by Lemma \ref{cofinal interpolants}, there are cofinally many of these interpolated $W$-levels below $\tau$.  This proves the Lemma. $\blacksquare$\\








\begin{lem} \label{S-0 D closed}
$D$ is closed in $\tau$.
\end{lem}

\textbf{Proof:}  Let $\tilde{\tau}$ be a limit point of $D$.  So $\tilde{\tau} \in \mathcal{S}$, as $\mathcal{S}$ is closed.  Form the direct limit $\langle \tilde{N} , \sigma_{\bar{\tau} \tilde{\tau} } \ | \ \bar{\tau} \in D \cap \tilde{\tau} \rangle$ of the diagram $\langle N_{\bar{\tau}} , \sigma_{ \tau^* \bar{\tau}} \ | \ \tau^* \leq \bar{\tau} \ \& \ \tau^* , \bar{\tau} \in D \cap \tilde{\tau} \rangle$.  This direct limit is well-founded, as there is a $\Sigma_0$-preserving embedding $\sigma : \tilde{N} \longrightarrow N_\tau$ defined by $\sigma : \sigma_{\bar{\tau} \tilde{\tau}} (x) \longrightarrow \sigma_{\bar{\tau} \tau} (x)$.  From now on consider $\tilde{N}$ to be transitive.  Notice that $\sigma \circ \sigma_{\bar{\tau} \tilde{\tau}} = \sigma_{\bar{\tau} \tau}$.  For $\xi < \tilde{\tau}$ we have $\sigma_{ \tilde{\tau} \tau} (\xi) = \sigma_{ \bar{\tau} \tau} (\xi) = \xi$ where $\bar{\tau}$ is such that $\xi < \bar{\tau} < \tilde{\tau}$, so $\sigma_{ \tilde{\tau} \tau} \restriction \tilde{\tau} = id$.  Also, the thread $\langle \bar{\tau} \ | \ \bar{\tau} \in D \ \cap \ \tilde{\tau} \rangle$ clearly represents $\tilde{\tau}$ in $\tilde{N}$, so $\tilde{\tau} = \sigma_{ \bar{\tau} \tilde{\tau} } (\bar{\tau})$ and $\sigma_{ \tilde{\tau} \tau} (\bar{\tau}) = \tau$.  It follows that $\tilde{\tau} = (\Lambda^+)^{\tilde{N}}$ and $\tilde{\tau} = crit(\sigma_{ \tilde{\tau} \tau})$.  It is easy to see that the maps $\sigma_{ \bar{\tau} \tilde{\tau} }$ are $\Sigma_0^{(n)}$-preserving, where $n = n_\tau$.  These preservation properties hold with respect to the language of coherent structures, as it is not clear that $\tilde{N}$ is a premouse of the same type as $N_\tau$ and the constant $\dot{\gamma}$ is correctly interpreted in $\tilde{N}$.  Our aim is to show that $\tilde{N} = N_{\tilde{\tau}}$, and again we intend to use the condensation lemma.  Thus, we have to show that its hypotheses are met.\\

The first step towards this is the verification that $\tilde{N}$ is a premouse of the same type as $N_\tau$.  To see that $\tilde{N}$ is a potential premouse, notice that $\Pi_2$-properties which hold on a tail-end are upward preserved under direct limit maps.  We know that each $N_{\bar{\tau}}$ is of the same type as $N_\tau$.  If they are type $A$, then so is $\tilde{N}$, as the statement 

\[
(\forall \bar{\lambda} < \lambda_N ) (\bar{\lambda} \text{ is not a cutpoint of $N$'s top extender})
\]

is $\Pi_2 (N)$ for any $N$.  If $N_\tau$ is type $B$, then $\sigma_{\tau^* \bar{\tau}} ( \gamma_{N_{\tau^*}} ) = \gamma_{N_{\bar{\tau}}}$, so $\sigma_{\tau^* \bar{\tau}} ( \lambda^*_{N_{\tau^*}}) = \lambda^*_{N_{\bar{\tau}}}$, as $\lambda^*_N$ is the largest cardinal in $J_{\gamma^N}^{E^N}$ for any $N$.  Set $(\tilde{\gamma} , \tilde{\lambda}^*) = \sigma_{ \bar{\tau} \tilde{\tau}} ( \gamma_{N_{\bar{\tau}}} , \lambda_{N_{\bar{\tau}}}^* )$; the preservation properties of the direct limit maps then guarantee that $\tilde{\lambda}^*$ is a cutpoint of $\tilde{N}$'s top extender $F$ and $F \restriction \tilde{\lambda}^* = E^{\tilde{N}}_{\tilde{\gamma}}$.  Now, exactly as in the case of type $A$ premice, we can show there are no cutpoints of $F$ larger than $\tilde{\lambda}^*$.  Hence $\tilde{N}$ is type $B$ as well, $\tilde{\lambda}^* = \lambda_{\tilde{N}}^*$ and $\tilde{\gamma} = \gamma_{\tilde{N}}$.  It remains to discuss the case where $N_\tau$ is type $C$.  Then $\varrho_1 (N_{\bar{\tau}}) = \lambda_{N_{\bar{\tau}}} > \Lambda$, as both $N_\tau$ and all $N_{\bar{\tau}}$ are in $\mathcal{S}$.  It follows that $n > 0$, so $\Pi_2^{(1)}$-statements which hold on a tail-end of $D \cap \tilde{\tau}$ are upwards preserved under the direct limit maps.  Notice also that $\varrho_1 (\tilde{N}) = \bigcup \{ \sigma``_{\bar{\tau} \tilde{\tau}} ( \varrho_1 (N_{\bar{\tau}}) ) \ | \ \bar{\tau} \in D \cap \tilde{\tau} \} = \lambda_{\bar{N}}$.  That $\tilde{N}$ is of type $C$ can be expressed by the $Q^{(1)}$-statement

\[
(\forall \zeta) (\exists \bar{\lambda} \geq \zeta ) (\bar{\lambda} \text{ is a cutpoint of $\tilde{N}$'s top extender}) \ ,
\]

so it is true in $\tilde{N}$ by the preservation properties of the direct limit maps.  Now given any $\bar{\lambda} < \lambda_{\tilde{N}}$, and letting $F$ be the top extender of $\tilde{N}$,\\

\[
F \restriction \bar{\lambda} = \{ \langle x , y \rangle \ | \ ( \exists y' ) ( \langle x , y' \rangle \in F \ \& \ y = y' \cap \bar{\lambda} \} \ ,
\]

so $F \restriction \bar{\lambda}$ is $\Sigma_1 (N)$-definable in $\bar{\lambda}$.  But we have seen that $\bar{\lambda} < \varrho_1 (\tilde{N}) = \lambda_{\tilde{N}}$, so $F \restriction \bar{\lambda}$, being a bounded subset of $\lambda_{\tilde{N}}$, must be in $\tilde{N}$.  This proves that $\tilde{N}$ is a type $C$ premouse.  From now on we know that the preservation properties of all maps $\sigma_{\bar{\tau} \tilde{\tau}}$ and $\sigma$ hold with respect to the language for premice.\\

Let $\tilde{p} = \sigma_{\bar{\tau} \tilde{\tau}} (p_{\bar{\tau}})$ for $\bar{\tau} \in D \cap \tilde{\tau}$.  Given an $x \in \tilde{N}$, there is a $\bar{\tau} \in D \cap \tilde{\tau}$ and an $\bar{x} \in N_{\bar{\tau}}$ such that $x = \sigma_{\bar{\tau} \tilde{\tau}} (\bar{x})$.  By the soundness of $N_{\bar{\tau}}$, there is a $\xi < \Lambda$ satisfying $\bar{x} = h_{\bar{\tau}} ( \xi , p_{\bar{\tau}})$.  This statement, being $\Sigma_1^{(n)}$, is upward preserved by $\sigma_{\bar{\tau} \tilde{\tau}}$, so $x = h_{n+1}^{\tilde{N}} (\xi , \tilde{p})$.  Since $x$ was arbitrary, $\tilde{N} = h_{n+1}^{\tilde{N}} (\Lambda \cup \{ \tilde{p} \} )$.  This shows that $\varrho_{\omega} ( \tilde{N}) = \varrho_{n+1} ( \tilde{N}) = \Lambda$ and $\tilde{p}$ is a very good parameter for $\tilde{N}$.  Now notice that $\sigma_{\tilde{\tau} \tau} (\tilde{p}) = p_\tau$ and that for any $\alpha \in p_{N_\tau}$, there is a generalized witness for $\alpha$ with respect to $N_\tau$ and $p_\tau$ in $range (\sigma)$.  The latter follows from the fact that $range(\sigma) \supset range ( \sigma_{\bar{\tau} \tau}) $ and that $range ( \sigma_{\bar{\tau} \tau})$ contains such witnesses, as is ensured by $c)$ in the definition of $D$.  But being a generalized solidity witness is a $\Pi_1^{(n)}$-property, so it is downward preserved by $\sigma$; this means $\tilde{N}$ has generalized solidity witnesses for $\tilde{p}$.  It now follows that $\tilde{N}$ is sound and $\tilde{p} = p_{\tilde{N}}$, and consequently that $\tilde{N} = N_{\tilde{\tau}}$, $n_{\tilde{\tau}} = n$ and $\sigma = \sigma_{\tilde{\tau} \tau}$.  Thus $\tilde{\tau} \in D$. $\blacksquare$\\

The following lemma contains the crucial combinatorial details which are needed for proofs of $\square$ in fine-structural inner models.  The main obstacle in these proofs seems to be the following: Many of our interpolated levels will be protomice of various sorts, and it is generally not too difficult to prove that these protomice correspond to levels of $W$, that is, there is a divisor of a level of $W$ which reproduces the protomouse.  The hard part is defining \textit{canonical} divisors of these levels of $W$, in such a way that we can ensure our interpolated protomice correspond to the canonical divisors.  The following lemma includes the key step in this argument: if cofinally many interpolated protomice were \textit{not} the canonical ones, then there are ``better" divisors of all the corresponding levels of $W$.  By pigeonholing, we want to turn these better divisors into something like a thread through the chain of interpolants, and prove that the image of this thread at the top of the chain will likewise be a better divisor than the one we used to form the interpolation-chain.  Therefore we should not have been using that divisor to form our chain in the first place.\\


\begin{lem} \label{divisor push-up}
$D$ is a subset of $\mathcal{S}^0$ on a tail-end, i.e., there is a $\bar{\tau} < \tau$ such that $D - \bar{\tau} \subset \mathcal{S}^0$.
\end{lem}

\textbf{Proof:}  Suppose the contrary, i.e., let $\langle \tau_\iota \ | \ \iota < \delta \rangle$ be an increasing sequence cofinal in $\tau$ such that each $\tau_\iota$ is in $\mathcal{S}^1$.  By the pigeonhole principle, we may assume that one of the following holds:\\

$i)$ Every $N_{\tau_\iota}$ has a canonical short divisor $(\kappa_{\tau_\iota} , q_{\tau_\iota} )$ and (by further application of the pigeonhole principle) for all $\iota < \delta$, $|q_{\tau_\iota}| = m$ for some fixed $m \in \omega$;\\

$ii)$ Every $N_{\tau_\iota}$ has a canonical long divisor $\nu_\iota$ and for all $\iota < \delta$, $|p_{\tau_\iota} \ \cap \ \nu_\iota| = m$ for some fixed $m \in \omega$;\\

$iii)$ Every $N_{\tau_\iota}$ has a canonical type $2$ long divisor $\nu_\iota$ and for all $\iota < \delta$, $|p_{\tau_\iota} \ \cap \ \nu_\iota| = m$ for some fixed $m \in \omega$.\\

In each of these three cases, we will show that the cofinal sequence of divisors can be ``pushed up the elementary chain" given by $D$, and prove that a divisor of the corresponding type is present in $N_\tau$ as well.  This will contradict the fact that $\tau \in S^0$, and thereby prove the lemma.\\



First we consider the case $i)$, in which there are cofinally many $N_{\tau_\iota}$ with canonical strong short divisors.  We can assume that the sequence $\langle \kappa_{\tau_\iota} \ | \ \iota < \delta \rangle$ is monotone (not necessarily strictly), as we can always replace it by a monotonic subsequence $\langle \kappa_{\tau_{\iota (\xi )}} \ | \ \xi < \delta' \rangle$, where $\iota (\xi)$ is inductively defined by

\[
\begin{split}
& \iota^* (\xi) = sup \{ \iota ( \bar{\xi} ) + 1 \ | \ \bar{\xi} < \xi \} \ ;\\
& \iota (\xi) = \text{ the least $\iota$ such that } \iota^* (\xi) \leq \iota < \delta \text{ and } \kappa_{\tau_\iota} = min \{ \kappa_{\tau_\eta} \ | \ \iota^* (\xi) \leq \eta < \delta \} \ ,
\end{split}
\]

where $sup(\emptyset) = 0$.  Let $\mu = sup \{ \kappa_{\tau_\iota} \ | \ \iota < \delta \}$, let $q$ be the bottom segment of $p_\tau$ with exactly $m$ elements and let $r = p_\tau - q$.  Notice that $\kappa \leq \Lambda$ and that $q = \sigma_{\tau_\iota \tau} (q_{\tau_\iota})$ and $r = \sigma_{\tau_\iota \tau} (r_{\tau_\iota})$ for all $\iota < \delta$.\\


We show that $( \kappa , q)$ is a divisor of $N_\tau$ by verifying all clauses in the definition of a divisor.  From the proof of the previous lemma we know that $\langle N_\tau , \sigma_{\tau_\iota \tau} \ | \ \iota < \delta \rangle$ is the direct limit of the diagram $\langle N_\tau , \sigma_{\tau_{\bar{\iota}}  \tau_\iota } \ | \ \bar{\iota} \leq \iota < \delta \rangle$.  Notice first that

\[
( * ) \ \ \ \ \ \varrho_\tau = \bigcup_{\iota < \delta } \sigma``_{\tau_\iota \tau} ( \varrho_{\tau_\iota } ) \ .
\]

We further observe that\\

\[
( \dagger ) \ \ \ \ \ h_\tau ( \kappa \cup \{ r \} ) = \bigcup_{\iota < \delta } \sigma``_{\tau_\iota \tau} h_{\tau_\iota } ( \kappa_{\tau_\iota} \cup \{ r_{\tau_\iota} \} ) \ .
\]

The inclusion $\supset$ follows from the fact that $\Sigma_1^{(n)}$-statements are upward preserved under $\sigma_{\tau_\iota \tau}$, so if $y = h_{\tau_\iota} ( \xi , r_{\tau_\iota} )$ for some $\xi < \kappa_{\tau_\iota}$ then $\sigma_{\tau_\iota \tau} (y) = h_\tau (\xi , r)$.  The converse follows from the property of direct limits that $\Sigma_1^{(n)}$-statements are downward preserved on a tail-end.  So if $y = h_\tau (\xi , r)$, then for a sufficiently large $\iota < \delta$, the value $y_\iota = h_{\tau_\iota} ( \xi , r_{\tau_\iota} )$ is defined and $\sigma_{\tau_\iota \tau} (y_\iota) = y$.  This proves $( \dagger )$.  Now as $(\kappa_{\tau_\iota } , q_{\tau_\iota} )$ is a divisor for $N_{\tau_\iota}$ for any $\iota < \delta$, each hull $h_{\tau_\iota} ( \kappa_{\tau_\iota} \cup \{ r_{\tau_\iota } \} )$ is cofinal in $\varrho_{\tau_\iota}$.  This together with $(*)$ and $(\dagger )$ yields that $h_\tau ( \kappa \cup \{ r \} )$ is cofinal in $\varrho_\tau$, which proves $b)$ in the definition of short divisor (Definition \ref{short divisor}).\\

To verify $c)$ of Definition \ref{short divisor}, we show

\[
h_\tau ( \kappa \cup \{ r \} ) \cap ( max (q) + 1 ) \subset \kappa \ .
\]

By convention we set $max ( \emptyset ) = \Lambda + 1$ here.  Fix a $\zeta \leq max (q)$ such that $\zeta = h_\tau (\xi , r)$ for some $\xi < \kappa$.  Pick an $\iota < \delta$ such that $\bar{\iota} = h_{\tau_\iota} ( \xi , r_{\tau_\iota})$ is defined and so $\zeta = \sigma_{\tau_\iota \tau} (\bar{\zeta})$.  Hence $\bar{\zeta} \leq max ( q_{\tau_\iota})$.  As $(\kappa_{\tau_\iota} , q_{\tau_\iota})$ is a divisor, $\bar{\zeta} < \kappa_{\tau_\iota}$.  So $\zeta = \sigma_{\tau_\iota \tau} ( \bar{\zeta}) = \bar{\zeta} < \kappa_{\tau_\iota} \leq \kappa$, which proves the above formula.  Thus, setting $\lambda$ to be the least ordinal in $h_\tau (\kappa , r ) - \kappa$, we have $\lambda > max (q_\tau)$.  On the other hand, $\lambda \leq \sigma_{\tau_\iota \tau} ( \lambda_{\tau_\iota}) < \varrho_\tau$; the former inequality is a consequence of the fact that $\sigma_{\tau_\iota \tau} ( \lambda_{\tau_\iota})$ is obviously in $h_\tau ( \kappa \cup \{ r \} )$.  This verifies that $(\kappa , q )$ is a short divisor of $N_\tau$.\\


It must be that $(\kappa , q)$ is not strong, as no $\tau \in \mathcal{S}^0$ admits a strong short divisor (or else it would have a canonical divisor and hence be in $\mathcal{S}^1$).  By Lemma \ref{characterization of strongness}, $p_{N'_\tau}$ is a proper lengthening of $r' = (\pi')^{-1} (r)$ where $\pi' : N'_\tau \longrightarrow N_\tau$ is the uncollapsing embedding associated with the $\Sigma_1^{(n)}$-hull $h_\tau ( \kappa_\tau \cup \{ p_\tau \} )$.  Let $\beta'$ be the largest element of $p_{N'_\tau} - r'$.  Then $W_{N'_\tau}^{\beta' , r'} \in N'_\tau$, as the preservation properties of $\pi'$ combined with the solidity theorem guarantee that $N'_\tau$ is solid.  Letting $Q = \pi' ( W_{N'_\tau}^{\beta' , r'} )$ and $t = \pi' (t')$ where $t'$ is the preimage of $r'$ under the associated canonical witness map, $\langle Q , t \rangle$ is a generalized witness for $\beta = \pi' (\beta')$ with respect to $N_\tau$ and $r$.  Let $\xi , \eta < \kappa$ be such that $\langle Q , t \rangle = h_\tau ( \xi , p_\tau )$ and $\beta = h_\tau (\eta , p_\tau )$; such ordinals exist, as $range ( \pi') = h_\tau (\kappa_\tau \cup \{ p_\tau \} )$.  Let $\iota < \delta$ be sufficiently large so that $\eta , \xi < \kappa_{\tau_\iota}$ and both $\bar{\beta} = h_{\tau_\iota} ( \eta , p_{\tau_\iota})$ and $\bar{T} = h_{\tau_\iota} ( \xi , p_{\tau_\iota})$ are defined.  Then $\sigma_{\tau_\iota \tau } ( \bar{\beta} , \bar{T} ) = ( \beta , \langle Q , t \rangle )$, so $\bar{T}$ is of the form $\langle \bar{Q} , \bar{t} \rangle$ and is a generalized witness for $\bar{\beta}$ with respect to $N_{\tau_\iota}$ and $r_{\tau_\iota}$; to see this, recall that being a generalized witness is a $\Pi_1^{(n)}$-property, and therefore is downward preserved under the $\Sigma_0^{(n)}$-preserving map $\sigma_{\tau_\iota \tau }$.  Moreover, both $\bar{\beta}$ and $\langle \bar{Q} , \bar{t} \rangle$ are in the range of $\pi'_{\tau_\iota} = h_{\tau_\iota} ( \kappa_\iota \cup \{ p_\iota \} )$ by our choice of $\eta$, $\xi$, and $\iota$.  Finally $\bar{\beta} \geq \kappa_{\tau_\iota}$, as $\beta \geq \kappa$ (this follows easily from the definition of $\beta$) and $\kappa_\iota \leq \kappa$.  Thus, if $( \bar{\beta} , \langle \bar{Q} , \bar{t} \rangle ) = \pi'_{\tau_\iota} ( \bar{\beta}' , \langle \bar{Q}' , \bar{t}' \rangle )$, then $\bar{\beta}' \geq \kappa_{\tau_\iota}$ and $\langle \bar{Q}' , \bar{t}' \rangle$ is a generalized witness for $\bar{\beta}'$ with respect to $N'_{\tau_\iota} (\kappa_{\tau_\iota})$ and $r'_{\tau_\iota}$, that is, an element of $N'_{\tau_\iota} (\kappa_{\tau_\iota})$.  Then the standard witness $W' = W_{N'_{\tau_\iota} (\kappa_{\tau_\iota})}^{\bar{\beta}' , r'_{\tau_\iota}}$ is an element of $N'_{\tau_\iota} (\kappa_{\tau_\iota})$ as well.  Let $t'_{\tau_\iota}$ be the preimage of $r'_{\tau_\iota}$ under the associated canonical witness map.  Since this map is $\Sigma_1^{(n)}$-preserving, for any set $A$ which is $\Sigma_1^{(n)} (N'_{\tau_\iota} (\kappa_{\tau_\iota}))$ in the parameter $r'_{\tau_\iota}$ we can find some $A'$ which is $\Sigma_1^{(n)} (W')$-definable in the parameter $t'_{\tau_\iota}$ such that $A \cap \bar{\beta}' = A' \cap \bar{\beta}' \in N'_{\tau_\iota} (\kappa_{\tau_\iota})$.  Then $r'_{\tau_\iota}$ is not a good parameter for $N_{\tau_\iota}$, as $\bar{\beta}' \geq \kappa_{\tau_\iota}$.  In other words, $p_{N'_{\tau_\iota} (\kappa_{\tau_\iota})}$ is a proper lengthening of $r'_{\tau_\iota}$.  By Lemma \ref{characterization of strongness}, the divisor $( \kappa_{\tau_\iota} , q_{\tau_\iota})$ is not strong, contradiction.\\


We have shown that $i)$ is impossible: only boundedly many of the $N_{\tau_\iota}$'s can have canonical short divisors.  We now give a similar argument to rule out $ii)$.  The argument is similar, but in fact simpler.  Because we are assuming $|p_{\tau_\iota} \cap  \nu_\iota| = m$ for some fixed $m \in \omega$, and the embedding-chain $\langle N_\tau , \sigma_{\tau_{\bar{\iota}}  \tau_\iota } \ | \ \bar{\iota} \leq \iota < \delta \rangle$ preserves standard parameters, it follows that the long divisors $\nu_\iota$ are a thread through this direct limit system.  Since each $\nu_\iota$ indexes an extender $E_{\nu_\iota}$ with $crit(E_{\nu_\iota}) = \kappa_\iota < \Lambda$, and the critical points of the maps $\sigma_{\tau_{\bar{\iota}}  \tau_\iota }$ are all $> \Lambda$, we have by $\Sigma_0^{(n)}$-elementarity of $\sigma_{\tau_{\bar{\iota}}  \tau_\iota }$ that all $\kappa_\iota$ are the same ordinal; call it $\kappa$.  Likewise, for $\alpha < \kappa^+$, $\alpha$ is below the critical points of all $\sigma_{\tau_{\bar{\iota}}  \tau_\iota }$, and is therefore a thread; and it follows that the short extenders $E_{\nu_\iota}$ have the property that $i_{E_{\nu_\iota}} (\alpha)$ is a thread, for every such $\alpha$ (that is, $i_{E_\nu} (\alpha) = \sigma_{\tau_{\iota}  \tau }( i_{E_{\nu_\iota}} (\alpha))$ for all $\alpha < \kappa^+$ and $\iota < \delta$).\\

We want to verify that $\nu = \sigma_{\tau_\iota  \tau} ( \nu_\iota )$ is a long divisor of $N_\tau$, so we must check that the clauses of \ref{long divisor} hold.  $a)$ follows trivially by elementarity.  Because $i_{E_\nu} (\alpha) = \sigma_{\tau_{\iota}  \tau }( i_{E_{\nu_\iota}} (\alpha))$ for all $\alpha < \kappa^+$ and $\iota < \delta$, it is easy to see that $b)$ holds as well.  To prove $c)$, we suppose $\xi \in Hull_{n+1}^{N_\tau} ( i_{E_\nu}`` ( \kappa^+))$, so for some $\alpha < \kappa^+$ we have $\xi = h_{n+1}^{N_\tau} ( i_{E_\nu} (\alpha) , r)$.  But then for sufficiently large $\iota$, $h_{n+1}^{N_{\tau_\iota}} ( i_{E_{\nu_\iota}} ( \alpha) , r_\iota) = \xi_\iota$ is defined, and since $\nu_\iota$ is a long divisor of $N_{\tau_\iota}$ we must have $\xi_\iota \in Hull_{n+1}^{N_{\tau_\iota}} ( i_{E_{\nu_\iota}}`` ( \kappa^+))$.  By elementarity, the same is true for $\xi$; this verifies $c)$.  Finally, $d)$ follows by a similar argument: if $\lambda^+$ were the space of an extender on the $N_\tau$-sequence, then for sufficiently large $\iota$, the same is true for $N_{\tau_\iota}$, contradicting the fact that all $\nu_\iota$ are long divisors.  (Here we are using the fact that $\langle \lambda_{E_{\nu_\iota}}^+ \ | \ \iota < \delta \rangle$ form a thread in our direct limit system, which is an easy consequence of elementarity.)  This finishes the verification that $\nu$ is a long divisor of $N_\tau$, contradiction.\\

Finally we must deal with the case $iii)$, in which a cofinal set of $N_{\tau_\iota}$'s have canonical type $2$ long divisors.  As in case $ii)$ above, the ordinals $\nu_\iota$ form a thread through $\langle N_\tau , \sigma_{\tau_{\bar{\iota}}  \tau_\iota } \ | \ \bar{\iota} \leq \iota < \delta \rangle$.  We want to verify that $\nu = \sigma_{\tau_\iota  \tau} ( \nu_\iota )$ is a type $2$ long divisor of $N_\tau$, so we must check that the clauses of Definition \ref{type 2 long divisor} hold.  As before, $a)$ follows trivially by elementarity, and $i_{E_\nu} (\alpha) = \sigma_{\tau_{\iota}  \tau }( i_{E_{\nu_\iota}} (\alpha))$ for all $\alpha < \kappa^+$ and $\iota < \delta$.  That is, using the notation of Definition \ref{type 2 long divisor}, the points in $Z$ are all threads through the direct limit system.  Also note that the ordinals $\mu_\iota = sup ( i_{E_{\nu_\iota}}``(\kappa^+))$ form a thread.  Since all $\nu_\iota$'s are type $2$ long divisors of their respective $N_{\tau_\iota}$'s, and the relations $R_\iota$ of Definition \ref{type 2 long divisor} for type $2$ long divisors are uniformly definable from $\mu_\iota$, we can argue exactly as before that $Hull_{n+1}^{( |N_\tau | , R )} ( Z \cup r )$ (the divisor-hull corresponding to $\nu$) satisfies $b)$ and $c)$ of Definition \ref{type 2 long divisor}.  Clause $d)$ also follows by elementarity exactly as before.\\
 


In all three cases, we have shown that cofinally many divisors in the embedding-chain would induce the existence of a divisor at the top, contradicting that $\tau \in \mathcal{S}^0$.  This proves the Lemma. $\blacksquare$\\



Let $\bar{\tau}$ be minimal with the property that $D - \bar{\tau} \subset \mathcal{S}^0$.  By the previous lemmata, $\bar{\tau} < \tau$ and $B_\tau - \bar{\tau} = D - \bar{\tau}$ is closed and unbounded in $\tau$.  This completes the proof of Lemma \ref{B-tau is a club on a tail-end} for $\tau \in \mathcal{S}^0$.\\










\subsection{Proof of Lemma \ref{B-tau is a club on a tail-end} for $\tau \in \mathcal{S}^1$}












Here we adopt the same strategy as for $\tau \in \mathcal{S}^0$.  We again assume that $\tau$ is a limit point of $\mathcal{S}$, of uncountable cofinality, but now, of course, $\tau \in \mathcal{S}^1$.  This means $M_\tau$, or  $M_\tau^{short}$ and $M_\tau^{long} $, are defined.  For the remainder of this subsection we fix one of these models and simply call it $M_\tau$.  There are five cases for what $M_\tau$ might look like:\\

Case $1)$ \ $M_\tau = N_\tau$ is a short pluripotent level of $W$;\\

Case $2)$ \ $M_\tau$ is the strong short protomouse associated with a canonical short divisor $(\kappa , q)$ of $N_\tau$;\\

Case $3)$ \ $M_\tau = N_\tau$ is a long pluripotent level of $W$;\\

Case $4)$ \ $M_\tau$ is the long protomouse associated with a canonical long divisor $\nu$ of $N_\tau$;\\

Case $5)$ \ $M_\tau$ is the type $2$ long protomouse associated with a canonical type $2$ long divisor $\nu$ of $N_\tau$.\\


In all five of these cases, our approach will be the same.  We first define $D$ to be the set of all $\bar{\tau} \in \mathcal{S}^1 \cap \tau$ such that there is an interpolant $M^*_{\bar{\tau}}$ of $M_\tau$ satisfying the conditions for $\bar{\tau} \in B_\tau$ (in particular, $M^*_{\bar{\tau}}$ is the protomouse associated with a divisor of $N_{\bar{\tau}}$), \textit{except} that $M^*_{\bar{\tau}}$ may not be a \textit{canonical} protomouse associated with $N_{\bar{\tau}}$.  We would like to prove that on a tail-end, these interpolants are in fact canonical protomice; so we suppose that cofinally many of our $M^*_{\bar{\tau}}$'s are `beaten' by some other divisor of $N_{\bar{\tau}}$ with a larger divisor-hull.  The heart of the entire $\square_\Lambda$ proof is the fact that this cofinal set of larger-hull divisors can be pushed up the elementary chain of $M^*_{\bar{\tau}}$'s to induce the existence of a larger divisor-hull of $N_\tau$ as well, which contradicts the fact that $M_\tau$ was the canonical divisor.\\

The basic structure of this proof is therefore similar to the proof given in the previous subsection for when $\tau \in \mathcal{S}^0$.  However, in that proof we were given an embedding-chain through our set of interpolants $D$ which were all levels of $W$, and had to `push up the existence of a divisor' through this chain of $W$-levels.  In the current context, the embedding-chain is through a set of interpolants $D$ which are all protomice associated with $W$-levels, and our goal is to `push up the existence of a divisor with larger divisor-hull' through this chain.  We must therefore make heavy use of the finestructural translation lemmas between protomice and their associated ppm's, because our embeddings are between protomice $M^*_{\bar{\tau}}$, but the divisors we want to push up are divisors of the associated ppm's $N_{\bar{\tau}}$.\\

Recall that we can treat short pluripotent levels as a limiting case of short protomice in which the top extender is in fact total, and similarly for long protomice.  In what follows we will treat Cases $1)$ and $2)$ together, and afterwards treat Cases $3)$, $4)$, and $5)$ together.\\





\textbf{Proof of Lemma \ref{B-tau is a club on a tail-end} in Cases $1)$ and $2)$}:\\







Recall that in Case $1)$, we set $(\kappa_\tau , q_\tau ) = (\kappa_{G_{N_\tau}} , d(N_\tau ) )$, where $G_{N_\tau}$ is the top extender of $N_\tau$.  In Case $2)$, $(\kappa_\tau , q_\tau )$ is the canonical short divisor of $N_\tau$.\\

Let $D$ be the set of all $\bar{\tau} \in \mathcal{S}^1 \cap \tau$ satisfying:


\begin{itemize}
\item{Setting $q_{\bar{\tau}^*}$ to be the bottom segment of $p_{\bar{\tau}}$ of length $= | q_\tau |$, the pair $(\kappa_\tau , q_{\bar{\tau}}^*)$ is a strong short divisor of $N_{\bar{\tau}}$;}
\item{Setting $M_{\bar{\tau}}^* = N_{\bar{\tau}} (\kappa_\tau , q_{\bar{\tau}}^*)$, there is a map $\sigma_{\bar{\tau} \tau } : M_{\bar{\tau}}^* \longrightarrow M_\tau$ that is $\Sigma_0$-preserving with respect to the language of coherent structures and such that}
\end{itemize}

\indent \indent $a)$ $\bar{\tau} = crit (\sigma_{\bar{\tau} \tau })$ and $\sigma_{\bar{\tau} \tau } (\bar{\tau}) = \tau $;\\

\indent \indent $b)$ $\sigma_{\bar{\tau} \tau } (\bar{q}_{\bar{\tau}}^*) = q_\tau$;\\

\indent \indent $c)$ for each $\alpha \in q_\tau$ there is a generalized solidity witness $Q_\tau (\alpha )$ for $\alpha$ with respect to $M_\tau$ and $q_\tau$ such that $Q_\tau (\alpha ) \in range ( \sigma_{\bar{\tau} \tau } )$.\\


Generalized witnesses in clause $c)$ are, of course, computed in the language for coherent structures.  Notice that $M_{\bar{\tau}}^*$ is \textit{never} a pluripotent level of $W$, even if $M_\tau$ is, as its top extender cannot measure all subsets of $\kappa_\tau$.  This follows from the fact that $\sigma_{\bar{\tau} \tau}$ is not cofinal.  We first prove that $D$ is closed and unbounded in $\tau$, and then that $( \kappa_\tau , q_{\bar{\tau}}^*) = ( \kappa_{\bar{\tau}} , q_{\bar{\tau}})$ on a tail-end of $D$.\\

\begin{lem}
$D$ is unbounded in $\tau$.
\end{lem}

\textbf{Proof:}  By Lemma \ref{cofinal interpolants}, there are cofinally many $\bar{\tau} < \tau$ such that there is an interpolant $M^*_{\bar{\tau}}$ of $M_\tau$ with the interpolation embedding mapping its critical point $\bar{\tau}$ to $\tau$.  By \ref{short pluripotent interpolates to short protomouse} for Case $1)$ or \ref{associated short protomouse interpolates to short protomouse} for Case $2)$, $M^*_{\bar{\tau}}$ is a short protomouse.  By \ref{short protomouse condensation}, the associated ppm of $M^*_{\bar{\tau}}$ is $N_{\bar{\tau}}$, the collapsing-level for $\bar{\tau}$ in $W$; this means $(\kappa_\tau , q_{\bar{\tau}}^*)$ is a short divisor of $N_{\bar{\tau}}$, with associated protomouse $M^*_{\bar{\tau}}$.  Now we must show that without loss of generality $(\kappa_\tau , q_{\bar{\tau}}^*)$ is a strong short divisor of $N_{\bar{\tau}}$: this will follow from Lemma \ref{strongness is the same as theta closed}, which tells us that $(\kappa_\tau , q_{\bar{\tau}}^*)$ is strong if and only if $\theta_{\bar{\tau}}$ is closed in $M^*_{\bar{\tau}}$ relative to $d(M^*_{\bar{\tau}})$.\\

Recall that $\theta_\tau$ is closed in $M_\tau$ relative to $d(M_\tau)$; if $M_\tau$ is a short protomouse this is immediate from our demand that the canonical short divisor of $N_\tau$ must be strong, and if $M_\tau$ is short pluripotent then $\theta_\tau = \kappa_\tau^+$ and the conclusion is trivial.  It follows by a simple closure argument that there are unboundedly many $\theta^* < \theta_\tau$ in $M_\tau$ which are closed in $M_\tau$ relative to $d(M_\tau)$ (see \cite{zeman square proof} Lemma 3.10 for further details).  Now when we form interpolants of $M_\tau$ using the construction of Lemma \ref{cofinal interpolants}, we first take a countable fully elementary hull $X' \prec_{\Sigma_\omega} M_\tau$, which will also have unboundedly many $\theta^* < \theta_\tau$ which are closed in $X'$ relative to $d(X')$.  Then we have a cofinal map $i : X \longrightarrow M^*_{\bar{\tau}}$ as given in Lemma \ref{cofinal interpolants}, where $\pi : X \longrightarrow X'$ is the uncollapse.  Note that $\pi$ maps $\bar{\theta} = dom (G_X) = \pi^{-1} ( \theta_\tau )$ cofinally into $\theta_{\bar{\tau}} = dom (G_{M_{\bar{\tau}}})$.  But then the cofinally many closure-points below $\bar{\theta}$ are mapped to cofinally many $\theta^* < \theta_{\bar{\tau}}$ which are closed in $M^*_{\bar{\tau}}$ relative to $d(M^*_{\bar{\tau}})$.  Therefore $\theta^*$ is so closed as well, which shows that it corresponds to a strong short divisor. $\blacksquare$\\






\begin{lem} \label{S-1 short D closed}
$D$ is closed in $\tau$.
\end{lem}

\textbf{Proof:}  We shall closely follow the proof of Lemma \ref{S-0 D closed}.  Let $\tilde{\tau} < \tau$ be a limit point of $D$.  Then $\tilde{\tau} \in \mathcal{S}$.  Form the direct limit $\langle \tilde{M} , \sigma_{\bar{\tau} \tilde{\tau}} \ | \ \bar{\tau} \in D \cap \tilde{\tau} \rangle$ of the diagram $\langle M_{\bar{\tau}}^* , \sigma_{\tau^* \bar{\tau}} \ | \ \tau^* \leq \bar{\tau} \ \& \ \tau^* , \bar{\tau} \in D \cap \tilde{\tau} \rangle$.  As before we have the $\Sigma_0$-preserving map $\sigma : \tilde{M} \longrightarrow M$ defined by $\sigma_{\bar{\tau} \tilde{\tau}} (x) \longrightarrow \sigma_{\bar{\tau} \tau} (x)$, so $\tilde{M}$ is well-founded and we can consider it to be transitive.  The arguments from the proof of Lemma \ref{S-0 D closed} can be modified in a straightforward way to obtain the following properties of $\tilde{M}$ and $\sigma$.  In the clauses below, $\bar{\tau}$ is an arbitrary element of $D \cap \tilde{\tau}$.

\begin{itemize}
\item{$\tilde{M}$ is a coherent structure.}
\item{$\sigma_{\bar{\tau} \tilde{\tau}} (\bar{\tau}) = \tilde{\tau}$ and $\sigma \circ \sigma_{\bar{\tau} \tilde{\tau}} = \sigma_{\bar{\tau} \tau}$.}
\item{$crit(\sigma) = \tilde{\tau}$ and $\sigma (\tilde{\tau}) = \tau$.}
\item{$h_{\tilde{M}} ( \Lambda \cup \{ \tilde{q} \} ) = \tilde{M}$, where $\tilde{q} = \sigma_{\bar{\tau} \tilde{\tau}} ( q_{\bar{\tau}}^* )$, so $\varrho_\omega (\tilde{M}) = \varrho_1 (\tilde{M}) = \Lambda$ and $\tilde{q}$ is a very good parameter for $\tilde{M}$.}
\item{$\sigma(\tilde{q}) = q_\tau$.}
\item{Let $\tilde{\alpha} \in \tilde{q}$ and $\sigma (\tilde{\alpha}) = \alpha$.  So $\alpha \in q_\tau$.  If $\langle Q ( \alpha ) , t(\alpha ) \rangle$ is a generalized witness for $\alpha$ with respect to $M_\tau$ and $q_\tau$ and $\langle Q (\alpha ) , t ( \alpha ) \rangle \in range ( \sigma_{\bar{\tau} \tau} )$ for some $\bar{\tau} \in D \cap \tau$ then $ \langle Q (\tilde{\alpha}), t ( \tilde{\alpha } ) \rangle = \sigma^{-1} ( \langle Q (\alpha ) , t (\alpha ) \rangle )$ is a generalized witness for $\tilde{\alpha}$ with respect to $\tilde{M}$ and $\tilde{q}$.}
\item{$\tilde{q} = p ( \tilde{M} )$.}
\item{$\tilde{M}$ is sound and solid.}
\end{itemize}

The very first clause follows from the fact that the direct limit $\tilde{M}$ satisfies $\Pi_2$-statements which hold on a tail-end of $D \cap \tilde{\tau}$; the rest is clear.  By Lemma \ref{short protomouse condensation}, $\tilde{M} = N_{\tilde{\tau}} ( \kappa_\tau , \tilde{q})$.\\  To complete the proof, we have to show that $(\kappa_\tau , \tilde{q})$ is a strong divisor of $\tilde{M}$.  We again show that $\tilde{\theta} = \theta ( \tilde{M})$ is closed in $\tilde{M}$ relative to $\tilde{q}$.  So pick an $f \in J_{\tilde{\theta}}^E$ and $\xi < \kappa_\tau$, and fix some $\bar{\tau} \in D \cap \tilde{\tau}$ such that $\tilde{F}(f) \in range ( \sigma_{\bar{\tau} \tilde{\tau}} )$, where $\tilde{F}$ is the top extender of $\tilde{M}$.  Letting $\bar{F}$ be the top extender of $M_{\bar{\tau}}^*$, we have

\[
\tilde{F} (f) ( \tilde{q}) , \xi) \cap \kappa_\tau = \sigma_{\bar{\tau} \tilde{\tau}} ( \bar{F} (f) (q_{\bar{\tau}}^* , \xi ) \cap \kappa_\tau ) = \bar{F} (f) ( q_{\bar{\tau}}^* , \xi ) \cap \kappa_\tau \in J_{\theta_{\bar{\tau}}}^E \subset J_{\tilde{\theta}}^E \ .
\]

The second equality is a consequence of the fact that $\sigma_{\bar{\tau} \tilde{\tau}} ( \kappa_\tau ) = \kappa_\tau$, and the membership relation follows from the assumption that $( \kappa_\tau , q_{\bar{\tau}}^* )$ is a strong divisor of $N_{\bar{\tau}}$.  $\blacksquare$\\


\begin{lem} \label{tail-end of interpolants are canonical, short chain}
There is a $\bar{\tau} < \tau$ such that for every $\tau' \in D - \bar{\tau}$ we have $( \kappa_\tau , q_{\tau'}^* ) = ( \kappa_{\bar{\tau}} , q_{\bar{\tau}})$.  (That is, our interpolant is one of the canonical divisors of $N_{\bar{\tau}}$.)  Consequently, $D - \bar{\tau} = B_\tau - \bar{\tau}$.
\end{lem}


\textbf{Proof:} We will again follow the proof of the corresponding Lemma \ref{divisor push-up}.  Suppose for a contradiction that $\langle \tau_\iota \ | \ \iota < \delta \rangle$ is an increasing sequence cofinal in $\tau$ such that $( \kappa_\tau , q_{\tau_\iota}^*)$ is not a canonical divisor of $N_{\tau_\iota}$.  This means each $N_{\tau_\iota}$ either has a strong short divisor $( \kappa_{\tau_\iota} , q_{\tau_\iota})$ with $q_{\tau_\iota}$ a bottom part of $q_{\tau_\iota}^*$ and $\kappa_{\tau_\iota} > \kappa_\tau$, or else it has a long or type $2$ long divisor $\nu_\iota$ with $\nu_\iota \in q_{\tau_\iota}^*$ and $\kappa_{\nu_\iota} \geq \kappa_\tau$.\\

By the pigeonhole principle, we may assume that one of the following holds:\\

$i)$ Every $N_{\tau_\iota}$ has a canonical short divisor $(\kappa_{\tau_\iota} , q_{\tau_\iota} )$ with $q_{\tau_\iota}$ a bottom part of $q_{\tau_\iota}^*$ and $\kappa_{\tau_\iota} > \kappa_\tau$, and also (by further application of the pigeonhole principle) for all $\iota < \delta$, $|q_{\tau_\iota}| = m$ for some fixed $m \in \omega$;\\

$ii)$ Every $N_{\tau_\iota}$ has a canonical long divisor $\nu_\iota$ with $\nu_\iota \in q_{\tau_\iota}^*$, and for all $\iota < \delta$, $|p_{\tau_\iota} \ \cap \ \nu_\iota| = m$ for some fixed $m \in \omega$;\\

$iii)$ Every $N_{\tau_\iota}$ has a canonical type $2$ long divisor $\nu_\iota$ with $\nu_\iota \in q_{\tau_\iota}^*$, and for all $\iota < \delta$, $|p_{\tau_\iota} \ \cap \ \nu_\iota| = m$ for some fixed $m \in \omega$.\\

In cases $i)$ and $ii)$, we will show that the cofinal sequence of divisors can be ``pushed up the elementary chain" given by $D$, and prove that a divisor of the corresponding type is present in $N_\tau$ as well.  This will contradict the fact that $(\kappa_\tau , q_\tau)$ was the canonical divisor of $N_\tau$, and thereby prove the lemma.  In case $iii)$ we will reach a contradiction, using the fact that type $2$ long divisors can never ``beat" our interpolated divisor $( \kappa_\tau , q_{\tau_\iota}^*)$.\\




 First we consider case $i)$.  We have that each $q_{\tau_\iota}$ is a bottom part of $q_{\tau_\iota}^*$, say $q_{\tau_\iota}^* = q_{\tau_\iota} \cup s_{\tau_\iota}$.  Moreover, $\kappa_{\tau_\iota} > \kappa_\tau$ for every $\iota < \delta$.  Arguing as in the proof of Lemma \ref{divisor push-up}, we can assume without loss of generality that $\langle \kappa_{\tau_\iota} \ | \ \iota < \delta \rangle$ is a monotonic sequence, and we have already arranged that $q_{\tau_\iota}$ have a fixed size $m$.  Then $\sigma_{\tau_{\bar{\iota}} \tau_\iota } ( q_{\tau_{\bar{\iota}}} , s_{\tau_{\bar{\iota}}}) = ( q_{\tau_\iota} , s_{\tau_\iota})$ whenever $\bar{\iota} \leq \iota \leq \delta$, and we can set

\[
\begin{split}
& q = \sigma_{\tau_\iota \tau } ( q_{\tau_\iota} ) ; \\
& s = \sigma_{\tau_\iota \tau } ( s_{\tau_\iota} ) ; \\
& r = r_\tau ; \\
& \kappa = sup_{\iota < \delta} ( \kappa_{\tau_\iota} ) .
\end{split}
\]

The former two values clearly do not depend on $\iota$.  Also, $s$ might be empty, in which case $q = q_\tau$.  We first observe that $(\kappa , q )$ is a short divisor of $N_\tau$.  As $q$ is a bottom part of $q_\tau$ and $\kappa > \kappa_\tau$, this short divisor cannot be strong by the definition of $( \kappa_\tau , q_\tau )$.  Using a reflection argument we then derive a contradiction to the fact that all $( \kappa_{\tau_\iota} , q_{\tau_\iota})$ are strong.\\

Recall that $M_\tau = N_\tau ( \kappa_\tau , q_\tau )$.  To see that $( \kappa , q )$ is a short divisor of $N_\tau$, we first verify clause $b)$ from Definition \ref{short divisor}.  But $h_\tau ( \kappa \cup \{ r \cup s \} ) \supset h_\tau ( \kappa_\tau \cup \{ r \} )$, and the latter hull is cofinal in $\varrho_\tau$, so the former is as well.  Now we need to see that $h_\tau ( \kappa \cup \{ r \cup s \} ) \cap ( max (q) +1 ) = \kappa$; this will verify the rest of the properties of short divisors.  Pick a $\zeta$ from the above intersection.  Let $F_\tau$ and $F_{\tau_\iota}$ be the top extenders of $M_\tau$ and $M_{\tau_\iota}^*$, respectively.  By Lemma \ref{translation1}, $\zeta$ is of the form $F_\tau (f) ( \xi , s )$ for some $f : [\kappa_\tau]^{< \omega} \longrightarrow \kappa_\tau$ and $\xi < \kappa$.  For $\iota$ sufficiently large such that $f \in dom ( F_{\tau_\iota})$ and $\xi < \kappa_{\tau_\iota}$, set $\zeta_\iota = F_{\tau_\iota} (f) (s_{\tau_\iota} , \xi )$, then clearly $\sigma_{\tau_\iota \tau } ( \zeta_\iota ) = \zeta$.  Since $\zeta \leq max (q)$, we have $\zeta_\iota \leq max ( q_{\tau_\iota})$.  Again by Lemma \ref{translation1}, $\zeta_\iota \in h_{\tau_\iota} ( \kappa_{\tau_\iota} \cup \{ r_{\tau_\iota} \} )$.  Hence $\zeta_\iota < \kappa_{\tau_\iota}$ and, consequently, $\zeta = \sigma_{\tau_\iota \tau} ( \zeta_\iota ) < \kappa_{\tau_\iota} \leq \kappa$.  This completes the proof that $( \kappa , q)$ is a short divisor.\\

Recall that for a strong short divisor $( \kappa , q)$,  $\pi' : N'_\tau ( \kappa ) \longrightarrow N_\tau$ is the uncollapsing map associated with the hull $h_\tau ( \kappa \cup \{ p_\tau \} )$.  Let $\pi' ( r' , s' ) = ( r , s )$.  As we have already mentioned, the short divisor $( \kappa , q)$ cannot be strong.  This means that for some $\beta'$ such that $\kappa \leq \beta' < min ( s' )$ we have

\[
W_{N'_{\tau} ( \kappa )}^{ \beta' r' \cup s' } \in N'_\tau ( \kappa ) \ .
\]

This follows from Lemma \ref{characterization of strongness}.\\



Le $\pi'_\iota : N'_{\tau_\iota} ( \kappa_{\tau_\iota} ) \longrightarrow N_{\tau_\iota}$ be the associated uncollapsing map and let

\[
( \beta'_\iota , r'_{\tau_\iota} , Q'_\iota , t'_\iota ) = (\pi'_\iota )^{-1} (\beta_\iota , r_{\tau_\iota} , W_{N_{\tau_\iota}}^{\beta_\iota , r_{\tau_\iota} \cup s_{\tau_\iota}} , t_\iota ) \ .
\]

Then $\langle Q'_\iota , t'_\iota \rangle$ is a generalized witness for $\beta'_\iota$ with respect to $N'_{\tau_\iota} ( \kappa_{\tau_\iota})$ and $r'_{\tau_\iota}$.  Notice also that $\beta'_\iota \geq \kappa_{\tau_\iota}$, as $\beta_\iota \geq \kappa_{\tau_\iota}$ and $\pi'_\iota \restriction \kappa_{\tau_\iota} = id$.  Now we can proceed exactly as in the proof of Lemma \ref{divisor push-up}.  First of all, once $\langle Q'_\iota , t'_\iota \rangle$ is in $N'_{\tau_\iota} ( \kappa_{\tau_\iota})$, we know that also the standard witness $W'_\iota$ for $\beta'_\iota$ with respect to $N'_{\tau_\iota} ( \kappa_{\tau_\iota} )$ and $r'_{\tau_\iota}$ is in $N'_{\tau_\iota}$.  Then every subset of $\beta'_\iota \geq \kappa_{\tau_\iota}$ which is $\Sigma_1^{(n_{\tau_\iota})} ( N'_{\tau_\iota} ( \kappa_{\tau_\iota}))$ in $r'_{\tau_\iota}$ is $\bf{\Sigma}_1^{( n_{\tau_\iota})}$$( W'_\iota)$, and therefore is an element of $N'_{\tau_\iota} ( \kappa_{\tau_\iota} )$.  Thus, $p_{N'_{\tau_\iota} ( \kappa_{\tau_\iota})}$ must be a \textit{proper} lengthening of $r'_{\tau_\iota}$.  By Lemma \ref{characterization of strongness}, this means $(\kappa_{\tau_\iota} , q_{\tau_\iota} )$ cannot be strong, contradiction. This completes the proof of Lemma \ref{tail-end of interpolants are canonical, short chain} under the assumption $i)$.\\

Next we consider assumption $ii)$: Suppose for a contradiction that $\langle \tau_\iota \ | \ \iota < \delta \rangle$ is an increasing sequence cofinal in $\tau$ such that for all $\iota < \delta$, $( \kappa_\tau , q_{\tau_\iota}^*)$ is not the canonical divisor of $N_{\tau_\iota}$ because $N_{\tau_\iota}$ has a canonical long divisor $\nu_\iota$ with $\nu_\iota \in q_{\tau_\iota}^*$.  (Recall also that unstable levels have two canonical divisors, and our current assumption is that $( \kappa_\tau , q_{\tau_\iota}^*)$ is \textit{not} one of them.)  We have also arranged that for all $\iota < \delta$, $|p_{\tau_\iota} \ \cap \ \nu_\iota| = m$ for some fixed $m \in \omega$.\\

As before, the ordinals $\nu_\iota$ form a thread through the direct limit system $\langle N_{\tau_\iota} , \sigma_{\tau_{\bar{\iota}} \tau_\iota} \ | \ \bar{\iota}, \iota < \delta \rangle$.  It follows that the critical points of the extenders $E_{\nu_\iota}$ also form a thread; and since the critical points of the embeddings $\sigma_{\tau_{\bar{\iota}} \tau_\iota}$ are all $> \Lambda > \kappa_{E_{\nu_\iota}}$, it follows that for all $\iota$, $\kappa_{E_{\nu_\iota}}$ is the same ordinal; call it $\kappa_\nu$.\\

If $\kappa_\nu < \kappa_\tau$, then all the levels $N_{\tau_\iota}$ are unstable.  But then $( \kappa_\tau , q_{\tau_\iota}^*)$ is a canonical divisor for $N_{\tau_\iota}$ after all (it is one of the two canonical divisors considered in the unstable case), contradiction.  So we have that $\kappa_{\nu_\iota} \geq \kappa_\tau$.  We wish to show that $\nu = \sigma_{\tau_\iota \tau } ( \nu_\tau )$ is a long divisor for $N_\tau$, with $\kappa_\nu = \kappa_{E_\nu}$ (so we are not in the unstable case with $(\kappa_\tau , q_\tau)$ and $\nu$ as the two canonical divisors of $N_\tau$).  This will contradict our assumption that $(\kappa_\tau , q_\tau)$ is a canonical divisor of $N_\tau$.\\

We need to verify that the properties of long divisors from Definition \ref{long divisor} hold for $\nu$ over $N_\tau$.  $a)$ follows from what we have already said.  $b)$ and $c)$ follow from arguments just like those used in the proof of $i)$ above: for $b)$, we know that the short divisor-hull $Hull_{n+1}^{N_\tau} ( \kappa_\tau \cup r_\tau )$ is cofinal in $\varrho_n (N_\tau)$, so certainly the strictly larger long divisor-hull $Hull_{n+1}^{N_\tau} ( Z \cup r_\tau )$, where $Z = i_{E_\nu}`` (\kappa_\nu^+)$, is cofinal as well.  For $c)$, we again use Lemma \ref{translation1}: Pick a $\zeta$ from $Hull_{n+1}^{N_\tau} ( Z \cup r_\tau ) \cap \lambda^+$.  Let $F_\tau$ and $F_{\tau_\iota}$ be the top extenders of $M_\tau$ and $M_{\tau_\iota}^*$, respectively.  By Lemma \ref{translation1}, $\zeta$ is of the form $F_\tau (f) ( \xi , s )$ for some $f : [\kappa_\tau]^{< \omega} \longrightarrow \kappa_\tau$ and $\xi \in Z$.  For $\iota$ sufficiently large such that $f \in dom ( F_{\tau_\iota})$, set $\zeta_\iota = F_{\tau_\iota} (f) (s_{\tau_\iota} , \sigma_{\tau_\iota \tau}^{-1} (\xi) )$.  Then $\zeta_\iota \in Z_\iota = i_{E_{\nu_\iota}}`` (\kappa_\nu^+)$.  Also we have $\sigma_{\tau_\iota \tau } ( \zeta_\iota ) = \zeta$, because all relevant points in the foregoing definition are threads through our direct limit system.  It follows that $\zeta \in Z$, which verifies $c)$ of Definition \ref{long divisor}.\\



Finally, to see $d)$, note that $\lambda_\nu$ is a thread in our direct limit system, and if there is a total extender in $N_\tau$ with critical point $\lambda_\nu$, it must be $\sigma_{\tau_\iota \tau} (F)$, where $F$ is a total extender in $N_{\tau_\iota}$ with critical point $\lambda_{\nu_\iota}$; but this violates $d)$ of long divisorhood for $\nu_\iota$ in $N_{\tau_\iota}$.  This completes the verification that $\nu$ is a long divisor, and hence proves Lemma \ref{tail-end of interpolants are canonical, short chain} under the assumption $ii)$.\\

At last we prove the lemma under assumption $iii)$.  This time we have that $\langle \tau_\iota \ | \ \iota < \delta \rangle$ is an increasing sequence cofinal in $\tau$ such that for all $\iota < \delta$, $( \kappa_\tau , q_{\tau_\iota}^*)$ is not a canonical divisor of $N_{\tau_\iota}$ because $N_{\tau_\iota}$ has a canonical type $2$ long divisor $\nu_\iota$ with $\nu_\iota \in q_{\tau_\iota}^*$.  As before, recall also that unstable levels have two canonical divisors, and our current assumption is that $( \kappa_\tau , q_{\tau_\iota}^*)$ is \textit{not} one of them.  But this contradicts Lemma \ref{type 2 long never beat other divisors}.\\


We have proved the Lemma in Cases $1)$ and $2)$.  $\blacksquare$\\








\textbf{Proof of Lemma \ref{B-tau is a club on a tail-end} in Cases $3)$, $4)$ and $5)$}:\\









Let $\nu$ be $\nu_{N_\tau}$ if $N_\tau = M_\tau$ is long pluripotent (Case $3)$), and $\nu$ be the canonical long or type $2$ long divisor of $N_\tau$ in Cases $4)$ or $5)$.  As before, we first define $D$ to be the set of all $\bar{\tau} \in \mathcal{S}^1 \cap \tau$ satisfying:


\begin{itemize}
\item{$\bar{\nu}$ is a long or type $2$ long divisor of $N_{\bar{\tau}}$, where $\bar{\nu} \in p(N_{\bar{\tau}})$ is such that $|p(N_\tau) \cap \nu| = |p(N_{\bar{\tau}} \cap \bar{\nu} )|$;}
\item{There is a map $\sigma_{\bar{\tau} \tau } : N_{\bar{\tau}}(\bar{\nu}) \longrightarrow M_\tau$ that is $\Sigma_0$-preserving with respect to the language of coherent structures and such that}
\end{itemize}

\indent \indent $a)$ $\bar{\tau} = crit (\sigma_{\bar{\tau} \tau })$ and $\sigma_{\bar{\tau} \tau } (\bar{\tau}) = \tau $;\\

\indent \indent $b)$ $\sigma_{\bar{\tau} \tau } (\bar{\nu}) = \nu$;\\

\indent \indent $c)$ for each $\alpha \in d(M_{\bar{\tau}})$ there is a generalized solidity witness $Q_\tau (\alpha )$ for $\alpha$ with respect to $M_\tau$ and $d(M_\tau)$ such that $Q_\tau (\alpha ) \in range ( \sigma_{\bar{\tau} \tau } )$.\\


In other words, $D$ is the set of $\bar{\tau}$ corresponding to interpolants of $M_\tau$ such that the interpolation embedding maps $\bar{\tau}$ to $\tau$.  We want to see that $D$ is club in $\tau$, and that on a tail-end of $\bar{\tau}$, the interpolants witnessing membership in $D$ are the canonical protomice corresponding to $N_{\bar{\tau}}$.

\begin{lem}
$D$ is unbounded in $\tau$.
\end{lem}

\textbf{Proof:}  By Lemma \ref{cofinal interpolants}, there are cofinally many $\bar{\tau} < \tau$ such that there is an interpolant $M^*_{\bar{\tau}}$ of $M_\tau$ with the interpolation embedding mapping its critical point $\bar{\tau}$ to $\tau$.  By \ref{long pluripotent interpolates to long protomouse} for Case $3)$, or \ref{associated long protomouse interpolates to long protomouse} for Cases $4)$ and $5)$, these interpolants are long protomice; and by \ref{long protomouse condensation}, their associated ppm's are levels $N_{\bar{\tau}}$ of $W$, so they witness $\bar{\tau} \in D$. $\blacksquare$\\


\begin{lem}
$D$ is closed in $\tau$.
\end{lem}

\textbf{Proof:}  We shall closely follow the proof of Lemma \ref{S-1 short D closed}.  Let $\tilde{\tau} < \tau$ be a limit point of $D$.  Then $\tilde{\tau} \in \mathcal{S}$.  Form the direct limit $\langle \tilde{M} , \sigma_{\bar{\tau} \tilde{\tau}} \ | \ \bar{\tau} \in D \cap \tilde{\tau} \rangle$ of the diagram $\langle M_{\bar{\tau}}^* , \sigma_{\tau^* \bar{\tau}} \ | \ \tau^* \leq \bar{\tau} \ \& \ \tau^* , \bar{\tau} \in D \cap \tilde{\tau} \rangle$.  As before we have the $\Sigma_0$-preserving map $\sigma : \tilde{M} \longrightarrow M$ defined by $\sigma_{\bar{\tau} \tilde{\tau}} (x) \longrightarrow \sigma_{\bar{\tau} \tau} (x)$, so $\tilde{M}$ is well-founded and we can consider it to be transitive.  The arguments from the proof of Lemma \ref{S-0 D closed} can be modified in a straightforward way to obtain the following properties of $\tilde{M}$ and $\sigma$.  In the clauses below, $\bar{\tau}$ is an arbitrary element of $D \cap \tilde{\tau}$.

\begin{itemize}
\item{$\tilde{M}$ is a coherent structure.}
\item{$\sigma_{\bar{\tau} \tilde{\tau}} (\bar{\tau}) = \tilde{\tau}$ and $\sigma \circ \sigma_{\bar{\tau} \tilde{\tau}} = \sigma_{\bar{\tau} \tau}$.}
\item{$crit(\sigma) = \tilde{\tau}$ and $\sigma (\tilde{\tau}) = \tau$.}
\item{$h_{\tilde{M}} ( \Lambda \cup \{ \tilde{q} \} ) = \tilde{M}$, where $\tilde{q} = \sigma_{\bar{\tau} \tilde{\tau}} ( q_{\bar{\tau}}^* )$, so $\varrho_\omega (\tilde{M}) = \varrho_1 (\tilde{M}) = \Lambda$ and $\tilde{q}$ is a very good parameter for $\tilde{M}$.}
\item{$\sigma(\tilde{q}) = q_\tau$.}
\item{Let $\tilde{\alpha} \in \tilde{q}$ and $\sigma (\tilde{\alpha}) = \alpha$.  So $\alpha \in q_\tau$.  If $\langle Q ( \alpha ) , t(\alpha ) \rangle$ is a generalized witness for $\alpha$ with respect to $M_\tau$ and $q_\tau$ and $\langle Q (\alpha ) , t ( \alpha ) \rangle \in range ( \sigma_{\bar{\tau} \tau} )$ for some $\bar{\tau} \in D \cap \tau$ then $ \langle Q (\tilde{\alpha}), t ( \tilde{\alpha } ) \rangle = \sigma^{-1} ( \langle Q (\alpha ) , t (\alpha ) \rangle )$ is a generalized witness for $\tilde{\alpha}$ with respect to $\tilde{M}$ and $\tilde{q}$.}
\item{$\tilde{q} = p ( \tilde{M} )$.}
\item{$\tilde{M}$ is sound and solid.}
\end{itemize}

The very first clause follows from the fact that the direct limit $\tilde{M}$ satisfies $\Pi_2$-statements which hold on a tail-end of $D \cap \tilde{\tau}$; the rest is clear.  By Lemma \ref{long protomouse condensation}, $\tilde{M} = N_{\tilde{\tau}} ( \tilde{\nu})$, where $\tilde{\nu} = \sigma^{-1} ( \nu )$. $\blacksquare$\\

Finally we show

\begin{lem} \label{tail-end of interpolants are canonical, long chain}
There is a $\bar{\tau} < \tau$ such that for every $\tilde{\tau} \in D - \bar{\tau}$ we have that $\tilde{\nu}$ is a canonical divisor of $N_{\tilde{\tau}}$,
where $\tilde{\nu} \in p(N_{\tilde{\tau}})$ is such that $|p(N_\tau) \cap \nu| = |p(N_{\tilde{\tau}} \cap \tilde{\nu} )|$.  (That is, our interpolant is one of the canonical divisors of $N_{\tilde{\tau}}$.)  Consequently, $D - \bar{\tau} = B_\tau - \bar{\tau}$.
\end{lem}



\textbf{Proof:} We will again follow the proof of the corresponding Lemma \ref{divisor push-up}.  Suppose for a contradiction that $\langle \tau_\iota \ | \ \iota < \delta \rangle$ is an increasing sequence cofinal in $\tau$ such that our interpolated divisor $\tilde{\nu}_\iota$ is not a canonical divisor of $N_{\tau_\iota}$.  This means each $N_{\tau_\iota}$ either has a canonical long or type $2$ long divisor $\nu_\iota < \tilde{\nu}_\iota$, or a strong short divisor $( \kappa_{\tau_\iota} , q_{\tau_\iota})$ with $max (q_{\tau_\iota}) < \tilde{\nu}_\iota$.\\


By the pigeonhole principle, we may assume that one of the following holds:\\




$i)$ Every $N_{\tau_\iota}$ has a canonical strong short divisor $(\kappa_{\tau_\iota} , q_{\tau_\iota} )$ with $max (q_{\tau_\iota}) < \tilde{\nu}_\iota$, and also (by further application of the pigeonhole principle) for all $\iota < \delta$, $|q_{\tau_\iota}| = m$ for some fixed $m \in \omega$;\\

$ii)$ Every $N_{\tau_\iota}$ has a canonical long divisor $\nu_\iota$ with $\nu_\iota < \tilde{\nu}_\iota$, and for all $\iota < \delta$, $|p_{\tau_\iota} \ \cap \ \nu_\iota| = m$ for some fixed $m \in \omega$;\\

$iii)$ Every $N_{\tau_\iota}$ has a canonical type $2$ long divisor $\nu_\iota$ with $\nu_\iota < \tilde{\nu}_\iota$, and for all $\iota < \delta$, $|p_{\tau_\iota} \ \cap \ \nu_\iota| = m$ for some fixed $m \in \omega$.\\

In cases $i)$ and $ii)$, we will show that the cofinal sequence of divisors can be ``pushed up the elementary chain" given by $D$, and prove that a divisor of the corresponding type is present in $N_\tau$ as well.  This will contradict the fact that $(\kappa_\tau , q_\tau)$ was the canonical divisor of $N_\tau$, and thereby prove the lemma.  Case $iii)$ is in fact impossible, using the fact that type $2$ long divisors can never ``beat" our interpolated divisor $\tilde{\nu}_\iota$; this is an easy application of Lemma \ref{type 2 long never beat other divisors}.\\





 First we consider case $i)$.  We have that each $q_{\tau_\iota}$ satisfies $max (q_{\tau_\iota}) < \tilde{\nu}_\iota$.  Moreover, $\kappa_{\tau_\iota} > \kappa_{\tilde{\nu}_\iota}$ for every $\iota < \delta$, because the divisor-hull corresponding to $(\kappa_{\tau_\iota} , q_{\tau_\iota} )$ contains $\nu$ and hence contains $\kappa_{\tilde{\nu}_\iota}$.  Arguing as in the proof of Lemma \ref{divisor push-up}, we can assume without loss of generality that $\langle \kappa_{\tau_\iota} \ | \ \iota < \delta \rangle$ is a monotonic sequence, and we have already arranged that all $q_{\tau_\iota}$ have a fixed size $m$.  Exactly as in Lemma \ref{tail-end of interpolants are canonical, short chain}, we set

\[
\begin{split}
& q = \sigma_{\tau_\iota \tau } ( q_{\tau_\iota} ) ; \\
& r = p(N_\tau) - q ; \\
& s = r \cap ( \nu + 1) ; \\
& \kappa = sup_{\iota < \delta} ( \kappa_{\tau_\iota} ) .
\end{split}
\]

We first observe that $(\kappa , q )$ is a short divisor of $N_\tau$.  Recall that $M_\tau = N_\tau ( \nu)$ is a long protomouse, and the divisor-hull corresponding to $\nu$ is $h_\tau ( Z \cup p(N_\tau) - (\nu + 1))$, where $Z = i_{E_\nu}`` (\kappa_\nu^+ )$.  To see that $( \kappa , q )$ is a short divisor of $N_\tau$, we first verify clause $b)$ from Definition \ref{short divisor}.  Notice that $h_\tau ( \kappa \cup r ) \supset h_\tau ( Z \cup p(N_\tau) - (\nu + 1))$, because $\nu \in r$ and $\kappa > \kappa_\nu$, so in fact $\kappa > \kappa_\nu^+$ and all points in $Z$ are definable from $\nu$ and ordinals $\xi < \kappa_\nu^+ < \kappa$.  The divisor-hull corresponding to $\nu$ is cofinal in $\varrho_\tau$, so the hull corresponding to $(\kappa , q )$ is as well.\\

Now we need to see that $h_\tau ( \kappa \cup r) \cap ( max (q) +1 ) = \kappa$; this will verify the rest of the properties of short divisors.  Pick a $\zeta$ from the above intersection.  Let $F_\tau$ and $F_{\tau_\iota}$ be the top extenders of $M_\tau$ and $M_{\tau_\iota}^*$, respectively.  By Lemma \ref{long translation1}, $\zeta$ is of the form $F_\tau (f) ( \xi , s )$ for some $f : [\kappa_\nu^+]^{< \omega} \longrightarrow \kappa_\nu^+$ and $\xi < \kappa$; this is because $\nu \in s$ and $\kappa_\nu^+ < \kappa$, so all points in $Z$ are definable from $s$ and ordinals $< \kappa$.  Thus Lemma \ref{long translation1} tells us that the points in the short divisor-hull $h_\tau ( \kappa \cup r )$ are \textit{exactly} the ones of this form.\\



For $\iota$ sufficiently large such that $f \in dom ( F_{\tau_\iota})$ and $\xi < \kappa_{\tau_\iota}$, set $\zeta_\iota = F_{\tau_\iota} (f) (\xi , s_{\tau_\iota} )$.  Then clearly $\sigma_{\tau_\iota \tau } ( \zeta_\iota ) = \zeta$.  Since $\zeta \leq max (q)$, we have $\zeta_\iota \leq max ( q_{\tau_\iota})$.  Again by Lemma \ref{long translation1}, $\zeta_\iota \in h_{\tau_\iota} ( \kappa_{\tau_\iota} \cup r_{\tau_\iota} )$.  Hence $\zeta_\iota < \kappa_{\tau_\iota}$ and, consequently, $\zeta = \sigma_{\tau_\iota \tau} ( \zeta_\iota ) < \kappa_{\tau_\iota} \leq \kappa$.  This completes the proof that $( \kappa , q)$ is a short divisor.\\

The verification that $( \kappa , q)$ is strong goes exactly like in the proof of Lemma \ref{tail-end of interpolants are canonical, short chain}.  Thus $(\kappa , q)$ should have been chosen as the canonical divisor of $N_\tau$ instead of $\nu$, which is a contradiction.  This completes the proof of Lemma \ref{tail-end of interpolants are canonical, long chain} under the assumption $i)$.\\


Next we consider assumption $ii)$: Suppose for a contradiction that $\langle \tau_\iota \ | \ \iota < \delta \rangle$ is an increasing sequence cofinal in $\tau$ such that for all $\iota < \delta$, $\tilde{\nu}_\iota$ is not the canonical divisor of $N_{\tau_\iota}$ because $N_{\tau_\iota}$ has a canonical long divisor $\nu_\iota$ with $\nu_\iota < \tilde{\nu}_\iota$.  We have also arranged that for all $\iota < \delta$, $|p_{\tau_\iota} \ \cap \ \nu_\iota| = m$ for some fixed $m \in \omega$.\\

As before, the ordinals $\nu_\iota$ form a thread through the direct limit system $\langle M^*_{\tau_\iota} , \sigma_{\tau_{\bar{\iota}} \tau_\iota} \ | \ \bar{\iota}, \iota < \delta \rangle$.  Let $\nu' = \sigma_{\tau_\iota \tau} (\nu_\iota)$ be the image of this thread in $M_\tau$.  It follows that the critical points of the extenders $E_{\nu_\iota}$ also form a thread; and since the critical points of the embeddings $\sigma_{\tau_{\bar{\iota}} \tau_\iota}$ are all $> \Lambda > \kappa_{E_{\nu_\iota}}$, we have that for all $\iota$, $\kappa_{E_{\nu_\iota}}$ is the same ordinal; call it $\kappa'$.  Notice that $\kappa' > \kappa_{E_\nu}$, because this is true in every $N_{\tau_\iota}$; whenever we have two long divisors $\nu_1 < \nu_2$ of a premouse $N$, it is immediate that $\kappa_{\nu_1} > \kappa_{\nu_2}$ because the divisor-hull corresponding to $\nu_1$ contains the ordinal $\nu_2$ and thus contains $\kappa_{\nu_2}$.  Additionally, notice that because $\kappa_{E_\nu}^+ < \kappa'$ and $\nu$ is a point in the divisor-hull corresponding to $\nu'$, in fact every element of $Z = i_{E_\nu}`` (\kappa_{E_\nu}^+)$ is in the divisor-hull corresponding to $\nu'$.  Thus $\nu'$ has a strictly greater divisor-hull than $\nu$.\\


We need to verify that the properties of long divisors from Definition \ref{long divisor} hold for $\nu'$ over $N_\tau$.  $a)$ follows from what we have already said.  $b)$ and $c)$ follow from arguments just like those used in the proof of $i)$ above: for $b)$, we know that the divisor-hull associated with $\nu$ is cofinal in $\varrho_n (N_\tau)$, so certainly the strictly larger divisor-hull associated with $\nu'$ is cofinal as well.  For $c)$, we again use Lemma \ref{long translation1}:

Pick a $\zeta$ from $Hull_{n+1}^{N_\tau} ( Z' \cup r_\tau ) \cap \lambda^+$, where $Z' = i_{\nu'}`` ( (\kappa')^+)$.  Let $F_\tau$ and $F_{\tau_\iota}$ be the top extenders of $M_\tau$ and $M_{\tau_\iota}^*$, respectively.  By Lemma \ref{long translation1}, $\zeta$ is of the form $F_\tau (f) ( \xi , s )$ for some $f : [\kappa_\nu^+]^{< \omega} \longrightarrow \kappa_\nu^+$ and $\xi \in Z'$; here $s = r_\tau - (p(N_\tau) - (\nu + 1))$, that is, the part of the parameter between $\nu'$ and $\nu$.  Because the points in $Z'$ correspond to threads through the direct limit system, there is $M_{\tau_\iota}^*$ such that $\sigma_{\tau_\iota \tau}^{-1}(\zeta)$ is defined.  Then because $\nu_\iota$ is a long divisor, $\sigma_{\tau_\iota \tau}^{-1}(\zeta) \in  i_{\nu_\iota}`` ( (\kappa')^+)$.  Pushing this fact up to $M_\tau$ proves $c)$.






Finally, to see $d)$, note that $\lambda_{\nu'}$ is a thread in our direct limit system, and if there is a total extender in $N_\tau$ with critical point $\lambda_{\nu'}$, it must be $\sigma_{\tau_\iota \tau} (F)$, where $F$ is a total extender in $N_{\tau_\iota}$ with critical point $\lambda_{\nu_\iota}$; but this violates $d)$ of long divisorhood for $\nu_\iota$ in $N_{\tau_\iota}$.  This completes the verification that $\nu'$ is a long divisor, and hence proves Lemma \ref{tail-end of interpolants are canonical, long chain} under the assumption $ii)$.\\



We have proved the Lemma in Cases $3)$, $4)$, and $5)$.  This completes the proof of Lemma \ref{B-tau is a club on a tail-end} for $\tau \in \mathcal{S}^1$, and the construction of our $\square_{\Lambda , 2}$-sequence.  $\blacksquare$\\








\bigskip
\bigskip

\begin{thebibliography}{1}


\bibitem{weak DJ} I. Neeman and J.R. Steel {\em A Weak Dodd-Jensen Lemma} 1999: The Journal of Symbolic Logic

\bibitem{PIPM} I. Neeman and J. R. Steel {\em Plus-one Premice} 2014: Handwritten notes

\bibitem{FSPIPM} I. Neeman and J. R. Steel {\em Fine Structure for Plus-one Premice} 2014: Handwritten notes


\bibitem{equiconsistencies} I. Neeman and J.R. Steel {\em Equiconsistencies at Subcompact Cardinals} 2016: Archive for Mathematical Logic


\bibitem{zeman square proof} E. Schimmerling and M. Zeman {\em Characterization of $\square_{\kappa}$ in Core Models} 2004: Journal of Mathematical Logic

\bibitem{ZS finestructure} R. Schindler and M. Zeman {\em Fine Structure} 2009: Handbook of Set Theory, Springer



\bibitem{steel outline} J.R. Steel  {\em An Outline of Inner Model Theory} 2009: Handbook of Set Theory, Springer


\bibitem{NITCIS} J. R. Steel {\em Normalizing Iteration Trees and Comparing Iteration Strategies} 2016

\bibitem{zeman book} M. Zeman {\em Inner Models and Large Cardinals} 2002: De Gruyter

\end{thebibliography}



\hfill \break
\\


\end{document}












